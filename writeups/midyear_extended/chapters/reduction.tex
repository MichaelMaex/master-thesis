This section is mostly based on \cite[sec.\ 2.4]{berkovichSpectralTheoryAnalytic2012}. 


In this section we assume banach algebras to be equipped with their spectral norm. 
Let $A$ be a $K$-Banach algebra with the residue norm, usually $K$-affinoid. Recall that we defined 
\begin{align*}
	A^{o} &=  \{a \in A \st |a| \le 1\}  \\
	A^{oo} &=  \{a \in A \st |a| < 1\}  \\
	\widetilde A &= A^{o} / A^{oo} 
.\end{align*}
As any morphism $\phi: A \to B$ of banach algebras, maps power bounded elements to powerbounded elements this map restricts to $\phi ^{o}: A ^{o} \to B^{o}$ and  $\phi ^{oo}: A ^{oo} \to B^{oo}$, and after quotienting this yields a map $\widetilde \phi: \widetilde A \to \widetilde B$. 

\begin{definition}
	Let $A$ be a Banach $K$-algebra. Then we define the \emph{reduction map} as 
	\begin{align*}
		\pi: \mathcal{M} (A) &\longrightarrow \spec \widetilde A \\
		\chi_x &\longmapsto \widetilde {\chi_x}
	\end{align*}
	where $\chi_x: A \to L$ is a character representing a point $x$ and $\widetilde {\chi_x}: \widetilde A \to \widetilde L$ is a character representing the image of $x$ in $\spec A$.  
\end{definition}
We will often write $X = \mathcal{M} (A)$ and $\widetilde X = \spec \widetilde A$. 
This notation might be a bit misleading as the functor $X \mapsto \widetilde X$ is only uniquely determined on the category affinoid spectra and not on $K$-analytic spaces in general. 
When $X = \bigcup_{i} \mathcal{M} (A_i)$ is an affinoid cover, then the reduction maps $\pi: \mathcal{M} (A_i) \to \spec \widetilde A_i$ do glue \todo{do they in general, or do we need conditions}, but the resulting space  $\widetilde X$ and map $\pi: X \to \widetilde X$ depends on the chosen cover.  
However for generic fibers $X = \mathfrak{X} _\eta$ of an admissible formal scheme we may always take $\widetilde X = \mathfrak{X} _k$, the special fibre of the formal scheme.

This reduction map will play a central role in the rest to thesis \todo{more elegant phrasing}. 
One way to understand the map is to understand how the preimages look like. 

\begin{lemma}
	Let $A$ be a Banach $K$-algebra and $I$ and ideal in $\widetilde A$ generated by $\widetilde{f_i}, i \in J$ with  $f_i \in A^{o}$, then 
	\[
		\pi^{-1}(V(I)) = \{x \in \mathcal{M} (A) \st |f_i|_x< 1, i \in I\} 
	.\] 

	If $\overline{f} \in \widetilde A$ then \[
		\pi^{-1}(D(f)) = \{x \in \mathcal{M}(A) | |f|_x = 1 \} 
	\] 
	and this set is non-empty. 
\end{lemma}
\begin{proof}
	\todo{this is not in \cite{berkovichSpectralTheoryAnalytic2012} but I think this needs a proof, even if it is unwinding definitions}
	This proof is mostly unwinding definitions, yet I think its insight full go give it anyway as it shows us how we can work with reduction map. 

	Let $x \in \mathcal{M} (A)$ and $\chi_x: A \to \mathcal{H} (x)$ be its minimal character. 
	Then 
	\begin{align*}
		\forall i: |f_i(x)| < 1 &\iff f_i(x) \in \mathcal{H}(x)^{oo}\\
					&\iff \widetilde{f_i(x) } = \widetilde{\chi_x}(f_i) = 0 \text{ in } \widetilde{\mathcal{H} (x)}\\
					&\iff f_i \in  \ker \widetilde{\chi_x} \\
					&\iff I \supset \ker \widetilde{\chi_x} = \pi(x)\\
					&\iff \pi(x) \in V(I)
	.\end{align*}
	Hence we find \[
		\pi^{-1}(V(I)) = \{x \in \mathcal{M} (A) \st |f_i|_x< 1, i \in I\} 
	.\] 
	Taking $I = (\overline{f})$ for $f \in A^{o} \setminus A^{oo}$ and looking at the complement we find \[
		\pi^{-1}(D(f)) = \{x \in \mathcal{M}(A) | |f|_x = 1 \} 
	\] 
	as well. 
	That this is non-empty follows from the maximum modulus principle. \todo{prove maximum modulus somewhere}


	\todo{We need a result that the spectral norm is the same as the sup-norm}
\end{proof}

\begin{corollary}
	If $A$ is Noetherian (e.g.\ an affinoid algebra) then
	the reduction map $\pi: \mathcal{M} (A) \to \spec \widetilde A$ is \emph{anti-continuous}, which means that inverse image of an open is closed. 
\end{corollary}
\begin{proof}
	It is equivalent to show that then inverse image of an closed set is open.
	Let $V(I)$ be any closed in $\spec \widetilde A$.
	As $A$ is Noetherian, so is $\widetilde A$ and hence $I = (\overline{f_1}, \ldots, \overline{f_n})$ for some $f_i \in A^{o}$.  
	Hence $\pi^{-1}(V(I)) = \bigcap_{i} \eval_f^{-1}([0, 1))$ which is a finite intersection of open sets, hence open. 
\end{proof}

\begin{definition}
	The \emph{Shilov boundary} of $\mathcal{M}(A)$ is the unique minimal subset of $\mathcal{M} (A)$ on which every $f \in A$ obtains is maximum. 
\end{definition}
Apriori it is not clear whether the Shilov boundary exists, because there might not be a globally minimal set with this property. But it will in our case. 

\begin{proposition}
	Let $A$ be a strikt $K$-affinoid algebra. We write $X = \mathcal{M} (A), \widetilde X = \spec \widetilde A$ and $\widetilde X _\text{gen} $ for the set of the generic points of the irreducible components of $\widetilde X$. 
	Then 
	\begin{enumerate}
		\item The reduction map $\pi: X \to \widetilde X$ is surjective
		\item Any generic point in $\widetilde X _\text{gen} $ has a unique preimage under the reduction map $\pi$. 

		Furthermore, if $|A|_\text{sup}  = |K|$ then there is an isomorphism $\tilde \kappa(\widetilde x) \simeq \widetilde{\mathcal{H} (x)}$

	\item The preimage of the generic points $\pi^{-1}(\widetilde X_\text{gen} )$ is the \emph{shilov boundary} of $\mathcal{M} (A)$ \todo{what is the shilov boundary. 
			 
		Is it the minimal set on which every $f \in A$ contains a maximum?}
	\end{enumerate}
\end{proposition}

\begin{proof}
\begin{enumerate}
	\item Lets assume this for now \todo{either understand proof or just refer to proof in thm 7.1.5.4 in \cite{siegfriedboschNonArchimedeanAnalysisSystematic1984}}

	\item 
\end{enumerate}	
\end{proof}
	

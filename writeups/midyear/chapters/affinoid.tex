While we have seen how a scheme over $K$ can be a analytified to obtain a Berkobich space, the real building blocks of these spaces are affinoid algebras, which rougly are completed finite type algebras. 
These are also crucial for defining the structure sheaf on Berkovich spaces. 

\subsection{Banach Algebras} \label{sec:banach_algebras}
\begin{definition}
	A $K$-banach space is a $K$-vector space equipped with a norm that is complete with respect to that norm. 
\end{definition}

\begin{definition}
	Let $A, B$ be $K$-banach spaces. 
	A $K$-vector space morphism $\phi: A \to B$ is \emph{bounded} if there is some $C\in \R$ such that $\|\phi(a)\|\le C \|a\| $. 
\end{definition}

\begin{lemma}
	A morphism $A \to B$ is bounded if and only if it is continuous. 
\end{lemma}

\begin{definition}
	A Banach $K$-algebra, $A$, is a commutative, associative, unital $K$-algebra with a submultiplicative norm that is a Banach $K$-algebra.
\end{definition}
\begin{definition}
	Let $A$ be a $K$-algebra. 
	Two norms $\|\cdot \|_1, \|\cdot \|_2$on $A$ are called equivalent if and only if they induce the same topology. 

	This means that there are non-zero constants $C, C'$ such that for every $ a\in A$ we have $C\|a\|_1 \le \|a\|_2 \le C'\|a\|_1$. 
\end{definition}

\begin{definition}
	Let $A$ be an $k$ banach algebra nd $I$ be a closed\todo{do we need closed here} ideal in $A$. 
	Then we define the \emph{residue norm} on $A / I$ as 
	\begin{align*}
		\|\cdot \|_\text{res} : A / I &\longrightarrow \R \\
		a + I &\longmapsto \inf_{x \in a + I} \|x\|
	.\end{align*}
\end{definition}

To describe the cateogory of $K$-banach algebras we still need morphisms.
\begin{definition}
	Let $A, B$ be $K$-banach algbras. 
	A $K$-algebra morphism $\phi: A \to B$ is \emph{admissible} if $\phi$ is bounded (continuous) if induced map $A / \ker \phi \to \im \phi$ is a homeomorphism, where $A / \ker \phi$ is equipped with the residue norm and $\im \phi$ with subspace norm of $B$. 
\end{definition}


\subsection{Affinoid algebras} \label{sec:affinoid_algebras}
We need restrict the category of banach algebras to those that are ``finitely generated'' in some appropriate sense. 

\begin{definition}
	[2.1.2, \cite{conrad2008several}]
	Let $A$ non-archimedean ring and $r_1, \ldots, r_n$ some positive real numbers. We define \[
		A \left<r_1^{-1}t_1, \ldots, r_{n}^{-1} t_n \right> = \left\{\sum_{\nu \in \N^{n}}^{} a_\nu t^{\nu} \in A[[t_1,\ldots, t_n]]\st a_\nu r^{\nu} \to 0, \text{ as } \nu \to \infty\right\} 
	.\] 
\end{definition}
This can be thought of as the completing of the polynomial ring $A[x_1, \ldots, x_n]$ with the norm defined as 
\begin{equation}\label{eq:norm_tate}
	\left\| \sum_{\nu \in \N^{n}} a_\nu y^{\nu}\right\| = \max_{\nu \in \N^{n}} \|a_\nu r^{\nu}\|
\end{equation}
This gives a similarly defined norm on $A\left<y_1, \ldots, y_n \right>$, but there are many other interesting norms on $A \left<y_1, \ldots, y_n \right>$, as we will discover. 

\begin{definition}
	A \emph{Tate algebra} in $n$ variables is the ring \[
		T_n(K) = K\left<T_1, \ldots, T_n \right>
	,\]
	equipped with the norm as in \eqref{eq:norm_tate}.
\end{definition}


\begin{definition}
	A affinoid algebra is a $K$-algebra $A$ with such that there is an admissible surjective morphism $f: T_n(K) \to A$ for. 
\end{definition}
This means that $A$ can be written as $K\left<T_1, \ldots, T_n \right> / I$ where $I$ is some closed ideal.
This does not mean that $A$ necessarily comes equipped with the residue norm, but this doesn't matter as the open mapping theorem shows that the norm on $A$ must be equivalent to residue norm.
\subsection{Affinoid domains} \label{sec:affinoid_domains}

there are different types of affinoid domains that we often encounter. 

\begin{definition}[multiple definitions in \cite{conrad2008several}]
		there are \emph{weierstrass domains, laurent domains} and \emph{rational domains}. 
	\begin{description}
		\item[weierstrass domain] let $a_1, \ldots, a_n$ be in $a$. 
			\[
				a\left<a_1, \ldots, a_n \right> := \frac{a\left<x_1, \ldots, x_n \right>}{(x_1-a_1, \ldots, x_n - a_n)}
			.\] 
			unlike for polynomial rings, this is not an evaulation map. you should think of this as forcing the $a_i$'s to become powerbounded. 
		\item [laurent domains]
			let $a_1, \ldots, a_n, b_1, \ldots, b_m \in a$ the \[
				a\left<a_1, \ldots, a_n, b_1^{-1}, \ldots, b_m^{-1} \right> := \frac{a\left<x_1, \ldots, x_n,y_1, \ldots, y_m  \right>}{(x_1 - a_1, \ldots, x_n - a_n, b_1 y_1 - 1, \ldots, b_m y_m - 1)}
			.\] 
		\item[rational domain] with $a_1, \ldots, a_n, a' \in a$ with no common zeros, i.e. there is no $\mathfrak{m} $ that contains both of these elements.   
			\[
				a \left<\frac{a_1}{a}, \ldots, \frac{a_n}{a'} \right> := \frac{a\left<x_1, \ldots, x_n \right>}{(a' x_1 - a_1, \ldots, a' x_n - a_n)}
			.\] 
			this is essentially the affinoid version of localisation.
			but it does more than just make $a'$ invertible. 
			is also makes $\frac{a_i}{ a'}$ act like it is power bounded in some sense. 
		\item 
	\end{description}

\end{definition}
many authors define these domains by describing their space of points, be it max spec or the berkovich functor. 
the following lemma gives shows that we can use these descriptions interchangeably
\begin{lemma}
	[2.1.8 in \cite{conrad2008several}]
	suppose $a$ is a $k$-affinoid algebra and $a_1, \ldots, a_n, a_0 \in a$ be elements with no common zero (i.e. at no point in $ x \in \maxspec a$ are all $a_{i} \in \mathfrak{m}$).
	 then a map of $k$-affinoid algebras $\phi: a \to b$ factors through $a \left<\frac{a_1}{a_0}, \ldots, \frac{a_n}{a_0} \right>$ like in the diagram \[
	 \begin{tikzcd}
		 a \left<\frac{a_1}{a_0}, \ldots, \frac{a_n}{a_0} \right> \drar[dashed]  \\
		 a \uar \rar[']{\phi} & b
	 \end{tikzcd}
	 \] 
	 if and only if  $m\phi: \maxspec b \to \maxspec a$ factors through the subset $x \in m(a)$ such that $|a_j(x)| \le |a'(x)|$ for all $j$. 
\end{lemma}

\subsection{admissible opens, covers and $g$-topology} \label{sec:admissible_opens,_covers_and_g-topology}

\begin{definition}
	[2.2.6, \cite{conrad2008several}]
	let $a$ be a  $k$ affinoid algebra. 
	a subset $u$ of $\maxspec(a)$ is an \emph{admissible open} if is has a cover $\{u_i\} $ of affinoid subdomains $u_i$ satisfying the following property: 
	for any map of $k$-affinoid algebras $\phi: a \to b$ such that the induced map $f: m(b) \to m(a)$ lands in $u$, the pullback of the cover $\{f^{-1}(u_i)\} $ has a finite subcovering. 


	a covering $v = \bigcup_{i \in  i} v_i$ is an \emph{admissible cover} if for any any map $\phi:a \to b$  of $k$-affinoid algebras that the image of  $f: m(b) \to m(a)$ lies in $v$, the pullback cover $\{f^{-1}(v_i)\} $ has a refinement by a covering consisting of finitely many affinoid subdomains.  
\end{definition}
\begin{proof}
	\todo{look at this proof}
\end{proof}

we need this to define the notion of a $g$-topology, which is a grothendieck topology on the set $\maxspec(a)$. 
i'm pretty sure later we will scrap $\maxspec$ in favour of the berkovich space.



\todo{Look at temkins notes for Gerritzen-Grauert, paper in mathematische analen.}
Let $K$ be a complete non-archimedean algebraically closed field. 

\begin{definition}
	
\end{definition}

\begin{definition}
	Tate algebra \todo{strict or non-strict?}
\end{definition}
\begin{definition}
	Affinoid algebra \todo{strict or non-strict}
\end{definition}

\begin{definition}
	Berkovich spectrum?
\end{definition}



In this section we'll describe the category of $K$-analytic spaces, which is where the Berkovich spaces of schemes will live. 
The main building blocks of these spaces are affinoind algebras which are a type of complete $K$-algebras. 
These will also crucial for defining the structure sheaf on Berkovich spaces. 

\subsection{Banach Algebras} \label{sec:banach_algebras}
\begin{definition}
	A $K$-Banach space is a $K$-vector space equipped with a norm that is complete with respect to that norm. 
\end{definition}

\begin{definition}
	Let $A, B$ be $K$-banach spaces. 
	A $K$-vector space morphism $\phi: A \to B$ is \emph{bounded} if there is some $C\in \R$ such that $\|\phi(a)\|\le C \|a\| $. 
\end{definition}

\begin{lemma}
	A morphism $A \to B$ is bounded if and only if it is continuous. 
\end{lemma}

\begin{definition}
	A Banach $K$-algebra, $A$, is a commutative, associative, unital $K$-algebra with a submultiplicative norm that is a $K$-Banach space.
\end{definition}
\begin{definition}
	Let $A$ be a $K$-algebra. 
	Two norms $\|\cdot \|_1, \|\cdot \|_2$on $A$ are called equivalent if and only if they induce the same topology. 

	This means that there are non-zero constants $C, C'$ such that for every $ a\in A$ we have $C\|a\|_1 \le \|a\|_2 \le C'\|a\|_1$. 
\end{definition}

\begin{definition}
	Let $A$ be an $k$ banach algebra and $I$ be a closed ideal in $A$. 
	Then we define the \emph{residue norm} on $A / I$ as 
	\begin{align*}
		\|\cdot \|_\text{res} : A / I &\longrightarrow \R \\
		a + I &\longmapsto \inf_{x \in a + I} \|x\|
	.\end{align*}
\end{definition}

To describe the cateogory of $K$-banach algebras we still need morphisms.
\begin{definition}
	Let $A, B$ be $K$-banach algbras. 
	A $K$-algebra morphism $\phi: A \to B$ is \emph{admissible} if $\phi$ is bounded (continuous) if induced map $A / \ker \phi \to \im \phi$ is a homeomorphism, where $A / \ker \phi$ is equipped with the residue norm and $\im \phi$ with subspace norm of $B$. 
\end{definition}

\begin{remark}\label{rem:uniqueness_norm_banach_algebra}
	Note that the identity  $\id: (A, \|\cdot \|_1) \to (A, \|\cdot \|_2)$ between two banach $K$-algebras is a isomorhism in the category of $K$-banach spaces if and only if the norms $\|\cdot \|_1, \|\cdot \|_2$ are equivalent. 
	So the norm is not intrinsic to an equivalence calss of object in the cateogry of banach  $K$-algebras, but it is well defined up to equivalence. 
\end{remark}

There is also a good way to equip the tensor product of Banach $K$-algebras with a submultiplicative norm, turning is again into a Banach $K$-algebra
\begin{definition}
	Let $A, B, C$ be banach $K$-algebras with morphisms  $C\to A, C \to B$. Then the \emph{completed tensor product}, written $A \widehat \otimes_C B$, is the ring $A \otimes_C B$ with norm defined by \[
		\|x\| = \inf \left\{\max_{i = 1}^{n} |a_i||b_i|  \st n \in \N, a_i \in A, b_i \in B, x = \sum_{i = 1}^{n} a_i \otimes_C b_i\right\} 
	\] 
	for any $x \in A \otimes_C B$. 
\end{definition}
\question{Is dit tensor product altijd goed gedefinieerd? Of hebben we extra voorwaarden nodig op $A, B, C$?}

\begin{proposition}\label{prop:universal_prop_complete_tensor}
	The completed tensor product is the pushout of the diagram $A \leftarrow C \to B$ in the category of  $K$-Banach algebras. I.e.\ it satisfies the universal property that for any  $D$ and morphism $A \to D$,  $B \to D$ making the square with $C$ commute there exists a unique map $A \widehat\otimes_C B \to D$ making the diagram commute\[
	\begin{tikzcd}
		C \rar \dar & A \dar \ar[bend left]{rdd} \\
		B \rar \ar[bend right]{drr} &  A \widehat{\otimes}_C B \ar[dashed]{dr}{\exists !} \\
		 & & D
	\end{tikzcd}
	.\] 
\end{proposition}
\begin{proof}
	See \cite[][section 2.1.7]{siegfriedboschNonArchimedeanAnalysisSystematic1984}. 
\end{proof}

\subsection{Berkovich spectra of Banach algebras} \label{sec:berkovich_spectra_of_banach_algebras}


To an affinoid algebra we can also associate a Berkovich space, which will be sligtly differently defined than our adhoc definition from \cref{sec:berkovich_spaces} for any space, because we need to introduce the concept of a bounded norm.

\begin{definition}
	Let $(A, \|\cdot \|)$ be a Banach $K$-algebra. Another norm $\|\cdot \|'$ on $A$ is \emph{bounded}  if there is some constant $C$ such that  $\|a\|' \le C \|a\|$ for all $a \in A$. 
\end{definition}


\begin{remark}
	If $\|\cdot \|'$ is multiplicative then we may assume the constant $C$ to be $1$. 
	I.e.\ a multiplicative norm $\|\cdot \|'$ is bounded if and only if  for all $a \in A: \|a\|' \le \|a\|$. 
\end{remark}
\begin{proof}
	For any $n \in \N, a \in A$ we find $\|a^{n}\|' \le C \|a^{n}\| \le C \|a\|^{n}$. 
	Taking the $n$ 'th root yields \[
		\|a\|' \le \sqrt[n]{C} \|a\|
	.\] 
	Taking the limit $n \to \infty$ yields the desired inequality. 
\end{proof}

\begin{definition}\label{def:spectrum_banach_algebra}
	Let $(A, \|\cdot \|)$ be a Banach $K$-algebra. 
	The Berchovich spectrum (as a topological space) is \[
		\mathcal{M} (A) = \{x: A \to\R^{+} \cup \{0\}  \st x \text{ is a bounded multiplicative seminorm}\} 
	\] 
	equipped with the coarsest topology that makes all maps $\mathcal{M} (A) \to [0, \infty): x \mapsto |a|_x$ continuous. 
\end{definition}
Note that due to \cref{rem:uniqueness_norm_banach_algebra}, the space $\mathcal{M} (A)$ only depends on the equivalence class of $A$ in the category of Banach $K$-algebras. 

\begin{proposition}\label{prop:norm_spectrum_extends_base_field}
	Every norm in $\mathcal{M} (A)$ extends the norm on $K$. 
\end{proposition}
\begin{proof}
	Suppose $(A, \|\cdot \|)$ is a Banach $K$-algebra. 
	Let $x \in \mathcal{M} (A)$ be a norm. 
	Then for any $\lambda \in K, \lambda \ne 0$ we have $|\lambda|_x \le  \|\lambda\cdot 1\| = |\lambda| \cdot \|1\| = |\lambda|$. 
	Similarly $|\lambda|_x^{-1} = |\lambda^{-1}|_x  \le |\lambda^{-1}| = |\lambda|^{-1}$. 
	From this it follows that $|\lambda_x| = |\lambda|$
\end{proof}


\begin{proposition}
	$\mathcal{M} (A)$ is non-empty. 
	In particularly for every maximal ideal $\mathfrak{m}  \subset  A$ there is a norm $x \in \mathcal{M} (A)$ such that $\ker x = \mathfrak{m} $.
\end{proposition}
\begin{proof}
	Let $\mathfrak{m}$ be a maximal ideal of $A$. 
	Then $\frac{A}{\mathfrak{m} }$ with at least one submultiplicative norm induced by the norm on  $A$. 
	So the set of submultiplicative norms on  $A/\mathfrak{m} $ is non-empty and may be partially ordered by $|\cdot |_1 \le |\cdot |_2$ if $|a|_1 \le |a|_2$ for all $a \in A / \mathfrak{m} $. 

	So by Zorn's lemma there is a minimal such norm $|\cdot |_\text{min} $ on $\mathcal{A}  / \mathfrak{m} $. 
	I claim that this norm is multiplicative. 
	First we will show that $|\cdot |_\text{min} $ is power multiplicative. 
	Define $\rho(a) = \lim_{n \to \infty} \sqrt[n]{|a^{n}|_\text{min} } $. Then $\rho$ is powermultiplicative and $\rho \le |\cdot |_\text{min} $. Hence $\rho = |\cdot |_\text{min} $. 
	Now we will show that $|f^{-1}| = |f|^{-1}$ for any $f \ne 0$. 
	\question{Is er een mannier om te zien dat dit waar is voor de spectrale norm? Er staat iets redelijk uitgebreid in \cite[][thm.\ 1.2.1]{berkovichSpectralTheoryAnalytic2012}, maar het lijkt me alsof het eenvoudiger kan aangezien powermultiplicative ook eenvoudiger kon.}

	Now we can easily show that $|\cdot |_\text{min} $ is multiplicative. 
	Let $f, g \ne 0$ be elements in $A / \mathfrak{m} $, then \[
		|fg| \le |f|\cdot |g| \le (|f^{-1}| \cdot |g^{-1}|)^{-1} \le |(fg)^{-1}|^{-1} = |fg|. 
	\]
	We can now pull back $|\cdot |_\text{min} $ to $A$ which yields the desired norm. 
\end{proof}

\begin{corollary}\label{cor:non_vanish_invertible}
	An element $a \in A$ is invertible if and only if $|a|_x > 0, \forall x \in \mathcal{M} (A)$.
\end{corollary}
\begin{proof}
	\ltr
	Suppose $a $ is invertible. Then for any  $x \in \mathcal{M} (A)$ we have that $1 = |1|_x = |a^{-1}a|_x = |a|_x |a^{-1}|_x$, from which it is clear tha $|a|_x \ne 0$. 

	\rtl Suppose $a$ is not invertible. Then there is some maximal $\mathfrak{m}  $ ideal containing $a$. 
	By the previous proposition there is some $x \in \mathcal{M} (A)$ such that $\ker x = \mathfrak{m} $ and thus $|a|_x = 0$. 
\end{proof}

\begin{proposition}
	The topology on $\mathcal{M} (A)$ is Haussdorf and compact. 
\end{proposition}
\begin{proof}
	Haussdorf is proven in exactly the same way as \cref{prop:spec_ring_haussdorf}.
	For compactness see \cite[][thm.\ 1.2.1]{berkovichSpectralTheoryAnalytic2012}.
\end{proof}
\subsection{Affinoid algebras} \label{sec:affinoid_algebras}
We need restrict the category of Banach algebras to those that are ``finitely generated'' in some appropriate sense. 

\begin{definition}
	[2.1.2, \cite{conrad2008several}]
	Let $A$ non-archimedean ring and $r_1, \ldots, r_n$ some positive real numbers. We define \[
		A \left<r_1^{-1}t_1, \ldots, r_{n}^{-1} t_n \right> = \left\{\sum_{\nu \in \N^{n}}^{} a_\nu t^{\nu} \in A[[t_1,\ldots, t_n]]\st a_\nu r^{\nu} \to 0, \text{ as } \nu \to \infty\right\} 
	.\] 
\end{definition}
This can be thought of as the completing of the polynomial ring $A[x_1, \ldots, x_n]$ with the norm defined as 
\begin{equation}\label{eq:norm_tate}
	\left\| \sum_{\nu \in \N^{n}} a_\nu y^{\nu}\right\| = \max_{\nu \in \N^{n}} \|a_\nu r^{\nu}\|
\end{equation}
This gives a similarly defined norm on $A\left<y_1, \ldots, y_n \right>$, but there are many other interesting norms on $A \left<y_1, \ldots, y_n \right>$, as we will discover. 



\begin{definition}
	A \emph{Tate algebra} in $n$ variables is the ring \[
		T_n(K) = K\left<T_1, \ldots, T_n \right>
	,\]
	equipped with the norm as in \eqref{eq:norm_tate}.
\end{definition}

Similarly to polynomial rings, these tate algebras are \emph{free} in a sense made precise by the following proposition. 
\begin{proposition}\label{prop:universal_property_tate_algebars}
	Take $r_1, \ldots, r_n \in \R^{ +}$ and let $A$ be a $K$-banach algebra. 
	The map
	\begin{align*}
		\Phi: \hom(K\left<\underline r^{-1} \underline T \right>, A) &\longrightarrow \prod_i A_{\le r_i}  \\
		f &\longmapsto (f(T_1), \ldots, f(T_n))
	\end{align*}
	is a well defined isomorphism. 
	The hom-set is in the category of $K$-Banach algebras
\end{proposition}
\begin{proof}
	Lets first check that the map is well defined. 
	Let $f: K\left<\underline r^{-1} \underline T \right> \to A$ be such a map. 
	We need to check that $|f(T_i)| \le r_i$. 
	Suppose that  $|f(T_i)| > r_i$, then there is some $a \in K$ such that $|f(T_i)| > |a| > r_i$ ( $K$ is algebraically closed so the value group is dense). 
	We know that $(a^{-1}T_i)^{m} \mapsto 0$ as $m \to 0$, but $|a f(T_i)| \to \infty $, which contradicts the continuity of $f$. 


	As $K[\underline T]$ is dense in $K\left<\underline r^{-1} \underline T \right>$, $f$ is uniquely determined by its restriction to $K$. So $\Phi$ is injective.
	To check that $\Phi$ is surjective, it is sufficient to note that for any $a = \sum_{\nu \in \N^{n}}^{} a_\nu \underline T^{\nu}$ with $r^{\nu} |a_\nu| \to 0$ the series $\sum_{\nu \in \N^{n}} a_\nu f(T^{\nu})$ is convergent in $A$. 
\end{proof}


\begin{definition}
	A affinoid algebra is a $K$-algebra $A$ with such that there is an admissible surjective morphism $f: K\left<\underline r ^{-1} \underline T \right> \to A$ for some $\underline r = (r_1, \ldots, r_n)$. 

	An affinoid algebra is said to be \emph{strict} if it admists such morphism with $r_i  \in |K^{\times }|$ for all $i$. 
	For any group $H$, with $|K^{\times }| \subset  H \subset \R^{ +}$ the algebra $A$ is said to be \emph{$H$-strict} if and it admists such morphism with $r_i \in H$ for $i$. 
\end{definition}
This means that $A$ can be written as $K\left<r_1^{-1}T_1, \ldots, r_n^{-1}T_n \right> / I$ where $I$ is some (closed) ideal.
This does not mean that $A$ necessarily comes equipped with the residue norm, but this doesn't matter as the open mapping theorem shows that the norm on $A$ must be equivalent to residue norm.


\begin{proposition}
	Affinoid algebras are Noetherian and all ideals are closed. 
\end{proposition}
\begin{proof}
	\todo{find reference. This is stated as fact 3.1.2.1 in \cite{temkinIntroductionBerkovichAnalytic2010} but without proof or reference to proof}	
\end{proof}

\subsection{Affinoid domains} \label{sec:affinoid_domains}

In order to turn $\mathcal{M} (A)$ into a ringed space and define a structure sheaf we need an analogue of affine opens on which we know what the sections of the structure sheaf should be. 
Suprisingly these analoges are actually closed sets.  
We will describe these domains by a universal property, which when interpreted in the language of scheme theory characterises the affine open subsets of schemes. 

\begin{definition}
	Let $\mathcal{M} (A)$ be the Berkovich spectrum of an affinoid algebra $A$. 
	An ($H$-strict) affinoid domain is a closed subset $V \subset  \mathcal{M} (A)$ together with bounded morphism of ($H$-strict) affinoid domain algebras $f:A \to A_V$ \emph{representing} $V$.
	This means that 
	\begin{itemize}
		\item the image of the induced map $\phi:\mathcal{M} (A_V) \to \mathcal{M} (A)$ is $V$ 
		\item every bounded $K$-banach algebra morphism $A \to B$ such that the image of $\mathcal{M} (B)$ in $\mathcal{M} (A)$ is contained in $V$ factors in a unique way through $A_V$ \[
	\begin{tikzcd}
		A \ar{rr}{\phi} \ar{dr}& & A_V \ar[dashed]{dl} \\
				& B
	\end{tikzcd}
	.\] 
	\end{itemize}
\end{definition}

Affinoid domains are closed under finite intersections. 
\begin{lemma}\label{lem:intersection_affinoid_domain}
	If $V_1, V_2$ are affinoid domains represented by $A_{V_1}, A_{V_2}$ then $V_1 \cap V_2$ is an affinoid domain represented by $A_{V_1} \widehat\otimes_A A_{V_2}$. 
\end{lemma}
\begin{proof}
	Let $f: A \to B$ be a morphism such that the corresponding morphism  $\phi: \mathcal{M} (B) \to \mathcal{M}(A) $ lands in $V_1 \cap V_2$. In particularly $\im \phi \subset  V_1$ and $\im \phi \subset V_2$. 
	So $f$ factors uniquely through $A_{V_1}$ and $A_{V_2}$.
	Applying \cref{prop:universal_prop_complete_tensor} yields that  $f$ factors unique through $A_{V_1} \cox_A A_{V_2}$ 
	\[
	\begin{tikzcd}
		A \rar \dar & A_{V_1} \dar \ar[bend left]{rdd} \\
		A_{V_2} \rar \ar[bend right]{drr} &  A_{V_1} \cox_A A_{V_2} \ar[dashed]{dr}{\exists !} \\
		 & & B
	\end{tikzcd}
	.\] 
\end{proof}

It is not easy to describe the affinoid domains for a given $\mathcal{M} (A)$ as explicitly as in the case of $\spec$ for rings. 
But there are a couple explicit ones that will be ``enough'' to develop the theory. 
\begin{definition}
	Let $A$ be an $K$-affinoid algebra. 
	There are ($H$-strict) \emph{weierstrass domains, laurent domains} and \emph{rational domains}. 
	\begin{description}
		\item[Weierstrass domain] Let $a_1, \ldots, a_n$ be in $A$ and $r_1, \ldots, r_n \in H$. 
			\[
				A\left<r_1^{-1}a_1, \ldots, r_n^{-1}a_n \right> := \frac{a\left<r^{-1}x_1, \ldots, r^{-1}_nx_n \right>}{(x_1-a_1, \ldots, x_n - a_n)}
			.\] 
			Unlike for polynomial rings, this is not an evaulation map. You should think of this as forcing the $a_i$'s to become powerbounded. 
		\item [Laurent domains]
			Let $a_1, \ldots, a_n, b_1, \ldots, b_m \in A$ without common zeroes\footnote{This means that there is no $x \in \mathcal{M} (A)$ that makes two evaluates two or more element to $0$} and $r_1,\ldots, r_n, s_1, \ldots, s_m \in H$ then \[
				A\left<r_1^{-1}a_1, \ldots, r_n^{-1}a_n, s^{-1}_1b_1^{-1}, \ldots, s^{-1}_mb_m^{-1} \right> := \frac{A\left<r^{-1}_1x_1, \ldots, r^{-1}_nx_n,s^{-1}_1y_1, \ldots,s_m^{-1} y_m  \right>}{(x_1 - a_1, \ldots, x_n - a_n, b_1 y_1 - 1, \ldots, b_m y_m - 1)}
			.\] 
			This is the affinoid analog of localisation. It makes the $r_i^{-1}a_i$ powerbound and $s_i^{-1}b_i$ invertible. 
		\item[Rational domain] With $a_1, \ldots, a_n, a' \in A$ with no common zeros, i.e. there is no $\mathfrak{m} $ that contains both of these elements and $r_1, \ldots, r_n \in H$. 
			\[
				A \left<r_1^{-1}\frac{a_1}{a'}, \ldots, r_n^{-1}\frac{a_n}{a'} \right> := \frac{A\left<r_1^{-1}x_1, \ldots, r_n^{-1}x_n \right>}{(a' x_1 - a_1, \ldots, a' x_n - a_n)}
			.\] 
			But it does more than just make $a'$ invertible. 
			It also makes $\frac{a_i}{ a'}$ act like it is power bounded in some sense. 
	\end{description}
\end{definition}

It is clear that Weierstrass domains and Laurent domains are specific types of rational domains. 
These are affinoid domains and the following theosem describes which set $V \subset \mathcal{M} (A)$ they repersent. 
\begin{proposition}
	Let $A$ be a an ($H$-strict) affinoid ring and $a_1, \ldots, a_n, a' \in A$ without common zero.
	Then the map $A \to A\left<\frac{a_1}{a'}, \ldots, \frac{a_n}{a'} \right>$ represents the domain \[
		\mathcal{M} \left(A \left<r^{-1}_1\frac{a_1}{a'}, \ldots, r^{-1}_n\frac{a_n}{a'}\right> \right) = \{x \in \mathcal{M} (A) \st |a_i|_x \le r_i|a'|, \forall i \in 1, \ldots, n\} 
	.\] 
\end{proposition}
\begin{proof}
	Lets denote the domain as above by $V$. 
	Let $f:A \to B$ be an admissible morphism such that for the reduced morphism $\phi: \mathcal{M} (B) \to \mathcal{M} (A)$ the image $\im \phi \subset  V$.
	Note that because there are no common zeroes, this means that $|a'|_x \ne 0$ for all $x \in V$. 
	So $|f(a')|_{y} \ne 0$ for all $y \in \mathcal{M} (B)$. 
	Hence $f(a')$ is invertible in $B$ by \cref{cor:non_vanish_invertible}. 

	Recall that \[
	A\left<r^{-1}_1\frac{a_1}{a'}, \ldots, r_n^{-1}\frac{a_n}{a'} \right> = \frac{A\left<r_1^{-1}x_1, \ldots, r_n^{-1}x_n \right>}{(a' x_1 - a_1, \ldots, a' x_n - a_n)}
	.\] 

	So to define a map to $B$ that extends the map $f$ from $A$ we need to define the images $x_i$. 
	We need to be able to quotient this map from by the ideal generated by $a'x_i - a_i$. 
	So its clear that the only way to define this map is like:
	\begin{align*}
		g: A\left<r_1^{-1}x_1, \ldots, r_n^{-1}x_n \right> &\longrightarrow B \\
		x_i &\longmapsto f(a_i) / f(a')
	.\end{align*}
	This is well defined as $f(a')$ is invertible and $|f(a_i) / f(a')|_x = |a_i|_{\phi(x)} / |a'|_{\phi(x)} \le r_i$ as $\phi(x) \in V$. 
	So by \cref{prop:universal_property_tate_algebars} this defines a unique map. 
	By construction $a'x_i - a_i \in \ker g$ for every $i$. Hence this defines a map $A \left<\frac{a_1}{a'}, \ldots, \frac{a_n}{a'} \right>$ as desired. 
\end{proof}

While we don't know what general affinoid domains look like, the following theorem shows that we can get by with just rational domains. 
\begin{proposition}
	[Gerritzen-Grauert Theorem]
	A $H$-affinoid domain $V \subset  \mathcal{M} (A)$ is a finite union of $H$-strict rational domains in $X$.
\end{proposition}
\begin{proof}
	There are many very techincal proofs that depend on the theory of rigid geometry (a predecessor of Berkovich geometry).
	The easiest proof and one that is enterely in the language of Berkovich geometry is one by Temkin \cite{temkinNewProofGerritzenGrauert2005}.
\end{proof}




\subsection{Interlude: $G$-topologies} \label{sec:interlude_g_topologies}

In the next section we will turn $\mathcal{M} (A)$ into a locally ringed space. 
But we will not do this with the cannonical topology as defined in \cref{def:spectrum_banach_algebra} (at first). 
In fact the topology that we will use is not a ordinary topology at all. 
It will be a $G$-topology, which is a notion somewhere in between an ordinary topology and a Grothendieck site. 
For a $G$ topology on $X$
the category of opens will still be a collection of subsets of $X$ with inclusion maps, closed under finite intersection (but not union). But the set of covers may be restricted. 
In this section we will follow \cite[][sec. 9.1.]{siegfriedboschNonArchimedeanAnalysisSystematic1984}

\begin{definition}
	[$G$-topology]
	Let $X$ be a set. 
	A $G$-topology $\mathcal{T} $ on $X $ consists of 
	\begin{itemize}
		\item  a collection $S$ of sets in $X$ we are called \emph{admissible opens},
		\item for every admissible open $U \in S$ a set $\cov U$ of \emph{admissible covers} consisting set theoretic covers of $U$ by admissible opens,
	\end{itemize}
	which are subject to the following conditions:
	\begin{enumerate}
		\item $S$ is closed under finite intersections
		\item For any admissible open $U$ is $\{U\}$ and admissible cover of $U$.
		\item If $\{U_i\}_{i \in I} \in \cov U $ and for each $U_i$ there is a an cover $\{V_{i, j}\}_{j \in J_i} \in \cov U_i$ then $\{V_{ij}\} _{i \in I, j \in J}$ is a an admissible cover of $U$
		\item If  $U, V$ are admissible opens with $V \subset  U$ and given an admissible cover $\{U_i\}_{i \in I} $ of $U$, then $\{V \cap U_i\} _{i \in I}$ is an admissible cover of $V$. 
	\end{enumerate}
\end{definition}
\begin{example}
	\begin{itemize}
		\item Any topological space is a $G$-topological space by letting $S$ be the set of opens and the for any open  $U$ define $\cov U$ to be all the set theoretic covers. 
		\item A less obvious $G$-topological space is the following. 
			Let $X$ be any topological space. 
			Let $S$ be the set of closed sets in $X$ and let the admissible covers be the finite set theoretic covers. 
	\end{itemize}
\end{example}
This last example is somewhat related to the topology that we will eventually define on $\mathcal{M} (A)$. 
Most definitions for topological spaces carry straight over to $G$-topological spaces, for example: 
\begin{definition}
	These definitions should not be suprising, but take note how reduced structure of $G$-topologies forces us to be slightly more precise.  
	\begin{itemize}
		\item  Let $X, Y$ be $G$-topological spaces. 
			A function $f: X \to Y$ is called \emph{continuous} if every inverse of an admissible open in $X$ is an admissible open, and every inverese of an admissible cover $\{U_i\} $ of $U$ gives an admissible cover $\{f^{-1}(U_i)\} $ of $f^{-1}(U)$. 
		\item A $G$-topological space $X$ is \emph{compact} if $X$ itself is an admissibl open and every admissible cover of $X$ can be reduced to a finite admissible cover. 
		\item A $G$-topological space $X$ is \emph{disconnected} if there is a an admissible cover $\{U_i\} _{i \in I}$ of $X$ such that there we can split the index set $I = I_1 \sqcup I_2$  such that \[
				\left(\bigcup_{i \in  I_1} U_i\right)\cap \left( \bigcup_{i \in I_2}  \right)  = \emptyset, \text{ and } \bigcup_{i \in I_1} U_i, \bigcup_{i \in I_2} U_i \text{ are nonempty}
		.\] 
	\end{itemize}
\end{definition}



\subsection{The structure sheaf on $\mathcal{M} (A)$} \label{sec:the_structure_sheaf_on_ma}

First we need to define a $G$-topology. In this section it doesn't really matter whether we work with strict or $H$-strict or non-strict affinoid algebras. So we will work in the most general case and let all algebras be non-strict unless specified otherwise.  

\begin{definition}
	Let $A$ be a affinoid algebra. 
	The \emph{weak $G$-topology} on $\mathcal{M} (A)$ is the topology where the admissible opens are the affinoid domains and the covers are finite set theoretic covers of affinoid domains. 
\end{definition}

We can define a presheaf $\mathcal{O}_X$ on $X =\mathcal{M} (A)$ with the weak  $G$-topology by defining $\mathcal{O}_X(V) = \mathcal{A} _V$ for any affinoid domain $V$ represented by $\mathcal{A}_V$.
The restiction maps are given by the defining property of affinoid domains.

This presheaf turns out to be a sheaf, which will follow from the following theorem. 
\begin{theorem}[Tate's acyclicity theorem]\label{thm:tate_acyclicity}
	Let $X = \mathcal{M} (A)$ and $\{V_i\} $ be a admissible cover of $X$. 
	Then for any finite finite banach $A$ module, $M$ the Čech complex \[
		0 \to M \to \prod_{i} M \cox_A A_{V_i} \to \prod_{i, j} M \cox_A A_{V_i} \cox_A A_{V_j} \to \ldots
	\] 
	is exact. 
\end{theorem}
\begin{proof}
	For a full proof see \cite[][prop.\ 2.2.5]{berkovichSpectralTheoryAnalytic2012}. 
	The entire proof boils down to reducing this to the case of a Laurant covering $A\left<f \right>, A\left<f^{-1} \right>$ of $A$ for some $f \in A$. and showing that \[
		0 \to A \to A\left<f \right>\times  A\left<f^{-1} \right> \to A\left<f, f^{-1} \right> \to 0
	.\]  
	is exact. 
	\question{Does it make sense to give a proof of this?}
\end{proof}


\begin{corollary}
	The presheaf $\mathcal{O}_X$ is a sheaf on the weak $G$-topology of $X$.
\end{corollary}
\begin{proof}
	Let $V$ be an affinoid domain and $\{V_i\} $ a finite cover. 
	Taking $M = A_V$ in \cref{thm:tate_acyclicity} yields \[
	0 \to A_V \to \prod_{i } A_V \otimes_A A_{V_i} \to \prod_{i,g} A_V \otimes A_{V_i} \otimes A_{V_j}
	.\] 
	Rewriting each term yields that \[
	0 \to A_V \to \prod_i A_{V_i} \to \prod_{i, j} A_{V_i \cap V_j}
	\] 
	is exact. Which is exactly the sheaf condition. 
\end{proof}

The $G$-topology and the structure sheaf can be slightly refined to allow finite unions. 
\begin{definition}
	Let $X = \mathcal{M} (A)$ for some $K$-affinoid algebra $A$. 
	A set $V\subset  X$ is \emph{special} if it is a finite union of affinoid domains. 
	We write $S(X)$ for the set of \emph{special sets} on $X$. 
\end{definition}

\begin{proposition}
	$S(X)$ with finite covers define a $G$-topology on $X$. 
	The structuse sheaf  $\mathcal{O}_X$ on $X$ can be extended to a sheaf on $S(X)$ by defining \[
		\mathcal{O}_X(V) = \ker\left(\prod_i A_{V_i} \to \prod_{ij} A_{V_{ij}}\right)
	.\] 
\end{proposition}
\begin{proof}
	That this is well defined is a consequence of Tate's aciclycity theorem (\ref{thm:tate_acyclicity}). 
	\todo{Either work this out or find a source that is earlier than Wojteks thesis}
\end{proof}


This structure sheaf can also be used to define a structure sheaf on the cannonical topology as well. 
\begin{definition}\label{def:structuresheaf_cannonical_topology}
	Let $X = \mathcal{M} (A)$. Then we can define a sheaf on $\mathcal{O}_X$ on $X$ via \[
		\mathcal{O}_X\left( U \right)  = \varprojlim_{V \subset U} \mathcal{O}_X(V)
	\]   
	where $V$ runs over all the special sets.
\end{definition}
\begin{proof}
	\todo{find a proof that this still is a sheaf to reference. Too builky to work out}
\end{proof}
This turns $X$ into a locally ringed topological space. 
It is a grave abuse of notation to use $\mathcal{O}_X$ for both the structure sheaf on the $G$-topology of $X$ as well as the cannonical topology. 
It is imporant keep in mind that there are two topologies with their respective structure sheaf, namely a locally ringed space $(X, \mathcal{O}_X)$ and a locally ringed $G$-topological space $(X_G, \mathcal{O}_{X_G})$. 


\subsection{$K$-analytic spaces} \label{sec:k_analytic_spaces}

The Berkovich spectra of ($H$-strict) $K$-affinoid algebras can be glued together to obtain ($H$-strict) $K$-anlytic spaces, similar to how schemes are made out of spectra of rings. 
However, the precise construction of this is more subtle than it looks at first glance and is therefore omitted from this document. 
They will be discussed in the final version of my thesis. 

For now it is important to know the following:
A \emph{good} $K $-analytic space, is a $K$-analytic space where every point $x$ has a neighborhood isomorphic to $\mathcal{M} (A)$ for some $K$-affinoid algebra $A$. 
On good  $K$-analytic spaces we can turn the structure sheaf on the $G$-topology into one on the ordinary topology, as in  \cref{def:structuresheaf_cannonical_topology}. 

For ordinary $K$-analytic spaces we only require that some finite union affinoid domains containing $x$ is neighborhood of $x$. 

\question{Thuilliers lessen \cite[les 4]{amaurythuillierBrazilFranceSchoolWorkshop}  doen its met pre-analytische ruimten. Hij zegt zijn definitie enken de Haussdorf (dus separated) $K$-analytische ruimten geeft. Enig idee waar hij dit vandaan haalt?}





I will introduce a preliminary definition of the Berkovich spectrum of a ring, just a topological space, which will serve build intution for berkovich spaces by exploring the Berkovich spectrum of two rings $\Z$ and $K[t]$ where $K$ is a complete algebraically closed field. 

\begin{definition}
	Let $A$ be a ring. Then we define \[
		\mathcal{M} (A) = \{x: A \to \R^{+} \st \text{multiplicative seminorm on } A\} 
	.\] 
	This comes with equipped with the coarsest topology such that for every $a \in A$ the map $\mathcal{M} (A) \to \R^{+}: \|\cdot \|_x \mapsto \|f\|_x $ is continuous. 

	We will almost always deal with a relative case where $A$ is a $K$ algebra over some real valued field. In this case adjust the definition \[
		\mathcal{M} (A) = \{x : A \to \R^{ + }  \mid \text{multiplicative seminorm on }A \text{ extending the norm on } K\} 
	.\] 
	Again $\mathcal{M} (A)$ comes with the coarsest topology making all maps $\mathcal{M} (A) \to \R^{+}: x \mapsto \|a\|_x, \forall a \in A$ continuous. 
\end{definition}
We think of $\mathcal{M} (A)$ as a geometric space and thus its elements are points, usually written by a letter, like $x$. 
For an element $f \in A$ we write $|f|_x$ or $f(x)$ for  $f$ evaluated in the norm $x$. 

\begin{remark}
	$\mathcal{M} (A)$ is not only a map, but it is also a contravariant functor. 
	Let $A, B$ be two rings and $f: A \to B$ a morphism. 
	Then a seminorm $x \in \mathcal{M} (B)$ turns into a seminorm on $\mathcal{M} (A)$ by defining $\mathcal{M} (f)(x): A \to \R^{ +}: a \mapsto |f(a)|_x$.


This is analogous to the $\spec$ functor and there even is a natural transformation $\mathcal{M} \implies \spec$ defined by $\mathcal{M} (A) \to \spec A: |\cdot |_x \mapsto \ker x$. This is well defined by \cref{rem:ker_norm_ideal}. 
This comes with a induced morphism $\chi_x: A \to \mathcal{H} (x)$.
\end{remark}

There also is a generalisation of the residue field at a prime ideal. 

\begin{definition}\label{def:completed_residue_field}
	Let $A$ be a ring (possibly over $K$) and $x \in \mathcal{M} (A)$ a point in its Berkovich spectrum. 
	Then the \emph{completed residue field at $x$} is defined as \[
		\mathcal{H} (x) = (A / \ker x, |\cdot |_x)^{\wedge}
	.\] 
	This is the completion of $A / \ker x$ with respect to the norm induced by $x$. 
	\todo{its not really necessary to take the kernel here right? Completing will already do that because they will be equivalent cauchysequences?}
\end{definition}

Like $\spec$, the points in $\mathcal{M} $ can be described as \emph{characters}, i.e.  maps from $\chi: A \to L$ (over $K$) where $L$ is a complete valued field subject to the condition that $\chi_1: A \to L_1, \chi_2: A \to L_2$ are equivalent if there exists another map to a complete valued field $\chi: A \to L$  and maps $L \to L_1, L \to L_2$ extending the norm on $L$ such that the diagram 
\[
\begin{tikzcd}
	& A \ar{dl}[']{\chi_1} \ar{dr}{\chi_2} \dar{\chi} \\
	L_1 & L \lar \rar & L_2
\end{tikzcd}
\] 
commutes. 
For any point $x$ the map $\chi_x: A \to \mathcal{H} (x)$ gives a minimal representative for the character corresponding to the point $x$. 


\begin{remark}
	Later we will consider other topologies on $X\an$, which will be necessary to turn $X\an $ into a ringed space. 
	There topologies will not be topologies like we know it. They will be Grothendieck topologies, where which unlike ordinary topologies restrict the notion of a cover. 
\end{remark}

\begin{proposition}
	Let $R$ be a ring then $\mathcal{M} (R)$ is Haussdorf. 
\end{proposition}
\begin{proof}
	Let $x, y \in \mathcal{M} (R)$ be two different points. 
	Then there is some $f \in R$ such that $f(x) \ne f(y)$. 
	So there are opens $U_x, U_y \subset  \R^{ +}$ that separate $f(x), f(y)$.
	Recall that $\phi: \mathcal{M} (A) \to \R^{ + }: x \mapsto f(x)$ is continuous by definition of the topology on $\mathcal{M} (A)$. 
	So $\phi^{-1}(U_x)$ and $\phi^{-1}(U_y)$ are opens in $\mathcal{M} (R)$ separating $x,y$. 
\end{proof}


\subsection{The Berkovich spectrum of $\Z$} \label{sec:the_berkovich_spectrum_of_z}
Classyfing all norms on $\Z$ was already done by Ostrowski.
\begin{theorem}
	[Ostrowski]
	Any multiplicative norm $|\cdot |'$ on $\Z$ is either
	\begin{itemize}
		\item \emph{the trivial norm} \[
				|\cdot |_0: n \mapsto \begin{cases}
					1 & n \ne 0 \\
					0 & n = 0
				\end{cases}
			\]
		\item an \emph{archimedean norm} \[
				|\cdot |_{\infty, \epsilon}: n \mapsto |n|^{\epsilon},
			\]
			for $0 <  \epsilon \le 1$
		\item a \emph{$p$-adic norm} \[
				|\cdot |_{p, \epsilon}: t \mapsto \epsilon ^{- v_p(t)},
			\]
			for $p$ a prime numbers and $0 < \epsilon < \infty$. 
		\item a $p$-adic trivial seminorm 
			\[
			|\cdot |_{p, \infty}: n \mapsto \begin{cases}
				1 & p \nmid n \\
				0 & p \mid n
			\end{cases}
			,\] 
			for $p$ prime. 
	\end{itemize}
\end{theorem}
So for every $p \in \{\text{primes}\} \cup \{\infty\} $ we find a continuous family of norms \[
	\epsilon \mapsto|\cdot |_{p, \epsilon}
\] 
for $\epsilon \in [0, 1]$ if $p = \infty$ and else $\epsilon \in [0, \infty]$. 
When $\epsilon = 0$ these all give the trivial norm. 
So these rays glue together at at point and the Berkovich space $\mathcal{M} (\Z)$ looks like \cref{fig:berkovich-space-of-z}.
We have to be careful with the topology here. 
While each of branches is homemorphic to a closed interval, they are not glued together along an open. 
Any open containing the point $|\cdot |_0$ has to contain all but finitely many of the branches. 

\begin{figure}[h]
    \centering
    \incfig{berkovich-space-of-z}
    \caption{berkovich space of $\Z$}
    \label{fig:berkovich-space-of-z}
\end{figure}


It is a good exercise to verify that 
\begin{itemize}
	\item  $\mathcal{H}(|\cdot |_0) = \Q$ 
	\item $\mathcal{H} (|\cdot |_{\infty, \epsilon}) = \R$ for $\epsilon \in (0, 1]$
	\item $\mathcal{H} (|\cdot |_{p, \epsilon}) = \Q_p$ for $\epsilon \in (0, \infty)$
	\item $\mathcal{H} (|\cdot |_{p, \infty}) = \F_p$ 
\end{itemize}



\subsection{Berkovich specturm of $\aff^{1}_K$} \label{sec:berkovich_specturm_of_affine_line}

We will now aim to describe the Berkovich space of the polynomials in one variable  $\mathcal{M} (K[T])$ where $K$ like usual is an algebraically closed, complete field. 
We will call $\aff_K^{1}\an$, the analityfication of the affine line. 

One way to find norms on $K[T]$ would taking a point $a \in K$ and define $|f|_a = |f(a)|_K$ where $|\cdot |_K$ is the norm on $K$. 

So there is a natural embedding $K \into \mathcal{M}(K[T])$ and we often implicitly identify these points.  

Anthor way to define a norm is by choosing closed disk $B(a, r) = \{x \in K \st |x - a| \le r\} $ with center $a$, and radius $r$ and define \[
	|f|_{B(a, r)} = \sup_{x \in B(a, r)} |f(x)|
.\] 
\begin{claim}
	The norm $|f|_{B(a, r)}$ as defined above is a well defined multiplicative seminorm extending the norm on $K$. 
	If $r \in |K|$ then the sumpremum is a maximum and the maximum is obtained for some point on the boundary of $B(a, r)$.
\end{claim}
\begin{proof}
	Let $f = \sum_{i = 0}^{n} a_i T^{i}$. 
	Then for $x \in B(a, r)$ we have $|x| \le |a| + r$. 
	So  $|f(x)| \le  \sum_{i = 0}^{n} |a_i| |x|^{i} \le \sum_{i = 0}^{n}|a_i| (|a| + r)^{i} $. 
	This last bound is indenpent of $x$. Hence the supremem exists. 

	Non-archimedean triangle inequality and extending the norm on $K$ is easily checked. 
	The multiplicativity is a little bit more suttle. 

	Without loss of generality we may translate the disk such that $a = 0$. 
	Lets assume for now that $r \in |K|$.
	Then we may also rescale such that $r = 1$. 

	Clearly the seminorm is multiplicative for elements in  $K$, i.e.\ for every $\lambda \in K, f \in K[t]: |\lambda \cdot f|_{B(a, r)} = |\lambda|\cdot |f|_{B(a, r)}$.  Let $f, g \in K[T]$, which we may rescale by an element of $K$, such that the largest coefficient of $f, g$ has norm $1$. 
	Then $f, g \in R[T]$ and their  reductions modulo  $\pi$ $\overline{f}, \overline{g} \in k[T]$ are nonzero.
	\todo{finish this}
\end{proof}

\begin{remark}\label{rem:norm_disk}
	From the proof it follows that if $f = \sum_{i = 0}^{n} b_i (T-a)^{i}$ is the taylor expansion of $f$ around $a$ then for any $r$
	\[
		|f|_{B(a, r)} = \max_{i}\{  |b_i|r^{i}\}
	.\] 
\end{remark}
\todo{maybe give a more rigorous proof of this}


So the space of closed discs in $K$, which includes points, aka discs of radius zero embeds as well in $\mathcal{M} (K[T])$. 
As we will see in a second these will make up almost all of the points on $\mathcal{M} (K[T])$. 
This means that $\mathcal{M}(K[T])$ is connected. 
Let $a \in K$ be any point. Then we can define a line $l_a: [0, \infty) \to  \aff_K^{1, \text{an}}: r\mapsto B(a, r)$. 


Because in a ultrametric space any point in a ball is a centre of ball, we see that for any two points $a, b$ the lines $l_a, l_b$ consinde for any $r \ge |a - b|$. 
So $\aff_K^{1,\text{an}}$ is not only connected, it is path connected!

We have been missing a few points on $\aff_K^{1, \text{an}}$. 
The Berkovich classification theorem gives all points. 





\todo{flesh this out}


\begin{theorem}
	[Berkovich Classifiaction theorem, \cite{bakerarizona}] \todo{find reference in Berkovich's paper}
	The points $x \in \aff_K^{1,\text{an}}$ is given by decreasing sequences of closed disks $B_n = B(a_n, r_n)$ in $K$ by the formula \[
	|f|_x = \lim_{n \to \infty} |f|_{B_n}
	.\] 
	and two such sequences $B_n, B'_n$ define the same norm if and only if:
	 \begin{itemize}
		\item Both sequences have the same non-emtpy intersection, i.e.\ $\bigcap_{n = 1}^{\infty} B_n = \bigcap_{n = 1}^{\infty} B_n' \ne \emptyset$. 
			In this case the intersection $B = \bigcap_{n \in N} B_n$ is a closed disk (possibly a point) and $x = |\cdot |_B$. 
		\item Both sequences have empty intersection, i.e.\ $\bigcap_{n = 1}^{\infty} B_n = \bigcap_{n = 1}^{\infty} B_n' \ne \emptyset$  and for every $n$ there are $m, m'> n$ such that $B_n \supset  B'_{m'}$  and $B'_n \supset B_m $.
	\end{itemize}
	This means that we can classify the points in $\aff ^{1,\text{an}}_K$ into four types depending on $B = \bigcap_{n \in \N} B_n$. 
	\begin{description}
		\item[Type I:] $B = \{a\} $ a point and the $|f|_x = |f(a)|$. 
		\item[Type II:] $B = B(a, r)$ with $r \in |K^* |$ and $|f|_x = |f|_{B(a, r)}$
		\item[Type III:] $B = B (a, r)$ with $r \not\in |K^*|$ and $|f|_x = |f|_{B(a, r)}$
		\item[Type IV:] $B = \emptyset$
	\end{description}
\end{theorem}
As you can see our previous discussion has excluded type four points. We will see an example of such point in \cref{ex:type4point}. 

\begin{proof}
	I will only show the proof that every norm is given by such a sequence as the proofs of all other points are technical and give little insight. 

	Let $x \in \aff_K^{1, \text{an}}$ be a multiplicative seminorm on $K[T]$. 
	As  $K$ is algebraically closed every polynomial factors into linear factors $(T-a)$. 
	Hence $x$ is uniquely detemrined by its value $|T-a|_x$ for each $a \in K$. 
	We will write $\rho_a = |T-a|_x$.
	Let  $a, b \in K$ such that $\rho_a \le \rho_b$. 
	Then  \begin{equation}\label{eq:proof_berk_class_1}
		|a - b|_x = |(T-b) - (T - a)|_x \le \max \{\rho_a, \rho_b\} = \rho_b 
	.\end{equation} 
	Hence $a \in B_{b, \rho_b}$ and thus  $B(a, \rho_a) \subseteq B(b, \rho_b)$. 

	Define $\rho := \inf_{a \in K} \rho_a$ and let $a_1, a_2, \ldots$ be a sequence such that $(\rho_{a_i})_{i \in \N}  $ is a non-increasing sequence converging to $\rho$. 
	By the previous remark this leads to decreasing sequence of disks \[
		B(a_1, \rho_{a_1}) \supset B(a_2, \rho_{a_2}) \supset B(a_3, \rho_{a_3}) \supset\ldots
	.\] 
	Now we will check that $|\cdot |_x = \lim_{n \to \infty} |\cdot |_{B(a_i, \rho_i)}$ by verifying that both norms agree on $T- b$ for every $b \in K$. 
	 There are two cases. Either $|T - b|_x = \rho_b = \rho$ or $\rho_b \ge \rho_{a_i}$ for some $i$. 

	 If $ \rho_b = \rho$ then by \eqref{eq:proof_berk_class_1} we find $|b - a_i|_x \le \rho_n$ for all $i$.
	 So \[
		 |T - b|_{B(a_i, \rho_i)} = |(T - a_i) - (b_i - a_i)|_{B(a_i, \rho_i)} = \max(\rho_n, |b - a_i|) = \rho_n
	 .\]
	 So $\lim_{i \to \infty} |T- b|_{B(a_i, \rho_i)} = \rho = |T - b|$. 

	 If $\rho_b > \rho$ then for $n \gg 0$ we have $\rho_b > \rho_a$ and thus $|b - a|_x \le \rho_b$, thus \[
		 |T - b|_{B(a_i, \rho_i)} = |(T - a_i) - (b -a_i)|_{B(a_i, \rho_i)} = \max (\rho_n, |b - a_i|) = |b - a_i| = |T - b|_x
	 .\]   
	 Taking the limit yields $\lim_{i \to \infty} |T- b| _{B(a_i, \rho_{i})} = |T - b|_x$.
\end{proof}

There are other ways of recognizing type 1, 2, 3 and 4 points, which will generalize better, when we look at berkovich spectra of other curves. One is topological and the other one is purely algebraic. 

\begin{proposition}
	Let $x \in \aff_K^{1,\text{an}}$. 
	Let $C = \pi_0(\aff_K^{1, \an}\setminus \{ x\} )$ be the set of connected components the punctured affine line. 
	\begin{itemize}
		\item If $x$ is of type I  or IV then $C$ has only one element.
		\item If $x$ is of type II then there a natural morphism $C \simeq k \cup \{\infty\} $. Recall that $k = \tilde K = K^{o} / K^{oo} $. 
			So $x$ is an branch point where infinitely many branches connect. 
		\item If $x = |\cdot |_{B(a, r)}$ is of type III then $|C| = 0$. The two components are all disks containing $B(a, r)$ and all disks contained in $B(a, b)$. 
	\end{itemize}
\end{proposition}\todo{should I prove this?}
\todo{what about tangents?}
The algebraic way to classify the points of $\aff_K^{1,\text{an}}$ is by using Abhyankar's inequality. 
\begin{definition}
	Let $K \subset L$ be a non-archimedean field extension. 
	Then we define \[
		E_{\frac{L}{ K}} = \dim \frac{|L^{\times }|}{|K^{\times }|} \otimes_\Z \Q, \qquad F_{L / K} = \trdeg(\tilde L / \tilde K)
	.\] 
\end{definition}

\begin{theorem}[Abhyankar inequality]
	Let $K \subset H \subset L$ be non-archimeadean rings extending each others norm. Assume that $L$ is algebraic over $\hat{H}$ and that $H$ is of trancendence degree $n$ over $K$. Then \[
	E_{L / K} + F_{L / K}  \le n
	.\] 
\end{theorem}

If $x \in \aff_K^{1, \text{an}}$ is a norm (not a seminorm) then its residue field $\mathcal{H} (x)$ is a completing on $K(T)$ with respect to the norm induced by $x$. As $K(T)$ is of trancendence degree 1 Abhyankar's inequality states that \[
	E_{\mathcal{H} (x) / K} + F_{\mathcal{H} (x) / K} \le 1
.\] 
This also allows us to classify points, because either both $E_{\mathcal{H} (x) / K}, F_{\mathcal{H} (x) / K}$ are zero, or exactly one is $1$. 
\begin{proposition}[2.3.3.3 in \cite{temkinIntroductionBerkovichAnalytic2010}]
	Let $x \in \aff_K^{1, \text{an}}$, then 
	\begin{itemize}
		\item $x$ is of type 1 if $\mathcal{H} (x) = K$
		\item $x$ is of type 2 if $F_{\mathcal{H} (x) / K} = 1$ and $E_{\mathcal{H} (x) / K} = 0$. 
		\item $x$ is of type 3 if $F_{\mathcal{H} (x) / K} = 0$ and $E_{\mathcal{H} (x) / K} = 1$.
		\item $x$ is of type 4 if $\mathcal{H} (x) \subsetneq K$ and  $F_{\mathcal{H} (x) / K} = E_{\mathcal{H} (x) / K} = 0$.
	\end{itemize}
\end{proposition}
\begin{proof}
	\todo{I believe this for type 1,2,3 but how do i see tyep 4? Just elimination?}
\end{proof}


\todo{picture of the affine line}

\begin{example}[type 4 point]\label{ex:type4point}
	\todo{work out an example}
\end{example}







\todo{Show the description of type I, II, III, IV points and space of disks}





Let $R$ be any ring. On rings like $\Z$ and $\Q$ and $\C$ the absolute value function gives away of measuring how low \emph{large} each element is. 
These types of functions are called (semi)norms, absolute values or valuations. 
\begin{definition}
	Let $R$ be a ring. A \emph{absolute value on $R$} is a function $|\cdot |: R \to \R^{+}$ satisfying for all $a, b \in R$
	\begin{enumerate}
		\item $|a| = 0 \iff a = 0$ 
		\item $|a \cdot b| = |a| \cdot |b|$ 
		\item $|a + b| \le |a| + |b|$
	\end{enumerate}
\end{definition}
Note that $R$ is not the zero ring, then $|1| = 1$ because $|1|^2 = |1| $ and $|1|\ne 0 $. 
\begin{example}
	The usual definition of absolute value on $\Z, \Q, \R$ is an absolute value. 
\end{example}
\begin{example}
	On any ring we can define the trivial absolute value as \[
	 |a| = \begin{cases}
		 1 & a \ne 0 \\
		 0 & a = 0
	 \end{cases}
	.\] 
\end{example}
\begin{example}
	A weirder absolute value on $\Z$ is the \emph{$p$-adic absolute value}, where $p$ is a prime. 
	Its defined as $|a|_p = p^{-n}$ where $n$ is the largest integers such that $p^{n} \divides a$, if $a \ne 0$ and $|0|_p = 0$. 


	This $p$-adic absolute value can be extended to an absolute value on $\Q$ by defining $|a / b|_p = \frac{|a|_p}{|b|_p}$. Let $\frac{a}{b}$ be any fraction. 
	I like to think about this in the following way. 
	Factor out the largest powers of $p$ out of $a$ and $b$ and write $a = p^{n} a', b = p^{m} b'$. Then $\frac{a}{ b} = p^{n - m} \frac{a'}{ b'}$ and $|a / b|_p = p^{m - n}$. 
	So the $p$-adic absolute values only cares about what powers of $p$ occur and doesn't care about numbers coprime to $p $.
\end{example}
As we will see in \cref{sec:berkovich_spectrum_of_Z}, we have esssentially listed all absolute values on $\Z$ and $\Q$ as examples above.  

Absolute values are a special case of (semi)norms
\begin{definition}
	Let $R$ be any ring. A \emph{seminorm on $R$} is a function $|\cdot |: R \to \R^{+}$ satisfying for all $a, b \in R$ 
	\begin{enumerate}
		\item $|0| = 0, |1| = 1$ 
		\item $|a \cdot b| \le |a | \cdot |b|$  (submultiplicativity)
		\item $|a + b| \le |a + b|$ (triangle inequality)
	\end{enumerate}
	A seminorm is called \emph{multiplicative} if it also satisfies a stronger version of 2. 
	\begin{enumerate}
		\item [4.]  $|a \cdot  b| = |a | \cdot |b|$
	\end{enumerate}
	A seminorm that is only satisfies $2$ is sometimes called a \emph{submultiplicative seminorm} if we need to stress that the norm is not required to be multiplicative. 
\end{definition}

\begin{definition}
	The \emph{kernel} of a seminorm $|\cdot |: R \to \R^{+}$ is defined as \[
		\ker(|\cdot |) = \{a \in \R \st |a| = 0\} 
	.\] 
\end{definition}
\begin{remark}\label{rem:ker_norm_ideal}
	Let $ |\cdot |$ be a norm on $R$. Then $\ker |\cdot |$ is an ideal of $R$. If $|\cdot |$ is multiplicative then $\ker |\cdot |$ is even a prime ideal!

	In some sense seminorms are analogous to ideals and and multiplicative seminorsm are analogous to prime ideals. 
\end{remark}
\begin{proof}
	Let $a, b \in \ker |\cdot |, c \in R$. Then $0 \le |a + b| \le |a|+|b| = 0$. So  $a + b \in \ker |\cdot |$. 
	Also $0 \le |c \cdot a| \le |c|\cdot |a| = 0$. So $c \cdot a \in \ker |\cdot |$. 
	This shows that $\ker |\cdot  |$ is an ideal. 
	
	Suppose that $|\cdot |$ is multiplicative. 
	Let $a, b\in R $ such that $a\cdot b \in \ker |\cdot |$ then  $|a|\cdot |b| = |a\cdot b| = 0$. 
	Hence either $|a| = 0$ or  $|b|=0$. 
	So either  $a \in \ker |\cdot |$ or $b \in \ker |\cdot |$. 
\end{proof}

\begin{definition}
	Let $R$ be a ring and $|\cdot |$ are seminorm. 
	Then $|\cdot |$ is called a \emph{norm} if $\ker |\cdot | = 0$, i.e.\ $|a| = 0 \iff a = 0$. 
\end{definition}

So an absolute value is a multiplicative norm. 

\begin{lemma}
	Any seminorm $|\cdot |$ on a field $F$ is a norm, i.e.\ has trivial kernel. 
\end{lemma}
\begin{proof}
	We have that $|1| = 1$. So $1 \not\in \ker |\cdot |$ and thus $\ker |\cdot |$ is an ideal of $F$ different from $F$. 
	As $F$ is a field this means that $\ker |\cdot | = (0)$. So $|\cdot |$ is norm. 
\end{proof}

\begin{definition}
	A (semi)norm $\|\cdot \|$ on $R$ is called \emph{non-archimedean} if it satisies the following stronger version of the triangle inequality, \[
	\forall a, b \in \R: \|a + b\| \le \max \{\|a\|, \|b\|\} 
	.\] 

	Confusingly a \emph{Archimedean (semi)norm} is a norm that is not non-archimedean.
\end{definition}

\begin{exercise}[\cite{nicaiseNONARCHIMEDEANGEOMETRY}]
	Let $k$ be a field.
	Show that a absolute value  $|\cdot |$ on $k$ is non-Archemedean if and only if the image of $\Z$ in $k$ is bounded. 
\end{exercise}
\begin{proof}
	Suppose $|\cdot |$ is non-archimedean. Then for any positive integers $n$ we have \[
	|n| = \left|\sum_{i = 1}^{n} 1 \right| \le \max \{1, 1, \ldots, 1\}  = 1
	.\] 
	And for negative integers we have $|n| = |-n| \le 1$. So $\|\cdot \|$ is bounded on the image of $\Z$ by $1$. 


	Suppose that $|\cdot |$ is bounded on the image of $\Z$ by $c$. 
	Let $x, y \in k$. Then \[
		|(x + y)^{n}| = \left| \sum_{i = 0}^{n} \binom{i}{n} x ^{i} y ^{n-i}\right| \le n\cdot c \max(x, y)^{n}
	.\] 
	So \[
		|x + y| \le \sqrt[n]{c\cdot n}  \max(x, y)
	.\] 
	Taking $n \to \infty$ yields the non-archimedean triangle inequality. 
\end{proof}

\begin{corollary}
	Let $A \subset  B$ an inclusion of rings  equipped with multiplicative such that the norm on $B$ extends the norm on $A$. 
	Then $A$ is non-archimedean if and only if $B$ is non-archimedean. 
\end{corollary}

\begin{remark}
	Suppose $R$ is non-archimedean and we have $a, b \in R$ such that $|a| \ne |b|$ then the triangle inequality is an equality \[
	|a + b| = \max \{|a|, |b|\} 
	.\] 
	So in some sense equality is the generic case. 
\end{remark}
It may sound like a really strong property to have, and that not many norms will be non-archimedean. 
But as you will see in this thesis, most norms that pop up are non-archimedean. 
In fact the only complete valued fields with archimedean norms are $\R$ and $\C$ \todo{find reference}.



\begin{definition}
	equivalent norm\todo{define this}
	{\color{red} This is a problem because equivalent norms and equivalent valuations are different as far as I understand}
\end{definition}

Note that the definition a non-archimedean norm only depends on the multiplicative structure on $\R^{+}$ and no longer depends on the additive structure. 
Sometimes it is more natural define norms in terms the isomorphic group $\R, +$. 
These are typically called valuationas. 


\begin{definition}
	Let $R$ be a ring. A valution on $R$ is a function $v:R \to \R \cup \{\infty\} $ such that 	
	\begin{itemize}
		\item $v(0) = \infty$, $v(1) = 0$. 
		\item $v(a\cdot b) = v(a) + v(b)$
		\item $v(a + b) \le \min\{v(a), v(b)\}$
	\end{itemize}
\end{definition}

\begin{remark}
	Note that there is essentially no difference between a non-archimedean absolute value and a valuation because one can take for absolute value $|\cdot |$ defines a valution $x \mapsto -\log |x|$ and a valution $v$ defines a non-archimedean absolute value $x\mapsto e^{- v(x)}$. 
\end{remark}



\subsection{Non-Archimedean rings and fields} \label{sec:non-archimedean_rings_and_fields}


\begin{theorem}\label{thm:norm_finite_field_ext}
	If $L$ is a finite field extension of a complete non-archimedean field $K$, then there is a unique extension of the norm of $K$ to $L$ and this extension is also non-archimedean.
\end{theorem}
\begin{proof}
	This a very technical result and for the proof we refer to \cite[][appendix A]{boschLecturesFormalRigid2014}. 
\end{proof}

\begin{corollary}
	If $K$ is a complete non-archimedean field then the norm of  $K$ extends uniquely to its algebraic closure $\overline{K}$.
\end{corollary}
\begin{proof}
	Recall that \[
	\overline{K} = \varprojlim_{L \supset K} L
	.\] 
	where $L$ ranges over all algebraic extensions of $K$. 
	As all these algebraic extensions have unique norms extending  the norm on $K$ and for $L_1 \supset L_2 \supset K$ the norm on  $L_1$ must restrict to norm on $L_2$ by uniqueness, it is clear that the norm functions $L \to \R^{+}$ glue to a map $\|\cdot \|: \overline{K} \to \R^{+}$. 

	It is easy to verify that this is a norm on $\overline{K}$ extending the norm on $K$. 
\end{proof}

\begin{theorem}
	[Krasner's Lemma]
	Let $K$ be a non-archimedean algebraically closed field. 
	Then its completion, $\widehat K$ is algebraically closed as well. 
\end{theorem}
\begin{proof}
	This follows from the continuity of roots. See \cite[][lem A.6]{boschLecturesFormalRigid2014} for a detailed argument. 
\end{proof}

This show that we can cannonically turn any non-archimedean field $K$ into an algebraically closed field by considering $\widehat{\overline{K}}$, which is first taking the algebraic closure of $K$ and then completing it. 

\begin{definition}
	For a prime $p$ we define the \emph{p-adic complex numbers} as \[
	\C_p = \widehat{\overline{\Q_p}}
	.\] 
\end{definition}

This is useful because Berkovich geometry works best when the base field is algebraically closed and complete. 

\begin{definition}
	Let $R$ be a non-archimedean ring. Then we write 
	\begin{align*}
		R^{0} &= \{a \in k \st |k| \le 1\}  \\
		R^{00} &=  \{a \in k \st |k| < 1\}  \\
		\tilde R &= R^{0} / R^{00}
	.\end{align*}
\end{definition}
If $k$ is a field then $k^{0}$ is a valuation ring and $ k^{00}$ is its maximal ideal. 

We will useally write $K$ for a non-arcimedean field, $R = K^{0}$ for its valuation ring, $k = \tilde K$ for its residue field. 

\begin{definition}
	Let $K$ be a non-archimedean ring. 
	The \emph{value group of $K$} is \[
	|K| = \{|a| \st \} 
	.\] 
\end{definition}


\subsection{Ultrametric spaces} \label{sec:ultrametric_spaces}

\begin{definition}
	A \emph{ultrametric space} is a topological spaces $(X, d)$ where the metric satisfies the non-archimedean triangle inequality. 
	\[
		d(x, y) \le \max \{d(x, z) ,d(z, y)\}, \quad \forall x, y ,z \in X
	.\] 
\end{definition}
\begin{exercise}
	If $d(x,z) \ne d(z,y)$ the inequality is an equality. 
\end{exercise}
\begin{proof}
	Suppose without loss of generality that $d(x, z) \ge d(z,y)$, so $d(x, y) \le \max \{d(x, z), d(z,y)\} = d(x, z) $. 
	Then $d(x, z) \le \max \{d(x, y), d(y,z)\}$. So $d(x, z) \le d(x, y) \le d(x, z)$. So we find equality.
\end{proof}
\begin{corollary}
	Any point of a ball (open or closed) is a center for that ball. 
\end{corollary}
\begin{corollary}
	If two balls have non-empty intersection that one must be included in the other. 
\end{corollary}
\begin{corollary}
	The topology of $X$ is totally disconneced. 
\end{corollary}

Non-archimedean fields are ultrametric spaces with the metric $d(x, y) = |x - y|$. 



\subsection{Analytfication of Vartieties} \label{sec:analytfication_of_vartieties}


Suppose $X$ be a scheme, locally of finite type over $K$.  We want define an analytification functor that takes such schemes to $K$-analytic spaces.  
\begin{definition}
	Let  $X$ be a locally finite $K$-scheme. 
	We define $X\an$ to be the unique $K$-analytic space, together with a morphism of ringed spaces $i: X\an \to X$ such for any good $K$-analytic space $Y$ and morphism of locally ringed spaces $Y \to X$ factors uniquely through $i: X\an \to X$. 
\end{definition}
So $X\an$ represents the functor from $K$-analytic spaces $Y \mapsto \hom(Y, X)$ where the hom-set is in the category of locally ringed spaces. 

This universal property is very useful, but it would be nice to have an more concrete understanding of the analytification of a $K$-variety. 
\begin{definition}
	The \emph{Berkovich analitification} of a locally finite type scheme $X$ over  $K$, as a set is \[
		X\an = \{(x, |\cdot |)  \mid x\in X, |\cdot | \text{ a norm on } k(x) \text{ extending the norm on }K \} 
	.\] 
	\todo{does this need to be smooth and proper and connected?}

	This comes equipped with a cannonical projection map $i: X\an \to X, (x, |\cdot |) \mapsto  x$.
	
	$X\an $ comes with a topology which we define to be the coarsest topology such that 
	\begin{itemize}
		\item $i: X\an \to X$ is continuous, i.e. $X\an$ is a finer space than  $X$. 
		\item For every open $U \subset X$ and $f \in \mathcal{O}_X(U)$ the map  \[
				|f|: i^{-1}(U) \to \R^{+}: (x, |\cdot |) \mapsto  |f(x)|
		\] 
		is continuous.
	\end{itemize}

	\todo{give definition of Berkovich analitification, maybe find an earlier reference than \cite{nicaiseBerkovichSkeletaBirational2016}}
\end{definition}

\begin{remark}
	If $X$ is an affine scheme $X = \spec A$ over $K$, then this construction agrees with our adhoc definition from \cref{sec:berkovich_spaces}.
	\[
		X^{\an} = \mathcal{M} (A)
	.\] 
\end{remark}


\begin{example}
	Berkovich affine line and projective line. 
	\todo{work this out}
\end{example}

\subsection{The generic fiver of formal schemes} \label{sec:the_generic_fiver_of_formal_schemes}




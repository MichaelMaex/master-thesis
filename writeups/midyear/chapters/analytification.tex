
\subsection{Analytification of Vartieties} \label{sec:analytfication_of_vartieties}


Suppose $X$ be a scheme, locally of finite type over $K$.  We want define an analytification functor that takes such schemes to $K$-analytic spaces.  
\begin{definition}
	Let  $X$ be a locally finite $K$-scheme. 
	We define $X\an$ to be the unique $K$-analytic space, together with a morphism of ringed spaces $i: X\an \to X$ such for any good $K$-analytic space $Y$ and morphism of locally ringed spaces $Y \to X$ factors uniquely through $i: X\an \to X$. 
\end{definition}
So $X\an$ represents the functor from $K$-analytic spaces $Y \mapsto \hom(Y, X)$ where the hom-set is in the category of locally ringed spaces. 

This universal property is very useful, but it would be nice to have an more concrete understanding of the analytification of a $K$-variety. 
\begin{definition}\label{def:berkovich_analytification_explicit}
	The \emph{Berkovich analitification} of a locally finite type scheme $X$ over  $K$, as a set is \[
		X\an = \{(x, |\cdot |)  \mid x\in X, |\cdot | \text{ a norm on } k(x) \text{ extending the norm on }K \} 
	.\] 
	\todo{does this need to be smooth and proper and connected?}

	This comes equipped with a cannonical projection map $i: X\an \to X, (x, |\cdot |) \mapsto  x$.
	
	$X\an $ comes with a topology which we define to be the coarsest topology such that 
	\begin{itemize}
		\item $i: X\an \to X$ is continuous, i.e. $X\an$ is a finer space than  $X$. 
		\item For every open $U \subset X$ and $f \in \mathcal{O}_X(U)$ the map  \[
				|f|: i^{-1}(U) \to \R^{+}: (x, |\cdot |) \mapsto  |f(x)|
		\] 
		is continuous.
	\end{itemize}

	\todo{give definition of Berkovich analitification, maybe find an earlier reference than \cite{nicaiseBerkovichSkeletaBirational2016}}
\end{definition}
\todo{is there some definition that also gives $G$-topology and structure sheaf?}

\begin{remark}
	If $X$ is an affine scheme $X = \spec A$ over $K$, then this construction agrees with our adhoc definition from \cref{sec:berkovich_spaces}.
	\[
		X^{\an} = \mathcal{M} (A)
	\] 
	by mapping a norm $x \in \mathcal{M} (A)$ onto $(\ker x, x')$ where $x'$ is the norm on $\kappa(\ker x)$ induced by $x$. 

	We will continue to write $\mathcal{M} (A)$ for the Berkovich analytification of $\spec A$, only this time it inclused the structure sheaf. 
\end{remark}

Like the analytification of Complex varieties there are cetrain GAGA results, which shows that under specific conditions all information is preserved by passing to the Berkovich analytification. 
\todo{write more on GAGA results}


\begin{example}[The Berkovic Affine line $\aff^{1, \text{an}}_K$]
	We already discussed the topology of $\aff^{1, \text{an}}_K$ in \cref{sec:the_berkovich_spectrum_of_z}. 
	But we can now discuss how $\aff^{1, \text{an}}$ behaves as a $K$-analytic space as well. 

	For any $r$ there is a natural inclusion $K[T] \into K\left<r^{-1} T \right>$ and this inclusion is dense. 
	Hence any (semi)norm on $K\left<r^{-1}T \right>$ is uniquely determined by its restriction to $K[T]$.
	Conversely any (semi)norm $|\cdot |_x$ on  $K[T]$ extends to a bounded (semi)norm $K\left<r^{-1}T \right>$ if and only if $r^{-1}|T|_x < 1$, i.e. $|T|_x \le r$. 

	This shows (that as sets at least)
	\[
		\aff^{1, \text{an}}_K = \bigcup_{r = 1} ^{\infty} \mathcal{M} (K\left<r^{-1} T \right>)
	.\] 

\begin{figure}[ht]
    \centering
    \incfig{affine-line-as-union-of-disks}
    \caption{Affine line as union of disks}
    \label{fig:affine-line-as-union-of-disks}
\end{figure}
\end{example}

\begin{example}
	[Berkovich Projective line]
	We can construct the analytification of $\pro^{1}_K$ the same way we construct its spectrum. 
Describe an affine cover $K[T], K[T^{-1}]$ and glue the spaces $\mathcal{M} (K[T]), \mathcal{M} (K[T^{-1})$ along $\mathcal{M} (K[T, T^{-1}])$. 
Using the description in \cref{def:berkovich_analytification_explicit} we see that $\mathcal{M} (K[T, T^{-1}]) = \aff^{1, \text{an}}_K \setminus \{0\} $.
See \cref{fig:berk_projective_line}. 
\end{example}

\begin{figure}[h]
	\centering
	\includegraphics[width=\textwidth]{figures/projective_line}
	\caption{The Berkovich projective constructed by gluing two copies of $\aff^{1, \text{an}}_K$}
	\label{fig:berk_projective_line}
\end{figure}


\subsection{The generic fiber of formal schemes} \label{sec:the_generic_fiber_of_formal_schemes}




\section{What is a model} \label{sec:what_is_a_model}

Let $K$ be a field and $S$ be a dedekind scheme whose function field is $K$. 
Let $X$ be an $n$-dimensional $K$-variety. 
The most general definition of a model is 
\begin{definition}
	A \emph{model of $X$} is a flat $S$-scheme $\mathscr X$ together with a map $i: X \to \mathscr X$, that induces an isomorphism on the generic fibre,i.e.\ $X \cong \mathscr X_\eta$.  
\end{definition}
So conceptually its a way of extending a generic fibre to a $n+1$-dimensional scheme over $S$. 
This definition is far too general for our purposes. First of all we will only be working in the case where we only pay attention to one closed point of $S$, i.e.\ we may assume that $S = \spec R$ where $R$ is a one dimensional local ring with residue field $K$.  

So that means that a model is a way to put a special fibre next to a given generic fibre. 


\section{(semi) stable models} \label{sec:(semi)_stable_models}

\section{(strict) normal crossings models} \label{sec:(strict)_normal_crossings_models}
\begin{definition}
	Let $X$ be a regular scheme and $D$ an effective divisor on $X$. 
	Then $D$ has normal crossings at a point $y \in Y$ if there exists local coordinates $f_1, \ldots, f_n \in \mathcal{O}_{Y, y}$ such that $\mathcal{O}(-D)_{Y, y}$ is generated by $f_1^{r_1} \cdots f_m^{r_m}$ for $m < n$ and $r_i$ positive integers. 
	I.e.\ the components of $D$ intersect transversally at $y$. 
\end{definition}
\todo{what about the lack of self intersections?, how does that play with etale morhpisms?}

\section{(minimal) regular models} \label{sec:(minimal)_regular_models}


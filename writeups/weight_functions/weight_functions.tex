\section{Recap on Berkovich analytification of varieties} \label{sec:recap_on_berkovich_analytification}

Let $K$ be a complete (non trivial)\todo{is this required} non-archimedean field, e.g. $K= \F_p, \C_p, \C\llbracket t \rrbracket$
We write $R$ for the ring of integers and $k$ for the residue field of $R$. 

\begin{definition}\label{def:berkovich_analytification_explicit}
	The \emph{Berkovich analitification} of a locally finite type scheme $X$ over  $K$, as a set is \[
		X\an = \{(x, |\cdot |)  \mid x\in X, |\cdot | \text{ a norm on } \kappa(x) \text{ extending the norm on }K \} 
	.\] 

	This comes equipped with a canonical projection map $i: X\an \to X, (x, |\cdot |) \mapsto  x$.
	$X\an $ comes with a topology which we define to be the coarsest topology such that 
	\begin{itemize}
		\item $i: X\an \to X$ is continuous, i.e. $X\an$ is a finer space than  $X$. 
		\item For every open $U \subset X$ and $f \in \mathcal{O}_X(U)$ the map  \[
				|f|: i^{-1}(U) \to \R^{+}: (x, |\cdot |) \mapsto  |f(x)|
		\] 
		is continuous.
	\end{itemize}
\end{definition}

\begin{definition}
	The \emph{birational points} of $X\an$ is  $X\bir = i ^{-1}(\eta_X)$ where $\eta_X$ is the generic point of $X$. 
\end{definition}
These are called the birational points of $X\an$ because they are schared by the analytifications of all varities birational to $X$.

\section{Recap on Models and Skeleta} \label{sec:recap_on_models_and_skeleta}
Let $X$ be a smooth connected $n$-dimensional $K$-variety. 
We will not assume that the scheme or the model is proper. 
\begin{definition}
	A model of $X$ is a flat separated finite type $R$-scheme $\mathscr X$, together with an isomorphism $X \simeq \mathscr X_K$.

	The model $\mathscr X$ is called a \emph{strict normal crossings divisor (sncd)}-model if its special fibre $\mathscr X_k$ is a strict normal crossings divisor of $\mathscr X$. 
\end{definition}

In a previous document I've constructed the reduction map $\pi: X\an \to \mathscr X_k$. 
There it was essential to assume that that $\mathscr X$ is proper, in order make sure that any map $\mathcal{H} (x) \to X$ extends to a map $\mathcal{H} ^{o}(x) \to \mathscr X$. 
If $\mathscr X$ is not proper, such an extension does not always exist hence the reduction map is only defined on a subset of $X\an$. 
\begin{definition}
	A point $x \in X\an $ is called a \emph{center} if the map $\mathcal{H} (x) \to X$ extends to $\mathcal{H}^{o} (x) \to \mathscr X$. 
	Note that this is unique by the valuative criterion of properness. 

	We write $\widehat{\mathscr X}_\eta$ for the set of all centers of $X\an$. 
\end{definition}

So the reduction map is defined from $\pi: \hat{\mathscr X}_\eta \to \mathscr X_k$. 

It turns out that $\hat{\mathscr X}_\eta$ is exactly the generic fibre of the completion of $\mathscr X$ introduced in a previous document. 


\subsection{Divisorial and monomial points} \label{sec:divisorial_and_monomial_points}

Let $\mathscr X_k = \sum_{i \in I} N_i E_i$ be the decomposition of $\mathscr X$ in its irreducible parts, and let $e_i$ be the generic point of $E_i$. 
As stated in \cite[thm 2.2.4]{berkovichSpectralTheoryAnalytic2012}, we know that the $e_i$ have unique inverses. 
We call $\pi^{-1}(e_i)$ the \emph{divisorial point} associated to $(\mathscr X, E_i)$.
\begin{definition}
	The \emph{divisorial points of $X\an$}, denoted by  $X^{\text{div}}$ is the set of all points that are the inverse of a generic point of irreducible component of a model of $X$. 
\end{definition}
\todo{give better characterisaton of divisorial points}

The monomial points are a way to interpolate between divisorial points. 
Let $J \subset  I$ such that $E_J := \bigcap_{j \in J} E_j \ne 0$.
Let $\xi$ be a generic point of $E_J$.

Let $f_i$ be local equations cutting out $E_i$.
Then we can choose an affine neighborhood $A = R[X_1, \ldots, X_m] / \left(\prod_{j \in J} f_j\right)$. 
Note that this means that the uniformizer can be expressed as $\pi = \prod_j f_j^{N_j}$. 
We can choose a map $R[Y_j, j \in J] \onto A, Y_j \mapsto f_j$, which is surjective by the chinese remainder theorem. \todo{check whether this is true?} 
We can now equip $R[Y_j, j\in J]$ with the norm such that $|Y_j| = \alpha_j$, and equip $A$ with the residue norm.\todo{why is this multiplicative? Why is $|f_i| = \alpha_i$?}
If $\sum_j \alpha_j \cdot N_j = 1$ this means that with respect to this norm  $|\pi| = 1$ and hence extend the norm on $R$. 
Now taking the 


\subsection{The skeleton associated to a model} \label{sec:the_skeleton_associated_to_a_model}

\section{What is a weight function} \label{sec:what_is_a_weight_function}

\subsection{Definition} \label{sec:definition}

\subsubsection{Arbitrary dimension} \label{sec:arbitrary_dimension}

\subsubsection{Curves} \label{sec:curves}
Let $\omega \in \omega_{X / K}^{\otimes m}$, i.e.\ a global section of some tensor power of the cannonical bundle. 
Now choose a divisorial point $x \in X\an$ associated to a model and divisor $(\mathscr X, E)$. 
Let $N$ be the multiplicity of $E$ in $\mathscr X_k$. 
The section $\omega$ extend to a global section of $\omega_{\mathscr X / R} ^{ \otimes m}$ on the model, also denoted by $\omega$. 
Then we define \[
	\wt_{\omega}\left(x  \right)  = \frac{\omega \cdot E + m}{N}
,\] 
where $\omega \cdot E$ is the intersection number \question{How does this work in higher dimension? In section 3.3 of \cite{nicaiseBerkovichSkeletaBirational2016} is it written as the multiplicity of $E$ in $\divisor_{\mathscr X}(\omega)$. What does that precisely mean?}
One can show that this is independent of the choise of model and divisor $(\mathscr X, E)$.  

\begin{definition}
	A divisorial point $x \in X^{\text{div}}$ is called \emph{$\omega$-essential} if has minial weight, i.e. \[
		\wt_\omega(x) = \inf \{\wt_\omega(x) | x \in X^{\text{div}}\} 
	.\] 
	The \emph{skeleton $\sk(X, \omega)$} is the closure of all $\omega$-essential points.  
\end{definition}

Maybe definitions?

\subsection{Intuition} \label{sec:intuition}


\section{Important properties} \label{sec:important_properties}


\section{Weight functions, (hyper) elliptic curves and more} \label{sec:weight_functions,_(hyper)_elliptic_curves_and_more}

Let $R$ be a complete DVR with fraction field $K$. 
Suppose $K$ is algebraically closed (for now). 
Let $C$ be a complete curve over $K$, together with a degree $n$ map $f: C\to \pro^{1}_K$, which has tame ramification. 
\begin{proposition}
Let $\omega \in \Omega _{\pro^{1}_K}^{\otimes n}(f_* R_f)$ a global section. 
Then  $f^* \omega \in \Omega^{\otimes n}_{C}$ and \[
\wt_{f^* \omega} = \wt_{\omega} \circ f^{\text{an}}
.\] 
\end{proposition}

\begin{proof}
	By the theory of weight functions it is sufficient to show that they are same on divisorial points, on some skeleton. 

	Let $x$ be a divisorial piont of $C\an$.
	Let $\mathscr{P} $ be an snc-model of $\pro^{1}_K$ such that the corresponding skeleton of $\mathscr P$ contains $f\an (x)$. 
	Then there is a induced morphism $\spec K(C) \to C \to P^{1}_K \into K$
	Let $\mathscr{C}'$ be the normalisation of $\mathcal{P} $ in $\spec K(C)$. 
	This is an $R$-scheme.
	As normalisation is local and  $C$ is complete this means that $\mathcal{C}' _\eta \simeq C$. 
	So hence $\mathscr{C}' $ is a model of $C$ and it fits into the commutate  diagram \[
	\begin{tikzcd}
		C \rar[hookrightarrow] \dar{f} & \mathscr C '\dar{\phi} \\
		\pro^{1}_K \rar[hookrightarrow] & \mathscr P
	\end{tikzcd}
	.\] 
	Where $\mathscr C' \to \mathscr P$ is dominant (even surjective). 
	Unfortunately  $\mathscr C'$ is not guaranteed to be a snc-model. 
	But by blowing up we may resolve the singularities and replace it with $\mathscr C$. 
	
	We now want to compare the weight functions 


\end{proof}


\subsection{Lets do all this again under simpler assumptions} \label{sec:lets_do_all_this_again_under_simpler_assumptions}

Let $R$ be a complete DVR with fraction field $K$. 

\begin{lemma}\label{lem:char_div_point}
	Let $X\an$ be a $K$-variety of dimension $n$. 
	The divisorial points of $X\an$ are precisely the discrete valuations $v$ on $K(X)$ that extend the valuation on $K$ and such that the $\trdeg [(K(X),v)^{\sim}, k] = n-1$.
\end{lemma}
\begin{proof}
	Let $x = (\mathscr X, E)$ be a divisorial point in $X\an$ corresponding to a model $\mathscr X$ with irreducible component $E$ in the special fibre.
	Let $\xi$ be the generic point of $E$. 
	Then clearly $(K(X), v_x)^{\sim} = (\mathcal{O}_{\xi, \mathscr X})^{\sim} = K(E)$ and hence the trancendence degree is going to be  $\dim E = n-1$.

	The converse is given by \cite[lem 2.45]{kollarBirationalGeometryAlgebraic1998}. 
\end{proof}
\begin{lemma}\label{lem:im_type_ii}
	If $f: X \to Y$ is a finite map and $x \in X\an$. Then
	\[
		x \in X^\text{div} \iff f(x) \in Y^\text{div}
	.\] 
\end{lemma}
\begin{proof}
	The map being finite means that $K(X)$ is a finite extension of $K(Y)$ and that $v_{f(x)}$ is the restriction of $v_x$ to $K(Y)$. 
	Then the residue fields of $K(X), K(Y)$ wrt.\ $v_x, v_{f(x)}$ are a finite extension \stacks{09E5}, hence of the same trancendence degree over $k$. 
	The claim now follows from \cref{lem:char_div_point}.
\end{proof}
\begin{remark}
	In the case of curves we can give a more topolgical argument for \cref{lem:im_type_ii}. 
	A divisorial point $x \in X\an$ is precisely a point with an infinite number of tangent directions, i.e.\ a branch point.
	The map $f\an: X\an \to Y\an $ is a finite branched cover. So topologically $f(x)$ is also such a branch point, hence divisorial. 	
\end{remark}

\begin{proposition}
Let $E$ be an elliptic curve  over $K$, together with a degree $2$ map $f: E \to \pro^{1}_K$ which has tame ramification (i.e. $\ch K \ne 2$). 
Let $R_f$ be the ramification divisor of $f$. 
	Let $\omega $ a global section of $\Omega_{\pro^{1}_K}^{\otimes 2}(-f_* R_f)$ (which we consider as a subsheaf of $\Omega_{\pro^{1}_K}$). 
	Then $f^* \omega$ is a global section of $\Omega_{E}^{\otimes 2}$ and \[
	\wt_{f^*\omega} = \wt_{\omega} \circ f\an
	.\] 
\end{proposition}
\begin{remark}
	I'm pretty sure the most general (tame) version of this is as follows. 
	Let $f: X \to Y $ be a morphism of smooth connected, projective curves over $K$ of degree $n$ with ramification divisor $R_f$. 
	Let $\omega \in \Omega_Y^{\otimes n}(-R_f)$, then \[
	\wt_{f^* \omega} = \wt_{\omega} \circ f\an
	.\] 
\end{remark}
\begin{proof}
	Lets first verify that $f^* \omega \in \Omega_E^{\otimes 2}$. 
	By definition of $R_f$ we have that $\Omega_E = f^*\Omega_{\pro ^{1}} \otimes R_f$.
	Hence 
	\begin{align*}
		f^* (\Omega_{\pro^{1}}^{\otimes 2}(-f_* R_f)) &= f^*(\Omega_{\pro^{1}})^{\otimes 2} \otimes f^*f_*(-R_f) \\
							      &= \Omega_{E}^{\otimes 2} \otimes \mathcal{O}(2 R_f) \otimes \mathcal{O}(-2 R_f) \\
							      &= \Omega_E 
	.\end{align*}

	As $X$ lies dense in in $X\an$ (with the metric topology) and the both the left and right hand side are continuious with respect to the metric topology, it follows that it suffices to obtain equality for $x \in X^\text{div}$.

	Let $x$ be a divisorial point in $E\an$. 
	Then $f(x)$ is divisorial in $\pro_K^{1}$. 
	Let $\mathscr{P} $ be a snc-model of $(\pro_K^{1},f_* R_f)$ whose skeleton contains $X$.
	Recall that because $E$ is a complete curve, hense $E$ is the normalisation of $\pro^{1}_K$.
	Let $N(\mathscr P, K(E))$ be the normalisation of $\mathscr P$ in $K(E)$. 
	Then this fits into a commutative diagram where the columns are relative localizations.
	As localisation is local we see that the map $E \to N(\mathscr P, K(E))$ is an isomorphism, so  $N(\mathscr P, K(E))$ is a model of $E$. 
	\[
	\begin{tikzcd}
		\spec K(E) \rar[equals] \dar[hookrightarrow]  & \spec K(E) \dar\\
		E \rar[hookrightarrow] \dar{f} & N(\mathscr P, K(E)) \dar{\phi} \\
		\pro^{1}_K \rar[hookrightarrow] & \mathscr P
	\end{tikzcd}
	.\] 
	By \stacks{0335}  $\mathscr P$ is nagata.\comment{It seems essential here that $K$ is a discretely valued. Can we salvage this in this algebraically closed case?}
	So by \stacks{0AVK} the morphism $\phi: N(\mathscr P, K(E)) \to \mathscr P$ is finite, hence projective. \todo{replace this with the easier argument form page 3 \cite{liuModelsCurvesFinitea}}
	So $N(\mathscr P, K(E))$ is a proper normal model of $E$. 
	It's crossings are locally complete intersections so  \question{why? I think we need the fact that $f$ is tamely ramified here. }
	\todo{Clearly we're missing somethign here.}

	Let call $\mathscr E$ the model that we eventually end up with. 
	Let $G$ be an irreducible component of $\mathscr E_s $ occuring with multiplicity $M$, which dominates a irreducible component $F$ of $\mathscr P_s$ occuring with multiplicity $N$. 
	The morphism $\phi$ is finite, hence its étale on an open dense subscheme of the special fibre.
	So for the purposes of computing the multiplicity of $F, G$ in $\divisor_\mathscr P ^{\text{log}}(\omega), \divisor _{\mathscr E}^{\text{ log}}(f^*\omega)$ we may assume that $\phi$ is étale. Let $\nu, \mu$ be their respective multiplicities. 
	Hence $f^*_s K_F = K_G$ and $(K_{\mathscr P} + F) = K_F$ by adjunction.
	So locally we have \[
		\nu f^*(F) = \divisor_{\mathscr E} ^{\text{log}} (f^*\omega) = \mu\cdot G  = \mu \frac{N}{M} f^* (F)  
	.\]
	Hence $\frac{\nu}{ N} = \frac{\mu}{M}$ whence $\wt_{f^*\omega}(F) = \wt_\omega(G)$. 
\end{proof}


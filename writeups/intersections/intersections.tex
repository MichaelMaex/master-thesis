\documentclass[a4paper]{article}
\usepackage[utf8]{inputenc}
\usepackage[T1]{fontenc}
\usepackage{textcomp}
\usepackage{amsmath, amssymb, amsthm}
\usepackage{geometry}
\usepackage{tikz-cd}
\usepackage{mdframed}
\usepackage{microtype}
\usepackage{hyperref}
\usepackage{cleveref}

\usepackage[backend = biber, style = alphabetic]{biblatex}
\addbibresource{../references.bib}

% figure support
\usepackage{import}
\usepackage{xifthen}
\usepackage[textsize=tiny,colorinlistoftodos,obeyDraft]{todonotes}
\newcommand{\question}[1]{\todo[color=green!40]{#1}}
\usepackage{pdfpages}
\usepackage{transparent}
\newcommand{\incfig}[1]{%
	\def\svgwidth{\columnwidth}
	\import{./figures/}{#1.pdf_tex}
}

\pdfsuppresswarningpagegroup=1


\newcommand{\N}{\mathbb{N}}
\newcommand{\Z}{\mathbb{Z}}
\newcommand{\Q}{\mathbb{Q}}
\newcommand{\C}{\mathbb{C}}
\newcommand{\R}{\mathbb{R}}
\newcommand{\F}{\mathbb{F}}
\newcommand{\pro}{\mathbb{P}}
\newcommand{\aff}{\mathbb{A}}
\newcommand{\ltr}{\par \noindent \framebox[1\width]{ $\implies$ } \hspace{.2cm}}
\newcommand{\rtl}{\par \noindent \framebox[1\width]{ $\impliedby$ } \hspace{.2cm} }

\DeclareMathOperator{\coker}{coker}
\DeclareMathOperator{\id}{Id}
\DeclareMathOperator{\im}{Im}
\DeclareMathOperator{\spec}{Spec}
\DeclareMathOperator{\maxspec}{MaxSpec}
\DeclareMathOperator{\spf}{Spf}
\DeclareMathOperator{\proj}{Proj}
\DeclareMathOperator{\red}{red}
\DeclareMathOperator{\trop}{trop}
\DeclareMathOperator{\trdeg}{TrDeg}
\DeclareMathOperator{\divisor}{Div}
\DeclareMathOperator{\cov}{Cov}

\newcommand{\into}{\hookrightarrow}
\newcommand{\onto}{\twoheadrightarrow}
\newcommand{\divides}{\mid}
\newcommand{\notdivides}{\nmid}
\newcommand{\Gan}{\ensuremath{\mathbb{G} _m ^{\mathrm{an}}}}
\newcommand{\an}{{}^{\text{an}}}
\newcommand{\cox}{\widehat{\otimes}}

\newcommand{\st}{%
  \nonscript\;
  \ifnum\currentgrouptype=16
    \,\middle|\,
  \else
    \,|\,
  \fi
  \nonscript\;}

\theoremstyle{definition}
\newtheorem{definition}{Definition}[subsection]
\newtheorem{remark}[definition]{Remark}
\newtheorem{theorem}[definition]{Theorem}
\newtheorem{example}[definition]{Example}
\newtheorem{lemma}[definition]{Lemma}
\newtheorem{claim}[definition]{Claim}
\newtheorem{proposition}[definition]{Proposition}
\newtheorem{exercise}[definition]{Exercise}
\newtheorem{corollary}[definition]{Corollary}


\author{Micha\"el Maex}


\title{Writeup Intersection Theory}
\begin{document}
	
\maketitle

The following theorem gives a good overview of some ``axioms'' of intersection theory over a basefield. 
Intersection theory for fibered surfaces is similar, but has other limitations. Nonetheless this serves as a good starting off point. 

\begin{theorem}
	[thm 1.50 \cite{kollar}]
	Suppose $S$ is a smooth projective surface over $k$ with divisors $C, D$. 
	Then there exists a pairing $(\underline\quad \cdot \underline\quad): \divisor(S) \times \divisor(S) \to \Z$ with the following properties, that also uniquely define the pairing (1-6)
	\begin{enumerate}
		\item $(C \cdot  D) \in \Z$ it is bilinear and symmetric. 
		\item The pairing factors through linear equivalence, i.e. if $C_1 \sim C_2$ then $(C_1 \cdot D) = (C_2 \cdot  D)$. 
		\item If $C, D$ are efficitive divisors with finite intersection $C \cap D$ then \[
				(C \cdot  D) = \sum_{p \in C \cap D} \dim_k \frac{\mathcal{O}_{S, p}}{(f_p, g_p)}
		\]
		where $f_p, g_p$ are the equations locally cutting out $C, D$ respectively. 
	\end{enumerate}
	The following properties concern how intersection numbers behave under birational maps, and thus are very useful in the theory of models of curves. 
	\begin{enumerate}
		\setcounter{enumi}{3}
		\item Let $h: S' \to S$ be a birational morphism. Then the intersection number doesn't change after pullback \[
				(h^* C \cdot  h^* D) = (C\cdot D)
		.\] 
		\item Let $h: S' \to S$ be a birational morphism and $E \subset S'$ an $h$-exceptional divisor, i.e.\ $h$ takes $E$ to a point. 
			Then $(h^* C \cdot E) = 0$. 
		\item Let $h: S' \to S$ be the blow-up of a smooth $k$-point and $E \subset  S'$ the exceptional divisor. 
			Then $(E \cdot E) = - 1$. 
	\end{enumerate}
\end{theorem}

\printbibliography
\end{document}

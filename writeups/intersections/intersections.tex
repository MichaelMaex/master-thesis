\documentclass[a4paper]{article}
\RequirePackage{fix-cm}
\documentclass[12pt,a4paper,oneside]{book}

\usepackage{graphicx,xcolor,textpos}
\usepackage{helvet}

\topmargin -10mm
\textwidth 160truemm
\textheight 240truemm
\oddsidemargin 0mm
\evensidemargin 0mm
% ---------------------- textpos settings ----------------------------
% Some additional settings for the cover
% --------------------------------------------------------------------

\definecolor{green}{RGB}{172,196,0}
\definecolor{bluetitle}{RGB}{29,141,176}
\definecolor{blueaff}{RGB}{0,0,128}
\definecolor{blueline}{RGB}{82,189,236}
\setlength{\TPHorizModule}{1mm}
\setlength{\TPVertModule}{1mm}


\title{Writeup Intersection Theory}
\begin{document}
	
\maketitle

The following theorem gives a good overview of some ``axioms'' of intersection theory over a basefield. 
Intersection theory for fibered surfaces is similar, but has other limitations. Nonetheless this serves as a good starting off point. 

\begin{theorem}
	[thm 1.50 \cite{kollar}]
	Suppose $S$ is a smooth projective surface over $k$ with divisors $C, D$. 
	Then there exists a pairing $(\underline\quad \cdot \underline\quad): \divisor(S) \times \divisor(S) \to \Z$ with the following properties, that also uniquely define the pairing (1-6)
	\begin{enumerate}
		\item $(C \cdot  D) \in \Z$ it is bilinear and symmetric. 
		\item The pairing factors through linear equivalence, i.e. if $C_1 \sim C_2$ then $(C_1 \cdot D) = (C_2 \cdot  D)$. 
		\item If $C, D$ are efficitive divisors with finite intersection $C \cap D$ then \[
				(C \cdot  D) = \sum_{p \in C \cap D} \dim_k \frac{\mathcal{O}_{S, p}}{(f_p, g_p)}
		\]
		where $f_p, g_p$ are the equations locally cutting out $C, D$ respectively. 
	\end{enumerate}
	The following properties concern how intersection numbers behave under birational maps, and thus are very useful in the theory of models of curves. 
	\begin{enumerate}
		\setcounter{enumi}{3}
		\item Let $h: S' \to S$ be a birational morphism. Then the intersection number doesn't change after pullback \[
				(h^* C \cdot  h^* D) = (C\cdot D)
		.\] 
		\item Let $h: S' \to S$ be a birational morphism and $E \subset S'$ an $h$-exceptional divisor, i.e.\ $h$ takes $E$ to a point. 
			Then $(h^* C \cdot E) = 0$. 
		\item Let $h: S' \to S$ be the blow-up of a smooth $k$-point and $E \subset  S'$ the exceptional divisor. 
			Then $(E \cdot E) = - 1$. 
	\end{enumerate}
\end{theorem}

\printbibliography
\end{document}

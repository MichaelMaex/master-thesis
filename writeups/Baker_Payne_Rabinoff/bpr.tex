\documentclass[a4paper]{article}
\usepackage{nomencl}
\makenomenclature
\RequirePackage{fix-cm}
\documentclass[12pt,a4paper,oneside]{book}

\usepackage{graphicx,xcolor,textpos}
\usepackage{helvet}

\topmargin -10mm
\textwidth 160truemm
\textheight 240truemm
\oddsidemargin 0mm
\evensidemargin 0mm
% ---------------------- textpos settings ----------------------------
% Some additional settings for the cover
% --------------------------------------------------------------------

\definecolor{green}{RGB}{172,196,0}
\definecolor{bluetitle}{RGB}{29,141,176}
\definecolor{blueaff}{RGB}{0,0,128}
\definecolor{blueline}{RGB}{82,189,236}
\setlength{\TPHorizModule}{1mm}
\setlength{\TPVertModule}{1mm}

\addbibresource{references.bib}
\renewcommand{\nomname}{List of Symbols}
\newcommand{\an}{\ensuremath{{}^{\mathrm{an}}}}
	

\title{Baker Payne Rabinoff and Prerequisites}
\begin{document}
\maketitle

The goal of this document is to understand the paper \cite{bakerStructureNonarchimedeanAnalytic2013}, which will contain the main ingredients of my thesis.  

	
\mbox{}
\nomenclature{\(K\)}{Algebraiclly closed field, complete wrt.\ a nonarchimedian valuation}
\nomenclature{\(v \)}{valuation on $K$}
\nomenclature{\(R \)}{valuation ring of $K$}
\nomenclature{\(\mathfrak{m}_R  \)}{maximal ideal of $R$}
\nomenclature{\(k \)}{residue field of $R$}
\nomenclature{\(\R^{+} \)}{the non-negative real numbers}


\printnomenclature

\part{Prequisites}

\section{Basic valuation theory} \label{sec:basic_valuation_theory}

Let $R$ be any ring. On rings like $\Z$ and $\Q$ we have ways of measuring how low \emph{large} each element is. We even have such functions on polynomials using the degree map. Depending on which properties these functions satisfy such ``measuring'' functions are called absolute values, norms or valuations. 
\begin{definition}
	Let $R$ be a ring. A \emph{absolute value on $R$} is a function $|\cdot |: R \to \R^{+}$ satisfying for all $a, b \in R$
	\begin{enumerate}
		\item $|a| = 0 \iff a = 0$ 
		\item $|a \cdot b| = |a| \cdot |b|$ 
		\item $|a + b| \le |a| + |b|$
	\end{enumerate}
\end{definition}
Note that $R$ is not the zero ring, then $|1| = 1$ because $|1|^2 = |1| $ and $|1|\ne 0 $. 
\begin{example}
	The usual definition of absolute value on $\Z, \Q, \R$ is an absolute value. 
\end{example}
\begin{example}
	On any ring we can define the trivial absolute value as \[
	 |a| = \begin{cases}
		 1 & a \ne 0 \\
		 0 & a = 0
	 \end{cases}
	.\] 
\end{example}
\begin{example}
	A weirder absolute value on $\Z$ is the \emph{$p$-adic absolute value}, where $p$ is a prime. 
	Its defined as $|a|_p = p^{-n}$ where $n$ is the largest integers such that $p^{n} \divides a$, if $a \ne 0$ and $|0|_p = 0$. 


	This $p$-adic absolute value can be extended to an absolute value on $\Q$ by defining $|a / b|_p = \frac{|a|_p}{|b|_p}$. Let $\frac{a}{b}$ be any fraction. 
	I like to think about this in the following way. 
	Factor out the largest powers of $p$ out of $a$ and $b$ and write $a = p^{n} a', b = p^{m} b'$. Then $\frac{a}{ b} = p^{n - m} \frac{a'}{ b'}$ and $|a / b|_p = p^{m - n}$. 
	So the $p$-adic absolute values only cares about what powers of $p$ occur and doesn't care about numbers coprime to $p $.
\end{example}
As we will see in \cref{sec:berkovich_spectrum_of_Z}, we have esssentially listed all absolute values on $\Z$ and $\Q$ as examples above.  

Absolute values are a special case of (semi)norms
\begin{definition}
	Let $R$ be any ring. A \emph{seminorm on $R$} is a function $\|\cdot \|: R \to \R^{+}$ satisfying for all $a, b \in R$ 
	\begin{enumerate}
		\item $\|0\| = 0, \|1\| = 1$
		\item $\|a \cdot b\| \le \|a \| \cdot \|b\|$ 
		\item $\|a + b\| \le \|a + b\|$ 
	\end{enumerate}
	Condition $2$ is called the submultiplicativity of $|\cdot |$ and condition 3 is called the triangle inequality. Both of these can be made stronger. 

	A seminorm is called \emph{multiplicative} if it also satisfies a stronger vrersion of 2. 
	\begin{enumerate}
		\item [4.]  $\|a \cdot  b\| = \|a \| \cdot \|b\|$
	\end{enumerate}
	A seminorm that is only satisfies $2$ is sometimes called a \emph{submultiplicative seminorm} if we need to stress that is doesn't have to multiplicative. 
\end{definition}

\begin{definition}
	The \emph{kernel} of a seminorm $\|\cdot \|: R \to \R^{+}$ is defined as \[
		\ker(\|\cdot \|) = \{a \in \R \st \|a\| = 0\} 
	.\] 
\end{definition}
\begin{remark}\label{rem:ker_norm_ideal}
	Let $\|\cdot \|$ be a norm on $R$. Then $\ker \|\cdot \|$ is an ideal of $R$. If $\|\cdot \|$ is multiplicative then $\ker \|\cdot \|$ is even a prime ideal!
	\todo{should I write a proof for this?}
\end{remark}

\begin{definition}
	Let $R$ be a ring and $\|\cdot \|$ are seminorm. 
	Then $\|\cdot \|$ is called a \emph{norm} if $\ker \|\cdot \| = 0$, i.e.\ $\|a\| = 0 \iff a = 0$. 
\end{definition}

So an absolute value is a multiplicative norm. 

\begin{lemma}
	Any seminorm $\|\cdot \|$ on a field $F$ is a norm, i.e.\ has trivial kernel. 
\end{lemma}
\begin{proof}
	We have that $\|1\| = 1$. So $1 \not\in \ker \|\cdot \|$ and thus $\ker \|\cdot \|$ is an ideal of $F$ different from $F$. 
	As $F$ is a field this means that $\ker \|\cdot \| = (0)$. So $\|\cdot \|$ is norm. 
\end{proof}

\begin{definition}
	A (semi)norm $\|\cdot \|$ on $R$ is called \emph{non-archimedean} if it satisies the following stronger version of the triangle inequality, \[
	\forall a, b \in \R: \|a + b\| \le \max \{\|a\|, \|b\|\} 
	.\] 

	Confusingly a \emph{Archimedean (semi)norm} is a norm that is not non-archimedean.
\end{definition}

\begin{exercise}[\cite{nicaiseNONARCHIMEDEANGEOMETRY}]
	Let $k$ be a field.
	Show that a absolute value  $|\cdot |$ on $k$ is non-Archemedean if and only if the image of $\Z$ in $k$ is bounded. 
\end{exercise}
\begin{proof}
	Suppose $|\cdot |$ is non-archimedean. Then for any positive integers $n$ we have \[
	|n| = \left|\sum_{i = 1}^{n} 1 \right| \le \max \{1, 1, \ldots, 1\}  = 1
	.\] 
	And for negative integers we have $|n| = |-n| \le 1$. So $\|\cdot \|$ is bounded on the image of $\Z$ by $1$. 


	Suppose that $|\cdot |$ is bounded on the image of $\Z$ by $c$. 
	Let $x, y \in k$. Then \[
		|(x + y)^{n}| = \left| \sum_{i = 0}^{n} \binom{i}{n} x ^{i} y ^{n-i}\right| \le n\cdot c \max(x, y)^{n}
	.\] 
	So \[
		|x + y| \le \sqrt[n]{c\cdot n}  \max(x, y)
	.\] 
	Taking $n \to \infty$ yields the non-archimedean triangle inequality. 
\end{proof}

\begin{remark}
	Suppose $R$ is non-archimedean and we have $a, b \in R$ such that $\|a\| \ne \|b\|$ then the triangle inequality is an equality \[
	\|a + b\| = \max \{\|a\|, \|b\|\} 
	.\] 
	So in some sense equality is the generic case. 
\end{remark}
It may sound like a really strong property to have, and that not many norms will be non-archimedean. 
But as you will see in this thesis, most norms that pop up are non-archimedean. 
In fact that $p$-adic norms are non-archimedian. 


\begin{definition}
	equivalent norm\todo{define this}
\end{definition}

Find
\begin{remark}
	dictionary norm $\leftrightarrow$ valuation
\end{remark} 

\todo{is this necessary?}

\subsection{Non-Archimedean rings and fields} \label{sec:non-archimedean_rings_and_fields}


\begin{theorem}\label{thm:norm_finite_field_ext}
	If $L$ is a finite \todo{should this be algebraic?} field extension of an non-archimedean field $K$, then there is a unique extension of the norm of $K$ to $L$ and this extension is also non-archimedean.
	\todo{find reference, this is not even true. $K$ needs to be complete }
\end{theorem}

\begin{corollary}
	If $K$ is a complete non-archimedean field then the norm of  $K$ extends uniquely to its algebraic closure $\overline{K}$.
\end{corollary}
\begin{proof}
	Recall that \[
	\overline{K} = \varprojlim_{L \supset K} L
	.\] 
	where $L$ ranges over all algebraic extensions of $K$. 
	As all these algebraic extensions have unique norms extending  the norm on $K$ and for $L_1 \supset L_2 \supset K$ the norm on  $L_1$ must restrict to norm on $L_2$ by uniqueness, it is clear that the norm functions $L \to \R^{+}$ glue to a map $\|\cdot \|: \overline{K} \to \R^{+}$. 

	It is easy to verify that this is a norm on $\overline{K}$ extending the norm on $K$. 
\end{proof}

\begin{definition}
	Let $R$ be a non-archimedean ring. Then we write 
	\begin{align*}
		R^{0} &= \{a \in k \st |k| \le 1\}  \\
		R^{00} &=  \{a \in k \st |k| < 1\}  \\
		\tilde R &= R^{0} / R^{00}
	.\end{align*}
\end{definition}
If $k$ is a field then $k^{0}$ is a valuation ring and $ k^{00}$ is its maximal ideal. 

\subsection{Ultrametric spaces} \label{sec:ultrametric_spaces}

\begin{definition}
	A \emph{ultrametric space} is a topological spaces $(X, d)$ where the metric satisfies the non-archimedean triangle inequality. 
	\[
		d(x, y) \le \max \{d(x, z) ,d(z, y)\}, \quad \forall x, y ,z \in X
	.\] 
\end{definition}
\begin{exericise}
	If $d(x,z) \ne d(z,y)$ the inequality is an equality. 
\end{exericise}
\begin{corollary}
	Any point of a ball (open or closed) is a center for that ball. 
\end{corollary}
\begin{corollary}
	If two balls have non-empty intersection that one must be included in the other. 
\end{corollary}
\begin{corollary}
	The topology of $X$ is totally disconneced. 
\end{corollary}

Non-archimedean fields are ultrametric spaces with the metric $d(x, y) = |x - y|$. 

\section{Berkovich Spaces} \label{sec:berkovich_spaces}

\subsection{Berkovich spectrum of a ring} \label{sec:berkovich_spectrum_of_a_ring}

\begin{definition}
	Let $A$ be a ring. Then we define \[
		\mathcal{M} (A) = \{x: A \to \R^{+} \st \text{multiplicative seminorm on } A\} 
	.\] 
	This comes with equipped with the coarsest topology such that for every $a \in A$ the map $\mathcal{M} (A) \to \R^{+}: \|\cdot \|_x \mapsto \|f\|_x $ is continuous. 

	We will almost always deal with a relative case where $A$ is a $K$ algebra over some real valued field. In this case adjust the definition \[
		\mathcal{M} (A) = \{x : A \to \R^{ + }  \mid \text{multiplicative seminorm on }A \text{ extending the norm on } K\} 
	.\] 
	Again $\mathcal{M} (A)$ comes with the coarsest topology making all maps $\mathcal{M} (A) \to \R^{+}: x \mapsto \|a\|_x, \forall a \in A$ continuous. 
\end{definition}

\subsection{Toy example: The Berkovich spectrum of $\Z$} \label{sec:toy_example:_the_berkovich_spectrum_of_Z}

\begin{theorem}
	[Ostrowski]
	Any (multiplicative) norm on  $\Z$ is either
	\begin{itemize}
		\item  trivial
		\item equivalent to  $|\cdot |$
		\item equivalent to $|\cdot |_p$
	\end{itemize}
\end{theorem}

We can make the list a bit more extensive. 
\begin{corollary}
	Any norm on $\Z$ is either  
\end{corollary}




Suppose $X$ be a scheme over $K$, and let $n = \dim X$. 
Then the 
\begin{definition}
	The \emph{Berkovich analitification} of a scheme $X$ over  $K$, as a set is \[
		X\an = \{(x, |\cdot |)  \mid x\in X, |\cdot | \text{ a norm on } k(x) \text{ extending the norm on }K \} 
	.\] 
	\todo{does this need to be smooth and proper and connected?}

	This comes equipped with a cannonical projection map $i: X\an \to X, (x, |\cdot |) \mapsto  x$.
	
	$X\an $ comes with a topology which we define to be the coarsest topology such that 
	\begin{itemize}
		\item $i: X\an \to X$ is continuous, i.e. $X\an$ is a finer space than  $X$. 
		\item For every open $U \subset X$ and $f \in \mathcal{O}_X(U)$ the map  \[
				|f|: i^{-1}(U) \to \R^{+}: (x, |\cdot |) \mapsto  |f(x)|
		\] 
		is continuous.
	\end{itemize}

	\todo{give definition of Berkovich analitification, maybe find an earlier reference than \cite{nicaiseBerkovichSkeletaBirational2016}}
\end{definition}
\begin{remark}
	Later we will consider other topologies on $X\an$, which will be necessary to turn $X\an $ into a ringed space. 
	There topologies will not be topologies like we know it. They will be Grothendieck topologies, where which unlike ordinary topologies restrict the notion of a cover. 
\end{remark}
\begin{remark}
	If $X$ is the spectrum of some $K$-algebra $A$, i.e. $X = \spec A$, then at topological spaces \todo{I think, maybe just as sets} we have \[
		X^{\an} = \mathcal{M} (A)
	.\] 

	\todo{prove this, maybe check whether this is true first}
\end{remark}



\subsection{Berkovich spectrum of $\Z$ or $\Q$} \label{sec:berkovich_spectrum_of_Z}

On $\Z$ we we know a couple norms. There of course is the trivial norm, which I will denote by $|\cdot |_0$ which is defined as \[
|n|_0 = \begin{cases}
	1 & n \ne 0 \\
	0 & n = 0
\end{cases}
.\]  
There is the absolute value norm defined as $|n| = \mathrm{abs}(n)$. You're also familiar with the $p$-adic norms, 



\subsection{Berkovich specturm of $\aff^{1}_K$} \label{sec:berkovich_specturm_of_affine_line}

\todo{Show the description of type I, II, III, IV points and space of disks}




\section{Alternative description of schemes and Berkovich spaces} \label{sec:alternative_description_of_schemes_and_berkovich_spaces}

Let $A$ be any ring. Can we describe the points in $\spec A$ without using prime ideals? It turns out we can!
To see this, pick any ideal prime ideal $\mathfrak{p} \in \spec A$. Then $\frac{A}{\mathfrak{p} }$ is a domain, which gives us a map from $A$ to a field as follows, \[
	A \to \frac{A}{\mathfrak{p} } \to Q\left( \frac{A}{\mathfrak{p} } \right) = k(\mathfrak{p} )
.\] 
\todo{Ask Johannes or Art for a reference for this}

\subsection{Reduction map} \label{sec:reduction_map}

\section{Affinoid algebras} \label{sec:affinoid_algebras}

While we have seen how a scheme over $K$ can be a analytified to obtain a Berkobich space, the real building blocks of these spaces are affinoid algebras, which rougly are completed finite type algebras. 
These are also crucial for defining the structure sheaf on Berkovich spaces. 

\subsection{Banach Algebras} \label{sec:banach_algebras}
\begin{definition}
	A $K$-banach space is a $K$-vector space equipped with a norm that is complete with respect to that norm. 
\end{definition}

\begin{definition}
	Let $A, B$ be $K$-banach spaces. 
	A $K$-vector space morphism $\phi: A \to B$ is \emph{bounded} if there is some $C\in \R$ such that $\|\phi(a)\|\le C \|a\| $. 
\end{definition}

\begin{lemma}
	A morphism $A \to B$ is bounded if and only if it is continuous. 
\end{lemma}

\begin{definition}
	A Banach $K$-algebra, $A$, is a commutative, associative, unital $K$-algebra with a submultiplicative norm that is a Banach $K$-algebra.
\end{definition}
\begin{definition}
	Let $A$ be a $K$-algebra. 
	Two norms $\|\cdot \|_1, \|\cdot \|_2$on $A$ are called equivalent if and only if they induce the same topology. 

	This means that there are non-zero constants $C, C'$ such that for every $ a\in A$ we have $C\|a\|_1 \le \|a\|_2 \le C'\|a\|_1$. 
\end{definition}

\begin{definition}
	Let $A$ be an $k$ banach algebra nd $I$ be a closed\todo{do we need closed here} ideal in $A$. 
	Then we define the \emph{residue norm} on $A / I$ as 
	\begin{align*}
		\|\cdot \|_\text{res} : A / I &\longrightarrow \R \\
		a + I &\longmapsto \inf_{x \in a + I} \|x\|
	.\end{align*}
\end{definition}

To describe the cateogory of $K$-banach algebras we still need morphisms.
\begin{definition}
	Let $A, B$ be $K$-banach algbras. 
	A $K$-algebra morphism $\phi: A \to B$ is \emph{admissible} if $\phi$ is bounded (continuous) if induced map $A / \ker \phi \to \im \phi$ is a homeomorphism, where $A / \ker \phi$ is equipped with the residue norm and $\im \phi$ with subspace norm of $B$. 
\end{definition}


\subsection{Affinoid algebras} \label{sec:affinoid_algebras}
We need restrict the category of banach algebras to those that are ``finitely generated'' in some appropriate sense. 

\begin{definition}
	[2.1.2, \cite{conrad2008several}]
	Let $A$ non-archimedean ring and $r_1, \ldots, r_n$ some positive real numbers. We define \[
		A \left<r_1^{-1}t_1, \ldots, r^{n}^{-1} t_n \right> = \left\{\sum_{\nu \in \N^{n}}^{} a_\nu t^{\nu} \in A[[t_1,\ldots, t_n]]\st a_\nu r^{\nu} \to 0, \text{ as } \nu \to \infty\right\} 
	.\] 
\end{definition}
This can be thought of as the completing of the polynomial ring $A[x_1, \ldots, x_n]$ with the norm defined as 
\begin{equation}\label{eq:norm_tate}
	\left\| \sum_{\nu \in \n^{n}} a_\nu y^{\nu}\right\| = \max_{\nu \in \n^{n}} \|a_\nu r^{\nu}\|
\end{equation}
This gives a similarly defined norm on $A\left<y_1, \ldots, y_n \right>$, but there are many other interesting norms on $A \left<y_1, \ldots, y_n \right>$, as we will discover. 

\begin{definition}
	A \emph{Tate algebra} in $n$ variables is the ring \[
		T_n(K) = K\left<T_1, \ldots, T_n \right>
	,\]
	equipped with the norm as in \eqref{eq:norm_tate}.
\end{definition}


\begin{definition}
	A affinoid algebra is a $K$-algebra $A$ with such that there is an admissible surjective morphism $f: T_n(K) \to A$ for. 
\end{definition}
This means that $A$ can be written as $K\left<T_1, \ldots, T_n \right> / I$ where $I$ is some closed ideal.
This does not mean that $A$ necessarily comes equipped with the residue norm, but this doesn't matter as the open mapping theorem shows that the norm on $A$ must be equivalent to residue norm.
\subsection{Affinoid domains} \label{sec:affinoid_domains}

there are different types of affinoid domains that we often encounter. 

\begin{definition}[multiple definitions in \cite{conrad2008several}]
		there are \emph{weierstrass domains, laurent domains} and \emph{rational domains}. 
	\begin{description}
		\item[weierstrass domain] let $a_1, \ldots, a_n$ be in $a$. 
			\[
				a\left<a_1, \ldots, a_n \right> := \frac{a\left<x_1, \ldots, x_n \right>}{(x_1-a_1, \ldots, x_n - a_n)}
			.\] 
			unlike for polynomial rings, this is not an evaulation map. you should think of this as forcing the $a_i$'s to become powerbounded. 
		\item [laurent domains]
			let $a_1, \ldots, a_n, b_1, \ldots, b_m \in a$ the \[
				a\left<a_1, \ldots, a_n, b_1^{-1}, \ldots, b_m^{-1} \right> := \frac{a\left<x_1, \ldots, x_n,y_1, \ldots, y_m  \right>}{(x_1 - a_1, \ldots, x_n - a_n, b_1 y_1 - 1, \ldots, b_m y_m - 1)}
			.\] 
		\item[rational domain] with $a_1, \ldots, a_n, a' \in a$ with no common zeros, i.e. there is no $\mathfrak{m} $ that contains both of these elements.   
			\[
				a \left<\frac{a_1}{a}, \ldots, \frac{a_n}{a'} \right> := \frac{a\left<x_1, \ldots, x_n \right>}{(a' x_1 - a_1, \ldots, a' x_n - a_n)}
			.\] 
			this is essentially the affinoid version of localisation.
			but it does more than just make $a'$ invertible. 
			is also makes $\frac{a_i}{ a'}$ act like it is power bounded in some sense. 
		\item 
	\end{description}

\end{definition}
many authors define these domains by describing their space of points, be it max spec or the berkovich functor. 
the following lemma gives shows that we can use these descriptions interchangeably
\begin{lemma}
	[2.1.8 in \cite{conrad2008several}]
	suppose $a$ is a $k$-affinoid algebra and $a_1, \ldots, a_n, a_0 \in a$ be elements with no common zero (i.e. at no point in $ x \in \maxspec a$ are all $a_{i} \in \mathfrak{m}$).
	 then a map of $k$-affinoid algebras $\phi: a \to b$ factors through $a \left<\frac{a_1}{a_0}, \ldots, \frac{a_n}{a_0} \right>$ like in the diagram \[
	 \begin{tikzcd}
		 a \left<\frac{a_1}{a_0}, \ldots, \frac{a_n}{a_0} \right> \drar[dashed]  \\
		 a \uar \rar[']{\phi} & b
	 \end{tikzcd}
	 \] 
	 if and only if  $m\phi: \maxspec b \to \maxspec a$ factors through the subset $x \in m(a)$ such that $|a_j(x)| \le |a'(x)|$ for all $j$. 
\end{lemma}

\subsection{admissible opens, covers and $g$-topology} \label{sec:admissible_opens,_covers_and_g-topology}

\begin{definition}
	[2.2.6, \cite{conrad2008several}]
	let $a$ be a  $k$ affinoid algebra. 
	a subset $u$ of $\maxspec(a)$ is an \emph{admissible open} if is has a cover $\{u_i\} $ of affinoid subdomains $u_i$ satisfying the following property: 
	for any map of $k$-affinoid algebras $\phi: a \to b$ such that the induced map $f: m(b) \to m(a)$ lands in $u$, the pullback of the cover $\{f^{-1}(u_i)\} $ has a finite subcovering. 


	a covering $v = \bigcup_{i \in  i} v_i$ is an \emph{admissible cover} if for any any map $\phi:a \to b$  of $k$-affinoid algebras that the image of  $f: m(b) \to m(a)$ lies in $v$, the pullback cover $\{f^{-1}(v_i)\} $ has a refinement by a covering consisting of finitely many affinoid subdomains.  
\end{definition}
\begin{proof}
	\todo{look at this proof}
\end{proof}

we need this to define the notion of a $g$-topology, which is a grothendieck topology on the set $\maxspec(a)$. 
i'm pretty sure later we will scrap $\maxspec$ in favour of the berkovich space.



\todo{Look at temkins notes for Gerritzen-Grauert, paper in mathematische analen.}
Let $K$ be a complete non-archimedean algebraically closed field. 

\begin{definition}
	
\end{definition}

\begin{definition}
	Tate algebra \todo{strict or non-strict?}
\end{definition}
\begin{definition}
	Affinoid algebra \todo{strict or non-strict}
\end{definition}

\begin{definition}
	Berkovich spectrum?
\end{definition}



\section{What comes next} \label{sec:what_comes_next}

Tate elliptic curves




\pagebreak

\part{The paper}

\section{The reduction map} \label{sec:the_reduction_map}
Let $X$ be a smooth connected proper curve over $K$. Let $\mathcal{X} $ be a semistable $R$-model\todo{Do we need semistable here?}, 
i.e.\ $\mathcal{X} $ is a proper $R$ such its sepical fiber $\mathcal{X}_K \cong X$.  

Then we can define extend the reduction map \todo{define reduction map in } from \cref{sec:reduction_map} to a map $X\an \to \mathcal{X} _k$ where where $\mathcal{X} _k$ is the special fibre of the model $\mathcal{X} $. 

To do this we take any point in $X\an$ which by our our discussion in \cref{sec:alternative_description_of_schemes_and_berkovich_spaces} corresponds to a  $\spec (L) \to X$, where $L$ is a valued extension of $K$ 


\begin{definition}
	Let $\mathcal{X} $ be a semistable $R$-model of $X$. Then we can extend the reduction map to $X\an \to \mathcal{X} _k$ as follows 
	\begin{align*}
		\red: X\an &\longrightarrow  \mathcal{X} _k \\
		(x: \spec L \to X) &\longmapsto (\red x: \spec \widetilde{\mathcal{O}_L}  \to \mathcal{X} _k)
	.\end{align*}
	Here $\mathcal{O}_L$ is the the ring of integers of the valued field $L$ and $\widetilde{ \mathcal{O}_L} $ is residue field. 
\end{definition}


There is a lot to unpack here. Recall from our discussion in \cref{sec:alternative_description_of_schemes_and_berkovich_spaces} that we can describe points of schemes and Berkovich spaces as maps from fields, resp.\ valued field extensions of $K$. 
So any point $x \in X\an$ is a  map $x: \spec L \to X = \mathcal{X} _K \subset \mathcal{X} $ where $L$ is a valued field extension of $K$ is just any point in $X\an$. 
As $\mathcal{X} $ is a proper model we may use the valuative criterion \[
\begin{tikzcd}
	\spec L \dar \rar{x} & \mathcal{X}  \dar \\
	\spec \mathcal{O}_L \urar[dashed,']{\tilde x} \rar & \spec R
\end{tikzcd}
\]  
to get a unique map $\tilde x: \mathcal{O}_L \to \mathcal{X}$.
In this diagram the bottom map is induced by the inclusion $R \into \mathcal{O}_L$
The map $\red x: \widetilde{\mathcal{O}_L} \to \mathcal{X} _k$ is obtained by restricting to the special fibre. 
\todo{Why is this independent of $L$.} 

We can see that this is independent of the our choice of $L$, by taking the minimal choice, the residue field $H(x)$, and consider the map $i:H(x) \into L$. 
Then \[
	\begin{tikzcd}[row sep = large]
		\spec L \rar{i} \dar & \spec H(x) \dar \rar{x} & \mathcal{X}  \dar \\
		\spec \mathcal{O}_L \rar{i} \ar[dashed]{urr}[near start]{x'} & \spec \mathcal{O}_{H(x)} \rar \urar[dashed,']{\tilde x} & \spec R
\end{tikzcd}
\]
Then by the uniqueness of valuative criterion we see that $x' = \tilde x \circ i$. In particular $\tilde x$ and $x'$ map to the same point. 

\begin{proposition}
	The map $\red$ is anti-continuous, meaning that the preimage of a open set it closed. 
\end{proposition}
\begin{proof}
	Do I even know how to begin with this? Or should I just cite this \todo{make up my mind}
\end{proof}


\section{Building blocks of analytic curves} \label{sec:building_blocks_of_analytic_curves}

Any Berkovich analitification $X\an$ of a smooth connected proper curve can be considered the as a union of Berkovich open balls and open annuli. 
Hence its worth to discuss these objects. 
These can be described as affinoid domains in $\aff^{1}\an$. \todo{write section on affinoid algebras. }

\begin{definition}
	The \emph{tropicalisation map} is the map 
	\begin{align*}
		\trop :  \mathcal{M} (K[T]) &\longrightarrow \R \cup \{\infty\}  \\
		x &\longmapsto -\log( \|T\|_x)
	.\end{align*}
	Here $\mathcal{M} (K[T])$ is the Berkovich affine line and equals $\aff^{1}_K\an$. 
\end{definition}
Intuitively you should think of this map as giving the inverse logarithmic distance from the origin of $\aff^{1}\an$. 
To visiualise this. This function if $\infty$ at $0$ and linearly with slope 1 increases along the path till $\infty$ (the extra point on the Berkovich projective line). 
The whole of $\aff^{1}\an $ retracts on this line, and takes the value of the point where it contracts to. Hence off the line between $0$ and $\infty$  $\trop$ is locally constant. 

\begin{figure}[ht]
    \centering
    \incfig{affine-line-ball-annuli}
    \caption{affine line ball annuli}
    \label{fig:affine-line-ball-annuli}
\end{figure}


\begin{definition}
	Let $r \in |K^{\times }|$. The \emph{standard closed ball} of radius $r$ is $B(r) = \trop^{-1}([-\log r, \infty])$, i.e. $B(a) = \{x \in \aff^{1}\an \st \|T\|_x \le r\} $. 
\end{definition}
The standard closed ball is also the Berkovich spectrum of the tate algebra \[
K\left<r^{-1}t \right> = \left\{\sum_{n = 0}^{\infty} a_n ^{n} \st r ^{n} \|a_n\| \to 0 \text{ as } n \to \infty \right\} 
.\] 
\begin{definition}
	Again let $a \in |K^{\times }|$. 
	The \emph{standard open ball} of radius $r$ is $B(a)_+ = \trop^{-1}((-\log r, \infty])$, i.e.\ $B(a) = \{x \in \aff^{1}\an \st \|T\|_x <  r\} $.
\end{definition}

\begin{definition}
	Let $r, s \in |K^{\times }|$. 
	The \emph{standard closed annulus with outer radius  $r$ and inner radius  $s$}  is $S(s, r) = \trop^{-1}([-\log r, -\log s])$, i.e.\ $S(s, r) = \{x \in \aff^{1}\an \st s \le \|T\|_x \le r\} $.

	The \emph{modulus} of this annulus is $s r^{-1}$. 
\end{definition}
The standard closed annulis is also the Berkovich specturm of the affinoid algebra \[
	K\left<r^{-1}t, st^{-1} \right> = \left\{\sum_{n \in \Z}^{} a_n ^{n} \st r^{n}|a_n| \to 0 \text{ as } n \to +\infty, s^n |a_n| \to 0 \text{ as } n \to - \infty\right\} 
.\] 
If $r = 1$ then we write $S(s) = S(s,1)$. 
\begin{definition}
	The \emph{standard open annulus of outer radius $r$ and inner radious $s$} is $S(s,r)_+ = \trop^{-1}((-\log s, -\log r))$. 
	I.e.\ $S(s,r)_+ = \{x \in \aff^{1}\an \st s < \|T\|_x < r\} $

	Similarly to the closed annulus we define the \emph{modulus} of this ball to be $s r^{-1}$. 
\end{definition}


Finally we define a domain that is not quite an annulus as defined above, but is very similar to one. 
\begin{definition}
	Let $r \in |K^{\times }|$. The \emph{standard punctured open ball of radius $r$ } is $S(0, a)_+ = \trop^{-1}((-\log a, \infty))$. 
	The \emph{standard punctured open ball of radious $r^{-1}$ around $\infty$} is $S(a, \infty)_+ = \trop^{-1}((-\infty, -\log a))$. 

	By definition we say these are annuli of modulus $\infty$. 
\end{definition}

All closed balls are isomorphic. Similarly all open balls are isomorphic. 
Two open (resp. closed) annuli/punctured balls are isomorpic if and only if they have the same modulus. 

\begin{remark}
	If we just remove $0$ from $\aff^{1}\an$ (and thus also $\infty$ of $\pro^{1}\an$) we gain obtain something similar to an annulus. But the outer radius is $\infty$ and the inner radius is $0$. 
	Note that this is also $\mathbb G^{\mathrm{an}}_m$, which how we will usually denote this space. 
\end{remark}



Why do we need proposition 2.2? \todo{figure out why this is relevant}

\subsection{The skeleton of a standard generalized annulus} \label{sec:the_skeleton_of_a_standard_generalized_annulus}
We define a section of the tropicalision map, which gives the line between $0$ and $\infty$ (or what lives in it the specific annulus). 
\begin{align*}
	\sigma: \R &\longrightarrow \mathbf G_m^{\mathrm{an}} \\
	r &\longmapsto \left[f \mapsto \sup_{z \in B(0,\exp(-r))} |f(z)|\right]
.\end{align*}
In the description of $\aff^{1}\an$ as a space of closed disks, this sections is a line made by disks with center $0$ of variying radius.


The following proposition helps us understand maps between annuli. 
\begin{proposition}
	[2.2 in paper]
	Let $a \in R, a \ne 0$.
	\begin{enumerate}
		\item 
			The units in $K \left<a t^{-1}, t \right>$ are functions of the form 
			\begin{equation}\label{eq:unit_annuli}
				f(t) = \alpha t ^{d}(1 + g(t))
			\end{equation}
			where $\alpha \in K^{\times }, d \in \Z$ and $|g|_\text{sup}  < 1$.
			\todo{Either understands Thuilliers proof or some other proof}
		\item Let $f(t)$ be a unit as in \eqref{eq:unit_annuli} with  $d > 0$.
			Then the morphism $\phi:S(a) \to \Gan$ induced by $K[x, x^{-1}] \to K\left<a t^{-1}, t \right> : x \mapsto \phi(t)$, factorst through a finite flat morphism $S(a) \to S(\alpha a^{d}, \alpha)$ of degree $d$. 

			Similarly if $f(t)$ is such a unit but with $d <0$ then the map $\phi: S(a) \to \Gan$ factors through $S(a) \to S(\alpha, \alpha a^{d})$ of degree $-d$. 

		\item If $d = 0$ is then the morphism $\phi: S(a) \to \Gan$ factors through a morphism $S(a) \to S(\alpha, \alpha)$ which is not finite. 
	\end{enumerate}

\end{proposition}
\begin{proof}
	\begin{itemize}
		\item \todo{Either refere to Thuillier again, or use the alternative proof Johannes suggested}
			Let $u$ be invertible with inverse 
		\item We can reduce to the case where $\alpha = 1$ and $d > 0$. \todo{Why?}

	\end{itemize}
\end{proof}

\printbibliography 
\end{document}

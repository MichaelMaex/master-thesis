\documentclass[a4paper]{article}
\usepackage{nomencl}
\makenomenclature
\usepackage[utf8]{inputenc}
\usepackage[T1]{fontenc}
\usepackage{textcomp}
\usepackage{amsmath, amssymb, amsthm}
\usepackage{geometry}
\usepackage{tikz-cd}
\usepackage{mdframed}
\usepackage{microtype}
\usepackage{hyperref}
\usepackage{cleveref}

\usepackage[backend = biber, style = alphabetic]{biblatex}
\addbibresource{../references.bib}

% figure support
\usepackage{import}
\usepackage{xifthen}
\usepackage[textsize=tiny,colorinlistoftodos,obeyDraft]{todonotes}
\newcommand{\question}[1]{\todo[color=green!40]{#1}}
\usepackage{pdfpages}
\usepackage{transparent}
\newcommand{\incfig}[1]{%
	\def\svgwidth{\columnwidth}
	\import{./figures/}{#1.pdf_tex}
}

\pdfsuppresswarningpagegroup=1


\newcommand{\N}{\mathbb{N}}
\newcommand{\Z}{\mathbb{Z}}
\newcommand{\Q}{\mathbb{Q}}
\newcommand{\C}{\mathbb{C}}
\newcommand{\R}{\mathbb{R}}
\newcommand{\F}{\mathbb{F}}
\newcommand{\pro}{\mathbb{P}}
\newcommand{\aff}{\mathbb{A}}
\newcommand{\ltr}{\par \noindent \framebox[1\width]{ $\implies$ } \hspace{.2cm}}
\newcommand{\rtl}{\par \noindent \framebox[1\width]{ $\impliedby$ } \hspace{.2cm} }

\DeclareMathOperator{\coker}{coker}
\DeclareMathOperator{\id}{Id}
\DeclareMathOperator{\im}{Im}
\DeclareMathOperator{\spec}{Spec}
\DeclareMathOperator{\maxspec}{MaxSpec}
\DeclareMathOperator{\spf}{Spf}
\DeclareMathOperator{\proj}{Proj}
\DeclareMathOperator{\red}{red}
\DeclareMathOperator{\trop}{trop}
\DeclareMathOperator{\trdeg}{TrDeg}
\DeclareMathOperator{\divisor}{Div}
\DeclareMathOperator{\cov}{Cov}

\newcommand{\into}{\hookrightarrow}
\newcommand{\onto}{\twoheadrightarrow}
\newcommand{\divides}{\mid}
\newcommand{\notdivides}{\nmid}
\newcommand{\Gan}{\ensuremath{\mathbb{G} _m ^{\mathrm{an}}}}
\newcommand{\an}{{}^{\text{an}}}
\newcommand{\cox}{\widehat{\otimes}}

\newcommand{\st}{%
  \nonscript\;
  \ifnum\currentgrouptype=16
    \,\middle|\,
  \else
    \,|\,
  \fi
  \nonscript\;}

\theoremstyle{definition}
\newtheorem{definition}{Definition}[subsection]
\newtheorem{remark}[definition]{Remark}
\newtheorem{theorem}[definition]{Theorem}
\newtheorem{example}[definition]{Example}
\newtheorem{lemma}[definition]{Lemma}
\newtheorem{claim}[definition]{Claim}
\newtheorem{proposition}[definition]{Proposition}
\newtheorem{exercise}[definition]{Exercise}
\newtheorem{corollary}[definition]{Corollary}


\author{Micha\"el Maex}

\addbibresource{references.bib}
\renewcommand{\nomname}{List of Symbols}
\newcommand{\an}{\ensuremath{{}^{\mathrm{an}}}}
	

\title{Baker Payne Rabinoff and Prerequisites}
\begin{document}
\maketitle

The goal of this document is to understand the paper \cite{bakerStructureNonarchimedeanAnalytic2013}, which will contain the main ingredients of my thesis.  

	
\mbox{}
\nomenclature{\(K\)}{Algebraiclly closed field, complete wrt.\ a nonarchimedian valuation}
\nomenclature{\(v \)}{valuation on $K$}
\nomenclature{\(R \)}{valuation ring of $K$}
\nomenclature{\(\mathfrak{m}_R  \)}{maximal ideal of $R$}
\nomenclature{\(k \)}{residue field of $R$}
\nomenclature{\(\R^{+} \)}{the non-negative real numbers}


\printnomenclature

\part{Prequisites}
Before we start the partial proof of the classification we gather the important results from intersection theory in the following lemma. 
\begin{lemma}
	\label{prop:irred_comp_of_prop_model}
	Let $E, \mathscr E$ be the elliptic curve and model from \cref{sec:notation_and_conventions}. 
	Let $K_{\mathscr E}$ be a canonical divisor on $E$. 
	\begin{enumerate}
		\item At least one irreducible component has multiplicity $N_i = 1$.  
		\item $\mathscr E_s$ is connected. 
		\item $K_{\mathcal{C} } \cdot \mathscr E_s = 0 $. 
			Here $K_{\mathcal{C} }$ is a canonical divisor. 
		\item If $r = 1$ (i.e.\ the special fiber is irreducible) then $F_1^2 = \mathscr{E} _s^2 = 0$ and $g (F_1) = 1$. 
		\item If $r \ge 2$ (i.e.\ the special fiber is reducible) then for each irreducible component $F_i$ we have \[
		F_i^2 = -2,\quad  F_i \simeq \pro^{1}_k \quad \text{ and }\quad \sum_{1 \le j \le r, j \ne i}^{} N_j F_j\cdot  F_i = 2 N_i
		.\]   
	\end{enumerate}
\end{lemma}
\begin{proof}
\begin{enumerate}
	\item As $E$ is an elliptic curve, it has at least one $K$-rational point $x$.
		Then by \cref{cor:closure_K_rational_point} we know that  $\red_{\mathscr E}(x)$ lies on a irreducible component of $\mathscr E_s$. 
	\item By \cite[cor.\ 9.1.24]{liuAlgebraicGeometryArithmetic2002} the map $\mathscr E \to R$ is $\mathcal{O}$-connected. 
		In particular  $\mathscr E_s$ is connected. 
	\item From \cref{prop:intersection_canonical_special} know that \[
			K_{\mathscr E} \cdot \mathscr E_{s} = 2g(E) - 2 = 0
	.\]  
	\item If $r = 1$ then $\mathscr E_s$ has only one irreducible component, $F_1$. 
		By the first part this component has multiplicity $N_1 = 1$. 
		So $\mathscr E_s = F_1$. 
		Then $\mathscr E_s^2 = F_1^2 = 0$. 
		As $\mathscr E \to \spec R$ is flat we know that $g(E)  = g(\mathscr E_s) = 1$.
	\item 	
		The adjunction formula (\cref{thm:adjunction_formula}) states that \begin{equation}\label{eq:proof_prereqs_1}
			F_i^2 + K_{\mathscr E} \cdot F_i = 2g(F_i) - 2
		.\end{equation}
		Suppose for the sake of contradiction that that $F_i^2 = 0$. 
		Then $F_i$ does not intersect any other components, and so $F_i$ is the only component of $\mathscr E_s$. 
		This contradicts $r \ge 2$. 
		We find that $F_i^2 \le -1$. 
		
		Suppose for the sake of contradiction that $K_\mathscr E \cdot F_i < 0$. 
		Then by \eqref{eq:proof_prereqs_1} this is only possible if $F_i^2 = -1$ and $g(F_i) =0$. 
		Thus $F_i$ is contractible by Castelnuovo's criterion, which contradicts the minimality of $\mathscr E$. 

		So for every $i$ we find that $F_i^2 \le 1, K_{\mathscr E} \cdot F_i \le 0$. 
		But \[
		0 = K_{\mathscr E} \cdot \mathscr E_s = \sum_{i = 1}^{r} K_{\mathscr E} \cdot F_i \le 0
		.\]  
		Hence  $K_{\mathscr E} \cdot F_i = 0$ for all $i$. 
		So \eqref{eq:proof_prereqs_1} becomes \[
			-1 \ge  F^2_i = 2g(F_i) - 2 \ge -2
		,\]  
		which can only hold if $F_i^2 = -2, g(F_i) = 0$. 

		The last equality is \cref{cor:compute_self_intersection}, together with $F_i^2 = -2$. 
\end{enumerate}	
\end{proof}




\pagebreak

\part{The paper}

\section{The reduction map} \label{sec:the_reduction_map}
Let $X$ be a smooth connected proper curve over $K$. Let $\mathcal{X} $ be a semistable $R$-model\todo{Do we need semistable here?}, 
i.e.\ $\mathcal{X} $ is a proper $R$ such its sepical fiber $\mathcal{X}_K \cong X$.  

Then we can define extend the reduction map \todo{define reduction map in } from \cref{sec:reduction_map} to a map $X\an \to \mathcal{X} _k$ where where $\mathcal{X} _k$ is the special fibre of the model $\mathcal{X} $. 

To do this we take any point in $X\an$ which by our our discussion in \cref{sec:alternative_description_of_schemes_and_berkovich_spaces} corresponds to a  $\spec (L) \to X$, where $L$ is a valued extension of $K$ 


\begin{definition}
	Let $\mathcal{X} $ be a semistable $R$-model of $X$. Then we can extend the reduction map to $X\an \to \mathcal{X} _k$ as follows 
	\begin{align*}
		\red: X\an &\longrightarrow  \mathcal{X} _k \\
		(x: \spec L \to X) &\longmapsto (\red x: \spec \widetilde{\mathcal{O}_L}  \to \mathcal{X} _k)
	.\end{align*}
	Here $\mathcal{O}_L$ is the the ring of integers of the valued field $L$ and $\widetilde{ \mathcal{O}_L} $ is residue field. 
\end{definition}


There is a lot to unpack here. Recall from our discussion in \cref{sec:alternative_description_of_schemes_and_berkovich_spaces} that we can describe points of schemes and Berkovich spaces as maps from fields, resp.\ valued field extensions of $K$. 
So any point $x \in X\an$ is a  map $x: \spec L \to X = \mathcal{X} _K \subset \mathcal{X} $ where $L$ is a valued field extension of $K$ is just any point in $X\an$. 
As $\mathcal{X} $ is a proper model we may use the valuative criterion \[
\begin{tikzcd}
	\spec L \dar \rar{x} & \mathcal{X}  \dar \\
	\spec \mathcal{O}_L \urar[dashed,']{\tilde x} \rar & \spec R
\end{tikzcd}
\]  
to get a unique map $\tilde x: \mathcal{O}_L \to \mathcal{X}$.
In this diagram the bottom map is induced by the inclusion $R \into \mathcal{O}_L$
The map $\red x: \widetilde{\mathcal{O}_L} \to \mathcal{X} _k$ is obtained by restricting to the special fibre. 
\todo{Why is this independent of $L$.} 

We can see that this is independent of the our choice of $L$, by taking the minimal choice, the residue field $H(x)$, and consider the map $i:H(x) \into L$. 
Then \[
	\begin{tikzcd}[row sep = large]
		\spec L \rar{i} \dar & \spec H(x) \dar \rar{x} & \mathcal{X}  \dar \\
		\spec \mathcal{O}_L \rar{i} \ar[dashed]{urr}[near start]{x'} & \spec \mathcal{O}_{H(x)} \rar \urar[dashed,']{\tilde x} & \spec R
\end{tikzcd}
\]
Then by the uniqueness of valuative criterion we see that $x' = \tilde x \circ i$. In particular $\tilde x$ and $x'$ map to the same point. 

\begin{proposition}
	The map $\red$ is anti-continuous, meaning that the preimage of a open set it closed. 
\end{proposition}
\begin{proof}
	Do I even know how to begin with this? Or should I just cite this \todo{make up my mind}
\end{proof}


\section{Building blocks of analytic curves} \label{sec:building_blocks_of_analytic_curves}

Any Berkovich analitification $X\an$ of a smooth connected proper curve can be considered the as a union of Berkovich open balls and open annuli. 
Hence its worth to discuss these objects. 
These can be described as affinoid domains in $\aff^{1}\an$. \todo{write section on affinoid algebras. }

\begin{definition}
	The \emph{tropicalisation map} is the map 
	\begin{align*}
		\trop :  \mathcal{M} (K[T]) &\longrightarrow \R \cup \{\infty\}  \\
		x &\longmapsto -\log( \|T\|_x)
	.\end{align*}
	Here $\mathcal{M} (K[T])$ is the Berkovich affine line and equals $\aff^{1}_K\an$. 
\end{definition}
Intuitively you should think of this map as giving the inverse logarithmic distance from the origin of $\aff^{1}\an$. 
To visiualise this. This function if $\infty$ at $0$ and linearly with slope 1 increases along the path till $\infty$ (the extra point on the Berkovich projective line). 
The whole of $\aff^{1}\an $ retracts on this line, and takes the value of the point where it contracts to. Hence off the line between $0$ and $\infty$  $\trop$ is locally constant. 

\begin{figure}[ht]
    \centering
    \incfig{affine-line-ball-annuli}
    \caption{affine line ball annuli}
    \label{fig:affine-line-ball-annuli}
\end{figure}


\begin{definition}
	Let $r \in |K^{\times }|$. The \emph{standard closed ball} of radius $r$ is $B(r) = \trop^{-1}([-\log r, \infty])$, i.e. $B(a) = \{x \in \aff^{1}\an \st \|T\|_x \le r\} $. 
\end{definition}
The standard closed ball is also the Berkovich spectrum of the tate algebra \[
K\left<r^{-1}t \right> = \left\{\sum_{n = 0}^{\infty} a_n ^{n} \st r ^{n} \|a_n\| \to 0 \text{ as } n \to \infty \right\} 
.\] 
\begin{definition}
	Again let $a \in |K^{\times }|$. 
	The \emph{standard open ball} of radius $r$ is $B(a)_+ = \trop^{-1}((-\log r, \infty])$, i.e.\ $B(a) = \{x \in \aff^{1}\an \st \|T\|_x <  r\} $.
\end{definition}

\begin{definition}
	Let $r, s \in |K^{\times }|$. 
	The \emph{standard closed annulus with outer radius  $r$ and inner radius  $s$}  is $S(s, r) = \trop^{-1}([-\log r, -\log s])$, i.e.\ $S(s, r) = \{x \in \aff^{1}\an \st s \le \|T\|_x \le r\} $.

	The \emph{modulus} of this annulus is $s r^{-1}$. 
\end{definition}
The standard closed annulis is also the Berkovich specturm of the affinoid algebra \[
	K\left<r^{-1}t, st^{-1} \right> = \left\{\sum_{n \in \Z}^{} a_n ^{n} \st r^{n}|a_n| \to 0 \text{ as } n \to +\infty, s^n |a_n| \to 0 \text{ as } n \to - \infty\right\} 
.\] 
If $r = 1$ then we write $S(s) = S(s,1)$. 
\begin{definition}
	The \emph{standard open annulus of outer radius $r$ and inner radious $s$} is $S(s,r)_+ = \trop^{-1}((-\log s, -\log r))$. 
	I.e.\ $S(s,r)_+ = \{x \in \aff^{1}\an \st s < \|T\|_x < r\} $

	Similarly to the closed annulus we define the \emph{modulus} of this ball to be $s r^{-1}$. 
\end{definition}


Finally we define a domain that is not quite an annulus as defined above, but is very similar to one. 
\begin{definition}
	Let $r \in |K^{\times }|$. The \emph{standard punctured open ball of radius $r$ } is $S(0, a)_+ = \trop^{-1}((-\log a, \infty))$. 
	The \emph{standard punctured open ball of radious $r^{-1}$ around $\infty$} is $S(a, \infty)_+ = \trop^{-1}((-\infty, -\log a))$. 

	By definition we say these are annuli of modulus $\infty$. 
\end{definition}

All closed balls are isomorphic. Similarly all open balls are isomorphic. 
Two open (resp. closed) annuli/punctured balls are isomorpic if and only if they have the same modulus. 

\begin{remark}
	If we just remove $0$ from $\aff^{1}\an$ (and thus also $\infty$ of $\pro^{1}\an$) we gain obtain something similar to an annulus. But the outer radius is $\infty$ and the inner radius is $0$. 
	Note that this is also $\mathbb G^{\mathrm{an}}_m$, which how we will usually denote this space. 
\end{remark}



Why do we need proposition 2.2? \todo{figure out why this is relevant}

\subsection{The skeleton of a standard generalized annulus} \label{sec:the_skeleton_of_a_standard_generalized_annulus}
We define a section of the tropicalision map, which gives the line between $0$ and $\infty$ (or what lives in it the specific annulus). 
\begin{align*}
	\sigma: \R &\longrightarrow \mathbf G_m^{\mathrm{an}} \\
	r &\longmapsto \left[f \mapsto \sup_{z \in B(0,\exp(-r))} |f(z)|\right]
.\end{align*}
In the description of $\aff^{1}\an$ as a space of closed disks, this sections is a line made by disks with center $0$ of variying radius.


The following proposition helps us understand maps between annuli. 
\begin{proposition}
	[2.2 in paper]
	Let $a \in R, a \ne 0$.
	\begin{enumerate}
		\item 
			The units in $K \left<a t^{-1}, t \right>$ are functions of the form 
			\begin{equation}\label{eq:unit_annuli}
				f(t) = \alpha t ^{d}(1 + g(t))
			\end{equation}
			where $\alpha \in K^{\times }, d \in \Z$ and $|g|_\text{sup}  < 1$.
			\todo{Either understands Thuilliers proof or some other proof}
		\item Let $f(t)$ be a unit as in \eqref{eq:unit_annuli} with  $d > 0$.
			Then the morphism $\phi:S(a) \to \Gan$ induced by $K[x, x^{-1}] \to K\left<a t^{-1}, t \right> : x \mapsto \phi(t)$, factorst through a finite flat morphism $S(a) \to S(\alpha a^{d}, \alpha)$ of degree $d$. 

			Similarly if $f(t)$ is such a unit but with $d <0$ then the map $\phi: S(a) \to \Gan$ factors through $S(a) \to S(\alpha, \alpha a^{d})$ of degree $-d$. 

		\item If $d = 0$ is then the morphism $\phi: S(a) \to \Gan$ factors through a morphism $S(a) \to S(\alpha, \alpha)$ which is not finite. 
	\end{enumerate}

\end{proposition}
\begin{proof}
	\begin{itemize}
		\item \todo{Either refere to Thuillier again, or use the alternative proof Johannes suggested}
			Let $u$ be invertible with inverse 
		\item We can reduce to the case where $\alpha = 1$ and $d > 0$. \todo{Why?}

	\end{itemize}
\end{proof}

\printbibliography 
\end{document}

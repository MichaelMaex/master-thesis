Our research in \cref{chap:a_berkovich_approach_to_classifying_elliptic_curves} opens many avenues that could be explored further. 
In this chapter we gather some of these loose ends that we either are working on, or think would be interesting to study for future research. 


\subsection*{Widening the class of elliptic curves considered} \label{sec:widening_the_class_of_elliptic_curves_considered}

In \cref{chap:a_berkovich_approach_to_classifying_elliptic_curves} we assumed that $E$ was of the form $y^2 = x(x-1)(x-\lambda)$ and that $v(\lambda) \le 0, v(1-\lambda) = 0$. 
It would be interesting to understand what happens if we choose $\lambda$ differently. 
What if $v(\lambda) > 0$? 
What  $E$ is of the form $y^2 = x(x^2 + \alpha)$. Then the ramification points are not necessarily $K$-points. 
In the most general form we might ask what $E$ looks like if it is given by the equation $y^2 = (x-\alpha)(x-\beta)(x-\gamma)$ where $\alpha, \beta, \gamma$ may not even lay in the ground field $K$. 

We expect that if $\ch k \ne 2$ it is possible to give a Berkovich geometric explanation for more reduction types, where $\sk(E)$ is a point. 
It would be interesting to see if by studying a neighborhood of $\sk(E)$ we can differentiate between $\mathrm {X}, \mathrm{X}^*$ where $\mathrm {X}$ is one of  $\mathrm{II}, \mathrm{III}, \mathrm{IV}$. 

If $\ch = 2$ we did not encounter curves of type $\mathrm I_m$ for some $m \ge 0$, which unfortunately limited our discussion in \cref{sec:wild_ramification}. 
We suspect that we would encounter these curves if $2 \mid v(2)$ or $v(\lambda) < 0$. 



\subsection*{Generalizing to arbitrary morphisms of curves} \label{sec:generalizing_to_arbitrary_morphisms_of_curves}

Most of the results we found here consider the morphism of genus 1 curve to $\pro^{1}_K$, but many results should generalize to arbitrary morphisms between curves. 
As a first step we might want to study hyper-elliptic curves, i.e.\ curves with a degree $2$ morphism to $\pro^{1}_K$.

\subsection*{Understanding the log different better} \label{sec:understanding_the_log_different_better}
The description of the log different that we gave in \cref{sec:weight_functions_and_wild_ramification} is very incomplete. 
If we understand how $\phi: E\an \to \pro^{1, \text{an}}_K$ looks on a neighborhood of $\sk(E)$, in particular how the metric behaves, we can compute the slopes $\wt_{f^*\omega}$ and $\wt_{\omega} \circ \phi$ using tools from \cite{bakerWeightFunctionsBerkovich2016}. 
This would tell us the slopes of $\mathfrak{d}_f $ in a neighborhood of $\sk(E)$, and not just their sign. 


If we find a way of computing these slopes directly, then it might be possible to obtain  $\sk(E)$ as $\minloc (\wt_{\omega} \circ \phi + \mathfrak{d}_f)$ directly, instead of using Tate's algorithm. 
It might be possible to compute $\omega_{X / Y}$ from the proof of \cref{prop:weightfunction_fullback} as a Weil divisor using an intersection theoretic argument, which when $\ch k \mid d$ would compute  $\mathfrak{d} _f$. 


\subsection*{Lengths of paths under morphisms} \label{sec:lenghts_of_paths_under_morphisms}

In \cref{rem:length_edge_skeleton_cover} we found that the length of a path is invariant under the analytification of a cover of curves, if the path lays outside the topological branch locus. 
In the proof of \cref{prop:m_when_E_type_Is} we showed how a chain of multiplicity 2 components maps onto a chain of multiplicity 1 components. In this case we see that length increases by multiple of 2 under the morphism. 
The main idea for both results is to count the number of divisorial points on paths. Can this be used to understand how the lengths of paths change under morphisms of curves more generally?

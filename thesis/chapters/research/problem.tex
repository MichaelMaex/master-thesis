Let $R$ be a complete discretely valued field with fraction field $K$ and algebraically closed residue field $k$. Suppose that  $\ch R = 0$ and $\ch k = p$ with $p$ possibly $0$. 
Let $E$ be an elliptic curve over $K$.

The minimal regular model determines the reduction type of $E$. 
The minimal regular model also determines the minimal snc-model $\mathscr E_\text{min} $, which in turn determines $\sk(\mathscr E_\text{min} )$ and the essential skeleton $\sk(E)$. 
So we may wonder whether we can determine the reduction type of $E$ by studying $E\an$, in particular the essential skeleton $\sk(E)$.
\Cref{tab:skeleton_by_kodaira_neron} gives the essential skeleton of $E$ by reduction type. 
Only considering the topology of $\sk E$ we can distinguish three sets of reduction types 
\begin{description}
	\item[point] $\mathrm I_0, \mathrm{II}, \mathrm{III},\mathrm{IV}, \mathrm{I}_0^*, \mathrm{II}^*, \mathrm{III}^*$ 
	\item[line] $\mathrm{I}_n^*$ for some $n > 0$
	\item[circle]  $\mathrm{I}_n$ for some  $n > 0$
\end{description}
If we know the \emph{combinatorical type of $\sk(E)$}, i.e.\ know all the multiplicities and genera of the divisorial points, then we can distuinguish between all types except $\mathrm{II}, \mathrm{II}^*$ and $\mathrm{III}, \mathrm{III}^*$ and  $\mathrm{IV}, \mathrm{IV}^*$.
In particular if the length of $\sk(E)$ is known, it is possible to find the $n$ in the reduction types $\mathrm I_n$ and $\mathrm I_n^*$. 


So the question becomes the following. Given an elliptic curve $E$ over $K$, can we get information about $\sk(E)$ like, the topology, lengths, genera,\ldots without computing a snc-model of  $E$?
The projective line $\pro^{1}_K$ is the only curve of which we know how to describe its Berkovich analytification without using models. 
As elliptic curves are hyper elliptic, there is a degree $2$ morphism $f: E \to \pro^{1}_K$. 
Our hope is that by studying the analytification of this morphism $\phi: E\an \to \pro^{1, \text{an}}_K$ we can recover information on the essential skeleton of $E\an$. 
We will limit ourselves to the following setting.
\begin{problem}\label{prob:main_problem}
	Let $E$ be the elliptic curve given by the Weierstrass equation 
	\begin{equation}\label{eq:weierstrass_problem}
		E: y^2 = x(x-1)(x-\lambda) = x^3 + (-1 - \lambda) x^2 + \lambda x
	,\end{equation}
	for some $\lambda \in K^\times $ such that $|\lambda| \le 1, |\lambda - 1| = 1$. 
	Let $f: E \to \pro^{1}_K$ given by projection to the $x$-axis, which a degree 2 morphism ramified over $4$ points. 
	What is $\sk (E)$? What is the reduction type of $E$?
\end{problem}
\setlength{\LTcapwidth}{5.7in}
\begin{longtable}{c|c|c|c|c}
		\caption{The minimal regular model, minimal snc-model $\mathscr E_\text{min} $, skeleton induced by $\mathscr E_\text{min} $ and the essential skeleton.
	Inessential components of the $\mathscr E_\text{min} $ are colored green. Components with multiplicity greater than one have their multiplicities labeled in blue. 
Points in the skeleton with genus greater than 0 are labeled in red. 
Lengths of line segments and circles in the skeleta are labeled in black.
If $\sk(E)$ is a point $x$, then it will be labeled with its multiplicity $N(x)$ in blue.}
	\label{tab:skeleton_by_kodaira_neron}
	\\
	type & minimal regular model	& minimal snc-model &  $\sk(\mathscr E_\text{min} )$ & $\sk(E)$ \\
	\hline
	$\mathrm{I}_0$ & \incfigsmall{reg-i-0} & \incfigsmall{reg-i-0} & 
	\incfigsmall{i0-sk-snc} & \incfigsmall{i0-sk-snc} \\
	\hline 
	$\mathrm I_1$  & \incfigsmall{i1-reg} & \incfigsmall{i1-snc} & \incfigsmall{i1-sk-snc} & \incfigsmall{i1-sk-snc} \\
	\hline
	$\mathrm I_n$ & \incfigsmall{in-reg} & 
	\incfigsmall{in-reg} & \incfigsmall{in-sk-scn} & \incfigsmall{in-sk-scn} \\
	\hline
	$\mathrm{II}$ &  \incfigsmall{ii-reg} & \incfigsmall{ii-snc} & \incfigsmall{ii-sk-snc} & \incfigsmall{ii-sk}
 \\ 
	\hline 
	$\mathrm{III}$ &  \incfigsmall{iii-reg} & \incfigsmall{iii-snc}   & \incfigsmall{iii-sk-snc}  &  \incfigsmall{iii-sk}
\\
	\hline 
	$\mathrm{IV}$ &  \incfigsmall{iv-reg} & \incfigsmall{iv-snc}   & \incfigsmall{iv-sk-snc}  &  \incfigsmall{iv-sk}
\\
	\hline 
	$\mathrm{I}_0^*$ &  \incfigsmall{i0s-reg}&  \incfigsmall{i0s-snc} & \incfigsmall{i0s-sk-snc} & \incfigsmall{i0s-sk} \\
	\hline 
	$\mathrm{I}_n^*$ & \incfigsmall{ins-reg} &\incfigsmall{ins-snc}  &\incfigsmall{ins-sk-snc} &\incfigsmall{ins-sk}\\
	\hline 
	$\mathrm{IV}^*$ & \incfigsmall{ivs-reg} & \incfigsmall{ivs-snc}  & \incfigsmall{ivs-sk-snc} & \incfigsmall{iv-sk} \\
	\hline 
	$\mathrm{III}^*$ & \incfigsmall{iiis-reg} & \incfigsmall{iiis-snc} &\incfigsmall{iiis-sk-snc} &\incfigsmall{iii-sk}\\
	\hline 
	$\mathrm{II}^*$ & \incfigsmall{iis-reg} & \incfigsmall{iis-snc} &\incfigsmall{iis-sk-snc} &\incfigsmall{ii-sk}\\
	\hline 
\end{longtable}



Requiring that $E$ is of the from in \eqref{eq:weierstrass_problem} puts some implicit assumptions on $E$ and the map $f: E \to \pro^{1}_K$.
In particular it forces the ramification points of $f$ to be defined over $K$, which as we will see in \cref{sec:expectations} limits the reduction types that can occur. 

The assumption that $|\lambda| \le 1, |1 - \lambda| = 1$ has the following Berkovich geometric interpretation.
We can consider $0, 1, \lambda, \infty$ as closed points of $\pro^{1}_K$, and thus also as type 1 points in $\pro^{1, \text{an}}_K$.
Let  $\Gamma_\lambda$ be the convex hull of the points $0, 1, \lambda, \infty$ in $\pro^{1, \text{an}}_K$. 
Then  $\Gamma_\lambda$ can have 4 different configurations depending on $|\lambda|$ and $|1-\lambda|$. 
Our assumptions enforce that $\Gamma_\lambda$ is of one of the first two configurations from \cref{fig:configurations_of_gamma_lambda}.
\begin{figure}[ht]
    \centering
    \incfig{configurations-of-gamma-lambda}
    \caption{configurations of $\Gamma_\lambda$ (in red)}
    \label{fig:configurations_of_gamma_lambda}
\end{figure}
If  $\Gamma_\lambda$ is not of the second configuration (as in \cref{fig:configurations_of_gamma_lambda}) then we may move to any other of the 3 remaining configurations by a Möbius transformation on $\pro^{1}_K$, taking $\lambda$ to $\lambda'$. 
This will induce an isomorphism $E_L \to E'_L$ where $E'$ is the elliptic curve defined by $y^2 = x(x-1)(x-\lambda')$ and $L$ is a finite extension of $K$. 
Note  that $L$ cannot always be taken to be $K$ itself
\footnote{Take the curves $E: y^2 = x(x -1)(x-3)$ and $E': y^2 = x(x-1)(x-1 /3)$ defined over $K = \hat{\Q_3^{\text{un}}}$.
An isomorphism is given by the coordinate transformation $y\mapsto 3\sqrt{3} y, x\mapsto 3x$, and hence is defined over $K(\sqrt{3} )$. However the curves $E, E'$ are not isomorphic over $K$ as they have different reduction type.}.

\begin{remark}
	\Cref{prob:main_problem} can be considered as the discretely valued analogue of a classical excercise in the setting where $K$ is algebraically closed.
	See for exampele \cite[exericise 6.1.3.3]{temkinIntroductionBerkovichAnalytic2010}.  
	In this exercise they compute $\mathscr E\an$ by determining the topological branch locus of $f\an$ in $\pro^{1, \text{an}}_K$, using analytic tools that only work when $K$ is algebraically closed. 
	As we will see the problem is much more involved in our discretely valued setting. 
\end{remark}











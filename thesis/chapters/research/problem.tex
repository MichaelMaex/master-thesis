Let $R$ be a complete discretly valued field with fraction field $K$ and algebraically closed residue field $k$. Suppose that  $\ch R = 0$ and $\ch k = p$ with $p$ possibly $0$. 
Let $E$ be an elliptic curve over $K$.

The minimal regular model determines the reduction type of $E$. 
The minimal regular model also determines the minimal snc-model $\mathscr E_\text{min} $, which inturn determines $\sk(\mathscr E_\text{min} )$ and the essential skeleton $\sk(E)$. 
So we may wonder whether we can determine the reduction type of $E$ by studying the $E\an$, in particular the essential skeleton $\sk(E)$.
\Cref{tab:skeleton_by_kodaira_neron} gives the essential skeleton of $E$ by reduction type. 
The topology of $\sk E$ we can distinguish three sets of reduction types 
\begin{description}
	\item[point] $\mathrm I_0, \mathrm{II}, \mathrm{III},\mathrm{IV}, \mathrm{I}_0^*$ 
	\item[line] $\mathrm{II}^*, \mathrm{III}^*, \mathrm{I}_n^*$ for some $n > 0$
	\item[circle]  $\mathrm{I}_1, \mathrm{I}_n$ for some  $n > 0$
	\item[three fan] $\mathrm {IV}^*$
\end{description}
By taking account more information of the skeleton like the lenghts, the genus of points, or the multiplicity of points, it is possible to divide these lists finer.
In particular the length of $\sk(E)$ is known, it is possible to find the $n$ in the reduction types $I_n$ and $I_n^*$. 


\begin{table}[t]
	\centering
	\caption{The minimal regular model, minimal scn model $\mathscr E_\text{min} $, skeleton induced by $\mathscr E_\text{min} $ and the essential skeleton.
	Inessential components of the $\mathscr E_\text{min} $ are coloured green. Components with multiplicity greater than one have their multiplicities labeled in blue. 
Points in the skeleton with genus greater than 0 are also labeled in blue. 
Lengts of line segments and circles in the skeleta are labeled in black.}
	\label{tab:skeleton_by_kodaira_neron}
	\begin{tabular}{c|c|c|c|c}
		type & minimal regular model	& miniman snc-model &  $\sk(\mathscr E_\text{min} )$ & $\sk(E)$ \\
		\hline
		$\mathrm{I}_0$ & \incfigsmall{reg-i-0} & \incfigsmall{reg-i-0} & 
		\incfigsmall{i0-sk-snc} & \incfigsmall{i0-sk-snc} \\
		\hline 
		$\mathrm I_1$  & \incfigsmall{i1-reg} & \incfigsmall{i1-snc} & \incfigsmall{i1-sk-snc} & \incfigsmall{i1-sk-snc} \\
		\hline
		$\mathrm I_n$ & \incfigsmall{in-reg} & 
		\incfigsmall{in-reg} & \incfigsmall{in-sk-scn} & \incfigsmall{in-sk-scn} \\
		\hline
		$\mathrm{II}$ &  \incfigsmall{ii-reg} & \incfigsmall{ii-snc} & \incfigsmall{ii-sk-snc} & \incfigsmall{ii-sk}
 \\ 
		\hline 
		$\mathrm{III}$ &  \incfigsmall{iii-reg} & \incfigsmall{iii-snc}   & \incfigsmall{ii-sk-snc}  &  \incfigsmall{ii-sk}
\\
		\hline 
		$\mathrm{I}_0^*$ &  \incfigsmall{i0s-reg}&  \incfigsmall{i0s-snc} &  & \incfigsmall{ii-sk} \\
		\hline 
		$\mathrm{I}_n^*$ \\
		\hline 
		$\mathrm{IV}^*$ \\
		\hline 
		$\mathrm{III}^*$ \\
		\hline 
		$\mathrm{II}^*$ \\
		\hline 
	\end{tabular}
\end{table}
\todo{finish this table on a brain dead moment. fix I1 }

So the question becomes: given an elliptic curve $E$ over $K$, can we get infomration of $\sk(E)$ like, the topology, lenghts, geni,\ldots without computing an snc model of  $E$?
There is only one curve, of which we know can study the berkovich analytification without resorting to models, $\pro^{1}_K$.
As elliptic curves are hyper elliptic, there is a degree $2$ morphism $f: E \to \pro^{1}_K$. 
So our hope is that by studying the analytification of this morphism $\phi: E\an \to \pro^{1, \text{an}}_K$ we can recover information on the essential skeleton of $E\an$. 
We will limit ourselves to the following setting.
\begin{problem}\label{prob:main_problem}
	Let $E$ be the elliptic curve given by the Weierstrass equation 
	\begin{equation}\label{eq:weierstrass_problem}
		y^2 = x(x-1)(x-\lambda) = x^3 + (-1 - \lambda) x^2 + \lambda x
	,\end{equation}
	for some $\lambda \in K^\times $ such that $|\lambda| \le 1, |\lambda - 1| = 1$. 
	Let $f: E \to \pro^{1}_K$ given by projection to the $x$-axis, which a degree 2 morphism ramified over $4$ points. 
	What is $\sk E$? What is the reduction type of $E$?
\end{problem}

Requiring that $E$ is of the from in \eqref{eq:weierstrass_problem} puts some implicit assumptions on $E$ and the map $f: E \to \pro^{1}_K$.
In particular it forces the ramification points of $f$ to be defined over $K$, which as we will see in \cref{sec:expectations} limits the reduction types that can occur. 

The assumption that $|\lambda| \le 1, |1 - \lambda| = 1$ has the following berkovich geometry interpretation.
We can consider $0, 1, \lambda, \infty$ as closed points of $\pro^{1}_K$, and thus also as type 1 points in $\pro^{1, \text{an}}_K$.
Let  $\Gamma_\lambda$ be the convex hull of the points $0, 1, \lambda, \infty$ in $\pro^{1}_K$. 
Then  $\Gamma_\lambda$ can have 4 different configurations depending on $|\lambda|$ and $|1-\lambda|$. 
Our assumptions enfore that $\Gamma_\lambda$ is of one of the first two configurations from \cref{fig:configurations_of_gamma_lambda}.
\begin{figure}[ht]
    \centering
    \incfig{configurations-of-gamma-lambda}
    \caption{configurations of $\Gamma_\lambda$ (in red)}
    \label{fig:configurations_of_gamma_lambda}
\end{figure}
If  $\Gamma_\lambda$ is not of the second configuration (as in \cref{fig:configurations_of_gamma_lambda}) then we may by move to any other of the 3 remaining configurations by a möbius transformation on $\pro^{1}_K$, taking $\lambda$ to $\lambda'$. 
This will induce an isomorphism $E_L \to E'_L$ where $E'$ is the elliptic curve defined by $y^2 = x(x-1)(x-\lambda')$ and $L$ is a finite extension of $K$. 
Note  that $L$ cannot always be takeng to be $K$ itself
\footnote{Take the curves $E: y^2 = x(x -1)(x-3)$ and $E': y^2 = x(x-1)(x-1 /3)$ defined over $K = \hat{\Q_3^{\text{un}}}$.
An isomorphism is given by the coordinate transformation $y\mapsto 3\sqrt{3} y, x\mapsto 3x$, and hence is defined over $K(\sqrt{3} )$. The curves $E, E'$ are not isomorphic over $K$ however as they have different reduction type.}.


\todo{do I need to expand on this more?}












In this chapter we use the theory of the previous chapters to study the relation between the reduction type of an elliptic curve and its Berkovich analytificication.  

\section{Problem statement and approach} \label{sec:problem_statement}
Let $R$ be a complete discretely valued field with fraction field $K$ and algebraically closed residue field $k$. Suppose that  $\ch R = 0$ and $\ch k = p$ with $p$ possibly $0$. 
Let $E$ be an elliptic curve over $K$.

The minimal regular model determines the reduction type of $E$. 
The minimal regular model also determines the minimal snc-model $\mathscr E_\text{min} $, which in turn determines $\sk(\mathscr E_\text{min} )$ and the essential skeleton $\sk(E)$. 
So we may wonder whether we can determine the reduction type of $E$ by studying $E\an$, in particular the essential skeleton $\sk(E)$.
\Cref{tab:skeleton_by_kodaira_neron} gives the essential skeleton of $E$ by reduction type. 
Only considering the topology of $\sk E$ we can distinguish three sets of reduction types 
\begin{description}
	\item[point] $\mathrm I_0, \mathrm{II}, \mathrm{III},\mathrm{IV}, \mathrm{I}_0^*, \mathrm{II}^*, \mathrm{III}^*$ 
	\item[line] $\mathrm{I}_n^*$ for some $n > 0$
	\item[circle]  $\mathrm{I}_n$ for some  $n > 0$
\end{description}
If we know the \emph{combinatorical type of $\sk(E)$}, i.e.\ know all the multiplicities and genera of the divisorial points, then we can distuinguish between all types except $\mathrm{II}, \mathrm{II}^*$ and $\mathrm{III}, \mathrm{III}^*$ and  $\mathrm{IV}, \mathrm{IV}^*$.
In particular if the length of $\sk(E)$ is known, it is possible to find the $n$ in the reduction types $\mathrm I_n$ and $\mathrm I_n^*$. 


So the question becomes the following. Given an elliptic curve $E$ over $K$, can we get information about $\sk(E)$ like, the topology, lengths, genera,\ldots without computing a snc-model of  $E$?
The projective line $\pro^{1}_K$ is the only curve of which we know how to describe its Berkovich analytification without using models. 
As elliptic curves are hyper elliptic, there is a degree $2$ morphism $f: E \to \pro^{1}_K$. 
Our hope is that by studying the analytification of this morphism $\phi: E\an \to \pro^{1, \text{an}}_K$ we can recover information on the essential skeleton of $E\an$. 
We will limit ourselves to the following setting.
\begin{problem}\label{prob:main_problem}
	Let $E$ be the elliptic curve given by the Weierstrass equation 
	\begin{equation}\label{eq:weierstrass_problem}
		E: y^2 = x(x-1)(x-\lambda) = x^3 + (-1 - \lambda) x^2 + \lambda x
	,\end{equation}
	for some $\lambda \in K^\times $ such that $|\lambda| \le 1, |\lambda - 1| = 1$. 
	Let $f: E \to \pro^{1}_K$ given by projection to the $x$-axis, which a degree 2 morphism ramified over $4$ points. 
	What is $\sk (E)$? What is the reduction type of $E$?
\end{problem}
\setlength{\LTcapwidth}{5.7in}
\begin{longtable}{c|c|c|c|c}
		\caption{The minimal regular model, minimal snc-model $\mathscr E_\text{min} $, skeleton induced by $\mathscr E_\text{min} $ and the essential skeleton.
	Inessential components of the $\mathscr E_\text{min} $ are colored green. Components with multiplicity greater than one have their multiplicities labeled in blue. 
Points in the skeleton with genus greater than 0 are labeled in red. 
Lengths of line segments and circles in the skeleta are labeled in black.
If $\sk(E)$ is a point $x$, then it will be labeled with its multiplicity $N(x)$ in blue.}
	\label{tab:skeleton_by_kodaira_neron}
	\\
	type & minimal regular model	& minimal snc-model &  $\sk(\mathscr E_\text{min} )$ & $\sk(E)$ \\
	\hline
	$\mathrm{I}_0$ & \incfigsmall{reg-i-0} & \incfigsmall{reg-i-0} & 
	\incfigsmall{i0-sk-snc} & \incfigsmall{i0-sk-snc} \\
	\hline 
	$\mathrm I_1$  & \incfigsmall{i1-reg} & \incfigsmall{i1-snc} & \incfigsmall{i1-sk-snc} & \incfigsmall{i1-sk-snc} \\
	\hline
	$\mathrm I_n$ & \incfigsmall{in-reg} & 
	\incfigsmall{in-reg} & \incfigsmall{in-sk-scn} & \incfigsmall{in-sk-scn} \\
	\hline
	$\mathrm{II}$ &  \incfigsmall{ii-reg} & \incfigsmall{ii-snc} & \incfigsmall{ii-sk-snc} & \incfigsmall{ii-sk}
 \\ 
	\hline 
	$\mathrm{III}$ &  \incfigsmall{iii-reg} & \incfigsmall{iii-snc}   & \incfigsmall{iii-sk-snc}  &  \incfigsmall{iii-sk}
\\
	\hline 
	$\mathrm{IV}$ &  \incfigsmall{iv-reg} & \incfigsmall{iv-snc}   & \incfigsmall{iv-sk-snc}  &  \incfigsmall{iv-sk}
\\
	\hline 
	$\mathrm{I}_0^*$ &  \incfigsmall{i0s-reg}&  \incfigsmall{i0s-snc} & \incfigsmall{i0s-sk-snc} & \incfigsmall{i0s-sk} \\
	\hline 
	$\mathrm{I}_n^*$ & \incfigsmall{ins-reg} &\incfigsmall{ins-snc}  &\incfigsmall{ins-sk-snc} &\incfigsmall{ins-sk}\\
	\hline 
	$\mathrm{IV}^*$ & \incfigsmall{ivs-reg} & \incfigsmall{ivs-snc}  & \incfigsmall{ivs-sk-snc} & \incfigsmall{iv-sk} \\
	\hline 
	$\mathrm{III}^*$ & \incfigsmall{iiis-reg} & \incfigsmall{iiis-snc} &\incfigsmall{iiis-sk-snc} &\incfigsmall{iii-sk}\\
	\hline 
	$\mathrm{II}^*$ & \incfigsmall{iis-reg} & \incfigsmall{iis-snc} &\incfigsmall{iis-sk-snc} &\incfigsmall{ii-sk}\\
	\hline 
\end{longtable}



Requiring that $E$ is of the form in \eqref{eq:weierstrass_problem} puts some implicit assumptions on $E$ and the map $f: E \to \pro^{1}_K$.
In particular it forces the ramification points of $f$ to be defined over $K$, which as we will see in \cref{sec:expectations} limits the reduction types that can occur. 

The assumption that $|\lambda| \le 1, |1 - \lambda| = 1$ has the following Berkovich geometric interpretation.
We can consider $0, 1, \lambda, \infty$ as closed points of $\pro^{1}_K$, and thus also as type 1 points in $\pro^{1, \text{an}}_K$.
Let  $\Gamma_\lambda$ be the convex hull of the points $0, 1, \lambda, \infty$ in $\pro^{1, \text{an}}_K$. 
Then  $\Gamma_\lambda$ can have 4 different configurations depending on $|\lambda|$ and $|1-\lambda|$. 
Our assumptions enforce that $\Gamma_\lambda$ is of one of the first two configurations from \cref{fig:configurations_of_gamma_lambda}.
\begin{figure}[ht]
    \centering
    \incfig{configurations-of-gamma-lambda}
    \caption{configurations of $\Gamma_\lambda$ (in red)}
    \label{fig:configurations_of_gamma_lambda}
\end{figure}
If  $\Gamma_\lambda$ is not of the second configuration (as in \cref{fig:configurations_of_gamma_lambda}) then we may move to any other of the 3 remaining configurations by a Möbius transformation on $\pro^{1}_K$, taking $\lambda$ to $\lambda'$. 
This will induce an isomorphism $E_L \to E'_L$ where $E'$ is the elliptic curve defined by $y^2 = x(x-1)(x-\lambda')$ and $L$ is a finite extension of $K$. 
Note  that $L$ cannot always be taken to be $K$ itself
\footnote{Take the curves $E: y^2 = x(x -1)(x-3)$ and $E': y^2 = x(x-1)(x-1 /3)$ defined over $K = \hat{\Q_3^{\text{un}}}$.
An isomorphism is given by the coordinate transformation $y\mapsto 3\sqrt{3} y, x\mapsto 3x$, and hence is defined over $K(\sqrt{3} )$. However the curves $E, E'$ are not isomorphic over $K$ as they have a different reduction type.}.

\begin{remark}
	\Cref{prob:main_problem} can be considered as the discretely valued analogue of a classical excercise in the setting where $K$ is algebraically closed.
	See for exampele \cite[exericise 6.1.3.3]{temkinIntroductionBerkovichAnalytic2010}.  
	In this exercise they compute $\mathscr E\an$ by determining the topological branch locus of $f\an$ in $\pro^{1, \text{an}}_K$, using analytic tools that only work when $K$ is algebraically closed. 
	As we will see the problem is much more involved in our discretely valued setting. 
\end{remark}












\section{Expectations} \label{sec:expectations}
Using Tate's algorithm we can find the answer to \cref{prob:main_problem} without studying the map $\phi: E\an \to \pro^{1, \text{an}}_K$. 
While the goal is to eliminate the use of Tate's algorithm and find as much information as possible about $E$ by studying its Berkovich geometry, 
it is useful to know what we expect to find, to guide our intuition and make conjectures. 

Let $E$ be the curve defined by \eqref{eq:weierstrass_problem}. 
Then using the notation from \cite[p.364]{silvermanAdvancedTopicsArithmetic1994} we have \[
a_2 = -1 - \lambda,\quad a_4 = \lambda, \quad a_1, a_3, a_5, a_6 = 0
,\] 
and \[
b_2 = -4 \lambda - 4, \quad b_4 = 2\lambda, \quad b_6 = 0, \quad b_8 = -\lambda ^2
.\] 
The discriminant is given by \[
	\Delta = 16\lambda^4 - 32\cdot \lambda^3 + 16\lambda^2 = 16 \lambda^2(\lambda - 1)^2
.\] 
Additionally we let $n = v(\lambda)$.
So $\lambda = u \cdot \pi^{n}$ for some unit $u$ of $R$.

To follow along with the computations it is best to have the description of Tate's algorithm in \cite[sec. IV.9]{silvermanAdvancedTopicsArithmetic1994} open at the same time, as we move through the steps. 

\subsection{Tate's algorithm when $\ch k$ is not 2} \label{sec:char_k_is_not_2}
We first treat the case where the characteristic of the residue field $k$ is different from 2. 

\step 1 
We have that $|\Delta| = |\lambda|^2 |1-\lambda|^2 = |\lambda|^2 $. 
So $\pi \mid \Delta$ if and only if  $|\lambda| < 1$. 
We find the following. 
\begin{tateconclusion}
	If  $n = 0$, then $E$ is of type $\mathrm I_0$. 
\end{tateconclusion}
We now continue under the assumption that $n > 0$, i.e.\ $\pi \mid \lambda$. 

\step 2 
The reduction of the Weierstrass model is given by the equation $y^2 = x^2 (x - 1)$ over $k$. 
So the singular point is already at $(0, 0)$ and we may continue the algorithm without coordinate change. 
As $\pi \nmid 4, \pi \mid \lambda$, we see that $\pi \nmid b_2 = -4\lambda - 4$. 
So $E$ has reduction type $\mathrm I_m$ with \[
	m = v(\Delta) = v(16 \lambda^2 (\lambda - 1)^2)  = 2v(\lambda) = 2n
.\]
\begin{tateconclusion}
	If $n > 1$ then $E$ is of type $\mathrm I_{2n}$.
\end{tateconclusion}
This concludes Tate's algorithm and we summarize our findings in the following proposition. 

\begin{proposition}\label{prop:conclusion_tate_tame}
	Suppose $\ch k \ne 2$ and $E$ be as in \cref{prob:main_problem}. 
	Then $E$ has reduction type $\mathrm I_{2n}$ with $n = v(\lambda)$. 
\end{proposition}

\subsection{Tate's algorithm when $\ch k$ is 2}  \label{sec:char_k_is_2}

Before starting Tate's algorithm we first perform an coordinate change on $E$ in order to reduce it to a minimal Weierstrass model in the cases that we work out and avoid getting stuck in an arbitrarily long loop inside Tate's algorithm. 
Recall that the curve in \cref{prob:main_problem} is given by \[
	E: y^2 = x(x-1)(x-\lambda) = x^3 + (-1 -\lambda) x^2 + \lambda x, \quad \lambda = u \cdot \pi^{n}, \quad 2 = v\cdot \pi^{v(2)}
.\] 
\begin{notation}
	We write $\lambda_i = \pi^{-i} \lambda = u\cdot \pi ^{n -i}$ and $2_i = \pi^{-i}2 = v \cdot \pi^{v(2) - i}$
\end{notation}
\nomenclature[i]{$a_i$}{$\pi^{i}a$ where $a \in K$ and $\pi$ is uniformiser of $K$}
\nomenclature[E]{$E$}{An elliptic curve over $K$}
We first substitute $y' = y + x$. 
Hence we obtain the curve with Weierstrass equation \[
	E: y'^2 - 2xy' = x^3 + (-2-\lambda) x^2 + \lambda x
.\] 
Now we want to reduce the Weierstrass equation by substituting $(x, y') = (\pi^{2i} x'', \pi^{3i}y'')$ for $i \ge 0$ as large as possible, such that the coefficients still lay in $R$. 
We easily see that \[
	i =  \min\left(v(2), \left\lfloor \frac{v(2 + \lambda)}{2} \right\rfloor,  \left\lfloor \frac{v(\lambda)}{4} \right\rfloor\right) = \min\left(v(2), \left\lfloor \frac{v(2 + \lambda)}{2} \right\rfloor,  \left\lfloor \frac{n}{4} \right\rfloor\right) 
\]
We will assume that $v(\lambda + 2) = \min(v(\lambda), v(2))$ (e.g.\ when $n \ne v(2)$), then we further see that \[
	i = \min\left(v(2), \left\lfloor \frac{\min (v(2), n)}{2} \right\rfloor,  \left\lfloor \frac{n}{4} \right\rfloor\right)  = \min\left(\left\lfloor \frac{v(2)}{2}\right\rfloor, \left\lfloor \frac{n}{4} \right\rfloor \right)
.\] 
From now on we'll assume that $n \ne 2v(2)$. 
Tate's algorithm requires a more complex coordinate change if $n = v(2)$ and there was insufficient time to work out this case. 

So the equation becomes (omitting the double primes after $x, y$) \[
	E: y^2 - 2_i xy = x^3 + (-2_{2i} - \lambda_{2i})x^2 + \lambda_{4i}x
.\] 
Then \[
a_1 = -2_i, \quad a_2 = -2_{2i} - \lambda_{2i}, \quad a_4 =  \lambda_{4i}, \quad a_3, a_6 = 0 
.\] 
Also 
\[
	b_2 = -2_{i}^2 (\lambda + 1), \quad b_4 = 2_{2i}\lambda_{2i}, \quad b_6 = 0,\quad  b_8 = -\lambda_{4i}^2
.\] 
And the discriminant is \[
	\Delta = \pi^{-12i} 2^{4}\cdot \lambda^2(1 - \lambda)^2
.\] 
Note that \[
	v(\Delta) = 4\cdot v(2)+ 2n - 12 \min\left(\left\lfloor \frac{v(2)}{2}\right\rfloor, \left\lfloor \frac{n}{4} \right\rfloor \right)
\]
\step 1 
We rewrite $\Delta = 2^{2} 2_{2i}^2 \lambda_{4i}^2 (1- \lambda)^2$. 
As $\pi\mid 2$ we see that $\pi \mid \Delta$. 
So we must continue to step 2.

We now split up in two cases, $n > 2v(2)$ and $n < 2v(2)$. 
Lets consider the case  $n > 2v(2)$ first.
\subsubsection{Case $n > 2v(2)$} \label{sec:case_n_>_2v2$}

\step 2
If $v(2)$ is even, then $i = v(2) / 2$. 
So the reduction of $E$ has the equation \[
	\tilde E: y^2 = x^3  + \overline{2_{2i}} x^2  = x^3 + vx^2 
.\]
We see that $\tilde E$ is singular at $(0,0)$. 
So there is no need to change coordinates.

Likewise, if $v(2)$ is odd, 
then $i = (v(2)-1) / 2$ and the reduction of $E$ looks like 
\[
\tilde E: y^2 = x^3 
.\] 
We see that  $\tilde E$ is singular $(0, 0)$. 
We find that $\pi \mid b_2$ so we have to continue to step 3. 

\step 3 
Recall $a_6 = 0$. So $\pi^2 \mid a_6$ and we continue to step 4. 

\step 4
We compute \[
	v(b_8) = 2(n - 4i) = \begin{cases}
		2(n - 2v(2) + 1) & 2 \nmid v(2) \\
		2(n - 2v(2)) & 2 \mid v(2) 
	\end{cases} 
\]
If $2\mid v(2)$ and $n = 2v(2) + 1$ then $v(b_8) = 2$ and $\pi^3 \nmid b_8$. 
So we find the following:
\begin{tateconclusion}
	If $2 \mid v(2)$ and $n = 2v(2) + 1$ then $E$ is of type $\mathrm{III}$.
\end{tateconclusion}
We will now continue under the assumption that either $2\nmid v(2) $ or $n \ge 2v(2) +2$.
Then we see that $v(b_8) \ge 3$. 
So $\pi^3\mid b_8$ and we continue to step 5.

\step 5 
Recall that $b_6 = 0$. So $\pi^3 \mid b_6$ and we continue to step 6. 

\step 6
We have  \[
	v(a_1) = v(2) - i > 0
,\] 
thus $\pi \mid a_1$. 
Further \[
	v(a_2) = v(2) - 2i = \begin{cases}
		0 & 2 \nmid v(2) \\
		1 & 2 \mid v(2)
	\end{cases}
.\] 
So $\pi \mid a_2$ if and only if $2 \nmid v(2)$.
We have $a_3 = 0$, hence $\pi^2 \mid a_3$. 
Finally \[
	v(a_4) = n - 4i = \begin{cases}
		n - 2v(2) & 2 \mid v(2) \\
		n - 2v(2) + 2 & 2 \nmid v(2)
	\end{cases}
.\] 
Recall that assumed that $n \ge 2v(2) + 2$ if $2 \mid v(2)$. 
So we find that $\pi^2 \mid a_4$ either way. 
Trivially $\pi^3 \mid a_6 = 0$. 
So there is no need to change coordinates in this step if $2\nmid v(2)$. 


From now on we assume that $2\nmid v(2)$ as for $2 \mid v(2) $ the algorithm leads to a lot of case distinctions and explicit computations in Sagemath result in some odd behavior. 
I strongly suspect that our Weierstrass equation is not be reduced in general if $2\mid v(2) $. 

For now assume that $2 \nmid v(2)$. 
Then the polynomial is given by \[
	P(T) = T^3 + \overline{\pi^{-1} 2_{2i}} T^2 = T^3 + vT^2
.\] 

\step 7
The polynomial  $P(T)$ has double root at $0$ and distinct single root $\overline{b}$ over $k$. 
So $E$ is of type $\mathrm I_m^*$. 
To determine $m$ we have to follow the subprocedure as $\ch k = 2$. 

Consider the families of polynomials 
\begin{align*}
	F_j(Y) &= Y^3 + \pi^{-i-2} a_3 Y - \pi^{-2i-3} a_6 = Y^2 \\
	G_j(X) &= -\pi^{-1}(-2_{2i} - \lambda_{2i}) X^2 + \pi^{-3 - j}\lambda_{4i} X
,\end{align*}
Note that $F_j$ has double root in $0$ for every $j$.  
As long as $3 + j + 4i \le n$ we find that $G_i(X)$ has coefficients in $R$. 
If $j + 3 + 4i <  n$ we obtain that $G_i(X) \equiv v X^2 \mod \pi$ has double root in $0$. 
If $3 + j + 4i = n$ we obtain that $G_i(X) \equiv v X^2 + uX$. Recall that $\lambda = u \cdot \pi^{n}$. 
So $G_j$ has distinct roots. 
This is exactly when $j = n - 3 - 4i = n - 1 - 2v(2)$. 
At this point we are $2(j + 1)$ steps into the subprocedure. 
So we find that 
\begin{tateconclusion}
	If $n > 2v(2)$ and $2\nmid v(2)$ then $E$ is of type $\mathrm I_m^*$ with  $m = 2(j + 1) = 2n - 4v(2)$.
\end{tateconclusion}


That concludes the case where $n > 2v(2)$. 
We now move to the case where $n < 2v(2)$. 

\subsubsection{Case $n < 2v(2)$} \label{sec:case_n_<_2v2}

In this case $i = \left\lfloor \frac{n}{4} \right\rfloor$.

\step 2
Note that $\pi\mid 2_{2i}$ so $\pi\mid b_2$. 
The reduction of $E$ is 
\[
\tilde E: y^2 = x^3 +  \overline{\lambda_{4i}}x
.\] 
If $n \equiv 0 \mod 4$ then $v(\lambda_{4i}) = 0$ and $\overline{\lambda_{4i}} = \overline{u}$. 
Then $\tilde E$ is singular at $(0, \sqrt{\overline{u}})$. 
So we must perform the translation $y' = y + s $ where $s \in R$ such that $\overline{s}^2 = u$. This results in $\overline{a_6} = \overline{u}$.
So briefly going into step $3$ we see that $\pi^2 \nmid a_6$ and find 
\begin{tateconclusion}
	If $n < 2v(2)$ and $n \equiv 0 \mod 4$ then $E$ is of type $\mathrm{II}$. 
\end{tateconclusion}

We continue step 2 under the assumption that $n \not\equiv 0 \mod 4$ then $v(\lambda_{4i}) = n - 4 \left\lfloor \frac{n}{4} \right\rfloor >  0$. 
So $\tilde E$ is singular at $(0, 0)$. And we do not need to do a coordinate transformation. 



\step 3
Recall that $a_6 = 0$. So trivially $\pi^3 \mid a_6$ and we move on to step 4.

\step 4
Trivially $\pi^3 \mid b_6 = 0$ so we continue with step 5.

\step 5
If $n \equiv 1$ then $v(\lambda_{4i}) = 1$. So $\pi^3 \nmid b_8 = - \lambda_{4i}^2$. 
The algorithm terminates and states the following. 
\begin{tateconclusion}
	If $n < 2v(2)$ and $n \equiv 1 \mod 4$ then $E$ is of type  $\mathrm {IV}$. 
\end{tateconclusion}
We continue under the assumption that $n \equiv 2$ or $n \equiv 3 \mod 4$. 

\step 6
We easily see that $\pi \mid a_1, a_2$ and trivially $\pi^2 \mid a_3, \pi^3 \mid a_6$ as $a_3 = a_6 = 0$. 
As we assume $n \equiv 2$ or $n \equiv 3 \mod 4$ then $\pi^2 \mid a_4$. 
So we don't need to change coordinates. 

The polynomial $P(T)$ is \[
	P(T) = T^3  -\pi^{-1}(2_{2i} - \lambda_{2i})T^2 + \pi^{-2}\lambda_{4i} T
.\] 
If $n \equiv 3 \mod 4$ then $\overline{\pi^{-2} \lambda_{4i}} = 0$. 
So $P(T)$ has at least double root $0$.
If $n = 2v(2) + 1$ then  $P(T) = T^3 - vT^2$. So $P(T)$ has a double root $0$ and single root $u$. 
If $n > 2v(2) + 1$ then $P(T) = T^3$ and has triple root $0$. 

If $n \equiv 2 \mod 4$ then $\pi^{-2} \lambda_{4i} = u$. 
Note that $n = 2v(2) +1 $ is impossible as $n$ is even. So $\overline{\pi^{-1} (2_{2i} - \lambda_{2i})} = 0$ and we find that \[
	P(T) = T^3 + u T
.\] 
Then $P(T)$ has double root $\sqrt{\overline{f}} $ and single root $0$ over $k$.

There is no case in which $P(T)$ has distinct roots. So we move to step 7.

\step 7
As computed in step 6 we find the following. 
\begin{tateconclusion}
	If $n \equiv 2 \mod 4$ or $n \equiv 3 \mod 4 \wedge n = 2v(2) - 1$ then $E$ is of type $I_m^*$ for some $m$. 
\end{tateconclusion}

Note that if we want to follow the subprocedure to find $m$ when $n \equiv 2 \mod 4$ we first have to perform a coordinate transformation $x' = x + s$, where $s \in R$ is some inverse image of $\sqrt{\overline{u}} $ in $k$

\todo{compute $m$ in this case}
If  $n \equiv 3 \mod 4$ and $n > 2v(2) - 1$ then $P(T)$ has triple root $0$ and we continue to step 8.

\step 8
As $a_3, a_6 = 0$ we see that $Y^2 + a_{3,2} Y - a_{6, 4} = Y^2 $ which has double root $0$. 
So we continue with step 9. 

\step 9
Suppose that $ \pi^{4} \mid a_4 = \lambda_{4i}$ then $i \ne \left\lfloor \frac{n}{4} \right\rfloor$.
So we find that $\pi^{4} \nmid a_4$. 
\begin{tateconclusion}
	If $n < 2v(2), n\equiv 3 \mod 4$ and $n < 2v(2) - 1$ then $E$ is of type $\mathrm{III}^*$. 
\end{tateconclusion}

This concludes the case where $\ch k = 2$. 
We recap our findings in the following proposition.

\begin{proposition}\label{prop:tate_char_is_2}
	Suppose $\ch k = 2$ and $E$ is as in \cref{prob:main_problem} and $n:= v(\lambda) \ne 2v(2), v(\lambda + 2) = \min(v(\lambda), v(2))$
	then $E$ has the type
	\begin{itemize}
		\item if $n > 2v(2)$ and $v(2)$ is odd:$\mathrm I_{2n - 4v(2)}^*$
		\item if $n < 2v(2)$ 
			\begin{itemize}
				\item and $n \equiv 0 \mod 4$:  $\mathrm {II}$
				\item and  $n \equiv 1 \mod 4$:  $\mathrm {IV}$
				\item and $n \equiv 2 \mod 4$:  $\mathrm I^*_m$ for some $m > 0$
				\item and $n\equiv 3 \mod 3$: 
					\begin{itemize}
						\item $n = 2v(2) - 1$:  $\mathrm I_m^*$ for some $m > 0$ 
						\item $n < 2v(2) - 1$: $\mathrm {III}^*$
					\end{itemize}
			\end{itemize}
	\end{itemize}
\end{proposition}




\section{Some preparatory lemmas} \label{sec:some_preparatory_lemmas}
The following lemma is stated much more generally than we will need it. 
But it is a neat result, and the proof is neither long, nor complicated. So we put it in. 
\begin{lemma}\label{lem:number_divisorial_points}
	Let $\mathscr C$ be a snc-model of  $C$ and suppose that $e$ is an edge between two divisorial points of degree $1$ in $\Delta(\mathscr C)$. 
	Let $f(n)$ be the number of divisorial points of degree $n$ on the edge  $e$ in $\sk(\mathscr C)$. 
	Then \[
		f(n) = \begin{cases}
			2 & n = 1 \\
			\varphi(n) & n > 1
		\end{cases}
	,\] 
	where $\varphi$ is the Euler totient function. 
\end{lemma}
\begin{proof}
	As the problem is local, we may assume that $\mathscr C$ is a (not necessarily proper) model with just two components of multiplicity $1$ intersecting transversely in one point. 
	So $\sk(\mathscr C)$ is just the edge $e$. 

	Let $x$ be any divisorial point on $e$. 
	Then by the factorization theorem $x$ can be obtained as the vertex in the dual graph of a snc-model, which is obtained by repeatedly blowing up in intersection points. 

	We construct a sequence of snc-models \[
	\ldots \to 	\mathscr C_n \to \mathscr C_{n -1} \to \ldots \to \mathscr C_1 = \mathscr C
,\]
each dominating the previous, as follows. 
The model  $\mathscr C_i$ is the blowup of $\mathscr C_{i-1}$ in all intersection points $\xi$ such that lies on two components with multiplicity $N, M$ and $N + M \le i$. 
Then $x$ is a vertex on $\Delta(\mathscr C_n)$ for sufficiently large $n$.
In the dual graph this repeatedly subdivides the edge by adding more vertices of higher multiplicity. 
\\
\noindent\incfig{dual-graphs-farey}
\\
We now obtain the result from noting that these multiplicities are exactly the denominators from the Farey sequence for which the result is folklore. 
\end{proof}


\begin{lemma}\label{lem:equality_lenghts_preserve_mult}
	Let $f: X \to Y$ be a finite morphism of curves and $p$ a path in $\sk(\mathscr X)\subset X\an $  such that $f\an|_p$ is injective and lands in $\sk(\mathscr Y)$ for some snc-models $\mathscr X, \mathscr Y$ of $X, Y$ respectively. 
	Suppose further that for every divisorial point $x \in p$  we have $N(x) = N(f\an(x))$. 
	Then $f\an(p)$ has the same length as $f\an$. 
\end{lemma}
\begin{proof}
	As the divisorial points are dense on the path $p$ we may assume that $p$ is a closed path that starts and ends in a divisorial point, $x, y$ respectively.
	By repeatedly blowing up in divisorial points we may replace $\mathscr X, \mathscr Y$ by models that such that $p$ and $\phi(p)$ are made up of a number of edges on $\sk(\mathscr X), \sk(\mathscr Y)$, respectively. 
	As the question is local we may replace $\mathscr X, \mathscr Y$ to opens in them such that $\sk(\mathscr X) = p, \sk(\mathscr Y) = f\an (p)$. (Note that $\mathscr X, \mathscr Y$ are no longer proper, but still have normal crossings).  
	Then $\Delta(\mathscr X)$ as a graph is a path ending and starting with vertex $x, y$ and $\Delta(\mathscr Y)$ is a path starting and ending with $f\an(x), f\an(y)$. 

	Let \[
		N = \max \{N(v) \st v \text{ is a vertex in } \Delta(\mathscr X) \text{ or } \Delta(\mathscr Y)\} 
	.\] 
	We will now describe how to blow up $\mathscr X$ and $\mathscr Y$ repeatedly in intersection points such that the vertices in the dual graphs are precisely the divisorial points of multiplicity less than or equal to $N$. 

	Let \[
	\mathscr X_N \to \mathscr X_{N - 1} \to \ldots \to \mathscr \mathscr X_1 = \mathscr X
	\] 
	be a sequence of locally snc-models of  $X$, where $\mathscr X_i$ is the blowup of $\mathscr X_{i-1}$ in all intersection points of laying on components with multiplicity $N_1, N_2$ such that $N_1 + N_2 \le i$. 
	Note that all components in the exceptional divisor have multiplicity at most $i$. 

	Note that $\Delta(\mathscr X_1)$ contains all divisorial points in $\sk(\mathscr X)$ of multiplicity 1. If not, let $z$ be such a divisorial points. 
	Then we can repeatedly blowup $\mathscr X$ in intersection points to obtain $\mathscr X'$ such that $z$ is in $\Delta(\mathscr X')$. 
	But $z$ is the result of a blowup in an intersection point, and thus has multiplicity at least $2$. 

	By induction we see that $\mathscr X_i$ contains all divisorial points in $\sk(\mathscr X)$ of multiplicity less than or equal to $i$, for $i = 1, \ldots, n$. 
	Indeed suppose that this is the case for $\mathscr X_{i -1}$ and $z$ is a divisorial point that we obtain by repeatedly blowing up intersection points.
	Then  $z$ lies between two adjacent vertices $v_1, v_2$ of multiplicity $N_1, N_2$ respectively in $\mathscr X_{i}$. 
	We easily see that $z$ has at least multiplicity $N_1 + N_2$. 
	So it is sufficient to show that $N_1 + N_2 > i$. 
	Suppose for the sake of contradiction that $N_1 + N_2 \le i$.
	Then both $N_1 \le i -1$ and $N_2 \le i-1$. 
	So $v_1, v_2$ are vertices that also belong to $\Delta(\mathscr X_{i-1})$, where they are also adjacent. 
	But as $N_1 + N_2 \le i$ we know that $\Delta(\mathscr X_{i})$ has a vertex in between $v_1, v_2$. This contradicts that $v_1, v_2$ are adjacent in $\Delta(\mathscr X_{i})$. 

	Now we see that $\Delta(\mathscr X_N)$ contains all divisorial points in $\sk(\mathscr X) = p$ of multiplicity less than or equal to $n$. 
	As in the process no vertices of multiplicity higher than $N$ were added, we see that $\Delta(\mathscr X_N)$ contains exactly the vertices of multiplicity less than or equal to $N$. 

	Similarly we construct a model $\mathscr Y_N$ such that the vertices on  $\Delta(\mathscr Y_N)$ are exactly all divisorial points on $\sk(\mathscr Y) = f\an(p)$ of multiplicity less than or equal to $N$. 

	Let $V$ be the set of vertices on $\Delta(\mathscr X_N)$. 
	As $f\an $ on $p$ preserves multiplicity we see that $f\an(V)$ are precisely the vertices of $\Delta(\mathscr Y_N)$ and $f\an$ even preserves the order of the vertices. 
	So $f\an$ induces an isomorphism of graphs $\Delta(\mathscr X_N) \to \Delta(\mathscr Y_N)$ where each vertex is labeled with its multiplicity. 
	It follows from the definition of the length on a graph that $|\Delta(\mathscr X_N)| = p$ has the same length as $|\Delta(\mathscr Y_N)| = f\an(p)$. 
\end{proof}

\begin{remark}\label{rem:length_edge_skeleton_cover}
	The main context in which the assumptions of \cref{lem:equality_lenghts_preserve_mult} are satisfied is when $f:X \to Y$ is a degree $d$ map of curves, $e$ is an edge in a skeleton of $Y$ such that $e \subset Y\an \setminus B_{f\an}$, and $e'$ is a lift of $e$ in $X\an$. 
	Here $B_{f\an}$ is the topological branch locus of the map $f\an: X\an \to Y\an$.
	Indeed, every divisorial point $x \in e$ has $d$ preimages, and by \cref{prop:balancing_finite_morphism} $x$ and it's preimages have the same multiplicity. 
	In this context we find that $e'$ and $e$ have the same length. 
\end{remark}

\begin{remark}
	\Cref{lem:number_divisorial_points,lem:equality_lenghts_preserve_mult} suggest that there is an intrinsic definition of the metric on a Berkovich analytic curve that does not depend on an analytification map $ C\an \to C$ or models of $C$, by counting the number of divisorial points of certain multiplicities on edges. 
	Indeed in \cite{jonssonConvergenceAdicPluricanonical2020a}, the measure on a Berkovich analytification is proven to be the same as a limit of Shilov measures. When the definitions are unrolled this comes down to counting divisorial points of certain degrees. 
\end{remark}

Finally we need a lemma that relates the description of $\pro^{1, \text{an}}_{K}$ from \cref{sec:the_berkovich_affine_line} so the language of models and divisorial points. 

\begin{lemma}\label{lem:model_disk_proj_line}
	Let $\mathscr P_{a,j} $ with $K, j \in \Z$ be the snc-model of $\mathscr \pro_{K}^{1}$ with only one irreducible component given on an affine chart by $\spec R[b], b = \pi^{-j}(x - a)$.
	Then the divisorial point associated to the unique component of the special fiber is the valuation corresponding to the disk $B(a, |\pi|^{j})$. 
\end{lemma}
\begin{proof}
	Note that a birational point in $X\an$, as a norm is uniquely determined by its valuation on polynomials. 
	Let $f \in K[X]$ be a polynomial and $f = \sum_{i = 0}^{n} a_i (x - a)^{i} $ be its Taylor expansion around $a$. 
	Let $\xi$ be the generic point of the special fiber of $\mathscr P_{a, j}$. 
	Note that this is the ideal $(\pi)$.
	Similarly the maximal ideal of $\mathcal{O}_{\xi}$ is generated by $(\pi)$ and let $v_{\xi}$ be a associated valuation.
	Note that a polynomial is only divisible by $\pi$ if all its coefficients are.
	So \[
		v_\xi(f) = v_\xi\left( \sum_{i = 0}^{n} a_{i} (x-a)^{i} \right) = v_\xi\left( \sum_{i = 0}^{n} a_i\pi^{ij}b^{i} \right) = \min_{i} v_{\xi}(a_i \pi^{ij}) 
	,\]
	which is the norm we constructed in \cref{sec:the_berkovich_affine_line} corresponding to the disk $B(a, |\pi|^{j})$.
\end{proof}




\section{ A weight function on $\pro^{1}_K$} \label{sec:a_weight_function_on_pro_1_k}
Recall $\Gamma_\lambda$ from \cref{fig:configurations_of_gamma_lambda}. 
Let $\delta = f_* \ram f$ which as a divisor on $\pro^{1}_K$ is the sum of the points on $\pro^{1}_K$ corresponding to $0, 1, \lambda, \infty$. 
\subsection{Finding a minimal scn model} \label{sec:finding_a_minimal_scn_model}
We can build a snc-model $\mathscr P$ of the pair $(\pro^{1}_K, \delta)$, i.e.\ $\mathscr P_s + \overline{\delta}$ is a strict normal crossings divisor, such that the associated skeleton $\sk(\mathscr P, \delta)$ in $\pro^{1}_K$ is exactly $\Gamma_\lambda$. 
The skeleton associated to a pair is defined in \cite[§3.2.1]{bakerWeightFunctionsBerkovich2016}.

We write $\pro^{1}_K = \spec K[x] \cup \spec K[\frac{1}{x}]$.
If $n = 0$ we can choose $\mathscr P = \pro^{1}_R$. Then the special fibre is $\pro^{1}_k$ and the points $0, 1, \lambda, \infty$ reduce to the distinct points $\overline{0}, \overline{1}, \overline{\lambda}, \overline{\infty}$.
So $\bar{\delta} + \pro^{1}_k$ is an snc-divisor on $\mathscr P$. 
See \cref{fig:model_pair_n_1}. 
\begin{figure}[h]
    \centering
    \incfig{the-model-pair-n-1}
    \caption{The minimal snc-model of the pair $(\pro^{1}_K, \overline{\delta})$ if $n =v(\lambda) = 0$.}
    \label{fig:model_pair_n_1}
\end{figure}

Let 
\begin{align}\label{eq:singular_model_pair}
	\mathscr P' &= \spec \frac{R[x, y]}{(xy - \pi^{n})} \cup \spec \frac{R\left[ \frac{1}{x},y \right]}{\left(y-\frac{\pi^{n}}{x}\right)} \cup \spec \frac{R\left[x, \frac{1}{y}\right]}{\left(x - \frac{\pi^{n}}{y}\right)} \\
	&=  \spec \frac{R[x, y]}{(xy - \pi^{n})} \cup \spec R\left[ \frac{1}{x} \right] \cup \spec R\left[ \frac{1}{y}\right] \nonumber 
 \end{align} 
This is a projective model of $\pro^{1}_K$. 
The special fibre on one the first affine chart is \[
	\spec k \times _{R} \spec \frac{R[x, y]}{(xy -\pi^{n})}=\spec  \frac{k[x, y]}{(xy)}
.\] 
Hence $\mathscr P'_s$ are two rational curves intersecting in one point. 
Let $F, G$ be the irreducible components cut out by $x, y$ respectively, which are given by the third and second chart in \eqref{eq:singular_model_pair} respectively. 
Note that $\mathscr P'$ is not regular at the intersection point.

The reduction of $0$ is the Zariksi closure of $V(x) = V(\pi ^{n} x ) = V(\frac{1}{y})$. 
So we see that $0$ reduces to the origin of the 3'rd affine chart in \eqref{eq:singular_model_pair}, i.e.\ a smooth point on $F$. 
Likewise  $V(x - \lambda) = V(\pi^{-n} x - \pi^{-n}\lambda) = V(\frac{1}{y} - \pi^{-n}\lambda)$. 
So $\lambda$ reduces to a point on $F$ as well, different from $\overline{0}$.
The reduction of $1$ is the Zariski closure of $V(x-1) = V(\frac{1}{x} - 1)$. 
So it lays on $G$. The reduction of $0$ is cut out by $V(\frac{1}{x})$ so it also lays on $G$. 

So we find that $0, \lambda$ reduce to distinct points on $F$ and $1, \infty$ reduce to distinct points on $G$. 
Let $\mathscr P$ be the minimal desingularisation of $\mathscr P'$. 
By \cite[cor.\ 9.3.25]{liuAlgebraicGeometryArithmetic2002} we find that $\mathscr P$ is $\mathscr P'$ where the intersection point is replaced by a chain of $n-1$ rational curves of multiplicity 1 and self intersection $-2$.  
So $\mathscr P$ is a chain of $n + 1$ rational curves. See \cref{fig:model_of_the_pair}. 

This is in fact the minimal snc-model of $(\pro^{1}_K, \delta)$ as the middle components can't be contracted and contracting $\tilde F$ or $\tilde G$ to get $\mathscr P''$, would cause either $0, \lambda$ or $1, \infty$ to be reduced to the same piont. Hence $\delta + \mathscr P''_s$ would no longer be an snc-divisor. 
\begin{figure}[h]
    \centering
    \incfig{the-model-of-the-pair}
    \caption{The minimal snc-model of the pair $(\pro^{1}_K, \overline{\delta})$ if $n =v(\lambda) > 0$. }
    \label{fig:model_of_the_pair}
\end{figure}

\subsection{Computing the skeleton} \label{sec:computing_the_skeleton}

The dual graph of $\mathscr P$ is a path of $n + 1$ vertices, where each edge has length $1$. 
So geometrically $\Delta(\mathscr P)$ is a line of length $n$ if $n > 0$ and a point if $n = 0$.
Let $a, b$ the start and end points of the line, corresponding to the components $\tilde F$ and $\tilde G$ respectively. If $n = 0$ (i.e. $\Delta(\mathscr P)$ is a point) then we let $a = b = \sk(\mathscr P)$. 

Then the paths from $0$ and  $\lambda$ to $\sk(\mathscr P)$ come together at $b$. We see this because $0$ and $\lambda$ both reduce to points on $\tilde G$. 
As they reduce to different points, they paths to $\sk(\mathscr P)$ come it at $b$ from different tangent directions. 
Similarly the paths from $0$ and $\infty$ to $\sk(\mathscr P)$ come together at $a$.

So $\sk(\mathscr P, \delta)$ consists of the paths running running between every pair of $0, 1, \lambda, \infty$. Thus $\Gamma_\lambda \subset \sk(\mathscr P, \delta)$ as $\Gamma_\lambda$ is the convex hull of $0, 1, \lambda, \infty$.
As $\sk(\mathscr P, \delta)$ only leaves are $0, 1, \lambda, \infty$ we also get the converse $\Gamma_\lambda \subset  \sk(\mathscr P, \delta)$. 
So we found a snc-model $\mathscr P$ of $(\pro^{1}_K, \delta)$ such the associated skeleton equals $\Gamma_\lambda$. 

\begin{figure}[ht]
    \centering
    \incfig{skeleton-of-the-pair}
    \caption{The skeleton associated to $\mathscr P$ and $(\mathscr P, \delta)$ (both in red).}
    \label{fig:skeleton_of_the_pair}
\end{figure}

\subsection{$\sk(\mathscr P)$ is a Kontsevich-Soibelman skeleton} \label{sec:$\sk(\mathscr_p)$_is_a_kontsevich-soibelman_skeleton}

\begin{lemma}\label{lem:unique_form_pair}
	There is a rational section $\omega$ of $\Omega_{\pro^{1}_K}^{\otimes 2}$ with poles at  $0, 1, \lambda, \infty$ and no other zeros or other poles. 
	Moreover this rational section is unique up to multiplication by $K$ and the poles are of order 1. 
\end{lemma}
\begin{proof}
	The statement is equivalent to showing that $\dim_K H^{0}(\Omega_{\pro^{1}_K}^{\otimes 2}(\delta))  = 1$.
	We have that $\Omega_{\pro^{1}_K} \simeq \mathcal{O}(-2)$ and $\mathcal{O}(\delta) \simeq \mathcal{O}(\deg \delta) = \mathcal{O}(4)$.
	So \[
		\Omega_{\pro^{1}_K} (\delta) \simeq \mathcal{O}(-1)^{\otimes 2} \otimes \mathcal{O}(4) \simeq \mathcal{O}\left(0 \right)  = \mathcal{O}_X
	.\] 
	Hence $H^{0}(\Omega_{\pro^{1}_K}^{\otimes 2}(\delta)$ is a $1$-dimensional $K$-vectorspace. 
\end{proof}

The proof of the following two results depend heavily on \cite[thm.\ 3.2.3]{bakerWeightFunctionsBerkovich2016}. 
	The metric on $\mathbb{H}(\pro^{1}_K)$ we defined in \cref{sec:}\todo{fill in the section/definition} (the stable metric) differs from the metric in used in \cite{bakerWeightFunctionsBerkovich2016} (the potential metric). 
	However, as we are interested in computing the minimal locus of $\wt_\omega$ by looking at its slopes, the spefic matters in our calculations. 
	So in the next two proof I'll use the potential metric instead of rescaling the slopes.

	The first lemma is a slight refinement of \cite[prop.\ 4.4.4]{mustataWeightFunctionsNonArchimedean2015} in the context of curves.
	\begin{lemma}\label{lem:well_behaved_pole_weight}
	Let $C$ be a curve over $K$ and $\omega$ be a $m$-pluricanonical form.
	Let $\mathrm{Pole}_\omega$ be the divisor that contains all points of $\omega$ and $\mathscr C$ a snc model of the pair $(\mathscr C, \omega)$.  
	If all the poles of $\omega$ are of degree less than $m$, then for all $x \in \pro^{1, \text{an}}_K$ \[
		\wt_\omega(x) \ge \wt_\omega(\rho_{\mathscr C}(x))
	\] 
	with equality if and only if $x \in \sk(\mathscr C)$. 
	\todo{define the reduction map in this sense}
\end{lemma}
\begin{proof}
	If $\red_{\mathscr C}(x)$ does not lie in the closure of $\mathrm{Pole}_\omega$ then the result follows from \cite[prop.\ 4.4.4.(2)]{mustataWeightFunctionsNonArchimedean2015} and the denseness of divisorial points. 

	If $\red_{\mathscr C}(x)$ lies in the closure of some pole $z \in \mathrm{Pole}_\omega$. 
	Let $\ell: [0, a] \to X\an$ be the path running from $\sk \mathscr C$ to $x$, parametrised by length. 
	By assumption this path coincides with the path $m: [0, \infty) : X\an$ from $\mathscr C$ to $z$, also parametrized by $\length$.
	Let $\epsilon$ be the length where $m, \ell$ branch. 
	The slope on  $\ell|_{[0, \epsilon]}$ is positive by \cite[thm.\ 3.2.3.(2)]{bakerWeightFunctionsBerkovich2016} , hence $\wt_\omega \ell(\epsilon) > \wt_\omega \red_{\mathscr C}(x)$.
	If $\epsilon = a$ we are done. 
	If $\epsilon < a$ then repeatedly blowing up $\mathscr C$ in the reduction of $z$ yields a model $\mathscr C'$ whose dual graph contains $\ell(\epsilon)$, but does not contain $x$.
	So the result now follows from applying \cite[prop.\ 4.4.4.(2)]{mustataWeightFunctionsNonArchimedean2015} to the model $\mathscr C'$. 
	\todo{cleanup this proof}
\end{proof}

\begin{proposition}
	Let $\omega$ be as in \cref{lem:unique_form_pair} and $\mathscr P$ the model constructed in \cref{sec:finding_a_minimal_scn_model}.
	Then $\sk(\pro^{1}_K, \omega) = \sk(\mathscr P)$. 
\end{proposition}
\begin{proof}
	\Cref{lem:well_behaved_pole_weight} implies that $\minloc \wt_{w} = \sk(C, \omega) \subset \sk(\mathscr P) $. 
	If $n = 0$ we are done as $\sk(\mathscr P) $ is a point. 
	Suppose that $n > 1$ ($\sk(\mathscr P)$ is a line), then 
	it suffices to show that $\wt_\omega$ is constant on $\sk(\mathscr P)$, i.e.\ constant on the line segment $[ab]$ in \cref{fig:skeleton_of_the_pair}. 
	By \cite[thm.\ 3.2.3.(3)]{bakerWeightFunctionsBerkovich2016} we know that the laplacian of $\wt_\omega$ restricted to $\sk(\mathscr P, \delta)$ is given by \[
		\Delta\left(\wt_\omega|_{\sk(\mathscr P, \delta)}\right) = 2\cdot \sum_{v \in \sk(\mathscr P, \delta)} N(v)(\val (g) - 2g(v) - 2)v = 2a + 2b
	.\] 
	At $a$ the lines $[0, a],[\lambda, a]$ and  $[a, b]$ meet. 
	The outgoing slopes of  $[0, a], [\lambda, a]$ are both $1$ by \cite[3.2.3.(2)]{bakerWeightFunctionsBerkovich2016}. 
	This implies that the outgoing slop at $a$ on the line$ [a, b]$ is $0$. 
	As the laplacian has no points on the interior of $[a, b]$ the slope remains constant on $[a, b]$ and we are done. 
\end{proof}





\section{Tame ramification} \label{sec:if_mk}
In the case where $\ch k \ne 2$ we can use \cref{prop:weightfunction_fullback} to compare the skeleton $\sk(\mathscr P)$ with the essential skeleton $\sk(E)$. 
\begin{lemma}\label{lem:sk_E_pulls_back}
	Let $\mathscr P$ be as in \cref{sec:a_weight_function_on_pro_1_k}, and $E$ from \cref{prob:main_problem}. 
	Then \[
		\sk(E) = \phi^{-1}(\sk(\mathscr P))
	.\] 
\end{lemma}
\begin{proof}
	In \cref{prop:model_P_is_KS} we saw that $\sk(\mathscr P) = \sk(\pro^{1}_K, \sigma)$, where $\sigma$ is the rational section of $\omega_{\pro^{1}_K}^{\otimes 2}$ with poles at $0, 1, \lambda, \infty$, which is unique up to multiplication by $K$. 
	Then $f^* \sigma$ is a $2$-pluricanonical form. 
	Indeed, we can take $\sigma$ as a global section of $\omega_{\pro^{1}_K}^{\otimes 2}(-f_* \ram _f)$. 
	And 
	\[
		f^* (\omega_{\pro^{1}_K / K}^{\otimes 2}(-f_* \ram _f)) = \omega_{E /K}^{\otimes 2} \otimes \mathcal{O}(2 \ram_f) \otimes f^*f_* \mathcal{O}(-\ram _f) = \omega_{E / K}^{\otimes 2}
	.\] 
	We also have 
	\[
		\sk(E, f^*\sigma) = \minloc \wt_{f^* \sigma} = \minloc \wt_{\sigma}\circ \phi = \phi^{-1}(\minloc \wt_\sigma) = \phi^{-1} (\sk(\pro^{1}_K, \sigma))
	.\] 
	Recall that $\omega_{E / K}$ is trivial as $E$ is an elliptic curve. 
	So all pluricanonical forms are the same up to multiplication. 
	So $\sk(E)$ can be computed using only one pluricanonical form, thus \[
		\sk(E) = \sk(E, f^*\sigma) = \phi^{-1}(\sk(\pro^{1}_K, \sigma)) = \phi^{-1}(\sk(\mathscr P))
	.\] 
\end{proof}

\subsection{The semistable case} \label{sec:the_semistable_case}
In this section we answer \cref{prob:main_problem} when $k$ has characteristic different from $2$, under the assumption that $E$ has semi-simple reduction, i.e.\  $E$ is of type $\mathrm I_m$ for some $m \ge 0$. 

\begin{remark}\label{rem:justification_semistable}
	The assumption that $E$ has semi-stable reduction is valid as this is what we have found \cref{sec:char_k_is_2} by means of Tate's algorithm. 
	However assumption also isn't valid for in the case where we choose $|\lambda| > 1$ (see \cref{fig:configurations_of_gamma_lambda}).
	If $|\lambda| > 1$, i.e.\ $v(\lambda) < 0$ then
	explicit computations of examples suggest that  $E$ has semistable reduction if $v(\lambda)$ is even and $E$ is of type $I^*_n$ if $n$ is odd. 
	I do not know of an explanation via Berkovich geometry of this. 
\end{remark}

\begin{lemma}\label{lem:semistable_skeleton}
	Let $\mathscr C$ be minimal snc-model of curve $C$ with semistable reduction.
	Then $\mathscr C$ has no inessential components. 
\end{lemma}
\begin{proof}
	Recall that $\mathscr C$ the minimal resolution of the minimal regular model $\mathscr C_\text{reg} $ of $C$. 
	Suppose for the sake of contradiction that  $C$ has inessential components.
	Then there is ``leaf''  $F$ which is a rational curve in $\mathscr C_s$. 
	If $F$ dominates a component $F'$ in $\mathscr C_\text{reg} $ then $F'$ is also a rational curve of multiplicity $1$ which intersects has self intersection $-1$,  which would contradict the minimality of  $\mathscr C _\text{reg} $ by Castelnouovo's criterion \cref{thm:castelnuovo}. 

	Suppose now that $F$ does not dominate a component in $\mathscr C_\text{reg} $.
	Then it is the $F$ is the exceptional divisor of a blowup in a smooth point of a intermediate model $\mathscr C \to \mathscr C' \to \mathscr C_\text{reg} $. 
	Then $\mathscr C'$ is also snc, which contradicts the minimality of $\mathscr C$. 
\end{proof}
\begin{corollary}
	Let $\mathscr C$ be ithe minimal snc-model of $C$. 
	Then the leaves of the dual graph  $\Delta (\mathscr C)$ have genus greater than $0$. 
\end{corollary}

Recall \cref{lemma:genus_semistable} which states that the Betti-number of $\Delta(C)$ plus the genus of vertices equals $g(C)$. 
In the case of $E$ this leaves two options. 
\begin{lemma}\label{lem:point_or_circle}
	Let $E$ be an elliptic curve with semi-stable reduction and $\mathscr E$ be its minimal semistable model. 
	Then $\Delta(\mathscr E) = \sk(\mathscr E) = \sk(E)$ is either 
	\begin{enumerate}
		\item a circle with every vertex of genus $0$,
		\item a single point of genus  $1$. 
	\end{enumerate}
\end{lemma}
\begin{proof}
	The fact $\sk(\mathscr E) = \sk(E)$ follows from \cref{lem:semistable_skeleton}. 
	We know that $\Delta(E)$ is homotopic to either a circle, or a point of genus $1$. 
	As it has no leaves of genus $0$ and is finite, this means that $\Delta(E)$ is either a circle or a point of genus 1, which proofs the claim.
\end{proof}

\begin{lemma}
	Let $E$ be as in \cref{prob:main_problem}. 
	Then $\sk(E)$ is a circle if $n > 0$ and $\sk(E)$ is a point if $n = 0$. 
\end{lemma}
\begin{proof}
	Recall that $\phi: E\an \to \pro^{1, \text{an}}_K$ is a finite map and that $\sk(E) = \sk(\mathscr P)$ with $\mathscr P$ the model from \cref{sec:finding_a_minimal_snc_model}.
	Suppose that $\sk(E)$ is a circle. 
	Then $\phi(\sk(E))= \sk(\mathscr P)$ is a not a point. 
	Hence $n \ne 0$. 
	Conversely, suppose that $\sk(E)$ is a point. 
	Then $\phi(\sk(E)) = \sk(\mathscr P)$ is a point, and thus $n  = 0$. 
\end{proof}

The only thing left to do is determine the length of the circle $\sk(E)$ when $n \ge 1$.
We know that $\phi|_{\sk(E)}:\sk (E) \to \sk(\mathscr P)$ is surjective map from a circle to a line, such that every point $x \in \sk(\mathscr P)$ has precisely 1 or 2 preimages. 
From this we see that $\phi|_{\sk(E)}$ is a branched cover with branch points in $a, b$, the end points of the line $\sk(\mathscr P)$ (see \cref{fig:skeleton_projective_line}).
\todo{topologically this is obvious, but how would you make this rigorous?}
Then the interior of $\sk(\mathscr P)$ lifts to two lines in $\mathscr  \sk(E)$ along the map $\phi$.
By \cref{rem:length_edge_skeleton_cover} each of these lifts has length $n$. 
So we find that whole circle has length $2n$. 
See \cref{fig:map_circle_line}
\begin{figure}[ht]
    \centering
    \incfig{map-circle-line}
    \caption{The map from $\sk(E)$ to $\sk(\mathscr P)$ inside $\pro^{1, \text{an}}_K$, with each of the lifts of $(a, b)$ in a different color.}
    \label{fig:map_circle_line}
\end{figure}
This gives gives us the following proposition.
\begin{proposition}
	Let $E$ be the curve from \cref{prob:main_problem}. 
	Suppose that $\ch k \ne 2$ and that $E$ has semistable reduction. 
	Then $E$ has reduction type $\mathrm I_{2n}$ with $n = v(\lambda)$. 
\end{proposition}

The picture we sketched above also helps us to understand what happens purely model theoretically. 
Let $\ell_1, \ell_2$ be the two lifts of $(a, b)$ along $\phi$. 
By \cref{prop:balancing_finite_morphism} we see that $\phi$ maps divisorial points in $\ell_1, \ell_2$ to divisorial points in $(a, b)$ of the same degree. 
Recall from \cref{sec:finding_a_minimal_snc_model} that $\mathscr P$ consist of a chain of $n+1$ rational curves of multiplicity 1. 
From this we see that $(a, b)$ contains exactly $n -1$ points of multiplicity 1. 
Hence each of the lines $\ell_1, \ell_2$ also contains $n-1$ points of multiplicity 1.
Together these are $2n - 2$ of the  $2n$ irreduducible components of $\mathscr E_\text{min} $ (the minimal regular/snc model). 

From \cref{prop:balancing_finite_morphism} we cannot see whether $\phi^{-1}(a), \phi^{-1}(b)$ are points of multiplicity 1 one 2. 
But as $\mathscr E_\text{min} $ has $2n$ components with multiplicity $1$, we find that  $\phi^{-1}(a), \phi^{-1}(b)$ must be the 2 missing degree 1 points. 
A more Berkovich theoric argument for this is the following. 
As $E$ has semistable reduction, we know that $\mathscr E_\text{min} $ is semistable, or the blowup of a semistable nc-model. 
As a result all the edges between two multiplicity 1 components are as described in \cref{lem:number_divisorial_points}.
So if $\phi ^{-1}(a)$ has degree 2 it must lay on an edge $e$ with ends points $x, y$ (possibly $x = y$) of multiplicity $1$ with no other multiplicity 1 points in between $x, \phi^{-1}(a)$ and $y, \phi^{-1}(a)$. 
By \cref{lem:number_divisorial_points} $\phi^{-1}(a)$ is the only multiplicity 2 point on $e$. 
But between every multiplicity one point on $\ell_1 \cup \{\phi^{-1}(b)\} $ and $\phi^{-1}(a)$ there is a multiplicity two point. 
This is a contradiction. 
So $\phi^{-1}(a)$ is a multiplicity 1 point, and similiarly $\phi^{-1}(b)$ is a multiplicity 1 point.  


From \cref{rem:balancing_galois_cover} we see that the map between components corresponding to $\phi^{-1}(a)$ and $a$ is a double cover of $\pro^{1}_k$ onto $\pro^{1}_k$.
Likewise for $\phi^{-1}(b)$ and $b$. 
This shows that there is the following morphism of models $\psi: \mathscr E_\text{min} \to \mathscr P$ extending the map $f: E \to \pro^{1}_K$, which we have illustrated in \cref{fig:morphism_models_semistable_tame}.

\begin{figure}[ht]
    \centering
    \incfig{morphism-models-semistable-tame}
    \caption{A morphimss }
    \label{fig:morphism_models_semistable_tame}
\end{figure}
From this we also see that $\mathscr E_\text{min} $ is actually the normalisation of $\mathscr P$ in $K(E)$. 
Such pair of models is sometimes referred to as a simultanious resolution of singularities \cite[sec.\ 6]{liuModelsCurvesFinitea}.



\subsection{Determining $m$ when $E$ is of type $\mathrm I_m^*$} \label{sec:determining_m_when_e_is_of_type_ims}
We now consider what happens in the hypothetical case that $E$ has reduction of type $I_m^*$ for some $m$. 
As we've discussed in \cref{rem:justification_semistable}, this does not happen in the context of \cref{prob:main_problem}, but it does happen when $n < 0$, i.e. $|\lambda| > 1$.  
In this case we can construct a similar model to $\mathscr P$, whose dual graph is the middle line in the case in the case $|\lambda| > 1$ from \cref{fig:configurations_of_gamma_lambda}. 


\begin{lemma}\label{lem:divisorial_points_Is}
	Let $x$ be a divisorial point in $\sk(E)$. 
	Then  \[
		N(x) \ge 2
	,\] 
	with equality if and only if $x$ is one of the vertices in $\sk(\mathscr E_\text{min} )$ where $\mathscr E_\text{min} $ is the minimal regular model of $E$. 
\end{lemma}
\begin{proof}
	Let $\Delta$ be the the dual graph of the essential components of $\mathscr E_\text{min} $. 
	Recall that $\Delta$ consists of $m + 1$ vertices of multiplicity $2$ and that there is a natural isomorphism $ |\Delta| \simeq \sk(E)$. 

	Let $x$ be a divisorial point on $\sk(E)$.
	Then $x$ can be obtained as a vertex of a model $\mathscr E$ by repeatedly blowing up the intersection points in $\mathscr E_\text{min} $ between components in $\sk(E)$. 
	Every exceptional divisor of such blowup will have multiplicity strictly greater than $2$. 
	So $x$ has multiplicity $2$ if it was one of the original vertices in $\Delta$, and multiplicity strictly greater than $2$ if not. 
\end{proof}

\begin{proposition}\label{prop:m_when_E_type_Is}
	Suppose that $|v(\lambda)| = n$ and $E$ is of type $I_m^*$, then $m = 2n$. 
\end{proposition}
\begin{proof}
	We start of locally so $\mathscr P'$ be a the non-proper model of $\pro^{1}_K$ that only contains to adjacent irreducible components $F, G$ of  $\mathscr P$.
	Let $x, y $ be the corresponding divisorial points in $\pro^{1, \text{an}}_K$. 
	Then $x, y$ are of multiplicity $1$, hence $\phi^{-1}(x), \phi^{-1}(y) \subset  \sk(E)$ are finite sets of points of multiplicity $1$ or $2$. 
	\Cref{lem:divisorial_points_Is} tells us that the points in $\phi^{-1}(x)$ are exactly of multiplicity 2, thus correspond to essential components in $\mathscr E_\text{min} $.
	From this we also see that $\phi^{-1}(x)$ is a point. 
	Similarly $\phi^{-1}(y)$ is a point of multiplicity 2 corresponding to an essential component in $\mathscr E$. 

\begin{figure}[ht]
    \centering
    \incfig{models-proof-is}
    \caption{The models and skeleta in the proof of \cref{prop:m_when_E_type_Is}.}
    \label{fig:models_proof_is}
\end{figure}

	Let $g: N(\mathscr P') \to \mathscr P'$ be the normalisation of $\mathscr P'$ in $K(E)$.
	From the computations above we see that $N(\mathscr P')_s$ contains two irreducible components, the strict transforms of $F, G$, both of multiplicity $2$. 
	We can see that $N(\mathscr P')$ is not regular. 
	Indeed if it was regular, by the adjunction formula we find that \[
		2 = 2F\cdot G =  (g^* F)\cdot (g^* G) =  (2 \tilde F) \cdot  (2 \tilde G)  \ge 4
	,\] 
	which clearly is a contradiction. 
	This tells us that $\tilde F, \tilde G$ are not adjacent in $\mathscr E_\text{min} $. 
	Let $H_1, \ldots, H_r$ be the missing components between $\tilde F, \tilde G$ of $\mathscr E_\text{min} $. 
	Let $\mathscr E'$ be the (non-proper) model of $E$ containig $\tilde F, H_1, \ldots, H_r, \tilde G$. See \cref{fig:models_proof_is}.
	
Then the images of $\phi(H_1), \ldots, \phi(H_r)$ are distinct divisorial points of multiplicity $1$ or $2$.
By \cref{lem:number_divisorial_points} we know that $\sk(\mathscr P')$ has only 3 points such points, two of which are $x, y$. 
Hence $r = 1$. 

This concludes the local computation. Globally this means the following. 
So we see that there is one component of degree 2 in $\mathscr E_\text{min} $ for every irreducible component of $\mathscr P$, and one for every intersection point. 
Hence there are $n + 1 + n = 2n + 1$ components of multiplicity $2$ in $\mathscr E_\text{min} $ from which it follows that $m = 2n$.  
\end{proof}


We can actually sketch a similar picture of a map of models as we did at the end of \cref{sec:the_semistable_case} of a map between models. 
Let $\mathscr E_\text{min} $ be the minimal snc model of $E$. 
Let $\mathscr E'_{\text{min}}$ the (singular) model of we obtain by contracting the four multiplicity 1 components in  $\mathscr E_\text{min} $. 
So $\mathscr E'_{\text{min}, s} $ is a chain of rational curves of multiplicity 2. 
Let $\mathscr P'$ be the model we obtain by blowing up each intresection point in $\mathscr P$. 
So between each two reduced rational cruves in $\mathscr P_s$ we place a multiplicity 2 rational curve.

By from our argument in the proof \cref{prop:m_when_E_type_Is}, we see that there is a morphism $\psi: \mathscr E'_\text{min}  \to \mathscr P'$ visualised in \cref{fig:dualgraphs_is_models} and that $\mathscr E'_\text{min} $ is the normalisation of $\mathscr P'$ in  $K(E)$. 
\todo{what happens to the weierstrass component? I suspect the image lays on on the lines $[a, 0]$, $[a, \lambda]$, $[b, \infty]$, $[b, 1]$}
\begin{figure}[ht]
    \centering
    \incfig{dualgraphs-is-models}
    \caption{Dualgraphs of the models $\mathscr E_\text{min} , \mathscr E_\text{min} ', \mathscr P', \mathscr P$ and the relations between them.}
    \label{fig:dualgraphs_is_models}
\end{figure}



\subsection{Distinguishing between $\mathrm I_m$ and  $\mathrm I_m^*$} \label{sec:distinguishing_between_im_and_ims}

Recall that $E$ is given by the Weierstrass equation 
\begin{equation}\label{eq:weierstrass_dist}
	E: y^2 = x(x-1)(x-\lambda)
,\end{equation}
with $\lambda = u \cdot \pi^{n}$ and $u$ some unit in $R$. 
If $n \ge 0$ we found that this Weierstrass equation is minimal in  \cref{sec:char_k_is_not_2} as Tate's algorithm ended after only one iteration. 
So the component of the Weierstrass model is a multiplicity 1 component of $\mathscr E_\text{min} $, the minimal snc-model. 
Let $w$ be the divisorial point associated to component.
If we consider the morphism of models (on affine charts) \[
	\spec \frac{R[x, y]}{(y^2 - x(x-1)(x-\lambda))} \to \spec R[x]
\]
we see that the map dominates on the special fiber. 
So $\phi(w)$ is the point corresponding to the disk $B(0, 1)$ (\cref{lem:model_disk_proj_line})
But this exactly the point $b$. So $\phi(w) = b$ and in particular $w \in \phi^{-1}(\mathscr P) = \sk(E)$. 
So $w$ is a multiplicity 1 essential component of $\mathscr E_\text{min} $.
This can only be the case if $E$ is of type $\mathrm I_{m}$ for some $m \ge 0$. 
As a bonus we also see that when $n = 0$,  $E$ is of type $\mathrm I_0$ and not of some other reduction type with $\sk(E)$ a point. 

Putting all information in this section together we have obtained an alternative and almost purely Berkovich geometric proof for \cref{prop:conclusion_tate_tame}, the conclusion of Tate's algorithm. 
\medskip

In \cref{rem:justification_semistable} we said that when $n < 0$ calculations in Sagemath suggest that whether $E$ is semistable depends on the parity of $n$. 
Here is a (conjectural) explanation for this. 
When $n < 0$, the Weierstrass equation \eqref{eq:weierstrass_dist} is no longer defined over $R$. 
After performing a coordinate transformation $(x_{2i}, y_{3i}) = (\pi^{-2i}x, \pi^{-3i}y)$ with $i = \left\lceil \frac{n}{2} \right\rceil $, the Weierstrass equation becomes \[
	E: y_{3i}^2 = x_{2i}(x_{2i}-\pi^{i})(x_{2i} - \pi^{i}\lambda)
.\] 
This Weierstrass equation is defined over $R$ and conjecturally minimal. 
Again let $w$ be the divisorial point associated to this Weierstrass model. 
From a similar map of models \[
	\spec \frac{R[x_{2i}, y_{2i}]}{\left(y_{3i}^2 - x_{2i}(x_{2i}-\pi^{i})(x_{2i} - \pi^{i}\lambda)\right)} 
	\to \spec R[x_{2i}]
,\] 
we see that $\phi(w)$ is the disk $B(0, |\pi|^{-n})$ if $n$ is even and $B(0,|\pi|^{-n-1}) $ when $n$ is odd. 
For this configuration of $\Gamma_\lambda$ (recall \cref{fig:configurations_of_gamma_lambda}) we can also construct a similar model $\mathscr P'$ of $\pro^{1}_K$ whose skeleton is the line from $B(0, 1)$ to $B(0, |\pi|^{-n})$ and such that $\sk(E) = \phi^{-1}(\sk(\mathscr P'))$.
In particular $w \in \sk(E)$  if and only if $i$ is even. 
So when $i$ is even the Weierstrass component is essential, thus the reduction type is $\mathrm I_m$ for some $m$. 
And when  $i$ is odd, the Weierstrass component is inessential. 
So the reduction type of $E$ is $\mathrm I_m^*$ for some $m$. 







\section{Wild ramification} \label{sec:wild_ramification}
If $\ch k =2$ we can no longer use \cref{prop:weightfunction_fullback} to show that  $\sk(E) = \phi^{-1}(\sk(\mathscr P))$. 
In fact, as we will see in this section, the conclusion $\sk(E) = \phi^{-1}(\sk(\mathscr P))$ will even be false. 
But by a closer examination of Tate's algorithm we can still identify $\phi(\sk(E))$ as a subbset of $ \pro_K^{1, \text{an}}$. 
This allows us to make some conjectures about the behaviour of error term $\delta^{\text{log}}$ from \cref{rem:weightfunction_fullback_art}. 


Recall that in \cref{sec:char_k_is_2} we found that \begin{equation}\label{eq:wild_minimal_weierstrass}
y_{3i}^2 - 2_i x_{2i} y_{3i} = {x_{2i}}^3 + (-2_{2i} - \lambda_{2i}) {x_{2i}}^2 + \lambda_{4i} x_{2i}
,\end{equation} 
with \[
	y_{3i} = \pi^{3i} y,\quad x_{2i} = \pi^{2i} x, \quad i = \min\left( \left\lfloor \frac{v(2)}{2}\right\rfloor, \left\lfloor \frac{n}{4} \right\rfloor  \right), \quad 2 = v\cdot \pi^{v(2)},\quad  \lambda = u \cdot \pi^{n}
.\] 
is a minimal Weierstrass equation for $E$ if $v(\lambda) \ne v(2)$, as Tate's algorithm ends after only one iteration. 
\todo{This might not be entirely accurate}

We will have to split up differerent cases. 
\subsection{Case $v(\lambda) > 2v(2)$ and $v(2)$ is odd} \label{sec:case_v_lambda_>_2_v_2_and_v_2_is_even}
In this case we have $i = (v(2) - 1) / 2$ and we computed that $E$ is of type $\mathrm I_{2n - 4v(2)}^*$ (\cref{prop:tate_char_is_2}).
Let $\mathscr E_\text{min} $ be the minimal regular model of $E$ (which for curves of type $\mathrm I_m^*$ this case coincides with the minimal snc-model).
The idea is to look at the proof of Tate's algorithm where they compute each component $\mathscr E_\text{min} $ explicitly by repeatedly blowing up the Weierstrass equation \eqref{eq:wild_minimal_weierstrass}. 
Then for each component, we can look for corresponding model of $\pro^{1}_K$ with a single component and a domininating map between the components. 
That way we can find the image of every divisorial point in $\Delta(\mathscr E_\text{min} )$ in $\pro^{1, \text{an}}_K$. 
Fortunately the necessary blowups are computed very clearly in the proof of Tate's algorithm in \cite[p.\ 369-379]{silvermanAdvancedTopicsArithmetic1994}

Note that in \cref{sec:char_k_is_2}, in the case where $n > 2v(2)$ and $v(2)$ is odd, we did not perform any coordinate translations by the time we reached step 7 of the algorithm. 
At the end op p.\ 372 Silverman writes that in his notation $x_{r} = \pi^{r}x, y_{r} = \pi^{r}y$, where $x, y$ are the original coordinates of the Weierstrass equation. 
We start with a weierstrass equation with variables $x_{2i}, y_{3i}$. 
So where Silverman writes $(x_r, y_r)$ in our notation that becomes $(x_{2i+ r}, y_{3i + r})$. 
In the proof of step 5, he starts with the model $\mathcal{V} $ which (on an affine chart) looks like 
\[
	\mathcal{V} = \spec \frac{R[x_{2i + 1}, y_{3i + 3}]}{(y_{3i + 3}^2 + 2_i x_{2i + 1}y_{3i + 3} = \pi x_{2i + 1}^3 + \pi v x_{2i + 1}^2 + \lambda_{4i + 1} x_{2i + 1})} \to \spec R[x_{2i + 1}]
,\] 
note that we added the projection to the $x$-axis. 
On the special fibres this looks like 
\[
	\mathcal{V}_s = \spec \frac{k[x_{2i + 1}, y_{3i + 3}]}{(y_{3i+3}^2)} \to \spec k[x_{2i + 1}]
.\] 
We see that the double line $y_{3i + 3}^2 = 0$ maps surjectively on $k[x _{2i + 1}]$. 
Then he blows up $\mathcal{V} $ in $(0, 0, 0)$ to obtain the scheme
\[
	\mathcal{V}_0  = \spec \frac{R[x_{n_0}, y_{m_0}]}{(\pi y_{n_0} ^2 + \pi2_{i + 1} x_{n_0} y_{n_0} = x_{n_0}^3 +  a_{2,1} x_{n_0}^2 + \lambda_{4i + 2} x_{n_0}^2)} \birat \spec \frac{R[x_{n_0}, z]}{(zx_{n_0}^2 - \pi)}
,\] 
where we used the notation \[
	n_j = 2i + 1 + \left\lfloor \frac{j + 1}{2} \right\rfloor,\quad  m_j = 3i + 2 + \left\lfloor \frac{j }{2} \right\rfloor
.\]
Let $a, b \in R$ be such the roots $0, a, b$ are the roots of $x_{n_0}^3 +  a_{2,1} x_{n_0}^2 + \lambda_{4i + 2} x_{n_0}^2$ and $\overline{a} = 0, \overline{b} = \overline{a_{2,1}} = \overline{v}$.  
These exists because over $k$ the polynomial looks like $x_{n_0}^3 + v x_{n_0}^2$ and by Hensel's lemma the roots lift. 

Then we can define a birational map of schemes \[
	\mathcal{V}_0 \birat \spec \frac{R[x_{n_0}, z]}{(z x_{n_0}(x_{n_0} - a) - \pi)},\quad
	x_{n_0} \mapsto x_{n_0}, z\mapsto \frac{(x - b)}{y_{n_0}^2 + 2_{i + 1} x_{n_0}y_{n_0} }
.\] 
Then the morphism on the special fibre is given by \[
	\mathcal{V} _{0, s} = \spec \frac{k[x_{n_0}, y_{m_0}]}{(x^3_{n_0} + \overline{v} x^2_{n_0})} \birat \frac{k[x_{n_1}, z_0]}{(z_0 \cdot x_{n_0}^2)}
.\] 
So we see that the double line cute out by $x_{n_0}^2$ in the source maps to the double line cut out by $(x_{n_0})^2$ in the image. 

In the proof of step 6 and 7 he continues with the models $\mathcal{V} _j$ which for our coefficients and $2 \nmid j, j\le 2n - 4v(2) - 1$ looks like . 
\[
	\mathcal{V} _j = \spec \frac{R[x_{n_{j}}, y_{m_j}]}{(y_{m_j}^2 + \pi 2_{i + 1} x_{n_j}y_{m_j} = \pi^2 x_{n_j}^3 + \pi a_{2, 1} x_{n_j}^2 + \lambda_{4i + 2 + \left\lfloor j / 2 \right\rfloor} x_{n_j}) } \to \spec R[x_{n_j}]
.\] 
Note that we added the projection to the $x_{n_j}$-axis. 
Then the map on special fibers looks like \[
	\mathcal{V} _{j, s} = \spec \frac{R[x_{n_j}, y_{m_j}]}{(y_{m_j}^2)} \to \spec k[x_{n_j}]
.\] 
Thus the double line given by $y_{m_j} = 0$ maps surjectively onto the line $\spec k[x_{n_j}]$. 

For $2 \mid j, j \le 2n - 4v(2) - 1$, the schemes $\mathcal{V} _j$ look like. 
\[
	\begin{aligned}
		\mathcal{V}_j = \spec \frac{R[x_{n_j},y_{m_j}]}{(\pi y^2_{m_j} + \pi 2_{{i + 1}}x_{2i}y_{n_j} = \pi x_{n_j}^3 + a_{2,1}x_{n_j}^2 + \lambda_{4i + 2 + \left\lfloor j / 2 \right\rfloor} x_{n_j})} \\ \dashrightarrow \spec \frac{R[ x_{n_j}, z_j]}{(z_j(a_{2,1} x_{n_j}^2 + \lambda_{4i + 2 + \left\lfloor j / 2 \right\rfloor})- \pi)} 
	\end{aligned}
,\]
where the birational map is given by $x_{n_j} \mapsto  x_{n_j}, z \mapsto (y^2_{m_j} + 2_{i + 1 }x_{m_j}y_{n_j} - \pi x_{n_j}^3)^{-1}$. 
On the level of special fibers this is given by \[
	\mathcal{V} _{j, s} = \spec \frac{k[x_{n_j}, y_{n_j}]}{(\overline{v}x^2_{n_j})} \birat \spec \frac{k[x_{n_j}, z_j]}{(z_j \cdot  x^2_{n_j})}
,\]
where we see that the double line cout out by  $x_{n_j} = 0$ in the source dominates the double line $x_{n_j} = 0$ in the image. 


\bigskip

The models $\mathcal{V} , \mathcal{V} _{0}, \ldots, \mathcal{V} _{2n - 4v(2) - 1}$ give all of the $2n - 4v(2) + 1$ multiplicity two components of  $\mathscr E_\text{min} $. 
Lets call these divisorial points $\alpha_{-1}, \alpha_0, \ldots, \alpha_{2n-2v(2) - 1}$ respectively. 
We write $\beta_j = \phi(\alpha_j)$. 
Furthermore, for each of these models, we have given a map to a normal model of $\pro^{1}_{K}$  and identified the image of the degree two component of the special fiber. 

If we understand what the divivisorial points of the models of $\pro^{1}_K$ are in $\pro^{1, \text{an}}_K$ we can identify the image of  $\sk(E)$ in $\pro^{1}_K$. 

\begin{lemma}
	Let $\mathscr P_{a,j}$ with $a \in \pro_K^{1}, j \in \Z$ be the model of $\mathscr \pro_{K}^{1}$ given on an affine chart by $\spec R[b], b = \pi^{j}(x - a)$.
	Then the divisorial point associated to the unique component of the special fiber is the valuation corresponding to the disk $B(a, |\pi|^{j})$. 
\end{lemma}
\begin{proof}
	Note that a birational point in $X\an$, as a norm is uniquely determined by its valuation on polynomials. 
	Let $f \in K[X]$ be a polynomial and $f = \sum_{i = 0}^{n} a_i (x - a)^{i} $ be its Taylor expansion around $a$. 
	Let $\xi$ be generic point of the irreducible component of $\mathscr P_{a, j}$. 
	Note that this is the ideal $(\pi, \pi^{j}(x-a))$.
	Similarly the maximal ideal of $\mathcal{O}_{\xi}$ is generated by $(\pi)$ and let $v_{\xi}$ be a associated valuation.
	Then \[
		v_\xi(f) = v_\xi\left( \sum_{i = 0}^{n} a_{i} (x-a)^{i} \right) = v_\xi\left( \sum_{i = 0}^{n} a_i\pi^{-ij}b^{i} \right) = \min_{i} v_{\xi}(a_i \pi^{-ij}) 
	,\]
	which is the norm we constructed in \cref{sec:the_berkovich_affine_line} corresponding to the disk $B(a, |\pi|^{-j})$.
	\todo{confused about the signs in the exponents of $\pi$ here}	
\end{proof}

So we see that the image of $\alpha_{j}$ is is $B(0, |\pi|^{\ldots})$ when $j$ is odd. \todo{figure out whats wrong here}
It should be possible to give a similar algebraic argument for $x_{j}$ when $j$ is even.
But instead of trying to determine the norm from the model of $\pro^{1}_K$, we can find $\phi(x_j)$ entirely from a Berkovich theoretic argument. 
Again we will use an argument that uses the number and distribution of divisorial points of a given degree on a line in $\sk(E)$ and $\pro^{1, \text{an}}_K$.
We do this to show that even in the wildly ramified case, we can use some Berkovich geometric arguments. 
The argument is quite very geometric, so it is best to follow along on \cref{fig:argument_ims_wild}

\medskip

Without loss of generality we will determine $\beta_0$ using what we know about $\beta_{-1}$ and $\beta_1$. 
As we've seen $\beta_{-1}, \beta_{1}$ are adjacent vetrices in $\Delta(\mathscr P)$. 
There is a unique point $\gamma$ on $(\beta_{-1}, \beta_1)$ such that $N(\gamma) = 2$.  
We expect that like in \cref{sec:determining_m_when_e_is_of_type_ims} $\phi(\alpha_0) = \beta_0 = \gamma$. 
Suppose for the sake of contradiction that $\beta_0 \ne \gamma$.



As $\pro^{1, \text{an}}_K$ is a uniquely geodesic space, we know that $\phi((\alpha_{-1}, \alpha_{1}))$ contains the entire line  $(\beta_{-1}, \beta_1)$. 
Further more, as $\alpha_{-1}$ is a multiplicity 2 point, mapping to the multiplicity 1 point $\beta_{-1}$, \cref{rem:balancing_galois_cover} tells us that $\alpha_{-1}$ is the unique preimage of $\beta_{-1}$. 
Likewise $\alpha_1$ is the unique preimage of $\beta_1$. 
So the $\phi((\alpha_{-1}, \alpha_{1}))$ is contained in the open annulus $A$ bounded by $\beta_{-1}, \beta_1$.
Note that the only divisorial points of multplicity 2 in $A$ are $\gamma$ and in horizontal branches originating in $\gamma$. There are no multiplicity 2 points in  $A$. 
\begin{figure}[ht]
    \centering
    \incfig{argument-ims-wild}
    \caption{Finding $\beta_0 = \phi(\alpha_0)$ in $\pro^{1,\text{an}}_{K}$. 
    Note that this figure is inaccurate as it is part of a proof by contradiction.}
    \label{fig:argument_ims_wild}
\end{figure}
As $\phi((\alpha_{-1}, \alpha_{1}))$ contains the entire line $(\beta_{-1}, \beta_1)$ we know that $\gamma$ has a preimage in that line, which by \cref{rem:balancing_galois_cover} is either of multiplicity $2$ or $4$. 
We've assumed it is not $\beta_0$, the only point of multiplity 2 in  $(\beta_{-1}, \beta_1)$.
So one of the two multiplicity 4 points must map to $\gamma$. 
Let $\delta$ be the multiplicity 4 point on $(\alpha_0, \alpha_1)$ and suppose without loss of generality that $\phi(\delta) = \gamma$. 
By \cref{rem:balancing_galois_cover} we know that $\delta$ is the unique preimage of $\gamma$. 

So $\phi(\alpha_0)$ must be point of multiplicity 1 or 2. As there are no points of multiplicity $1$ in $A$ and we asummed $\phi(\alpha_0)\ne \gamma$ we find that that $\phi(\alpha_0) = \beta_0$ must be one on one of the branches coming of $\gamma$. 
Again using the fact that $\pro^{1, \text{an}}_K$ is uniquely geodesic we know that $(\beta_0, \gamma] \cup [\gamma, \beta_{-1}) \subset  \phi((\gamma, \alpha_{-1}))$. 
So $\gamma$ has a preimage different from $\delta$ on $(\gamma, \alpha_{-1})$. 
This is a contradiction. 
Thus $\phi(\beta_0) = \gamma$. 

Pooling together everything we know about $\alpha_{-1}, \ldots, \alpha_{2n - 4v(2)-1}$ we see that $\sk(E)$ maps onto a line inside $\sk(\mathscr P)$ with length $n- 2v(2)$ whose start and end points are at distance $v(2)$ from $a, b$.
See \cref{fig:image_between_skeleta_wild_ims}.
\begin{figure}[h]
    \centering
    \incfig{image-between-skeleta-wild-ims}
    \caption{The image of $\sk(E)$ in  $\pro^{1, \text{an}}_K$ in the case where $n > 2v(2)$}
    \label{fig:image_between_skeleta_wild_ims}
\end{figure}

\subsection{The case where $n < 2 v(2)$} \label{sec:the_case_where_n_<_2_v2}

\subsubsection{If $n$ is even} \label{sec:if_$n$_is_even}

If we will make the simplifying assumption that $\sqrt{\lambda} $ exists. 
\begin{remark}
	This is equivalent to saying that $\sqrt{u} $ exists.
	This is not guaranteed to be the case as we cannot use Hensel's lemma to lift the root $\sqrt{\overline{u}} $ from $k$ to $R$, because the derivative vanishes $(x^2 - u)' = 2x = 0$. 
	I'm fairly sure that if $\sqrt{u} $ does not exits in $ K$, the argument can be salvaged by passing to the field extension $K(\sqrt{u} )$. 
	This is an immediate extension, i.e.\ it is unramified and isomorphism on the residue fields. 
	But to show that the argument is still valid needs to be checked.
\end{remark}

Let $g: \pro^{1}_K \to \pro^{1}_K$ be the automorphism induced by $x \mapsto \lambda / x$. 
This extends to an automorphism $h: E \to E$ induced by the coordinate transformation $x \mapsto  \lambda / x, y\mapsto \sqrt{\lambda ^3} x y$. 
So we have the following commutative diagram.
\[
\begin{tikzcd}
	E\an \rar{h\an} \dar{\phi} & E\an \dar{\phi} \\
	\pro^{1, \text{an }}_K \rar{g\an} & \pro^{1, \text{an}}_K
\end{tikzcd}
.\] 

On $\pro^{1}_K$, $g$ is the map that maps $0 \leftrightarrow \infty, 1 \leftrightarrow \lambda$. 
So we see that $g\an$ leaves $\Gamma_\lambda$ invariant. In particular $\sk(\mathscr P)$ is a fixed set of $g\an$, but $g\an$ flips the orientation of $\sk(\mathscr P)$, i.e. $g\an(a) = b, g\an(b) = a$. 
As $h$ is an automorphism we have and the essential skeleton is canonical, $\sk(E)$ a fixed set of $h\an$. 
Thus $\phi(\sk(E))$ is a fixed set of $g\an$.
Let $c$ be the mid point of $\sk(\mathscr P)$. 
As $n$ is even we see that $c$ is a multiplicity 1 point. 

\medskip

If  $n \equiv 0 \mod 4$ then $E$ is of type $\mathrm{II}$.  
So $\sk(E)$ is a single point of multiplicity 6. 
Therefore  $\phi(\sk(E))$ is a fixed point of $g\an$. 
Then fixed points of $g\an$ are the line  $[\sqrt{\lambda}, c] \cup [-\sqrt{\lambda} , c]$ where $\sqrt{\lambda} $ is the closed point in $\pro^{1}_K$. 
Note that these lines branch off in the same tangent direction at $c$ as the reduction map at that point maps both directions to  $\sqrt{\overline{u}} $ in $k$. 
So $\sk(E) \in [\sqrt{\lambda}, c] \cup [-\sqrt{\lambda}, c] $, but $\phi(\sk(E)) \ne c$ as a multiplicity 6 point must map to a multiplicity 3 or 6 point, under a degree 2 Galois cover. 
So $\phi(\sk(E))$ is a point off the line $\sk(\mathscr P)$. 

\begin{figure}[h]
    \centering
    \incfig{image-skeleton-wild-n-le-2v2-n-even}
    \caption{The embedding of $\sk(E)$ in $\pro^{1, \text{an}}_K$ when $n < 2v(2)$ and  $n$ is even.}
    \label{fig:image-skeleton-wild-n-le-2v2-n-even}
\end{figure}
\medskip 
If $n \equiv 2 \mod 4$ then $E$ is of type $\mathrm I_{m}^*$ for some $m$. 
Note that at at step 7 we had to change coordinates $x_{i+1}' = x_{i + 1}  + s$ where $\overline{s} = \sqrt{\overline{u}} $. 
Following the same argument as in \cref{sec:case_v_lambda_>_2_v_2_and_v_2_is_even} we see that degree two components of $\mathscr E_\text{min} $ lie in a branch of $c$. 
But by as we changed coordinates, they lay in the branch of $c$ with the same tangent direction as $\sqrt{\lambda} $. 
As the map $\phi|_{\mathscr E}$ is injective by the same argument as in \cref{sec:case_v_lambda_>_2_v_2_and_v_2_is_even} we further see that the $\phi(\sk(E))$ must lay on the line $[\sqrt{\lambda}, c] $ or $[-\sqrt{\lambda} , c]$
So again we find that $\phi(\sk (E))$ is not contained in $\sk(\mathscr P)$. 


\subsubsection{If $n \equiv 1 \mod 4$} \label{sec:if_n_equiv_1_mod_4$}

Let $c$ again be the mid point of $\sk(\mathscr P)$. 
As $n$ is odd we see that $c$ is a multiplicity 2 point. 
In this case we found that $E$ has reduction type $\mathrm{IV}$. 
So $\sk(E)$ is a multiplicity 3 point. 
So $\phi(\sk(E))$ is a multiplicity 
Note that there are no multiplicity 3 points branching of $c$ away from $\sk(\mathscr P)$. 
So in this case  $\phi(\sk(E))$ is not a fixed point of $g\an$.  
Curiously the symmetry is broken. 


\subsubsection{If $n \equiv 3 \mod 4$ } \label{sec:if_n_equiv_3_mod_4}
In this case it seems to be more difficult to determine. 
It should be possible by tracing the blowups in the proof of steps 8,9 of Tate's algorithm \cite[p. 374-376]{silvermanAdvancedTopicsArithmetic1994}.
But there was not sufficient time to work out this case. 



\section{Weight functions and wild ramification} \label{sec:weight_functions_and_wild_ramification}
The computations in the previous section give us a glimpse of the behavior of the weight function under wildly ramified covers of curves. 

Indeed we know that $\sk(E) = \sk(E, \sigma)$ for any $m$-pluricanonical form on $E$, as the canonical bundle is trivial on $E$. 
Like in the proof of \cref{lem:sk_E_pulls_back} we can choose a non-zero section $\sigma$ of $\omega_{\pro^{1}_K  / R} ^{\otimes 2}$ with poles in $0, 1, \lambda, \infty$. 
Then $\sk(E) = \sk(E, f^*\sigma) = \minloc \wt_{f^* \sigma}$ with \[
\wt_{f^*\sigma} = \wt_{\sigma} + \mathfrak{d}_f
,\] 
where $\mathfrak{d}_f $ is the error term from \cref{rem:weightfunction_fullback_art}. 

In \cref{fig:image_between_skeleta_wild_ims} we saw that $\phi(\sk(E))$ maps onto a subset of  $\sk(\mathscr P)$ when $n > 2v(2)$ and $v(2)$ is odd. We continue with notation from the figure.
This suggests that on the inverse image $(\beta_{-1}, b)$ the error term $\mathfrak{d}_f $ increases. 
Similarly for the inverse image of  $(\beta_1, a)$. 
Note that away from the preimage of $(\beta_{-1}, \beta_1)$, the function $\wt_{f^*\sigma}$ strictly increases by \cref{prop:weight_function_increase}.
So $\mathfrak{d} _f$ does not decrease there, or the slope of is not high enough to counteract the decrease away from $(\beta_{-1}, \beta_1)$ of $\wt_{\sigma}$. 
See \cref{fig:slope_log_different}.

When $n < 2v(2)$, these two lines of length $v(2)$ starting on $a, b$ overlap. See \cref{fig:image_between_skeleta_wild_ims}. 
We conjecture that $\mathfrak{d} _f$ decreases on the inverse images of the lines $(c, b), (c, a)$. 
When $n$ is odd we found that $\phi(\sk(E))$ is no longer contained in $\sk(\mathscr P)$, but instead lays in the lines $(\sqrt{\lambda}, c]\cup (-\sqrt{\lambda} , c]  $. 
This suggests that $\wt_{f^*\sigma}$ is negative or 0 at the end of the inverse images of the lines $(\sqrt{\lambda}, c], (-\sqrt{\lambda}, c]$. 
Thus $\mathfrak{d} _f$ must decrease here. 
See \cref{fig:slope_log_different}.

\begin{figure}[ht]
    \centering
    \incfig{slope-log-different}
    \caption{The conjectured slopes of $\mathfrak{d} _f$ on the preimages of parts of $\pro^{1, \text{an}}_K$.
    On the red lines the slope is conjectured to be non-zero with the arrows pointing in the direction where $\mathfrak{d}_f $ is increasing. }
    \label{fig:slope_log_different}
\end{figure}




Using Tate's algorithm we can find the anwer to \cref{prob:main_problem} without studying the map $\phi: E\an \to \pro^{1, \text{an}}_K$. 
While the goal is to eliminate the use of tate's algorithm and find as much information about $E$ by studying its berkovich geometry, 
it is useful to know what we expect to find, to guide our intuition and make conjectures. 

Let $E$ be the curve defined by \eqref{eq:weierstrass_problem}. 
Then using the notation from \cite[p.364]{silvermanAdvancedTopicsArithmetic1994} we have \[
a_2 = -1 - \lambda,\quad a_4 = \lambda, \quad a_1, a_3, a_5, a_6 = 0
.\] 
And \[
b_2 = -4 \lambda - 4, \quad b_4 = 2\lambda, \quad b_6 = 0, \quad b_8 = -\lambda ^2
.\] 
The discriminant and $j$-invariant are given by \[
	\Delta = 16\lambda^4 - 32\cdot \lambda^3 + 16\lambda^2 = 16 \lambda^2(\lambda - 1)^2
.\] 
Additionally we let $n = v(\lambda)$.
So $\lambda = u \cdot \pi^{n}$ for some unit $u$ of $R$.
\todo{do we need the $j$ invaraiant?}

\subsection{$\ch k$ is not 2} \label{sec:char_k_is_not_2}
We first treat the case where the characteristic of the residue field $k$ is different from 2. 
We label the steps in Tate's algorithm as in \cite[sec.\ IV.9]{silvermanAdvancedTopicsArithmetic1994}.

\step 1 
We have that $|\Delta| = |\lambda|^2 |1-\lambda|^2 = |\lambda|^2 $. 
So $\pi \mid \Delta$ if and only if  $|\lambda| < 1$. 
So if  $n = 0$ we find that $E$ is of type $\mathrm I_0$. 
We now continue under the assumption that $n > 0$, i.e $\pi \mid \lambda$. 

\step 2 
The reduction of the weierstrass model is given by the equation $y^2 = x^2 (x - 1)$ over $k$. 
So the sinuglar point is already at $(0, 0)$ and we may continue the algorithm without coordinate change. 
As $\pi \nmid 4, \pi \mid \lambda$, we see that $\pi \nmid b_2 = -4\lambda - 4$. 
So $E$ has reduction type $I_m$ with \[
	m = v(\Delta) = v(16 \lambda^2 (\lambda - 1)^2)  = 2v(\lambda) = 2n
.\]
This concludes Tates algorithm and we obtain the following result. 

\begin{proposition}
	Suppose $\ch k \ne 2$ and $E$ be as in \cref{prob:main_problem}. 
	Then $E$ has reduction type $\mathrm I_{2n}$ with $n = v(\lambda)$. 
\end{proposition}

\subsection{$\ch k$ is 2} \label{sec:char_k_is_2}
Again we label the steps as in \cite[sec.\ IV.9]{silvermanAdvancedTopicsArithmetic1994}.

\step 1 
We have that $|\Delta| = |2|^{4} |\lambda|^2 |1-\lambda^2| = 2^{-4}|\lambda|^2$.
As $|\lambda|\le 1$, we find that $|\Delta| \le |2|^{-4} < 1$. 
Hence $\pi \mid \Delta$. So $E$ is not of type $\mathrm I_0$ and we must continue to step 2.

\step 2
As $\pi \mid 2$ and $2 \mid b_2 = -4 - 4\lambda$ we find that $\pi \mid b_2$. 
Suppose that $n \ge 0$. We will handle the case where $n = 0$ later. 
Then the reduced curve is given by $y^2 = x^2(x+1)$ which is singular in $(0, 0)$. 
So we can continue to step  $3$ without changing coordinates. 

\step 3
We have $a_6 = 0$, hence $\pi^2 \mid a_6$ and we continue on to step 4.

\step 4
Recall that  $b_8 = -\lambda^2$. 
If  $n = 1$, then $\pi^3 \nmid b_8$ so we find that $E$ is of type $\mathrm{III}$.
If  $n > 1$, then $\pi^3 \mid b_8$, so we have to continue to step 5.

\step 5
Recall $b_6 = 0$. So $\pi^3 \imd b_6$ and we have to continue to step 6.

\step 6
The coordinate transformation $y' = y + x$ yields the weierstrass equation  \[
	y'^2 - 2xy'  = x^3 + (-2 - \lambda)x^2 + \lambda x
\]
The new values of $a_i$ are \[
a_1 = -2,\quad a_2 = -2 -\lambda,\quad a_4 = \lambda,\quad a_3,a_6 =0
,\] 
which satisfy \[
\pi \mid a_1,\quad \pi\mid a_2,\quad  \pi^2 \mid a_3,\quad \pi^2 \mid a_4,\quad \pi^3 \mid a_6
.\] 
Take the polynomial \[
	P(T) = T^3 + \pi^{-1}(-2 -\lambda ) T^2 + \pi^{-2}\lambda T 
\] 
If $n = 2$, then $\pi^{-2}\lambda \not\equiv 0 \mod \pi$. 
So $P(T)$ has no double roots over $k$.
Hence $E$ has has type $I_0^*$. 
If $n >0$, then $\pi^{-2} \lambda \equiv 0 \mod \pi$.
So $0$ is a double root of $P(T)$ over $k$. 
We continue to step 7.

\step 7
If $P(T)$ has double root  $0$ and simple root $1$. 
So  $E$ has reduction type $\mathrm I_m^*$ for some $m$.
To determine $m$ we have to follow the subprocedure as $\ch k = 2$.

Consider the family of polynomails \[
	P_i(T) = T^2 - 
.\] 










\bigskip
Now suppose that $n = 0$.  
The reduced curve is given by \[
	\overline{f}(x, y) = y^2 - x (x-1)(x-\overline{\lambda}) = y^2 + x^3 + (1 + \overline{\lambda})x^2 + \overline{\lambda}x
\]
We have $\partial_y \overline{f} = 2y = 0, \partial_x \overline{f} = 3x^2 + \overline{\lambda} = x^2 + \overline{\lambda}$. 
So the singular component has $x$-coordinate $\sqrt{\overline{\lambda}} $, and then the $y$-coordinate must be $\sqrt{} $
So the curve is singular at the point $(,\sqrt{\overline{\lambda / 3}}) $










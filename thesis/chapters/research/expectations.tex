Using Tate's algorithm we can find the answer to \cref{prob:main_problem} without studying the map $\phi: E\an \to \pro^{1, \text{an}}_K$. 
While the goal is to eliminate the use of Tate's algorithm and find as much information as possible about $E$ by studying its Berkovich geometry, 
it is useful to know what we expect to find, to guide our intuition and make conjectures. 

Let $E$ be the curve defined by \eqref{eq:weierstrass_problem}. 
Then using the notation from \cite[p.364]{silvermanAdvancedTopicsArithmetic1994} we have \[
a_2 = -1 - \lambda,\quad a_4 = \lambda, \quad a_1, a_3, a_5, a_6 = 0
,\] 
and \[
b_2 = -4 \lambda - 4, \quad b_4 = 2\lambda, \quad b_6 = 0, \quad b_8 = -\lambda ^2
.\] 
The discriminant is given by \[
	\Delta = 16\lambda^4 - 32\cdot \lambda^3 + 16\lambda^2 = 16 \lambda^2(\lambda - 1)^2
.\] 
Additionally we let $n = v(\lambda)$.
So $\lambda = u \cdot \pi^{n}$ for some unit $u$ of $R$.

To follow along with the computations it is best to have the description of Tate's algorithm in \cite[sec. IV.9]{silvermanAdvancedTopicsArithmetic1994} open at the same time, as we move through the steps. 

\subsection{Tate's algorithm when $\ch k$ is not 2} \label{sec:char_k_is_not_2}
We first treat the case where the characteristic of the residue field $k$ is different from 2. 

\step 1 
We have that $|\Delta| = |\lambda|^2 |1-\lambda|^2 = |\lambda|^2 $. 
So $\pi \mid \Delta$ if and only if  $|\lambda| < 1$. 
We find the following. 
\begin{tateconclusion}
	If  $n = 0$, then $E$ is of type $\mathrm I_0$. 
\end{tateconclusion}
We now continue under the assumption that $n > 0$, i.e.\ $\pi \mid \lambda$. 

\step 2 
The reduction of the Weierstrass model is given by the equation $y^2 = x^2 (x - 1)$ over $k$. 
So the singular point is already at $(0, 0)$ and we may continue the algorithm without coordinate change. 
As $\pi \nmid 4, \pi \mid \lambda$, we see that $\pi \nmid b_2 = -4\lambda - 4$. 
So $E$ has reduction type $\mathrm I_m$ with \[
	m = v(\Delta) = v(16 \lambda^2 (\lambda - 1)^2)  = 2v(\lambda) = 2n
.\]
\begin{tateconclusion}
	If $n > 1$ then $E$ is of type $\mathrm I_{2n}$.
\end{tateconclusion}
This concludes Tate's algorithm and we summarise our findings in the following proposition. 

\begin{proposition}\label{prop:conclusion_tate_tame}
	Suppose $\ch k \ne 2$ and $E$ be as in \cref{prob:main_problem}. 
	Then $E$ has reduction type $\mathrm I_{2n}$ with $n = v(\lambda)$. 
\end{proposition}

\subsection{Tate's algorithm when $\ch k$ is 2}  \label{sec:char_k_is_2}

Before starting Tate's algorithm we first perform an coordinate change on $E$ in order to reduce it to a minimal Weierstrass model in the cases that we work out and avoid getting stuck in an arbitrarily long loop inside Tate's algorithm. 
Recall that the curve in \cref{prob:main_problem} is given by \[
	E: y^2 = x(x-1)(x-\lambda) = x^3 + (-1 -\lambda) x^2 + \lambda x, \quad \lambda = u \cdot \pi^{n}, \quad 2 = v\cdot \pi^{v(2)}
.\] 
\begin{notation}
	We write $\lambda_i = \pi^{-i} \lambda = u\cdot \pi ^{n -i}$ and $2_i = \pi^{-i}2 = v \cdot \pi^{v(2) - i}$
\end{notation}
\nomenclature[i]{$a_i$}{$\pi^{i}a$ where $a \in K$ and $\pi$ is uniformiser of $K$}
\nomenclature[E]{$E$}{An elliptic curve over $K$}
We first substitute $y' = y + x$. 
Hence we obtain the curve with Weierstrass equation \[
	E: y'^2 - 2xy' = x^3 + (-2-\lambda) x^2 + \lambda x
.\] 
Now we want to reduce the Weierstrass equation by substituting $(x, y') = (\pi^{2i} x'', \pi^{3i}y'')$ for $i \ge 0$ as large as possible, such that the coefficients still lay in $R$. 
We easily see that \[
	i =  \min\left(v(2), \left\lfloor \frac{v(2 + \lambda)}{2} \right\rfloor,  \left\lfloor \frac{v(\lambda)}{4} \right\rfloor\right) = \min\left(v(2), \left\lfloor \frac{v(2 + \lambda)}{2} \right\rfloor,  \left\lfloor \frac{n}{4} \right\rfloor\right) 
\]
We will assume that $v(\lambda + 2) = \min(v(\lambda), v(2))$ (e.g.\ when $n \ne v(2)$), then we further see that \[
	i = \min\left(v(2), \left\lfloor \frac{\min (v(2), n)}{2} \right\rfloor,  \left\lfloor \frac{n}{4} \right\rfloor\right)  = \min\left(\left\lfloor \frac{v(2)}{2}\right\rfloor, \left\lfloor \frac{n}{4} \right\rfloor \right)
.\] 
From now on we'll assume that that $n \ne 2v(2)$. 
Tate's algorithm requires a more complex coordinate change if $n = v(2)$ and there was insufficient time to work out this case. 

So the equation becomes (omitting the double primes after $x, y$) \[
	E: y^2 - 2_i xy = x^3 + (-2_{2i} - \lambda_{2i})x^2 + \lambda_{4i}x
.\] 
Then \[
a_1 = -2_i, \quad a_2 = -2_{2i} - \lambda_{2i}, \quad a_4 =  \lambda_{4i}, \quad a_3, a_6 = 0 
.\] 
Also 
\[
	b_2 = -2_{i}^2 (\lambda + 1), \quad b_4 = 2_{2i}\lambda_{2i}, \quad b_6 = 0,\quad  b_8 = -\lambda_{4i}^2
.\] 
And the discriminant is \[
	\Delta = \pi^{-12i} 2^{4}\cdot \lambda^2(1 - \lambda)^2
.\] 
Note that \[
	v(\Delta) = 4\cdot v(2)+ 2n - 12 \min\left(\left\lfloor \frac{v(2)}{2}\right\rfloor, \left\lfloor \frac{n}{4} \right\rfloor \right)
\]
\step 1 
We rewrite $\Delta = 2^{2} 2_{2i}^2 \lambda_{4i}^2 (1- \lambda)^2$. 
As $\pi\mid 2$ we see that $\pi \mid \Delta$. 
So we must continue to step 2.

We now split up in two cases, $n > 2v(2)$ and $n < 2v(2)$. 
Lets consider the case  $n > 2v(2)$ first.
\subsubsection{Case $n > 2v(2)$} \label{sec:case_n_>_2v2$}

\step 2
If $v(2)$ is even, then $i = v(2) / 2$. 
So the reduction of $E$ has the equation \[
	\tilde E: y^2 = x^3  + \overline{2_{2i}} x^2  = x^3 + vx^2 
.\]
We see that $\tilde E$ is singular at $(0,0)$. 
So there is no need to change coordinates.

Likewise, if $v(2)$ is odd, 
then $i = (v(2)-1) / 2$ and the reduction of $E$ looks like 
\[
\tilde E: y^2 = x^3 
.\] 
We see that  $\tilde E$ is singular $(0, 0)$. 
We find that $\pi \mid b_2$ so we have to continue to step 3. 

\step 3 
Recall $a_6 = 0$. So $\pi^2 \mid a_6$ and we continue to step 4. 

\step 4
We compute \[
	v(b_8) = 2(n - 4i) = \begin{cases}
		2(n - 2v(2) + 1) & 2 \nmid v(2) \\
		2(n - 2v(2)) & 2 \mid v(2) 
	\end{cases} 
\]
If $2\mid v(2)$ and $n = 2v(2) + 1$ then $v(b_8) = 2$ and $\pi^3 \nmid b_8$. 
So we find the following:
\begin{tateconclusion}
	If $2 \mid v(2)$ and $n = 2v(2) + 1$ then $E$ is of type $\mathrm{III}$.
\end{tateconclusion}
We will now continue under the assumption that either $2\nmid v(2) $ or $n \ge 2v(2) +2$.
Then we see that $v(b_8) \ge 3$. 
So $\pi^3\mid b_8$ and we continue to step 5.

\step 5 
Recall that $b_6 = 0$. So $\pi^3 \mid b_6$ and we continue to step 6. 

\step 6
We have  \[
	v(a_1) = v(2) - i > 0
,\] 
thus $\pi \mid a_1$. 
Further \[
	v(a_2) = v(2) - 2i = \begin{cases}
		0 & 2 \nmid v(2) \\
		1 & 2 \mid v(2)
	\end{cases}
.\] 
So $\pi \mid a_2$ if and only if $2 \nmid v(2)$.
We have $a_3 = 0$, hence $\pi^2 \mid a_3$. 
Finally \[
	v(a_4) = n - 4i = \begin{cases}
		n - 2v(2) & 2 \mid v(2) \\
		n - 2v(2) + 2 & 2 \nmid v(2)
	\end{cases}
.\] 
Recall that assumed that $n \ge 2v(2) + 2$ if $2 \mid v(2)$. 
So we find that $\pi^2 \mid a_4$ eitherway. 
Trivially $\pi^3 \mid a_6 = 0$. 
So there is no need to change coordinates in this step if $2\nmid v(2)$. 


From now on we assume that $2\nmid v(2)$ as for $2 \mid v(2) $ the algorithm leads to a lot of case distinctions and explicit computations in Sagemath result in in some odd behaviour. 
I strongly suspect that our Weierstrass equation might not be reduced in general if $2\mid v(2) $. 

For now assume that $2 \nmid v(2)$. 
Then the polynomial is given by \[
	P(T) = T^3 + \overline{\pi^{-1} 2_{2i}} T^2 = T^3 + vT^2
.\] 

\step 7
The polynomial  $P(T)$ has double root at $0$ and distinct single root $\overline{b}$ over $k$. 
So $E$ is of type $\mathrm I_m^*$. 
To determine $m$ we have to follow the subprocedure as $\ch k = 2$. 

Consider the families of polynomials 
\begin{align*}
	F_j(Y) &= Y^3 + \pi^{-i-2} a_3 Y - \pi^{-2i-3} a_6 = Y^2 \\
	G_j(X) &= -\pi^{-1}(-2_{2i} - \lambda_{2i}) X^2 + \pi^{-3 - j}\lambda_{4i} X
,\end{align*}
Note that $F_j$ has double root in $0$ for every $j$.  
As long as $3 + j + 4i \le n$ we find that $G_i(X)$ has coefficients in $R$. 
If $j + 3 + 4i <  n$ we obtain that $G_i(X) \equiv v X^2 \mod \pi$ has double root in $0$. 
If $3 + j + 4i = n$ we obtain that $G_i(X) \equiv v X^2 + uX$. Recall that $\lambda = u \cdot \pi^{n}$. 
So $G_j$ has distinct roots. 
This occurs is exactly when $j = n - 3 - 4i = n - 1 - 2v(2)$. 
At this point we are $2(j + 1)$ steps into the subprocedure. 
So we find that 
\begin{tateconclusion}
	If $n > 2v(2)$ and $2\nmid v(2)$ then $E$ is of type $\mathrm I_m^*$ with  $m = 2(j + 1) = 2n - 4v(2)$.
\end{tateconclusion}


That concludes the case where $n > 2v(2)$. 
We now move to the case where $n < 2v(2)$. 

\subsubsection{Case $n < 2v(2)$} \label{sec:case_n_<_2v2}

In this case $i = \left\lfloor \frac{n}{4} \right\rfloor$.

\step 2
Note that $\pi\mid 2_{2i}$ so $\pi\mid b_2$. 
The reduction of $E$ is 
\[
\tilde E: y^2 = x^3 +  \overline{\lambda_{4i}}x
.\] 
If $n \equiv 0 \mod 4$ then $v(\lambda_{4i}) = 0$ and $\overline{\lambda_{4i}} = \overline{u}$. 
Then $\tilde E$ is singular at $(0, \sqrt{\overline{u}})$. 
So we must perform the translation $y' = y + s $ where $s \in R$ such that $\overline{s}^2 = u$ this results in $\overline{a_6} = \overline{u}$.
So briefly going into step $3$ we see that $\pi^2 \nmid a_6$ and find 
\begin{tateconclusion}
	If $n < 2v(2)$ and $n \equiv 0 \mod 4$ then $E$ is of type $\mathrm{II}$. 
\end{tateconclusion}

We continue step 2 under the assumption that $n \not\equiv 0 \mod 4$ then $v(\lambda_{4i}) = n - 4 \left\lfloor \frac{n}{4} \right\rfloor >  0$. 
So $\tilde E$ is singular at $(0, 0)$. And we do not need to do a coordinate transformation. 



\step 3
Recall that $a_6 = 0$. So trivially $\pi^3 \mid a_6$ and we move on to step 4.

\step 4
Trivially $\pi^3 \mid b_6 = 0$ so we continue with step 5.

\step 5
If $n \equiv 1$ then $v(\lambda_{4i}) = 1$. So $\pi^3 \nmid b_8 = - \lambda_{4i}^2$. 
The algorithm terminates and states the following. 
\begin{tateconclusion}
	If $n < 2v(2)$ and $n \equiv 1 \mod 4$ then $E$ is of type  $\mathrm {IV}$. 
\end{tateconclusion}
We continue under the assumption that $n \equiv 2$ or $n \equiv 3 \mod 4$. 

\step 6
We easily see that $\pi \mid a_1, a_2$ and trivially $\pi^2 \mid a_3, \pi^3 \mid a_6$ as $a_3 = a_6 = 0$. 
As we assume $n \equiv 2$ or $n \equiv 3 \mod 4$ then $\pi^2 \mid a_4$. 
So we don't need to change coordinates. 

The polynomial $P(T)$ is \[
	P(T) = T^3  -\pi^{-1}(2_{2i} - \lambda_{2i})T^2 + \pi^{-2}\lambda_{4i} T
.\] 
If $n \equiv 3 \mod 4$ then $\overline{\pi^{-2} \lambda_{4i}} = 0$. 
So $P(T)$ has at least double root $0$.
If $n = 2v(2) + 1$ then  $P(T) = T^3 - vT^2$. So $P(T)$ has a double root $0$ and single root $u$. 
If $n > 2v(2) + 1$ then $P(T) = T^3$ and has triple root $0$. 

If $n \equiv 2 \mod 4$ then $\pi^{-2} \lambda_{4i} = u$. 
Note that $n = 2v(2) +1 $ is impossible as $n$ is even. So $\overline{\pi^{-1} (2_{2i} - \lambda_{2i})} = 0$ and we find that \[
	P(T) = T^3 + u T
.\] 
Then $P(T)$ has double root $\sqrt{\overline{f}} $ and single root $0$ over $k$.

There is no case in which $P(T)$ has distinct roots. So we move to step 7.

\step 7
As computed in step 6 we find the following. 
\begin{tateconclusion}
	If $n \equiv 2 \mod 4$ or $n \equiv 3 \mod 4 \wedge n = 2v(2) - 1$ then $E$ is of type $I_m^*$ for some $m$. 
\end{tateconclusion}

Note that if we want to follow the subprocedure to find $m$ in the where $n \equiv 2 \mod 4$ then we first have to perform a coordinate transformation $x' = x + s$ where $s \in R$ is some inverse image of $\sqrt{u} $ in $k$

\todo{compute $m$ in this case}
If  $n \equiv 3 \mod 4$ and $n > 2v(2) - 1$ then $P(T)$ has triple root $0$ and we continue to step 8.

\step 8
As $a_3, a_6 = 0$ we see that $Y^2 + a_{3,2} Y - a_{6, 4} = Y^2 $ which has double root $0$. 
So we continue with step 9. 

\step 9
Suppose that $ \pi^{4} \mid a_4 = \lambda_{4i}$ then $i \ne \left\lfloor \frac{n}{4} \right\rfloor$.
So we find that $\pi^{4} \nmid a_4$. 
\begin{tateconclusion}
	If $n < 2v(2), n\equiv 3 \mod 4$ and $n < 2v(2) - 1$ then $E$ is of type $\mathrm{III}^*$. 
\end{tateconclusion}

This concludes the case where $\ch k = 2$. 
We recap our findings in the following proposition.

\begin{proposition}\label{prop:tate_char_is_2}
	Suppose $\ch k = 2$ and $E$ is as in \cref{prob:main_problem} and $n:= v(\lambda) \ne 2v(2), v(\lambda + 2) = \min(v(\lambda), v(2))$
	then $E$ has the type
	\begin{itemize}
		\item if $n > 2v(2)$ and $v(2)$ is odd:$\mathrm I_{2n - 4v(2)}^*$
		\item if $n < 2v(2)$ 
			\begin{itemize}
				\item and $n \equiv 0 \mod 4$:  $\mathrm {II}$
				\item and  $n \equiv 1 \mod 4$:  $\mathrm {IV}$
				\item and $n \equiv 2 \mod 4$:  $\mathrm I^*_m$ for some $m > 0$
				\item and $n\equiv 3 \mod 3$: 
					\begin{itemize}
						\item $n = 2v(2) - 1$:  $\mathrm I_m^*$ for some $m > 0$ 
						\item $n < 2v(2) - 1$: $\mathrm {III}^*$
					\end{itemize}
			\end{itemize}
	\end{itemize}
\end{proposition}

Recall $\Gamma_\lambda$ from \cref{fig:configurations_of_gamma_lambda}. 
Let $\delta = f_* (\ram_f)$ which as a divisor on $\pro^{1}_K$ is the sum of the points on $\pro^{1}_K$ corresponding to $0, 1, \lambda, \infty$. 
\subsection{Finding a minimal snc-model} \label{sec:finding_a_minimal_snc_model}
We can build a snc-model $\mathscr P$ of the pair $(\pro^{1}_K, \delta)$, i.e.\ $\mathscr P_s + \overline{\delta}$ is a strict normal crossings divisor, such that the associated skeleton $\sk(\mathscr P, \delta)$ in $\pro^{1}_K$ is exactly $\Gamma_\lambda$. 
The skeleton associated to a pair is defined in \cite[§3.2.1]{bakerWeightFunctionsBerkovich2016}.

We write $\pro^{1}_K = \spec K[x] \cup \spec K[\frac{1}{x}]$.
If $n = 0$ we can choose $\mathscr P = \pro^{1}_R$. Then the special fibre is $\pro^{1}_k$ and the points $0, 1, \lambda, \infty$ reduce to the distinct points $\overline{0}, \overline{1}, \overline{\lambda}, \overline{\infty}$.
So $\bar{\delta} + \pro^{1}_k$ is an snc-divisor on $\mathscr P$. 
See \cref{fig:model_pair_n_1}. 
\begin{figure}[h]
    \centering
    \incfig{the-model-pair-n-1}
    \caption{The minimal snc-model of the pair $(\pro^{1}_K, \overline{\delta})$ if $n =v(\lambda) = 0$.}
    \label{fig:model_pair_n_1}
\end{figure}

Let 
\begin{align}\label{eq:singular_model_pair}
	\mathscr P' &= \spec \frac{R[x, y]}{(xy - \pi^{n})} \cup \spec \frac{R\left[ \frac{1}{x},y \right]}{\left(y-\frac{\pi^{n}}{x}\right)} \cup \spec \frac{R\left[x, \frac{1}{y}\right]}{\left(x - \frac{\pi^{n}}{y}\right)} \\
	&=  \spec \frac{R[x, y]}{(xy - \pi^{n})} \cup \spec R\left[ \frac{1}{x} \right] \cup \spec R\left[ \frac{1}{y}\right] \nonumber 
 \end{align} 
This is a projective model of $\pro^{1}_K$. 
The special fibre on one the first affine chart is \[
	\spec k \times _{R} \spec \frac{R[x, y]}{(xy -\pi^{n})}=\spec  \frac{k[x, y]}{(xy)}
.\] 
Hence $\mathscr P'_s$ are two rational curves intersecting in one point. 
Let $F, G$ be the irreducible components cut out by $x, y$ respectively, which are given by the third and second chart in \eqref{eq:singular_model_pair} respectively. 
Note that $\mathscr P'$ is not regular at the intersection point.

The reduction of $0$ is the Zariksi closure of $V(x) = V(\pi ^{n} x ) = V(\frac{1}{y})$. 
So we see that $0$ reduces to the origin of the 3'rd affine chart in \eqref{eq:singular_model_pair}, i.e.\ a smooth point on $F$. 
Likewise  $V(x - \lambda) = V(\pi^{-n} x - \pi^{-n}\lambda) = V(\frac{1}{y} - \pi^{-n}\lambda)$. 
So $\lambda$ reduces to a point on $F$ as well, different from $\overline{0}$.
The reduction of $1$ is the Zariski closure of $V(x-1) = V(\frac{1}{x} - 1)$. 
So it lays on $G$. The reduction of $0$ is cut out by $V(\frac{1}{x})$ so it also lays on $G$. 

So we find that $0, \lambda$ reduce to distinct points on $F$ and $1, \infty$ reduce to distinct points on $G$. 
Let $\mathscr P$ be the minimal desingularisation of $\mathscr P'$. 
By \cite[cor.\ 9.3.25]{liuAlgebraicGeometryArithmetic2002} we find that $\mathscr P$ is $\mathscr P'$ where the intersection point is replaced by a chain of $n-1$ rational curves of multiplicity 1 and self intersection $-2$.  
So $\mathscr P$ is a chain of $n + 1$ rational curves. See \cref{fig:model_of_the_pair}. 

This is in fact the minimal snc-model of $(\pro^{1}_K, \delta)$ as the middle components can't be contracted and contracting $\tilde F$ or $\tilde G$ to get $\mathscr P''$, would cause either $0, \lambda$ or $1, \infty$ to be reduced to the same point. Hence $\delta + \mathscr P''_s$ would no longer be an snc-divisor. 
\begin{figure}[h]
    \centering
    \incfig{the-model-of-the-pair}
    \caption{The minimal snc-model of the pair $(\pro^{1}_K, \overline{\delta})$ if $n =v(\lambda) > 0$. }
    \label{fig:model_of_the_pair}
\end{figure}

\subsection{Computing the skeleton} \label{sec:computing_the_skeleton}

The dual graph of $\mathscr P$ is a path of $n + 1$ vertices, where each edge has length $1$. 
So geometrically $\Delta(\mathscr P)$ is a line of length $n$ if $n > 0$ and a point if $n = 0$.
Let $a, b$ the start and end points of the line, corresponding to the components $\tilde F$ and $\tilde G$ respectively. If $n = 0$ (i.e. $\Delta(\mathscr P)$ is a point) then we let $a = b = \sk(\mathscr P)$. 

Then the paths from $0$ and  $\lambda$ to $\sk(\mathscr P)$ come together at $b$. We see this because $0$ and $\lambda$ both reduce to points on $\tilde G$. 
As they reduce to different points, the paths to $\sk(\mathscr P)$ come in at $b$ from different tangent directions. 
Similarly the paths from $0$ and $\infty$ to $\sk(\mathscr P)$ come together at $a$.

So $\sk(\mathscr P, \delta)$ consists of the paths running running between every pair of $0, 1, \lambda, \infty$. Thus $\Gamma_\lambda \subset \sk(\mathscr P, \delta)$ as $\Gamma_\lambda$ is the convex hull of $0, 1, \lambda, \infty$.
As  the only leaves of $\sk(\mathscr P, \delta)$ are $0, 1, \lambda, \infty$ we also get the converse $\Gamma_\lambda \subset  \sk(\mathscr P, \delta)$. 
So we found a snc-model $\mathscr P$ of $(\pro^{1}_K, \delta)$ such the associated skeleton equals $\Gamma_\lambda$. 
See \cref{fig:skeleton_of_the_pair}

\begin{figure}[ht]
    \centering
    \incfig{skeleton-of-the-pair}
    \caption{The skeleton associated to $\mathscr P$ and $(\mathscr P, \delta)$ (both in red).}
    \label{fig:skeleton_of_the_pair}
\end{figure}

\subsection{$\sk(\mathscr P)$ is a Kontsevich-Soibelman skeleton} \label{sec:sk_P_is_a_kontsevich-soibelman_skeleton}

\begin{lemma}\label{lem:unique_form_pair}
	There is a rational section $\omega$ of $\Omega_{\pro^{1}_K}^{\otimes 2}$ with poles at  $0, 1, \lambda, \infty$ and no other zeros or other poles. 
	Moreover this rational section is unique up to multiplication by $K$ and the poles are of order 1. 
\end{lemma}
\begin{proof}
	The statement is equivalent to showing that $\dim_K H^{0}\left(\Omega_{\pro^{1}_K}^{\otimes 2}(\delta)\right)  = 1$.
	We have that $\Omega_{\pro^{1}_K} \simeq \mathcal{O}(-2)$ and $\mathcal{O}(\delta) \simeq \mathcal{O}(\deg \delta) = \mathcal{O}(4)$.
	So \[
		\Omega_{\pro^{1}_K} (\delta) \simeq \mathcal{O}(-1)^{\otimes 2} \otimes \mathcal{O}(4) \simeq \mathcal{O}\left(0 \right)  = \mathcal{O}_X
	.\] 
	Hence $H^{0}\left(\Omega_{\pro^{1}_K}^{\otimes 2}(\delta)\right)$ is a $1$-dimensional $K$-vector space. 
\end{proof}

The proof of the following two results depend heavily on \cite[thm.\ 3.2.3]{bakerWeightFunctionsBerkovich2016}. 
	The metric on $\mathbb{H}(\pro^{1}_K)$ we defined in \cref{sec:}\todo{fill in the section/definition} (the stable metric) differs from the metric in used in \cite{bakerWeightFunctionsBerkovich2016} (the potential metric). 
	However in our specific application the difference does not matter as we are interested in computing the minimal locus of $\wt_\omega$ by looking at its slopes, for which only the sign is important.
	So in the next two proof I'll use the potential metric instead of rescaling the slopes.

	The first lemma is a slight refinement of \cite[prop.\ 4.4.4]{mustataWeightFunctionsNonArchimedean2015} in the context of curves.
	\begin{lemma}\label{lem:well_behaved_pole_weight}
	Let $C$ be a curve over $K$ and $\omega$ be a $m$-pluricanonical form.
	Let $\mathrm{Pole}_\omega$ be the divisor that contains all points of $\omega$ and $\mathscr C$ a snc-model of the pair $(\mathscr C, \omega)$.  
	If all the poles of $\omega$ are of degree less than $m$, then for all $x \in \pro^{1, \text{an}}_K$ \[
		\wt_\omega(x) \ge \wt_\omega(\rho_{\mathscr C}(x))
	\] 
	with equality if and only if $x \in \sk(\mathscr C)$. 
	%\todo{define the reduction map in this sense}
\end{lemma}
\begin{proof}
	If $\red_{\mathscr C}(x)$ does not lie in the closure of $\mathrm{Pole}_\omega$ then the result follows from \cite[prop.\ 4.4.4.(2)]{mustataWeightFunctionsNonArchimedean2015} and the denseness of divisorial points. 

	If $\red_{\mathscr C}(x)$ lies in the closure of some pole $z \in \mathrm{Pole}_\omega$ and $x \not\in \sk(\mathscr C)$. 
	Let $\ell: [0, a] \to X\an$ be the path running from $\sk \mathscr C$ to $x$, parametrised by length. 
	By assumption this path coincides with the path $m: [0, \infty) \to X\an$ from $\mathscr C$ to $z$, also parametrized by $\length$.
	Let $\epsilon$ be the length where $m, \ell$ branch. 
	The slope on  $\ell|_{[0, \epsilon]}$ is positive by \cite[thm.\ 3.2.3.(2)]{bakerWeightFunctionsBerkovich2016} , hence $\wt_\omega \ell(\epsilon) > \wt_\omega \red_{\mathscr C}(x)$.
	If $\epsilon = a$ we are done. 
	If $\epsilon < a$ then repeatedly blowing up $\mathscr C$ in the reduction of $z$ yields a model $\mathscr C'$ whose dual graph contains $\ell(\epsilon)$, but does not contain $x$.
	So the result now follows from applying \cite[prop.\ 4.4.4.(2)]{mustataWeightFunctionsNonArchimedean2015} to the model $\mathscr C'$. 
	\todo{cleanup this proof}
\end{proof}

\begin{proposition}\label{prop:model_P_is_KS}
	Let $\omega$ be as in \cref{lem:unique_form_pair} and $\mathscr P$ the model constructed in \cref{sec:finding_a_minimal_scn_model}.
	Then $\sk(\pro^{1}_K, \omega) = \sk(\mathscr P)$. 
\end{proposition}
\begin{proof}
	\Cref{lem:well_behaved_pole_weight} implies that $\minloc \wt_{w} = \sk(C, \omega) \subset \sk(\mathscr P) $. 
	If $n = 0$ we are done as $\sk(\mathscr P) $ is a point. 
	Suppose that $n > 1$ ($\sk(\mathscr P)$ is a line), then 
	it suffices to show that $\wt_\omega$ is constant on $\sk(\mathscr P)$, i.e.\ constant on the line segment $[ab]$ in \cref{fig:skeleton_of_the_pair}. 
	By \cite[thm.\ 3.2.3.(3)]{bakerWeightFunctionsBerkovich2016} we know that the laplacian of $\wt_\omega$ restricted to $\sk(\mathscr P, \delta)$ is given by \[
		\Delta\left(\wt_\omega|_{\sk(\mathscr P, \delta)}\right) = 2\cdot \sum_{v \in \sk(\mathscr P, \delta)} N(v)(\val (g) - 2g(v) - 2)v = 2a + 2b
	.\] 
	At $a$ the lines $[0, a],[\lambda, a]$ and  $[a, b]$ meet. 
	The outgoing slopes of  $[0, a], [\lambda, a]$ are both $1$ by \cite[3.2.3.(2)]{bakerWeightFunctionsBerkovich2016}. 
	This implies that the outgoing slop at $a$ on the line$ [a, b]$ is $0$. 
	As the laplacian has no points on the interior of $[a, b]$ the slope remains constant on $[a, b]$ and we are done. 
\end{proof}




The computations in the previous section give us a glimpse of the behaviour of the weight function under wildly ramified covers of curves. 

Indeed we know that $\sk(E) = \sk(E, \omega)$ for any $m$-pluricannonical form on $E$, as the cannonical bundle is trivial on $E$. 
Like in the proof of \cref{lem:sk_E_pulls_back} we can choose a section of $\omega_{\pro^{1}_K  / R} ^{\otimes 2}$ with poles in $0, 1, \lambda, \infty$. 
Then $\sk(E) = \sk(E, f^*\omega) = \minloc \wt_{f^* \omega}$ with \[
\wt_{f^*\omega} = \wt_{f} + \mathfrak{d}_f
,\] 
where $\mathfrak{d} $ is the error term from \cref{rem:weightfunction_fullback_art}. 

In \cref{fig:image_between_skeleta_wild_ims} we saw that $\phi(\sk(E))$ maps onto a subset of  $\pro^{1}_K$ when $n > 2v(2)$ and $v(2)$ is odd. We continue with notation from the figure.
This suggests that on the inverse image $(\beta_{-1}, b)$ the error term $\mathfrak{d}_f $ increases. 
Similarly for the inverse image of  $(\beta_1, a)$. 
Note that away from the primage of $(\beta_{-1}, \beta_1)$, the slopes of $\wt_{f^*\omega}$. 
So $\mathfrak{d} _f$ does not decrease there, or the slope of is not high enough to conteract the decrease away from $(\beta_{-1}, \beta_1)$ of $\wt_{\omega}$. 

When $n < 2v(2)$, these two lines of length $v(2)$ starting on $a, b$ overlap. See \cref{fig:image_between_skeleta_wild_ims}. 
We conjecture that $\mathfrak{\delta} _f$ decreases on the inverse images of the lines $(c, b), (c, a)$. 
When $n$ is odd we found that $\phi(\sk(E))$ is no longer contained in $\sk(\mathscr P)$, but instead lies in the lines $(\sqrt{\lambda}, c]\cup (-\sqrt{\lambda} , c]  $. 
This suggests that $\wt_{f^*\omega}$ is negative or 0 at the end of the inverse images of the lines $(\sqrt{\lambda}, c], (-\sqrt{\lambda}, c]$. 
Thus $\mathfrak{d} _f$ must decrease here. 

\begin{figure}[ht]
    \centering
    \incfig{slope-log-different}
    \caption{The conjectured slopes of $\mathfrak{d} _f$ on the preimages of parts of $\pro^{1, \text{an}}_K$.
    On the red lines the slope is conjectured to be non-zero with the arrows pointing in the direction where $\mathfrak{d}_f $ is increasing. }
    \label{fig:slope-log-different}
\end{figure}


 

In this section we answer \cref{prob:main_problem} when $k$ has characteristic different from $2$, under the assumption that $E$ has semi-simple reduction, i.e.\  $E$ is of type $\mathrm I_m$ for some $m \ge 0$. 

\begin{remark}\label{rem:justification_semistable}
	The assumption that $E$ has semi-stable reduction is valid as this is what we have found \cref{sec:char_k_is_2} by means of Tate's algorithm. 
	However assumption also isn't valid for in the case where we choose $|\lambda| > 1$ (see \cref{fig:configurations_of_gamma_lambda}).
	If $|\lambda| > 1$, i.e.\ $v(\lambda) < 0$ then
	explicit computations of examples suggest that  $E$ has semistable reduction if $v(\lambda)$ is even and $E$ is of type $I^*_n$ if $n$ is odd. 
	I do not know of an explanation via Berkovich geometry of this. 
\end{remark}

\begin{lemma}\label{lem:semistable_skeleton}
	Let $\mathscr C$ be minimal snc-model of curve $C$ with semistable reduction.
	Then $\mathscr C$ has no inessential components. 
\end{lemma}
\begin{proof}
	Recall that $\mathscr C$ the minimal resolution of the minimal regular model $\mathscr C_\text{reg} $ of $C$. 
	Suppose for the sake of contradiction that  $C$ has inessential components.
	Then there is ``leaf''  $F$ which is a rational curve in $\mathscr C_s$. 
	If $F$ dominates a component $F'$ in $\mathscr C_\text{reg} $ then $F'$ is also a rational curve of multiplicity $1$ which intersects has self intersection $-1$,  which would contradict the minimality of  $\mathscr C _\text{reg} $ by Castelnouovo's criterion \cref{thm:castelnuovo}. 

	Suppose now that $F$ does not dominate a component in $\mathscr C_\text{reg} $.
	Then it is the $F$ is the exceptional divisor of a blowup in a smooth point of a intermediate model $\mathscr C \to \mathscr C' \to \mathscr C_\text{reg} $. 
	Then $\mathscr C'$ is also snc, which contradicts the minimality of $\mathscr C$. 
\end{proof}
\begin{corollary}
	Let $\mathscr C$ be ithe minimal snc-model of $C$. 
	Then the leaves of the dual graph  $\Delta (\mathscr C)$ have genus greater than $0$. 
\end{corollary}

Recall \cref{lemma:genus_semistable} which states that the Betti-number of $\Delta(C)$ plus the genus of vertices equals $g(C)$. 
In the case of $E$ this leaves two options. 
\begin{lemma}\label{lem:point_or_circle}
	Let $E$ be an elliptic curve with semi-stable reduction and $\mathscr E$ be its minimal semistable model. 
	Then $\Delta(\mathscr E) = \sk(\mathscr E) = \sk(E)$ is either 
	\begin{enumerate}
		\item a circle with every vertex of genus $0$,
		\item a single point of genus  $1$. 
	\end{enumerate}
\end{lemma}
\begin{proof}
	The fact $\sk(\mathscr E) = \sk(E)$ follows from \cref{lem:semistable_skeleton}. 
	We know that $\Delta(E)$ is homotopic to either a circle, or a point of genus $1$. 
	As it has no leaves of genus $0$ and is finite, this means that $\Delta(E)$ is either a circle or a point of genus 1, which proofs the claim.
\end{proof}

\begin{lemma}
	Let $E$ be as in \cref{prob:main_problem}. 
	Then $\sk(E)$ is a circle if $n > 0$ and $\sk(E)$ is a point if $n = 0$. 
\end{lemma}
\begin{proof}
	Recall that $\phi: E\an \to \pro^{1, \text{an}}_K$ is a finite map and that $\sk(E) = \sk(\mathscr P)$ with $\mathscr P$ the model from \cref{sec:finding_a_minimal_snc_model}.
	Suppose that $\sk(E)$ is a circle. 
	Then $\phi(\sk(E))= \sk(\mathscr P)$ is a not a point. 
	Hence $n \ne 0$. 
	Conversely, suppose that $\sk(E)$ is a point. 
	Then $\phi(\sk(E)) = \sk(\mathscr P)$ is a point, and thus $n  = 0$. 
\end{proof}

The only thing left to do is determine the length of the circle $\sk(E)$ when $n \ge 1$.
We know that $\phi|_{\sk(E)}:\sk (E) \to \sk(\mathscr P)$ is surjective map from a circle to a line, such that every point $x \in \sk(\mathscr P)$ has precisely 1 or 2 preimages. 
From this we see that $\phi|_{\sk(E)}$ is a branched cover with branch points in $a, b$, the end points of the line $\sk(\mathscr P)$ (see \cref{fig:skeleton_projective_line}).
\todo{topologically this is obvious, but how would you make this rigorous?}
Then the interior of $\sk(\mathscr P)$ lifts to two lines in $\mathscr  \sk(E)$ along the map $\phi$.
By \cref{rem:length_edge_skeleton_cover} each of these lifts has length $n$. 
So we find that whole circle has length $2n$. 
See \cref{fig:map_circle_line}
\begin{figure}[ht]
    \centering
    \incfig{map-circle-line}
    \caption{The map from $\sk(E)$ to $\sk(\mathscr P)$ inside $\pro^{1, \text{an}}_K$, with each of the lifts of $(a, b)$ in a different color.}
    \label{fig:map_circle_line}
\end{figure}
This gives gives us the following proposition.
\begin{proposition}
	Let $E$ be the curve from \cref{prob:main_problem}. 
	Suppose that $\ch k \ne 2$ and that $E$ has semistable reduction. 
	Then $E$ has reduction type $\mathrm I_{2n}$ with $n = v(\lambda)$. 
\end{proposition}

The picture we sketched above also helps us to understand what happens purely model theoretically. 
Let $\ell_1, \ell_2$ be the two lifts of $(a, b)$ along $\phi$. 
By \cref{prop:balancing_finite_morphism} we see that $\phi$ maps divisorial points in $\ell_1, \ell_2$ to divisorial points in $(a, b)$ of the same degree. 
Recall from \cref{sec:finding_a_minimal_snc_model} that $\mathscr P$ consist of a chain of $n+1$ rational curves of multiplicity 1. 
From this we see that $(a, b)$ contains exactly $n -1$ points of multiplicity 1. 
Hence each of the lines $\ell_1, \ell_2$ also contains $n-1$ points of multiplicity 1.
Together these are $2n - 2$ of the  $2n$ irreduducible components of $\mathscr E_\text{min} $ (the minimal regular/snc model). 

From \cref{prop:balancing_finite_morphism} we cannot see whether $\phi^{-1}(a), \phi^{-1}(b)$ are points of multiplicity 1 one 2. 
But as $\mathscr E_\text{min} $ has $2n$ components with multiplicity $1$, we find that  $\phi^{-1}(a), \phi^{-1}(b)$ must be the 2 missing degree 1 points. 
A more Berkovich theoric argument for this is the following. 
As $E$ has semistable reduction, we know that $\mathscr E_\text{min} $ is semistable, or the blowup of a semistable nc-model. 
As a result all the edges between two multiplicity 1 components are as described in \cref{lem:number_divisorial_points}.
So if $\phi ^{-1}(a)$ has degree 2 it must lay on an edge $e$ with ends points $x, y$ (possibly $x = y$) of multiplicity $1$ with no other multiplicity 1 points in between $x, \phi^{-1}(a)$ and $y, \phi^{-1}(a)$. 
By \cref{lem:number_divisorial_points} $\phi^{-1}(a)$ is the only multiplicity 2 point on $e$. 
But between every multiplicity one point on $\ell_1 \cup \{\phi^{-1}(b)\} $ and $\phi^{-1}(a)$ there is a multiplicity two point. 
This is a contradiction. 
So $\phi^{-1}(a)$ is a multiplicity 1 point, and similiarly $\phi^{-1}(b)$ is a multiplicity 1 point.  


From \cref{rem:balancing_galois_cover} we see that the map between components corresponding to $\phi^{-1}(a)$ and $a$ is a double cover of $\pro^{1}_k$ onto $\pro^{1}_k$.
Likewise for $\phi^{-1}(b)$ and $b$. 
This shows that there is the following morphism of models $\psi: \mathscr E_\text{min} \to \mathscr P$ extending the map $f: E \to \pro^{1}_K$, which we have illustrated in \cref{fig:morphism_models_semistable_tame}.

\begin{figure}[ht]
    \centering
    \incfig{morphism-models-semistable-tame}
    \caption{A morphimss }
    \label{fig:morphism_models_semistable_tame}
\end{figure}
From this we also see that $\mathscr E_\text{min} $ is actually the normalisation of $\mathscr P$ in $K(E)$. 
Such pair of models is sometimes referred to as a simultanious resolution of singularities \cite[sec.\ 6]{liuModelsCurvesFinitea}.


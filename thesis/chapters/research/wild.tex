If $\ch k =2$ we can no longer use \cref{prop:weightfunction_fullback} to show that  $\sk(E) = \phi^{-1}(\sk(\mathscr P))$. 
In fact, as we will see in this section, the conclusion $\sk(E) = \phi^{-1}(\sk(\mathscr P))$ will even be false. 
But by a closer examination of Tate's algorithm we can still identify $\phi(\sk(E))$ as a subbset of $ \pro_K^{1, \text{an}}$. 
This allows us to make some conjectures about the behaviour of error term $\delta^{\text{log}}$ from \cref{rem:weightfunction_fullback_art}. 


Recall that in \cref{sec:char_k_is_2} we found that \begin{equation}\label{eq:wild_minimal_weierstrass}
y_{3i}^2 - 2_i x_{2i} y_{3i} = {x_{2i}}^3 + (-2_{2i} - \lambda_{2i}) {x_{2i}}^2 + \lambda_{4i} x_{2i}
,\end{equation} 
with \[
	y_{3i} = \pi^{3i} y,\quad x_{2i} = \pi^{2i} x, \quad i = \min\left( \left\lfloor \frac{v(2)}{2}\right\rfloor, \left\lfloor \frac{n}{4} \right\rfloor  \right), \quad 2 = v\cdot \pi^{v(2)},\quad  \lambda = u \cdot \pi^{n}
.\] 
is a minimal Weierstrass equation for $E$ if $v(\lambda) \ne v(2)$, as Tate's algorithm ends after only one iteration. 
\todo{This might not be entirely accurate}

We will have to split up differerent cases. 
\subsection{Case $v(\lambda) > 2v(2)$ and $v(2)$ is odd} \label{sec:case_v_lambda_>_2_v_2_and_v_2_is_even}
In this case we have $i = (v(2) - 1) / 2$ and we computed that $E$ is of type $\mathrm I_{2n - 4v(2)}^*$ (\cref{prop:tate_char_is_2}).
Let $\mathscr E_\text{min} $ be the minimal regular model of $E$ (which for curves of type $\mathrm I_m^*$ this case coincides with the minimal snc-model).
The idea is to look at the proof of Tate's algorithm where they compute each component $\mathscr E_\text{min} $ explicitly by repeatedly blowing up the Weierstrass equation \eqref{eq:wild_minimal_weierstrass}. 
Then for each component, we can look for corresponding model of $\pro^{1}_K$ with a single component and a domininating map between the components. 
That way we can find the image of every divisorial point in $\Delta(\mathscr E_\text{min} )$ in $\pro^{1, \text{an}}_K$. 
Fortunately the necessary blowups are computed very clearly in the proof of Tate's algorithm in \cite[p.\ 369-379]{silvermanAdvancedTopicsArithmetic1994}

Note that in \cref{sec:char_k_is_2}, in the case where $n > 2v(2)$ and $v(2)$ is odd, we did not perform any coordinate translations by the time we reached step 7 of the algorithm. 
At the end op p.\ 372 Silverman writes that in his notation $x_{r} = \pi^{r}x, y_{r} = \pi^{r}y$, where $x, y$ are the original coordinates of the Weierstrass equation. 
We start with a weierstrass equation with variables $x_{2i}, y_{3i}$. 
So where Silverman writes $(x_r, y_r)$ in our notation that becomes $(x_{2i+ r}, y_{3i + r})$. 
In the proof of step 5, he starts with the model $\mathcal{V} $ which (on an affine chart) looks like 
\[
	\mathcal{V} = \spec \frac{R[x_{2i + 1}, y_{3i + 3}]}{(y_{3i + 3}^2 + 2_i x_{2i + 1}y_{3i + 3} = \pi x_{2i + 1}^3 + \pi v x_{2i + 1}^2 + \lambda_{4i + 1} x_{2i + 1})} \to \spec R[x_{2i + 1}]
,\] 
note that we added the projection to the $x$-axis. 
On the special fibres this looks like 
\[
	\mathcal{V}_s = \spec \frac{k[x_{2i + 1}, y_{3i + 3}]}{(y_{3i+3}^2)} \to \spec k[x_{2i + 1}]
.\] 
We see that the double line $y_{3i + 3}^2 = 0$ maps surjectively on $k[x _{2i + 1}]$. 
Then he blows up $\mathcal{V} $ in $(0, 0, 0)$ to obtain the scheme
\[
	\mathcal{V}_0  = \spec \frac{R[x_{n_0}, y_{m_0}]}{(\pi y_{n_0} ^2 + \pi2_{i + 1} x_{n_0} y_{n_0} = x_{n_0}^3 +  a_{2,1} x_{n_0}^2 + \lambda_{4i + 2} x_{n_0}^2)} \birat \spec \frac{R[x_{n_0}, z]}{(zx_{n_0}^2 - \pi)}
,\] 
where we used the notation \[
	n_j = 2i + 1 + \left\lfloor \frac{j + 1}{2} \right\rfloor,\quad  m_j = 3i + 2 + \left\lfloor \frac{j }{2} \right\rfloor
.\]
Let $a, b \in R$ be such the roots $0, a, b$ are the roots of $x_{n_0}^3 +  a_{2,1} x_{n_0}^2 + \lambda_{4i + 2} x_{n_0}^2$ and $\overline{a} = 0, \overline{b} = \overline{a_{2,1}} = \overline{v}$.  
These exists because over $k$ the polynomial looks like $x_{n_0}^3 + v x_{n_0}^2$ and by Hensel's lemma the roots lift. 

Then we can define a birational map of schemes \[
	\mathcal{V}_0 \birat \spec \frac{R[x_{n_0}, z]}{(z x_{n_0}(x_{n_0} - a) - \pi)},\quad
	x_{n_0} \mapsto x_{n_0}, z\mapsto \frac{(x - b)}{y_{n_0}^2 + 2_{i + 1} x_{n_0}y_{n_0} }
.\] 
Then the morphism on the special fibre is given by \[
	\mathcal{V} _{0, s} = \spec \frac{k[x_{n_0}, y_{m_0}]}{(x^3_{n_0} + \overline{v} x^2_{n_0})} \birat \frac{k[x_{n_1}, z_0]}{(z_0 \cdot x_{n_0}^2)}
.\] 
So we see that the double line cute out by $x_{n_0}^2$ in the source maps to the double line cut out by $(x_{n_0})^2$ in the image. 

In the proof of step 6 and 7 he continues with the models $\mathcal{V} _j$ which for our coefficients and $2 \nmid j, j\le 2n - 4v(2) - 1$ looks like . 
\[
	\mathcal{V} _j = \spec \frac{R[x_{n_{j}}, y_{m_j}]}{(y_{m_j}^2 + \pi 2_{i + 1} x_{n_j}y_{m_j} = \pi^2 x_{n_j}^3 + \pi a_{2, 1} x_{n_j}^2 + \lambda_{4i + 2 + \left\lfloor j / 2 \right\rfloor} x_{n_j}) } \to \spec R[x_{n_j}]
.\] 
Note that we added the projection to the $x_{n_j}$-axis. 
Then the map on special fibers looks like \[
	\mathcal{V} _{j, s} = \spec \frac{R[x_{n_j}, y_{m_j}]}{(y_{m_j}^2)} \to \spec k[x_{n_j}]
.\] 
Thus the double line given by $y_{m_j} = 0$ maps surjectively onto the line $\spec k[x_{n_j}]$. 

For $2 \mid j, j \le 2n - 4v(2) - 1$, the schemes $\mathcal{V} _j$ look like. 
\[
	\begin{aligned}
		\mathcal{V}_j = \spec \frac{R[x_{n_j},y_{m_j}]}{(\pi y^2_{m_j} + \pi 2_{{i + 1}}x_{2i}y_{n_j} = \pi x_{n_j}^3 + a_{2,1}x_{n_j}^2 + \lambda_{4i + 2 + \left\lfloor j / 2 \right\rfloor} x_{n_j})} \\ \dashrightarrow \spec \frac{R[ x_{n_j}, z_j]}{(z_j(a_{2,1} x_{n_j}^2 + \lambda_{4i + 2 + \left\lfloor j / 2 \right\rfloor})- \pi)} 
	\end{aligned}
,\]
where the birational map is given by $x_{n_j} \mapsto  x_{n_j}, z \mapsto (y^2_{m_j} + 2_{i + 1 }x_{m_j}y_{n_j} - \pi x_{n_j}^3)^{-1}$. 
On the level of special fibers this is given by \[
	\mathcal{V} _{j, s} = \spec \frac{k[x_{n_j}, y_{n_j}]}{(\overline{v}x^2_{n_j})} \birat \spec \frac{k[x_{n_j}, z_j]}{(z_j \cdot  x^2_{n_j})}
,\]
where we see that the double line cout out by  $x_{n_j} = 0$ in the source dominates the double line $x_{n_j} = 0$ in the image. 


\bigskip

The models $\mathcal{V} , \mathcal{V} _{0}, \ldots, \mathcal{V} _{2n - 4v(2) - 1}$ give all of the $2n - 4v(2) + 1$ multiplicity two components of  $\mathscr E_\text{min} $. 
Lets call these divisorial points $\alpha_{-1}, \alpha_0, \ldots, \alpha_{2n-2v(2) - 1}$ respectively. 
We write $\beta_j = \phi(\alpha_j)$. 
Furthermore, for each of these models, we have given a map to a normal model of $\pro^{1}_{K}$  and identified the image of the degree two component of the special fiber. 

If we understand what the divivisorial points of the models of $\pro^{1}_K$ are in $\pro^{1, \text{an}}_K$ we can identify the image of  $\sk(E)$ in $\pro^{1}_K$. 


So we see that the image of $\alpha_{j}$ is is $B(0, |\pi|^{\ldots})$ when $j$ is odd. \todo{figure out whats wrong here}
It should be possible to give a similar algebraic argument for $x_{j}$ when $j$ is even.
But instead of trying to determine the norm from the model of $\pro^{1}_K$, we can find $\phi(x_j)$ entirely from a Berkovich theoretic argument. 
Again we will use an argument that uses the number and distribution of divisorial points of a given degree on a line in $\sk(E)$ and $\pro^{1, \text{an}}_K$.
We do this to show that even in the wildly ramified case, we can use some Berkovich geometric arguments. 
The argument is quite very geometric, so it is best to follow along on \cref{fig:argument_ims_wild}

\medskip

Without loss of generality we will determine $\beta_0$ using what we know about $\beta_{-1}$ and $\beta_1$. 
As we've seen $\beta_{-1}, \beta_{1}$ are adjacent vetrices in $\Delta(\mathscr P)$. 
There is a unique point $\gamma$ on $(\beta_{-1}, \beta_1)$ such that $N(\gamma) = 2$.  
We expect that like in \cref{sec:determining_m_when_e_is_of_type_ims} $\phi(\alpha_0) = \beta_0 = \gamma$. 
Suppose for the sake of contradiction that $\beta_0 \ne \gamma$.



As $\pro^{1, \text{an}}_K$ is a uniquely geodesic space, we know that $\phi((\alpha_{-1}, \alpha_{1}))$ contains the entire line  $(\beta_{-1}, \beta_1)$. 
Further more, as $\alpha_{-1}$ is a multiplicity 2 point, mapping to the multiplicity 1 point $\beta_{-1}$, \cref{rem:balancing_galois_cover} tells us that $\alpha_{-1}$ is the unique preimage of $\beta_{-1}$. 
Likewise $\alpha_1$ is the unique preimage of $\beta_1$. 
So the $\phi((\alpha_{-1}, \alpha_{1}))$ is contained in the open annulus $A$ bounded by $\beta_{-1}, \beta_1$.
Note that the only divisorial points of multplicity 2 in $A$ are $\gamma$ and in horizontal branches originating in $\gamma$. There are no multiplicity 2 points in  $A$. 
\begin{figure}[ht]
    \centering
    \incfig{argument-ims-wild}
    \caption{Finding $\beta_0 = \phi(\alpha_0)$ in $\pro^{1,\text{an}}_{K}$. 
    Note that this figure is inaccurate as it is part of a proof by contradiction.}
    \label{fig:argument_ims_wild}
\end{figure}
As $\phi((\alpha_{-1}, \alpha_{1}))$ contains the entire line $(\beta_{-1}, \beta_1)$ we know that $\gamma$ has a preimage in that line, which by \cref{rem:balancing_galois_cover} is either of multiplicity $2$ or $4$. 
We've assumed it is not $\beta_0$, the only point of multiplity 2 in  $(\beta_{-1}, \beta_1)$.
So one of the two multiplicity 4 points must map to $\gamma$. 
Let $\delta$ be the multiplicity 4 point on $(\alpha_0, \alpha_1)$ and suppose without loss of generality that $\phi(\delta) = \gamma$. 
By \cref{rem:balancing_galois_cover} we know that $\delta$ is the unique preimage of $\gamma$. 

So $\phi(\alpha_0)$ must be point of multiplicity 1 or 2. As there are no points of multiplicity $1$ in $A$ and we asummed $\phi(\alpha_0)\ne \gamma$ we find that that $\phi(\alpha_0) = \beta_0$ must be one on one of the branches coming of $\gamma$. 
Again using the fact that $\pro^{1, \text{an}}_K$ is uniquely geodesic we know that $(\beta_0, \gamma] \cup [\gamma, \beta_{-1}) \subset  \phi((\gamma, \alpha_{-1}))$. 
So $\gamma$ has a preimage different from $\delta$ on $(\gamma, \alpha_{-1})$. 
This is a contradiction. 
Thus $\phi(\beta_0) = \gamma$. 

Pooling together everything we know about $\alpha_{-1}, \ldots, \alpha_{2n - 4v(2)-1}$ we see that $\sk(E)$ maps onto a line inside $\sk(\mathscr P)$ with length $n- 2v(2)$ whose start and end points are at distance $v(2)$ from $a, b$.
See \cref{fig:image_between_skeleta_wild_ims}.
\begin{figure}[h]
    \centering
    \incfig{image-between-skeleta-wild-ims}
    \caption{The image of $\sk(E)$ in  $\pro^{1, \text{an}}_K$ in the case where $n > 2v(2)$}
    \label{fig:image_between_skeleta_wild_ims}
\end{figure}

\subsection{The case where $n < 2 v(2)$} \label{sec:the_case_where_n_<_2_v2}

\subsubsection{If $n$ is even} \label{sec:if_$n$_is_even}

If we will make the simplifying assumption that $\sqrt{\lambda} $ exists. 
\begin{remark}
	This is equivalent to saying that $\sqrt{u} $ exists.
	This is not guaranteed to be the case as we cannot use Hensel's lemma to lift the root $\sqrt{\overline{u}} $ from $k$ to $R$, because the derivative vanishes $(x^2 - u)' = 2x = 0$. 
	I'm fairly sure that if $\sqrt{u} $ does not exits in $ K$, the argument can be salvaged by passing to the field extension $K(\sqrt{u} )$. 
	This is an immediate extension, i.e.\ it is unramified and isomorphism on the residue fields. 
	But to show that the argument is still valid needs to be checked.
\end{remark}

Let $g: \pro^{1}_K \to \pro^{1}_K$ be the automorphism induced by $x \mapsto \lambda / x$. 
This extends to an automorphism $h: E \to E$ induced by the coordinate transformation $x \mapsto  \lambda / x, y\mapsto \sqrt{\lambda ^3} x y$. 
So we have the following commutative diagram.
\[
\begin{tikzcd}
	E\an \rar{h\an} \dar{\phi} & E\an \dar{\phi} \\
	\pro^{1, \text{an }}_K \rar{g\an} & \pro^{1, \text{an}}_K
\end{tikzcd}
.\] 

On $\pro^{1}_K$, $g$ is the map that maps $0 \leftrightarrow \infty, 1 \leftrightarrow \lambda$. 
So we see that $g\an$ leaves $\Gamma_\lambda$ invariant. In particular $\sk(\mathscr P)$ is a fixed set of $g\an$, but $g\an$ flips the orientation of $\sk(\mathscr P)$, i.e. $g\an(a) = b, g\an(b) = a$. 
As $h$ is an automorphism we have and the essential skeleton is canonical, $\sk(E)$ a fixed set of $h\an$. 
Thus $\phi(\sk(E))$ is a fixed set of $g\an$.
Let $c$ be the mid point of $\sk(\mathscr P)$. 
As $n$ is even we see that $c$ is a multiplicity 1 point. 

\medskip

If  $n \equiv 0 \mod 4$ then $E$ is of type $\mathrm{II}$.  
So $\sk(E)$ is a single point of multiplicity 6. 
Therefore  $\phi(\sk(E))$ is a fixed point of $g\an$. 
Then fixed points of $g\an$ are the line  $[\sqrt{\lambda}, c] \cup [-\sqrt{\lambda} , c]$ where $\sqrt{\lambda} $ is the closed point in $\pro^{1}_K$. 
Note that these lines branch off in the same tangent direction at $c$ as the reduction map at that point maps both directions to  $\sqrt{\overline{u}} $ in $k$. 
So $\sk(E) \in [\sqrt{\lambda}, c] \cup [-\sqrt{\lambda}, c] $, but $\phi(\sk(E)) \ne c$ as a multiplicity 6 point must map to a multiplicity 3 or 6 point, under a degree 2 Galois cover. 
So $\phi(\sk(E))$ is a point off the line $\sk(\mathscr P)$. 

\begin{figure}[h]
    \centering
    \incfig{image-skeleton-wild-n-le-2v2-n-even}
    \caption{The embedding of $\sk(E)$ in $\pro^{1, \text{an}}_K$ when $n < 2v(2)$ and  $n$ is even.}
    \label{fig:image-skeleton-wild-n-le-2v2-n-even}
\end{figure}
\medskip 
If $n \equiv 2 \mod 4$ then $E$ is of type $\mathrm I_{m}^*$ for some $m$. 
Note that at at step 7 we had to change coordinates $x_{i+1}' = x_{i + 1}  + s$ where $\overline{s} = \sqrt{\overline{u}} $. 
Following the same argument as in \cref{sec:case_v_lambda_>_2_v_2_and_v_2_is_even} we see that degree two components of $\mathscr E_\text{min} $ lie in a branch of $c$. 
But by as we changed coordinates, they lay in the branch of $c$ with the same tangent direction as $\sqrt{\lambda} $. 
As the map $\phi|_{\mathscr E}$ is injective by the same argument as in \cref{sec:case_v_lambda_>_2_v_2_and_v_2_is_even} we further see that the $\phi(\sk(E))$ must lay on the line $[\sqrt{\lambda}, c] $ or $[-\sqrt{\lambda} , c]$
So again we find that $\phi(\sk (E))$ is not contained in $\sk(\mathscr P)$. 


\subsubsection{If $n \equiv 1 \mod 4$} \label{sec:if_n_equiv_1_mod_4$}

Let $c$ again be the mid point of $\sk(\mathscr P)$. 
As $n$ is odd we see that $c$ is a multiplicity 2 point. 
In this case we found that $E$ has reduction type $\mathrm{IV}$. 
So $\sk(E)$ is a multiplicity 3 point. 
So $\phi(\sk(E))$ is a multiplicity 
Note that there are no multiplicity 3 points branching of $c$ away from $\sk(\mathscr P)$. 
So in this case  $\phi(\sk(E))$ is not a fixed point of $g\an$.  
Curiously the symmetry is broken. 


\subsubsection{If $n \equiv 3 \mod 4$ } \label{sec:if_n_equiv_3_mod_4}
In this case it seems to be more difficult to determine. 
It should be possible by tracing the blowups in the proof of steps 8,9 of Tate's algorithm \cite[p. 374-376]{silvermanAdvancedTopicsArithmetic1994}.
But there was not sufficient time to work out this case. 


In the case where $\ch k \ne 2$ we can use \cref{prop:weightfunction_fullback} to compare the skeleton $\sk(\mathscr P)$ with the essential skeleton $\sk(E)$. 
\begin{lemma}
	Let $\mathscr P$ be as in \cref{sec:a_weight_function_on_pro_1_k}, and $E$ from \cref{prob:main_problem}. 
	Then \[
		\sk(E) = \phi^{-1}(\sk(\mathscr P))
	.\] 
\end{lemma}
\begin{proof}
	In \cref{prop:model_P_is_KS} we saw that $\sk(\mathscr P) = \sk(P, \omega)$, where $\omega$ is the rational section of $\Omega_{\pro^{1}_K}^{\otimes 2}$ with poles at $0, 1, \lambda, \infty$, which is unique up to multiplication by $K$. 
	Then $f^* \omega$ is a $2$ pluricanonical form. 
	Indeed, we can take $\omega$ as a global section of $\Omega_{\pro^{1}_K}^{\otimes 2}(f_* \ram _f)$. 
	And 
	\[
		f^* (\Omega_{\pro^{1}_K / K}^{\otimes 2}(-f_* \ram _f)) = \Omega_{E /K}^{\otimes 2} \otimes \mathcal{O}(2 \ram_f) \otimes f^*f_* \mathcal{O}(-\ram _f) = \Omega_{E / K}
	.\] 
	We also have 
	\[
		\sk(E, f^*\omega) = \minloc \wt_{f^* \omega} = \minloc \wt_{\omega}\circ \phi = \phi^{-1}(\minloc \wt_\omega) = \phi^{-1} (\sk(\pro^{1}_K, \omega)
	.\] 
	Recall that $\Omega_{E / K}$ is trivial as $E$ is an elliptic curve. 
	So all pluricanonnical forms are the same up to multiplication. 
	So $\sk(E)$ can be computed using only one pluricanonical form, thus \[
		\sk(E) = \sk(E, f^*\omega) = \phi^{-1}(\sk(\pro^{1}_K, \omega)) = \phi^{-1}(\mathscr P)
	.\] 
\end{proof}

The following lemma is stated much more generally than we will need it. 
But it is a neat result, and the proof is neither long, nor complicated. So we put it in. 
\begin{lemma}\label{lem:number_divisorial_points}
	Let $\mathscr C$ be a snc-model of  $C$ and suppose that $e$ is an edge between two divisorial pionts of degree $1$ in $\Delta(\mathscr C)$. 
	Let $f(n)$ be the number of divisorial points of degree $n$ on the edge  $e$ in $\sk(\mathscr C)$. 
	Then \[
		f(n) = \begin{cases}
			2 & n = 1 \\
			\varphi(n) & n > 1
		\end{cases}
	,\] 
	where $\varphi$ is the Euler totient function. 
\end{lemma}
\begin{proof}
	As the problem is local, we may assume that $\mathscr C$ is a (not proper) model with just two component of multiplicity $1$ intersecting in one point. 
	So $\sk(\mathscr C)$ is just the edge $e$. 

	Let $x$ be any divisorial point on $e$. 
	Then by the factorisation theorem $x$ can be obtained as the vertex in the dual graph of a snc-model, which is obtained by repeatedly blowing up in intersection points. 

	We now construct a sequence of snc-models \[
	\ldots \to 	\mathscr C_n \to \mathscr C_{n -1} \to \ldots \to \mathscr C_1 = \mathscr C
,\]
each dominating the previous as follows. 
The model  $\mathscr C_i$ is the blowup of $\mathscr C_{i-1}$ in all intersection points $\xi$ such that lies on two components with multiplicity $N, M$ and $N + M \le i$. 
Then $x$ is a vertex on $\Delta(\mathscr C_n)$ for sufficiently large $n$.
In the dual graph this repeatedly subdivides the edge in $e$ by adding more vertices of higher multiplicity. 
\\
\noindent\incfig{dual-graphs-farey}
\\
We now obtain the result from noting that these multiplicities are exactly the denominators from the Farey sequence for which the result is a well known theorem. 
\end{proof}


\begin{lemma}\label{lem:equality_lenghts_preserve_mult}
	Let $f: X \to Y$ be a finite morphism of curves and $p$ a path in $\sk(\mathscr X)\subset X\an $  such that $f\an|_p$ is injective and lands in $\sk(\mathscr Y)$ for some snc-models $\mathscr X, \mathscr Y$ of $X, Y$ respectively. 
	Suppose further that for every divisorial point $x \in p$  we have $N(x) = N(f\an(x))$. 
	Then $f\an(p)$ has the same length as $f\an$. 
\end{lemma}
\begin{proof}
	As the divisorial points are dense on the path $p$ we may assume that $p$ is a closed path that starts and ends in a divisorial point, $x, y$ respectively.
	By repeatedly blowing up in divisorial points we may replace $\mathscr X, \mathscr Y$ by models that such that $p$ and $\phi(p)$ are made up of a number of edges on $\sk(\mathscr X), \sk(\mathscr Y)$, respectively. 
	As the question is local we may replace $\mathscr X, \mathscr Y$ to opens in them such that $\sk(\mathscr X) = p, \sk(\mathscr Y) = f\an (p)$. (Note that $\mathscr X, \mathscr Y$ are no longer proper, but still have normal crossings).  
	Then $\Delta(\mathscr X)$ as a graph is a path ending and starting with vertex $x, y$ and $\Delta(\mathscr Y)$ is a path starting and ending with $f\an(x), f\an(y)$. 

	Let \[
		N = \max \{N(v) \st v \text{ is a vertex in } \Delta(\mathscr X) \text{ or } \Delta(\mathscr Y)\} 
	.\] 
	We will now describe how blowup $\mathscr X$ and $\mathscr Y$ repeatedly in intersection points such that the vertices in the dual graphs are precisely the divisorial points of multiplicity less than or equal to $N$. 

	Let \[
	\mathscr X_n \to \mathscr X_{n - 1} \to \ldots \to \mathscr \mathscr X_1 = \mathscr X
	\] 
	be a sequence of locally snc-models of  $X$, where $\mathscr X_i$ is the blowup of $\mathscr X_{i-1}$ in all intersection points of laying on components with multiplicity $N_1, N_2$ such that $N_1 + N_2 \le i$. 
	Note that all components in the expecptional divisor have multiplicity at most $i$. 

	Note that $\Delta(\mathscr X_1)$ contains all divisorial points in $\sk(\mathscr X)$ of multiplicity 1. If not, let $z$ be such a divisorial points. 
	Then we can repeatedly blowup $\mathscr X$ in intersection points to obtain $\mathscr X'$ such that $z$ is in $\Delta(\mathscr X')$. 
	But $z$ is the result of a blowup in an intersection point, and thus has multiplicity atleast $2$. 

	By induction we see that $\mathscr X_i$ contains all divisorial points in $\sk(\mathscr X)$ of multiplicity less than or equal to $i$, for $i = 1, \ldots, n$. 
	Indeed suppose that this is the case for $\mathscr X_{i -1}$ and $z$ is a divisorial point that we obtain by repeatedly blowing up intersection points.
	Then  $z$ lies between two adjacent vertices $v_1, v_2$ of multiplicity $N_1, N_2$ respectively in $\mathscr X_{i}$. 
	We easily see that $z$ has atleast multiplicity $N_1 + N_2$. 
	So it is sufficient to show that $N_1 + N_2 > i$. 
	Suppose for the sake of contradiction that $N_1 + N_2 \le i$.
	Then both $N_1 \le i -1$ and $N_2 \le i-1$. 
	So $v_1, v_2$ are vertices that also belong to $\Delta(\mathscr X_{i-1})$, and they are also adjacent. 
	But as $N_1 + N_2 \le i$ we know that $\Delta(\mathscr X_{i})$ has a vertex in between $v_1, v_2$. This contradicts that $v_1, v_2$ are adjacent in $\Delta(\mathscr X_{i})$. 

	Now we see that $\Delta(\mathscr X_n)$ contains all divisorial points in $\sk(\mathscr X) = p$ of multiplicity less than or equal to $n$. 
	As in the process no vertices of multiplicity higher than $n$ were added, we see that $\Delta(\mathscr X_n)$ contains exactly the vertices of multiplicity less than or equal to $n$. 

	Similarly we construct a model $\mathscr Y_n$ such that the vertices on  $\Delta(\mathscr Y_n)$ are exactly all divisorial points on $\sk(\mathscr Y) = f\an(p)$ of multiplicity less than or equal to $n$. 

	Let $V$ be the set of vertices on $\Delta(\mathscr X_n)$. 
	As $f\an $ on $p$ preserves multiplicity we see that $f\an(V)$ are precisely the vertices of $\Delta(\mathscr Y_n)$ and $f\an$ even preserves the order of the vertices. 
	So $f\an$ induces an isomorphism of graphs $\Delta(\mathscr X_n) \to \Delta(\mathscr Y_n)$ where each vertex is labeled with its multiplicity. 
	It follows from the definition of the length on a graph that $|\Delta(\mathscr X_n)| = p$ has the same length as $|\Delta(\mathscr Y_n)| = f\an(p)$. 
\end{proof}

\begin{remark}\label{rem:length_edge_skeleton_cover}
	The main context in which the assumptions of \cref{lem:equality_lenghts_preserve_mult} are satisfied is when $f:X \to Y$ is a degree $d$ map of curves, $e$ is an edge in a skeleton of $Y$ such that $e \subset Y\an \setminus B_{f\an}$, and $e'$ is a lift of $e$ in $X\an$. 
	Here $B_{f\an}$ is the topological branch locus of the map $f\an: X\an \to Y\an$.
	Indeed, every divisorial point $x \in e$ has $d$ preimages, and by \cref{lem:}\todo{link back to this lemma} $x$ and it's preimages have the same multiplicity. 
	In this context we find that $e'$ and $e$ have the same length. 
\end{remark}

\begin{remark}
	\todo{suggests intrinsic definition. Refer to Matthias Johnsson}
\end{remark}
\subsection{The semistable case} \label{sec:the_semistable_case}
In this section we answer \cref{prob:main_problem} when $k$ has characteristic different from $2$, under the assumption that $E$ has semi-simple reduction, i.e.\  $E$ is of type $\mathrm I_m$ for some $m \ge 0$. 

\begin{remark}\label{rem:justification_semistable}
	The assumption that $E$ has semi-stable reduction is valid as this is what we have found \cref{sec:char_k_is_2} by means of Tate's algorithm. 
	However assumption also isn't valid for in the case where we choose $|\lambda| > 1$ (see \cref{fig:configurations_of_gamma_lambda}).
	If $|\lambda| > 1$, i.e.\ $v(\lambda) < 0$ then
	explicit computations of examples suggest that  $E$ has semistable reduction if $v(\lambda)$ is even and $E$ is of type $I^*_n$ if $n$ is odd. 
	I do not know of an explanation via Berkovich geometry of this. 
\end{remark}

\begin{lemma}\label{lem:semistable_skeleton}
	Let $\mathscr C$ be minimal snc-model of curve $C$ with semistable reduction.
	Then $\mathscr C$ has no inessential components. 
\end{lemma}
\begin{proof}
	Recall that $\mathscr C$ the minimal resolution of the minimal regular model $\mathscr C_\text{reg} $ of $C$. 
	Suppose for the sake of contradiction that  $C$ has inessential components.
	Then there is ``leaf''  $F$ which is a rational curve in $\mathscr C_s$. 
	If $F$ dominates a component $F'$ in $\mathscr C_\text{reg} $ then $F'$ is also a rational curve of multiplicity $1$ which intersects has self intersection $-1$,  which would contradict the minimality of  $\mathscr C _\text{reg} $ by Castelnouovo's criterion \cref{thm:castelnuovo}. 

	Suppose now that $F$ does not dominate a component in $\mathscr C_\text{reg} $.
	Then it is the $F$ is the exceptional divisor of a blowup in a smooth point of a intermediate model $\mathscr C \to \mathscr C' \to \mathscr C_\text{reg} $. 
	Then $\mathscr C'$ is also snc, which contradicts the minimality of $\mathscr C$. 
\end{proof}
\begin{corollary}
	Let $\mathscr C$ be ithe minimal snc-model of $C$. 
	Then the leaves of the dual graph  $\Delta (\mathscr C)$ have genus greater than $0$. 
\end{corollary}

Recall \cref{lemma:genus_semistable} which states that the Betti-number of $\Delta(C)$ plus the genus of vertices equals $g(C)$. 
In the case of $E$ this leaves two options. 
\begin{lemma}\label{lem:point_or_circle}
	Let $E$ be an elliptic curve with semi-stable reduction and $\mathscr E$ be its minimal semistable model. 
	Then $\Delta(\mathscr E) = \sk(\mathscr E) = \sk(E)$ is either 
	\begin{enumerate}
		\item a circle with every vertex of genus $0$,
		\item a single point of genus  $1$. 
	\end{enumerate}
\end{lemma}
\begin{proof}
	The fact $\sk(\mathscr E) = \sk(E)$ follows from \cref{lem:semistable_skeleton}. 
	We know that $\Delta(E)$ is homotopic to either a circle, or a point of genus $1$. 
	As it has no leaves of genus $0$ and is finite, this means that $\Delta(E)$ is either a circle or a point of genus 1, which proofs the claim.
\end{proof}

\begin{lemma}
	Let $E$ be as in \cref{prob:main_problem}. 
	Then $\sk(E)$ is a circle if $n > 0$ and $\sk(E)$ is a point if $n = 0$. 
\end{lemma}
\begin{proof}
	Recall that $\phi: E\an \to \pro^{1, \text{an}}_K$ is a finite map and that $\sk(E) = \sk(\mathscr P)$ with $\mathscr P$ the model from \cref{sec:finding_a_minimal_snc_model}.
	Suppose that $\sk(E)$ is a circle. 
	Then $\phi(\sk(E))= \sk(\mathscr P)$ is a not a point. 
	Hence $n \ne 0$. 
	Conversely, suppose that $\sk(E)$ is a point. 
	Then $\phi(\sk(E)) = \sk(\mathscr P)$ is a point, and thus $n  = 0$. 
\end{proof}

The only thing left to do is determine the length of the circle $\sk(E)$ when $n \ge 1$.
We know that $\phi|_{\sk(E)}:\sk (E) \to \sk(\mathscr P)$ is surjective map from a circle to a line, such that every point $x \in \sk(\mathscr P)$ has precisely 1 or 2 preimages. 
From this we see that $\phi|_{\sk(E)}$ is a branched cover with branch points in $a, b$, the end points of the line $\sk(\mathscr P)$ (see \cref{fig:skeleton_projective_line}).
\todo{topologically this is obvious, but how would you make this rigorous?}
Then the interior of $\sk(\mathscr P)$ lifts to two lines in $\mathscr  \sk(E)$ along the map $\phi$.
By \cref{rem:length_edge_skeleton_cover} each of these lifts has length $n$. 
So we find that whole circle has length $2n$. 
See \cref{fig:map_circle_line}
\begin{figure}[ht]
    \centering
    \incfig{map-circle-line}
    \caption{The map from $\sk(E)$ to $\sk(\mathscr P)$ inside $\pro^{1, \text{an}}_K$, with each of the lifts of $(a, b)$ in a different color.}
    \label{fig:map_circle_line}
\end{figure}
This gives gives us the following proposition.
\begin{proposition}
	Let $E$ be the curve from \cref{prob:main_problem}. 
	Suppose that $\ch k \ne 2$ and that $E$ has semistable reduction. 
	Then $E$ has reduction type $\mathrm I_{2n}$ with $n = v(\lambda)$. 
\end{proposition}

The picture we sketched above also helps us to understand what happens purely model theoretically. 
Let $\ell_1, \ell_2$ be the two lifts of $(a, b)$ along $\phi$. 
By \cref{prop:balancing_finite_morphism} we see that $\phi$ maps divisorial points in $\ell_1, \ell_2$ to divisorial points in $(a, b)$ of the same degree. 
Recall from \cref{sec:finding_a_minimal_snc_model} that $\mathscr P$ consist of a chain of $n+1$ rational curves of multiplicity 1. 
From this we see that $(a, b)$ contains exactly $n -1$ points of multiplicity 1. 
Hence each of the lines $\ell_1, \ell_2$ also contains $n-1$ points of multiplicity 1.
Together these are $2n - 2$ of the  $2n$ irreduducible components of $\mathscr E_\text{min} $ (the minimal regular/snc model). 

From \cref{prop:balancing_finite_morphism} we cannot see whether $\phi^{-1}(a), \phi^{-1}(b)$ are points of multiplicity 1 one 2. 
But as $\mathscr E_\text{min} $ has $2n$ components with multiplicity $1$, we find that  $\phi^{-1}(a), \phi^{-1}(b)$ must be the 2 missing degree 1 points. 
A more Berkovich theoric argument for this is the following. 
As $E$ has semistable reduction, we know that $\mathscr E_\text{min} $ is semistable, or the blowup of a semistable nc-model. 
As a result all the edges between two multiplicity 1 components are as described in \cref{lem:number_divisorial_points}.
So if $\phi ^{-1}(a)$ has degree 2 it must lay on an edge $e$ with ends points $x, y$ (possibly $x = y$) of multiplicity $1$ with no other multiplicity 1 points in between $x, \phi^{-1}(a)$ and $y, \phi^{-1}(a)$. 
By \cref{lem:number_divisorial_points} $\phi^{-1}(a)$ is the only multiplicity 2 point on $e$. 
But between every multiplicity one point on $\ell_1 \cup \{\phi^{-1}(b)\} $ and $\phi^{-1}(a)$ there is a multiplicity two point. 
This is a contradiction. 
So $\phi^{-1}(a)$ is a multiplicity 1 point, and similiarly $\phi^{-1}(b)$ is a multiplicity 1 point.  


From \cref{rem:balancing_galois_cover} we see that the map between components corresponding to $\phi^{-1}(a)$ and $a$ is a double cover of $\pro^{1}_k$ onto $\pro^{1}_k$.
Likewise for $\phi^{-1}(b)$ and $b$. 
This shows that there is the following morphism of models $\psi: \mathscr E_\text{min} \to \mathscr P$ extending the map $f: E \to \pro^{1}_K$, which we have illustrated in \cref{fig:morphism_models_semistable_tame}.

\begin{figure}[ht]
    \centering
    \incfig{morphism-models-semistable-tame}
    \caption{A morphimss }
    \label{fig:morphism_models_semistable_tame}
\end{figure}
From this we also see that $\mathscr E_\text{min} $ is actually the normalisation of $\mathscr P$ in $K(E)$. 
Such pair of models is sometimes referred to as a simultanious resolution of singularities \cite[sec.\ 6]{liuModelsCurvesFinitea}.



\subsection{Determining $m$ when $E$ is of type $\mathrm I_m^*$} \label{sec:determining_m_when_e_is_of_type_ims$}
We now consider what happens in the hypothetical case that $E$ has reduction of type $I_m^*$ for some $m$. 
As we've discussed in \cref{rem:justification_semistable}, this does not happen in the context of \cref{prob:main_problem}, but it does happen when $n < 0$, i.e. $|\lambda| > 1$.  
In this case we can construct a similar model to $\mathscr P$, whose dual graph is the middle line in the case in the case $|\lambda| > 1$ from \cref{fig:configurations_of_gamma_lambda}. 


\begin{lemma}\label{lem:divisorial_points_Is}
	Let $x$ be a divisorial point in $\sk(E)$. 
	Then  \[
		N(x) \ge 2
	,\] 
	with equality if and only if $x$ is one of the vertices in $\sk(\mathscr E_\text{min} )$ where $\mathscr E_\text{min} $ is the minimal regular model of $E$. 
\end{lemma}
\begin{proof}
	Let $\Delta$ be the the dual graph of the essential components of $\mathscr E_\text{min} $. 
	Recall that $\Delta$ consists of $m + 1$ vertices of multiplicity $2$ and that there is a natural isomorphism $ |\Delta| \simeq \sk(E)$. 

	Let $x$ be a divisorial point on $\sk(E)$.
	Then $x$ can be obtained as a vertex of a model $\mathscr E$ by repeatedly blowing up the intersection points in $\mathscr E_\text{min} $ between components in $\sk(E)$. 
	Every exceptional divisor of such blowup will have multiplicity strictly greater than $2$. 
	So $x$ has multiplicity $2$ if it was one of the original vertices in $\Delta$, and multiplicity strictly greater than $2$ if not. 
\end{proof}

\begin{proposition}\label{prop:m_when_E_type_Is}
	Suppose that $|v(\lambda)| = n$ and $E$ is of type $I_m^*$, then $m = 2n$. 
\end{proposition}
\begin{proof}
	We start of locally so $\mathscr P'$ be a the non-proper model of $\pro^{1}_K$ that only contains to adjacent irreducible components $F, G$ of  $\mathscr P$.
	Let $x, y $ be the corresponding divisorial points in $\pro^{1, \text{an}}_K$. 
	Then $x, y$ are of multiplicity $1$, hence $\phi^{-1}(x), \phi^{-1}(y) \subset  \sk(E)$ are finite sets of points of multiplicity $1$ or $2$. 
	\Cref{lem:divisorial_points_Is} tells us that the points in $\phi^{-1}(x)$ are exactly of multiplicity 2, thus correspond to essential components in $\mathscr E_\text{min} $.
	From this we also see that $\phi^{-1}(x)$ is a point. 
	Similarly $\phi^{-1}(y)$ is a point of multiplicity 2 corresponding to an essential component in $\mathscr E$. 

\begin{figure}[ht]
    \centering
    \incfig{models-proof-is}
    \caption{The models and skeleta in the proof of \cref{prop:m_when_E_type_Is}.}
    \label{fig:models_proof_is}
\end{figure}

	Let $g: N(\mathscr P') \to \mathscr P'$ be the normalisation of $\mathscr P'$ in $K(E)$.
	From the computations above we see that $N(\mathscr P')_s$ contains two irreducible components, the strict transforms of $F, G$, both of multiplicity $2$. 
	We can see that $N(\mathscr P')$ is not regular. 
	Indeed if it was regular, by the adjunction formula we find that \[
		2 = 2F\cdot G =  (g^* F)\cdot (g^* G) =  (2 \tilde F) \cdot  (2 \tilde G)  \ge 4
	,\] 
	which clearly is a contradiction. 
	This tells us that $\tilde F, \tilde G$ are not adjacent in $\mathscr E_\text{min} $. 
	Let $H_1, \ldots, H_r$ be the missing components between $\tilde F, \tilde G$ of $\mathscr E_\text{min} $. 
	Let $\mathscr E'$ be the (non-proper) model of $E$ containig $\tilde F, H_1, \ldots, H_r, \tilde G$. See \cref{fig:models_proof_is}.
	
Then the images of $\phi(H_1), \ldots, \phi(H_r)$ are distinct divisorial points of multiplicity $1$ or $2$.
By \cref{lem:number_divisorial_points} we know that $\sk(\mathscr P')$ has only 3 points such points, two of which are $x, y$. 
Hence $r = 1$. 

This concludes the local computation. Globally this means the following. 
So we see that there is one component of degree 2 in $\mathscr E_\text{min} $ for every irreducible component of $\mathscr P$, and one for every intersection point. 
Hence there are $n + 1 + n = 2n + 1$ components of multiplicity $2$ in $\mathscr E_\text{min} $ from which it follows that $m = 2n$.  
\end{proof}


We can actually sketch a similar picture of a map of models as we did at the end of \cref{sec:the_semistable_case} of a map between models. 
Let $\mathscr E_\text{min} $ be the minimal snc model of $E$. 
Let $\mathscr E'_{\text{min}}$ the (singular) model of we obtain by contracting the four multiplicity 1 components in  $\mathscr E_\text{min} $. 
So $\mathscr E'_{\text{min}, s} $ is a chain of rational curves of multiplicity 2. 
Let $\mathscr P'$ be the model we obtain by blowing up each intresection point in $\mathscr P$. 
So between each two reduced rational cruves in $\mathscr P_s$ we place a multiplicity 2 rational curve.

By from our argument in the proof \cref{prop:m_when_E_type_Is}, we see that there is a morphism $\psi: \mathscr E'_\text{min}  \to \mathscr P'$ visualised in \cref{fig:dualgraphs_is_models} and that $\mathscr E'_\text{min} $ is the normalisation of $\mathscr P'$ in  $K(E)$. 
\todo{what happens to the weierstrass component? I suspect the image lays on on the lines $[a, 0]$, $[a, \lambda]$, $[b, \infty]$, $[b, 1]$}
\begin{figure}[ht]
    \centering
    \incfig{dualgraphs-is-models}
    \caption{Dualgraphs of the models $\mathscr E_\text{min} , \mathscr E_\text{min} ', \mathscr P', \mathscr P$ and the relations between them.}
    \label{fig:dualgraphs_is_models}
\end{figure}






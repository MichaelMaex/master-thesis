In the case where $\ch k \ne 2$ we can use \cref{prop:weightfunction_fullback} to compare the skeleton $\sk(\mathscr P)$ with the essential skeleton $\sk(E)$. 
\begin{lemma}\label{lem:sk_E_pulls_back}
	Let $\mathscr P$ be as in \cref{sec:a_weight_function_on_pro_1_k}, and $E$ from \cref{prob:main_problem}. 
	Then \[
		\sk(E) = \phi^{-1}(\sk(\mathscr P))
	.\] 
\end{lemma}
\begin{proof}
	In \cref{prop:model_P_is_KS} we saw that $\sk(\mathscr P) = \sk(\pro^{1}_K, \omega)$, where $\omega$ is the rational section of $\Omega_{\pro^{1}_K}^{\otimes 2}$ with poles at $0, 1, \lambda, \infty$, which is unique up to multiplication by $K$. 
	Then $f^* \omega$ is a $2$-pluricanonical form. 
	Indeed, we can take $\omega$ as a global section of $\Omega_{\pro^{1}_K}^{\otimes 2}(f_* \ram _f)$. 
	And 
	\[
		f^* (\Omega_{\pro^{1}_K / K}^{\otimes 2}(-f_* \ram _f)) = \Omega_{E /K}^{\otimes 2} \otimes \mathcal{O}(2 \ram_f) \otimes f^*f_* \mathcal{O}(-\ram _f) = \Omega_{E / K}
	.\] 
	We also have 
	\[
		\sk(E, f^*\omega) = \minloc \wt_{f^* \omega} = \minloc \wt_{\omega}\circ \phi = \phi^{-1}(\minloc \wt_\omega) = \phi^{-1} (\sk(\pro^{1}_K, \omega))
	.\] 
	Recall that $\Omega_{E / K}$ is trivial as $E$ is an elliptic curve. 
	So all pluricanonical forms are the same up to multiplication. 
	So $\sk(E)$ can be computed using only one pluricanonical form, thus \[
		\sk(E) = \sk(E, f^*\omega) = \phi^{-1}(\sk(\pro^{1}_K, \omega)) = \phi^{-1}(\mathscr P)
	.\] 
\end{proof}

\subsection{The semistable case} \label{sec:the_semistable_case}
In this section we answer \cref{prob:main_problem} when $k$ has characteristic different from $2$, under the assumption that $E$ has semi-simple reduction, i.e.\  $E$ is of type $\mathrm I_m$ for some $m \ge 0$. 

\begin{remark}\label{rem:justification_semistable}
	The assumption that $E$ has semi-stable reduction is valid as this is what we have found \cref{sec:char_k_is_2} by means of Tate's algorithm. 
	However assumption also isn't valid for in the case where we choose $|\lambda| > 1$ (see \cref{fig:configurations_of_gamma_lambda}).
	If $|\lambda| > 1$, i.e.\ $v(\lambda) < 0$ then
	explicit computations of examples suggest that  $E$ has semistable reduction if $v(\lambda)$ is even and $E$ is of type $I^*_n$ if $n$ is odd. 
	I do not know of an explanation via Berkovich geometry of this. 
\end{remark}

\begin{lemma}\label{lem:semistable_skeleton}
	Let $\mathscr C$ be minimal snc-model of curve $C$ with semistable reduction.
	Then $\mathscr C$ has no inessential components. 
\end{lemma}
\begin{proof}
	Recall that $\mathscr C$ the minimal resolution of the minimal regular model $\mathscr C_\text{reg} $ of $C$. 
	Suppose for the sake of contradiction that  $C$ has inessential components.
	Then there is ``leaf''  $F$ which is a rational curve in $\mathscr C_s$. 
	If $F$ dominates a component $F'$ in $\mathscr C_\text{reg} $ then $F'$ is also a rational curve of multiplicity $1$ which intersects has self intersection $-1$,  which would contradict the minimality of  $\mathscr C _\text{reg} $ by Castelnouovo's criterion \cref{thm:castelnuovo}. 

	Suppose now that $F$ does not dominate a component in $\mathscr C_\text{reg} $.
	Then it is the $F$ is the exceptional divisor of a blowup in a smooth point of a intermediate model $\mathscr C \to \mathscr C' \to \mathscr C_\text{reg} $. 
	Then $\mathscr C'$ is also snc, which contradicts the minimality of $\mathscr C$. 
\end{proof}
\begin{corollary}
	Let $\mathscr C$ be ithe minimal snc-model of $C$. 
	Then the leaves of the dual graph  $\Delta (\mathscr C)$ have genus greater than $0$. 
\end{corollary}

Recall \cref{lemma:genus_semistable} which states that the Betti-number of $\Delta(C)$ plus the genus of vertices equals $g(C)$. 
In the case of $E$ this leaves two options. 
\begin{lemma}\label{lem:point_or_circle}
	Let $E$ be an elliptic curve with semi-stable reduction and $\mathscr E$ be its minimal semistable model. 
	Then $\Delta(\mathscr E) = \sk(\mathscr E) = \sk(E)$ is either 
	\begin{enumerate}
		\item a circle with every vertex of genus $0$,
		\item a single point of genus  $1$. 
	\end{enumerate}
\end{lemma}
\begin{proof}
	The fact $\sk(\mathscr E) = \sk(E)$ follows from \cref{lem:semistable_skeleton}. 
	We know that $\Delta(E)$ is homotopic to either a circle, or a point of genus $1$. 
	As it has no leaves of genus $0$ and is finite, this means that $\Delta(E)$ is either a circle or a point of genus 1, which proofs the claim.
\end{proof}

\begin{lemma}
	Let $E$ be as in \cref{prob:main_problem}. 
	Then $\sk(E)$ is a circle if $n > 0$ and $\sk(E)$ is a point if $n = 0$. 
\end{lemma}
\begin{proof}
	Recall that $\phi: E\an \to \pro^{1, \text{an}}_K$ is a finite map and that $\sk(E) = \sk(\mathscr P)$ with $\mathscr P$ the model from \cref{sec:finding_a_minimal_snc_model}.
	Suppose that $\sk(E)$ is a circle. 
	Then $\phi(\sk(E))= \sk(\mathscr P)$ is a not a point. 
	Hence $n \ne 0$. 
	Conversely, suppose that $\sk(E)$ is a point. 
	Then $\phi(\sk(E)) = \sk(\mathscr P)$ is a point, and thus $n  = 0$. 
\end{proof}

The only thing left to do is determine the length of the circle $\sk(E)$ when $n \ge 1$.
We know that $\phi|_{\sk(E)}:\sk (E) \to \sk(\mathscr P)$ is surjective map from a circle to a line, such that every point $x \in \sk(\mathscr P)$ has precisely 1 or 2 preimages. 
From this we see that $\phi|_{\sk(E)}$ is a branched cover with branch points in $a, b$, the end points of the line $\sk(\mathscr P)$ (see \cref{fig:skeleton_projective_line}).
\todo{topologically this is obvious, but how would you make this rigorous?}
Then the interior of $\sk(\mathscr P)$ lifts to two lines in $\mathscr  \sk(E)$ along the map $\phi$.
By \cref{rem:length_edge_skeleton_cover} each of these lifts has length $n$. 
So we find that whole circle has length $2n$. 
See \cref{fig:map_circle_line}
\begin{figure}[ht]
    \centering
    \incfig{map-circle-line}
    \caption{The map from $\sk(E)$ to $\sk(\mathscr P)$ inside $\pro^{1, \text{an}}_K$, with each of the lifts of $(a, b)$ in a different color.}
    \label{fig:map_circle_line}
\end{figure}
This gives gives us the following proposition.
\begin{proposition}
	Let $E$ be the curve from \cref{prob:main_problem}. 
	Suppose that $\ch k \ne 2$ and that $E$ has semistable reduction. 
	Then $E$ has reduction type $\mathrm I_{2n}$ with $n = v(\lambda)$. 
\end{proposition}

The picture we sketched above also helps us to understand what happens purely model theoretically. 
Let $\ell_1, \ell_2$ be the two lifts of $(a, b)$ along $\phi$. 
By \cref{prop:balancing_finite_morphism} we see that $\phi$ maps divisorial points in $\ell_1, \ell_2$ to divisorial points in $(a, b)$ of the same degree. 
Recall from \cref{sec:finding_a_minimal_snc_model} that $\mathscr P$ consist of a chain of $n+1$ rational curves of multiplicity 1. 
From this we see that $(a, b)$ contains exactly $n -1$ points of multiplicity 1. 
Hence each of the lines $\ell_1, \ell_2$ also contains $n-1$ points of multiplicity 1.
Together these are $2n - 2$ of the  $2n$ irreduducible components of $\mathscr E_\text{min} $ (the minimal regular/snc model). 

From \cref{prop:balancing_finite_morphism} we cannot see whether $\phi^{-1}(a), \phi^{-1}(b)$ are points of multiplicity 1 one 2. 
But as $\mathscr E_\text{min} $ has $2n$ components with multiplicity $1$, we find that  $\phi^{-1}(a), \phi^{-1}(b)$ must be the 2 missing degree 1 points. 
A more Berkovich theoric argument for this is the following. 
As $E$ has semistable reduction, we know that $\mathscr E_\text{min} $ is semistable, or the blowup of a semistable nc-model. 
As a result all the edges between two multiplicity 1 components are as described in \cref{lem:number_divisorial_points}.
So if $\phi ^{-1}(a)$ has degree 2 it must lay on an edge $e$ with ends points $x, y$ (possibly $x = y$) of multiplicity $1$ with no other multiplicity 1 points in between $x, \phi^{-1}(a)$ and $y, \phi^{-1}(a)$. 
By \cref{lem:number_divisorial_points} $\phi^{-1}(a)$ is the only multiplicity 2 point on $e$. 
But between every multiplicity one point on $\ell_1 \cup \{\phi^{-1}(b)\} $ and $\phi^{-1}(a)$ there is a multiplicity two point. 
This is a contradiction. 
So $\phi^{-1}(a)$ is a multiplicity 1 point, and similiarly $\phi^{-1}(b)$ is a multiplicity 1 point.  


From \cref{rem:balancing_galois_cover} we see that the map between components corresponding to $\phi^{-1}(a)$ and $a$ is a double cover of $\pro^{1}_k$ onto $\pro^{1}_k$.
Likewise for $\phi^{-1}(b)$ and $b$. 
This shows that there is the following morphism of models $\psi: \mathscr E_\text{min} \to \mathscr P$ extending the map $f: E \to \pro^{1}_K$, which we have illustrated in \cref{fig:morphism_models_semistable_tame}.

\begin{figure}[ht]
    \centering
    \incfig{morphism-models-semistable-tame}
    \caption{A morphimss }
    \label{fig:morphism_models_semistable_tame}
\end{figure}
From this we also see that $\mathscr E_\text{min} $ is actually the normalisation of $\mathscr P$ in $K(E)$. 
Such pair of models is sometimes referred to as a simultanious resolution of singularities \cite[sec.\ 6]{liuModelsCurvesFinitea}.



\subsection{Determining $m$ when $E$ is of type $\mathrm I_m^*$} \label{sec:determining_m_when_e_is_of_type_ims}
We now consider what happens in the hypothetical case that $E$ has reduction of type $I_m^*$ for some $m$. 
As we've discussed in \cref{rem:justification_semistable}, this does not happen in the context of \cref{prob:main_problem}, but it does happen when $n < 0$, i.e. $|\lambda| > 1$.  
In this case we can construct a similar model to $\mathscr P$, whose dual graph is the middle line in the case in the case $|\lambda| > 1$ from \cref{fig:configurations_of_gamma_lambda}. 


\begin{lemma}\label{lem:divisorial_points_Is}
	Let $x$ be a divisorial point in $\sk(E)$. 
	Then  \[
		N(x) \ge 2
	,\] 
	with equality if and only if $x$ is one of the vertices in $\sk(\mathscr E_\text{min} )$ where $\mathscr E_\text{min} $ is the minimal regular model of $E$. 
\end{lemma}
\begin{proof}
	Let $\Delta$ be the the dual graph of the essential components of $\mathscr E_\text{min} $. 
	Recall that $\Delta$ consists of $m + 1$ vertices of multiplicity $2$ and that there is a natural isomorphism $ |\Delta| \simeq \sk(E)$. 

	Let $x$ be a divisorial point on $\sk(E)$.
	Then $x$ can be obtained as a vertex of a model $\mathscr E$ by repeatedly blowing up the intersection points in $\mathscr E_\text{min} $ between components in $\sk(E)$. 
	Every exceptional divisor of such blowup will have multiplicity strictly greater than $2$. 
	So $x$ has multiplicity $2$ if it was one of the original vertices in $\Delta$, and multiplicity strictly greater than $2$ if not. 
\end{proof}

\begin{proposition}\label{prop:m_when_E_type_Is}
	Suppose that $|v(\lambda)| = n$ and $E$ is of type $I_m^*$, then $m = 2n$. 
\end{proposition}
\begin{proof}
	We start of locally so $\mathscr P'$ be a the non-proper model of $\pro^{1}_K$ that only contains to adjacent irreducible components $F, G$ of  $\mathscr P$.
	Let $x, y $ be the corresponding divisorial points in $\pro^{1, \text{an}}_K$. 
	Then $x, y$ are of multiplicity $1$, hence $\phi^{-1}(x), \phi^{-1}(y) \subset  \sk(E)$ are finite sets of points of multiplicity $1$ or $2$. 
	\Cref{lem:divisorial_points_Is} tells us that the points in $\phi^{-1}(x)$ are exactly of multiplicity 2, thus correspond to essential components in $\mathscr E_\text{min} $.
	From this we also see that $\phi^{-1}(x)$ is a point. 
	Similarly $\phi^{-1}(y)$ is a point of multiplicity 2 corresponding to an essential component in $\mathscr E$. 

\begin{figure}[ht]
    \centering
    \incfig{models-proof-is}
    \caption{The models and skeleta in the proof of \cref{prop:m_when_E_type_Is}.}
    \label{fig:models_proof_is}
\end{figure}

	Let $g: N(\mathscr P') \to \mathscr P'$ be the normalisation of $\mathscr P'$ in $K(E)$.
	From the computations above we see that $N(\mathscr P')_s$ contains two irreducible components, the strict transforms of $F, G$, both of multiplicity $2$. 
	We can see that $N(\mathscr P')$ is not regular. 
	Indeed if it was regular, by the adjunction formula we find that \[
		2 = 2F\cdot G =  (g^* F)\cdot (g^* G) =  (2 \tilde F) \cdot  (2 \tilde G)  \ge 4
	,\] 
	which clearly is a contradiction. 
	This tells us that $\tilde F, \tilde G$ are not adjacent in $\mathscr E_\text{min} $. 
	Let $H_1, \ldots, H_r$ be the missing components between $\tilde F, \tilde G$ of $\mathscr E_\text{min} $. 
	Let $\mathscr E'$ be the (non-proper) model of $E$ containig $\tilde F, H_1, \ldots, H_r, \tilde G$. See \cref{fig:models_proof_is}.
	
Then the images of $\phi(H_1), \ldots, \phi(H_r)$ are distinct divisorial points of multiplicity $1$ or $2$.
By \cref{lem:number_divisorial_points} we know that $\sk(\mathscr P')$ has only 3 points such points, two of which are $x, y$. 
Hence $r = 1$. 

This concludes the local computation. Globally this means the following. 
So we see that there is one component of degree 2 in $\mathscr E_\text{min} $ for every irreducible component of $\mathscr P$, and one for every intersection point. 
Hence there are $n + 1 + n = 2n + 1$ components of multiplicity $2$ in $\mathscr E_\text{min} $ from which it follows that $m = 2n$.  
\end{proof}


We can actually sketch a similar picture of a map of models as we did at the end of \cref{sec:the_semistable_case} of a map between models. 
Let $\mathscr E_\text{min} $ be the minimal snc model of $E$. 
Let $\mathscr E'_{\text{min}}$ the (singular) model of we obtain by contracting the four multiplicity 1 components in  $\mathscr E_\text{min} $. 
So $\mathscr E'_{\text{min}, s} $ is a chain of rational curves of multiplicity 2. 
Let $\mathscr P'$ be the model we obtain by blowing up each intresection point in $\mathscr P$. 
So between each two reduced rational cruves in $\mathscr P_s$ we place a multiplicity 2 rational curve.

By from our argument in the proof \cref{prop:m_when_E_type_Is}, we see that there is a morphism $\psi: \mathscr E'_\text{min}  \to \mathscr P'$ visualised in \cref{fig:dualgraphs_is_models} and that $\mathscr E'_\text{min} $ is the normalisation of $\mathscr P'$ in  $K(E)$. 
\todo{what happens to the weierstrass component? I suspect the image lays on on the lines $[a, 0]$, $[a, \lambda]$, $[b, \infty]$, $[b, 1]$}
\begin{figure}[ht]
    \centering
    \incfig{dualgraphs-is-models}
    \caption{Dualgraphs of the models $\mathscr E_\text{min} , \mathscr E_\text{min} ', \mathscr P', \mathscr P$ and the relations between them.}
    \label{fig:dualgraphs_is_models}
\end{figure}



\subsection{Distinguishing between $\mathrm I_m$ and  $\mathrm I_m^*$} \label{sec:distinguishing_between_im_and_ims}

Recall that $E$ is given by the Weierstrass equation 
\begin{equation}\label{eq:weierstrass_dist}
	E: y^2 = x(x-1)(x-\lambda)
,\end{equation}
with $\lambda = u \cdot \pi^{n}$ and $u$ some unit in $R$. 
If $n \ge 0$ \cref{sec:char_k_is_not_2} we found that this Weierstrass equation is minimal.
So the component of the Weierstrass model is a multiplicity 1 component of $\mathscr E_\text{min} $, the minimal snc-model. 
Let $w$ be the divisorial point associated to component.
If we consider the morphism of models (on affine charts) \[
	\spec \frac{R[x, y]}{(y^2 - x(x-1)(x-\lambda)} \to \spec R[x]
\]
we see that the map dominates on the special fiber. 
So $\phi(w)$ is the point corresponding to the disk $B(0, 1)$ (\cref{lem:model_disk_proj_line})
But this exactly the point $b$. So $\phi(w) = b$ and in particular $w \in \phi^{-1}(\mathscr P) \in \sk(E)$. 
So $w$ is a multiplicity 1 essential component of $\mathscr E_\text{min} $.
This can only be the case if $E$ is of type $\mathrm I_{m}$ for some $m \ge 0$. 
As a bonus we also see that when $n = 0$,  $E$ is of type $I_0$ and not of some other reduction type with $\sk(E)$ a point. 

Putting all information in this section together we have reached obtained an alternative and almost purely Berkovich geometric proof for \cref{prop:conclusion_tate_tame}, the conclusion of Tate's algorithm. 
\medskip

In \cref{rem:justification_semistable} we said that when $n < 0$ we that whether $E$ is semistable depends on the parity of $n$. 
Here is a (conjectural) explanation for this. 
When $n < 0$, the Weierstrass equation \eqref{eq:weierstrass_dist} is no longer defined over $R$. 
After performing a coordinate transformation $(x_{2i}, y_{3i}) = (\pi^{-2i}x, \pi^{-3i}y)$ with $i = \left\lceil \frac{n}{2} \right\rceil $, the Weierstrass equation becomes \[
	E: y_{3i}^2 = x_{2i}(x_{2i}-\pi^{i})(x_{2i} - \pi^{i}\lambda)
.\] 
This Weierstrass equation is defined over $R$ and conjecturally minimal. 
Again let $w$ be the divisorial point associated to this Weierstrass model. 
From a similar map of models \[
	\spec \frac{R[x_{2i}, y_{2i}]}{\left(y_{3i}^2 - x_{2i}(x_{2i}-\pi^{i})(x_{2i} - \pi^{i}\lambda)\right)} 
	\to \spec R[x_{2i}]
,\] 
we see that $\phi(w)$ is the disk $B(0, |\pi|^{-n})$ if $n$ is even and $B(0,|\pi|^{-n-1}) $ when $n$ is odd. 
For this configuration of $\Gamma_\lambda$ (recall \cref{fig:configurations_of_gamma_lambda}) we can also construct a similar model $\mathscr P'$ of $\pro^{1}_K$ whose skeleton is the line from $B(0, 1)$ to $B(0, |\pi|^{-n})$ and such that $\sk(E) = \phi^{-1}(\sk(\mathscr P'))$.
In particular $w \in \sk(E)$  if and only if $i$ is even. 
So when $i$ is even the Weierstrass component is essential, thus the reduction type is $\mathrm I_m$ for some $m$. 
And when  $i$ is odd, the Weierstrass component is inessential. 
So the reduction type of $E$ is $\mathrm I_m^*$ for some $m$. 






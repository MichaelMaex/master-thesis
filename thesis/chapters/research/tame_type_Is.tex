We now consider what happens in the hypothetical case that $E$ has reduction of type $\mathrm I_m^*$ for some $m$. 
As we've discussed in \cref{rem:justification_semistable}, this does not happen in the context of \cref{prob:main_problem}, but it does happen when $n < 0$, i.e. $|\lambda| > 1$.  
In this case we can construct a similar model to $\mathscr P$, whose dual graph is the middle line in the case $|\lambda| > 1$ from \cref{fig:configurations_of_gamma_lambda}. 


\begin{lemma}\label{lem:divisorial_points_Is}
	Let $x$ be a divisorial point in $\sk(E)$. 
	Then  \[
		N(x) \ge 2
	,\] 
	with equality if and only if $x$ is one of the vertices in $\sk(\mathscr E_\text{min} )$ where $\mathscr E_\text{min} $ is the minimal regular model of $E$. 
\end{lemma}
\begin{proof}
	Let $\Delta$ be the dual graph of the essential components of $\mathscr E_\text{min} $. 
	Recall that $\Delta$ consists of $m + 1$ vertices of multiplicity $2$ and that there is a natural isomorphism $ |\Delta| \simeq \sk(E)$. 

	Let $x$ be a divisorial point on $\sk(E)$.
	Then $x$ can be obtained as a vertex of a model $\mathscr E$ by repeatedly blowing up the intersection points in $\mathscr E_\text{min} $ between components in $\sk(E)$. 
	Every exceptional divisor of such blowup will have multiplicity strictly greater than $2$. 
	So $x$ has multiplicity $2$ if it was one of the original vertices in $\Delta$, and multiplicity strictly greater than $2$ if not. 
\end{proof}

\begin{proposition}\label{prop:m_when_E_type_Is}
	Suppose that $|v(\lambda)| = n > 0$ and $E$ is of type $\mathrm I_m^*$, then $m = 2n$. 
\end{proposition}

\begin{proof}
	Let $\mathscr Y$ be the normalisation of $\mathscr P$ in $K(E)$.  
	From the arguments in the proof of \cref{lem:snc_models_morphism_curves} we see that $\mathscr Y$ is a proper but not necessarily regular model of $E$. 
	Moreover the divisorial points of the components in $\sY$ are the preimages from the divisorial points in $\sX$, which are of multiplicity 1.
	By \cref{rem:balancing_galois_cover} we know that the components of $\sY_s$ are of multiplicity 1 or 2. 
	Then \cref{lem:divisorial_points_Is} shows that the components of $\sY_s$ are of multiplicity exactly two, and thus belong to $\mathscr E_\text{min} $ (the minimal snc model). 
	So $\mathscr E_\text{min}  \to \sY$ is the minimal desingularisation of $\sY$. 

	We move on to show that the intersection points between the components are not regular.
	Indeed, suppose that $\tilde G, \tilde F$ are adjacent components in $\sY_s$, which are given by the strict transforms of adjacent components  $G, F$ in $\mathscr P_s$.
	As $\sY$ is a contraction of a chain of rational curves, we see that  $\tilde G, \tilde F$ intersect in only one point. 
	Suppose for the sake of contradiction that $\sY$ is regular in that intersection point. 
	Then the map $\mathscr E_\text{min} \to \sY$ is an isomorphism locally around the intersection point.  
	Let $F', G'$ be the strict transforms of $\tilde F, \tilde G$ in $\mathscr E_\text{min} $. 
	Then $F', G'$ are adjacent in $\mathscr E_\text{min} $ and intersect in one point. 
	Let $g: \mathscr E_\text{min}  \to \mathscr P$. 
	Then by the projection formula we have \[
		2 = 2 \cdot  F\cdot  G = g^* F \cdot g^* G = (2  F')\cdot (2 G') \ge 4
	.\] 
	This is a contradiction, which tells us that $ F',  G'$ are not adjacent in $\mathscr E_\text{min} $. 

\begin{figure}[ht]
    \centering
    \incfig{models-proof-is}
    \caption{The models and skeleta in the proof of \cref{prop:m_when_E_type_Is}. 
    Note that this illustrates the reasoning in the proof, not final determined model. In actuality $r = 1$, i.e.\ there is only one component between $G', F'$ in $\mathscr E_{\text{min},s}$. }
    \label{fig:models_proof_is}
\end{figure}

	Let $H_1, \ldots, H_r$ be the missing components between $\tilde F, \tilde G$ of $\mathscr E_\text{min} $. 
	These are all multiplicity 2 rational curves.
	The images of $\phi(H_1), \ldots, \phi(H_r)$ are distinct divisorial points of multiplicity $1$ or $2$ on the edge $e$ between $G, F$ in $\sk(\mathscr P)$.
By \cref{lem:number_divisorial_points} we know that $e$ has only 3 such points, two of which are $x, y$. 
Hence $r = 1$. 

 Globally this means the following. 
There is one component of degree 2 in $\mathscr E_\text{min} $ for every irreducible component of $\mathscr P$, and one for every intersection point in $\mathscr P$. 
Hence there are $n + 1 + n = 2n + 1$ components of multiplicity $2$ in $\mathscr E_\text{min} $ from which it follows that $m = 2n$.  




\end{proof}


We can actually sketch a similar picture of a map of models as we did at the end of \cref{sec:the_semistable_case}. 
Let $\mathscr E_\text{min} $ be the minimal snc model of $E$. 
Let $\mathscr E'_{\text{min}}$ the (singular) model of we obtain by contracting the four multiplicity 1 components in  $\mathscr E_\text{min} $. 
So $\mathscr E'_{\text{min}, s} $ is a chain of rational curves of multiplicity 2. 
Let $\mathscr P'$ be the model we obtain by blowing up each intersection point in $\mathscr P$. 
So between each two reduced rational curves in $\mathscr P_s$ we place a multiplicity 2 rational curve.

From our arguments in the proof of \cref{prop:m_when_E_type_Is}, we see that there is a morphism $\psi: \mathscr E'_\text{min}  \to \mathscr P'$ visualized in \cref{fig:dualgraphs_is_models} and that $\mathscr E'_\text{min} $ is the normalisation of $\mathscr P'$ in  $K(E)$. 
\todo{what happens to the Weierstrass component? I suspect the image lays on on the lines $[a, 0]$, $[a, \lambda]$, $[b, \infty]$, $[b, 1]$}
\begin{figure}[ht]
    \centering
    \incfig{dualgraphs-is-models}
    \caption{Dual graphs of the models $\mathscr E_\text{min} , \mathscr E_\text{min} ', \mathscr P', \mathscr P$ and the relations between them.}
    \label{fig:dualgraphs_is_models}
\end{figure}


What makes algebraic geometry over $\C$ different from algebraic geometry over other fields is that there is an analytification functor that turn $\C$-varieties into $\C$-analytic spaces, i.e.\ complex manifolds that may have singularities. 
If we restrict this functor to the category of proper  $\C$-varieties, this functor is even fully faithful. This fact and a collection of results that act as a dictionary between the algebraic and analytic world are collectively known as the GAGA principles. 
They lead to an interesting interplay between complex analysis and algebraic geometry, which gives rise to topics like Hodge theory. 

A analysis is not limited to $\C$. In fact the only thing we need to work out a theory of analysis is a normed field, preferably a complete normed field. 
If this field is not $\C$ or $\R$ then it has to be either trivially valued or non-archimedean. \todo{find a citation for this}
The best known examples of these are the $p$-adic rationals $\Q_p$. 
There is in fact an entire theory of $p$-adic analysis. 

Varieties over $p$-adic fields and rings lay at the heart of arithmetic geometry. 
So it is a natural question to ask whether $p$-adic varieties, or more generally varieties over complete non-archimedean fields can be ``analytified'' in a suitable way, giving us an interplay between algebraic geometry and analysis that has been very fruitful in complex geometry. 

The first question that needs to be resolved, is what such an analytic space might be.
Over $\C$ we already had a theory of complex manifolds.
But if we try to construct $p$-adic manifolds, we quickly run into a problem.
The topology of $\Q_p$ or any non-archimedean field, is totally disconnected. 
This means that $p$-adic manifolds are very non-rigid and in particular the principle of analytic continuation fails. 
We will use $K$ to mean any complete non-archimedean field. Think $K = \Q_p$ or $K = \C((t))$. 

In 1960\todo{source} Tate was was the first one propose a solution. 
His idea was to still work with $p$-adic manifolds, but replace the topology, by $G$-topology. 
This is somewhat like an ordinary topology, but not al set theoretic covers of opens are topological covers. 
The topological covers are called admissible.
This works, but it hides a lot of the topology of the space in set of covers, which is difficult to grasp.

In 1990 Berkovich proposed his theory of $K$-analytic spaces - often called Berkovich spaces - which solves the disconnectedness problem by adding extra points to the space, which connect the disconnected components. 
The resulting space actually is a very nice topological space - hausdorff, locally compact, path-connected, \ldots - which is in stark contrast to the topology of $K$ itself.  



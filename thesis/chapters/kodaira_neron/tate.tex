It is a natural to question, whether given an elliptic curve $E$, one can determine its reduction type. 
It is a well known result that any elliptic curve $E / K$ can be written in the form of a Weierstrass equation \begin{equation}\label{eq:weierstrass_eq_KN_tate}
E: y^2 +a_1 xy + a_3 y= x^3 + a_2 x^2 + a_4 x + a_6
,\end{equation}
with $a_1, \ldots, a_6 \in K$. 
After applying a coordinate transformation $(x', y') = (\pi^{-2n}x, \pi^{-3n} y)$ for sufficiently high $n$, one sees that one may choose  $a_1, \ldots, a_6 \in R$.
Thus \eqref{eq:weierstrass_eq_KN_tate} defines a relative curve (model of $E$) over  $R$. 

Starting from such description of $E$, J.\ Tate found an algorithm that finds the reduction type of $E$ if \eqref{eq:weierstrass_eq_KN_tate} is a minimal Weierstrass equation of  $E$. 
Otherwise it outputs an another Weierstrass equation that can be further reduced. 
Repeatedly applying this algorithm will terminate and result in the reduction type of $E$. 
A description and proof of Tate's algorithm is described in \cite[sec.\ IV.9]{silvermanAdvancedTopicsArithmetic1994}.
We will use Tate's algorithm in \cref{sec:expectations} to determine the reduction type of elliptic curves defined by a specific type of Weierstrass equation.

\medskip

If we assume that \eqref{eq:weierstrass_eq_KN_tate} is a reduced Weierstrass equation. 
Let $\mathscr W $ be the $R$-scheme defined by this equation. 
This is a normal, but non-proper model of $E$, with reduced special fiber. 
One can show that $\mathscr E$ can be obtained as a desingularisation of $\mathscr W$. 
In particular one of the irreducible components of $\mathscr E$ is the strict transform of $\mathscr W_s$. 
If $\mathscr W_s$ is a node, with a type $A_n$ singularity, then $E$ is of type $\mathrm I_n$. 

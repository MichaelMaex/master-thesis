In the previous chapter we have seen that for curves of genus $g(C) > 0$ there is a unique minimal regular model, $\mathscr C_\text{reg} $.
We can describe this model just by its combinatorial properties, i.e.\ irreducible components in $\mathscr C_s$, their multiplicities, their genera and their intersections.
This gives a way of classifying arithmetic curves by their minimal regular models. 
For genus $1$ curves, i.e.\ elliptic curves this was first done by K.\ Kodaira \cite{kodairaCompactAnalyticSurfaces1963} and later refined by A. Neron \cite{neronModelesMinimauxVarietes1964}.

This chapter is mostly based on \cite[][\S 8]{silvermanAdvancedTopicsArithmetic1994}. Occasionally \cite[sec.\ 10.2]{liuAlgebraicGeometryArithmetic2002} was referenced, where they do not assume that the residue field $k$ is algebraically closed. 
We will assume that $k$ is algebraically closed as that is the context that we use when we apply the results of this chapter in \cref{chap:a_berkovich_approach_to_classifying_elliptic_curves}.

\section{Notation and conventions} \label{sec:notation_and_conventions}

Throughout the this chapter $R$ will be a discrete valuation ring with maximal ideal $\mathfrak{m} = (\pi) $, such that the residue field $k := R / \mathfrak{m} $ is algebraically closed and $K$ is the fraction field of $R$. 

We let $E$ be an elliptic curve over $K$, and we let $\mathscr E$ be its minimal regular model. 
The irreducible components of $\mathscr E$ are $F_1, \ldots, F_r$ with respective multiplicities $N_1, \ldots, N_r$. 
So as divisors we have  
\begin{equation}\label{eq:decomposition_fibre}
	\mathscr E_s = \sum_{i = 0}^{r} N_i F_i.
\end{equation}

\section{The reduction types of an elliptic curve} \label{sec:the_reduction_types_of_an_elliptic_curve}
The classification puts the elliptic curve $E$ into one of the following types 
\[
	\mathrm{I}_0,\quad \mathrm{I}_1,\quad \mathrm{I}_n,\quad \mathrm{II},\quad \mathrm{III},\quad \mathrm{IV},\quad \mathrm{I}_0^*,\quad \mathrm{I}_{n}^*,\quad \mathrm{II}^*,\quad \mathrm{III}^*, \quad \mathrm{IV}^*
,\]
with $n \in \Z_{\ge 2}$ for type $\mathrm I_n$ and $n \in \Z_{\ge 1}$ for type $\mathrm I_n^*$. 
These are visualized in \cref{fig:reduction_types_of_e}. 

\begin{figure}[hbtp]
    \centering
    \incfig{reduction-types-of-e}
    \caption{The reduction types of elliptic curves. Components with multiplicity higher than 1 are labeled with their multiplicities. This figure is almost an exact copy of \cite[fig.\ 4.4]{silvermanAdvancedTopicsArithmetic1994}.}
    \label{fig:reduction_types_of_e}
\end{figure}

The elliptic curves with semistable reduction are precisely those of type $\mathrm I_n$ with $n \ge 0$. 
Note that while elliptic curves of type  $\mathrm{II}$, $\mathrm{III}$, $\mathrm{IV}$ have minimal regular models with reduced special fibers, they are not semistable as their special fibers have singular points that are not ordinary double points. 
So curves of type  $\mathrm{II}$, $\mathrm{III}$, $\mathrm{IV}$ are not semistable. 

The reduction type of a curve reveals a lot of arithmetic properties of the curve. 
A good overview of these properties can be found in \cite[tab.\ 4.1, p.\ 365]{silvermanAdvancedTopicsArithmetic1994}.
For the applications that we consider in this thesis we are only interested in the combinatorial properties of the special fiber, in particular how it desingularizes to a minimal snc-model. 
Later in \cref{tab:skeleton_by_kodaira_neron} we also show how the special fiber of the minimal snc-model looks like for each type. 


\section{Necessary results from intersection theory} \label{sec:necessary_results_from_intersection_theory}
Before we start the partial proof of the classification we gather the important results from intersection theory in the following lemma. 
\begin{lemma}
	\label{prop:irred_comp_of_prop_model}
	Let $E, \mathscr E$ be the elliptic curve and model from \cref{sec:notation_and_conventions}. 
	Let $K_{\mathscr E}$ be a canonical divisor on $E$. 
	\begin{enumerate}
		\item At least one irreducible component has multiplicity $N_i = 1$.  
		\item $\mathscr E_s$ is connected. 
		\item $K_{\mathcal{C} } \cdot \mathscr E_s = 0 $. 
			Here $K_{\mathcal{C} }$ is a canonical divisor. 
		\item If $r = 1$ (i.e.\ the special fiber is irreducible) then $F_1^2 = \mathscr{E} _s^2 = 0$ and $g (F_1) = 1$. 
		\item If $r \ge 2$ (i.e.\ the special fiber is reducible) then for each irreducible component $F_i$ we have \[
		F_i^2 = -2,\quad  F_i \simeq \pro^{1}_k \quad \text{ and }\quad \sum_{1 \le j \le r, j \ne i}^{} N_j F_j\cdot  F_i = 2 N_i
		.\]   
	\end{enumerate}
\end{lemma}
\begin{proof}
\begin{enumerate}
	\item As $E$ is an elliptic curve, it has at least one $K$-rational point $x$.
		Then by \cref{cor:closure_K_rational_point} we know that  $\red_{\mathscr E}(x)$ lies on a irreducible component of $\mathscr E_s$. 
	\item By \cite[cor.\ 9.1.24]{liuAlgebraicGeometryArithmetic2002} the map $\mathscr E \to R$ is $\mathcal{O}$-connected. 
		In particular  $\mathscr E_s$ is connected. 
	\item From \cref{prop:intersection_canonical_special} know that \[
			K_{\mathscr E} \cdot \mathscr E_{s} = 2g(E) - 2 = 0
	.\]  
	\item If $r = 1$ then $\mathscr E_s$ has only one irreducible component, $F_1$. 
		By the first part this component has multiplicity $N_1 = 1$. 
		So $\mathscr E_s = F_1$. 
		Then $\mathscr E_s^2 = F_1^2 = 0$. 
		As $\mathscr E \to \spec R$ is flat we know that $g(E)  = g(\mathscr E_s) = 1$.
	\item 	
		The adjunction formula (\cref{thm:adjunction_formula}) states that \begin{equation}\label{eq:proof_prereqs_1}
			F_i^2 + K_{\mathscr E} \cdot F_i = 2g(F_i) - 2
		.\end{equation}
		Suppose for the sake of contradiction that that $F_i^2 = 0$. 
		Then $F_i$ does not intersect any other components, and so $F_i$ is the only component of $\mathscr E_s$. 
		This contradicts $r \ge 2$. 
		We find that $F_i^2 \le -1$. 
		
		Suppose for the sake of contradiction that $K_\mathscr E \cdot F_i < 0$. 
		Then by \eqref{eq:proof_prereqs_1} this is only possible if $F_i^2 = -1$ and $g(F_i) =0$. 
		Thus $F_i$ is contractible by Castelnuovo's criterion, which contradicts the minimality of $\mathscr E$. 

		So for every $i$ we find that $F_i^2 \le 1, K_{\mathscr E} \cdot F_i \le 0$. 
		But \[
		0 = K_{\mathscr E} \cdot \mathscr E_s = \sum_{i = 1}^{r} K_{\mathscr E} \cdot F_i \le 0
		.\]  
		Hence  $K_{\mathscr E} \cdot F_i = 0$ for all $i$. 
		So \eqref{eq:proof_prereqs_1} becomes \[
			-1 \ge  F^2_i = 2g(F_i) - 2 \ge -2
		,\]  
		which can only hold if $F_i^2 = -2, g(F_i) = 0$. 

		The last equality is \cref{cor:compute_self_intersection}, together with $F_i^2 = -2$. 
\end{enumerate}	
\end{proof}




\section{A partial proof of the classification} \label{sec:a_partial_proof_of_the_classification}

The proof of the classification is very repetitive as keeps using the same combinatorial techniques to determine all possible. 
So we will only cover a part of the proof to get a feel of the combinatorics involved. 
For a full proof, see \cite[sec.\ IV.8]{silvermanAdvancedTopicsArithmetic1994} or \cite[sec.\ 10.2.1]{liuAlgebraicGeometryArithmetic2002}.

\medskip 
We start with the usual decomposition of the special fiber as in \cref{sec:notation_and_conventions}  \[
	\mathscr E_s = \sum_{i = 0}^{r} N_i F_i
.\]  
By point 1 of \cref{prop:irred_comp_of_prop_model} and reordering the fibers we may assume that $N_1 = 1$ and that $F_1$ contains the image of $0 \in E (K)$ under the reduction map. 
From now on the proof uses a lot of case distinctions. 
\subsection{$\mathscr E _s$ is irreducible, i.e. $r = 1$} \label{sec:C_s_is_irreducible,_i.e._r_=_1}
We have $\mathscr E _s = N_1 F_1 = F_1$. Then by point 4 of \cref{prop:irred_comp_of_prop_model} we see that  $\mathscr E_s$ is a irreducible curve of genus one over $k$. 
Then from general theory of curves of genus 1 we see that $\mathscr E _s$ is either smooth, has a cusp or has a node, which give type $\text{I}_0, \text{I}_1$ and $\text{II}$ respectively.  

\begin{figure}[ht]
    \centering
    \incfig{type-i2-iii}
    \caption{The special fibers of elliptic curves of type $\text{I}_0 \text{, I}_0$ and $\text{II}$,}
    \label{fig:type-i2-iii}
\end{figure}

\bigskip
In what follows we consider cases where $r \ge 2$. We will make frequent use the formula 
\begin{equation}\label{eq:sum_intersections}
	\sum_{1 \le j \le r, j \ne i}^{} N_j F_i \cdot F_j = 2 N_i
\end{equation}
The main clue is that all the terms in this sequence are non negative, which will really restrict the how large the $N_j$ 's and intersection numbers  $F_i \cdot  F_j$ can be.
Recall that we also have that $N_1 = 1$ and that $\mathscr E _s$ is connected. We may relabel such that $F_2$ intersects $F_1$, i.e.\  $F_1 \cdot F_2 > 0$.  
From now on the proof boils down to using combinatorics to find which cases are possible, similar to proofs of A, (B, C), D, E classifications once one has obtained the right symmetric bilinear form (the intersection pairing in our case).  



\subsection{$\mathscr E _s$ has two irreducible components, i.e.\ $r  = 2$} \label{sec:C_s_has_two_irreducible_components}
In this case we have $\mathscr E _s = F_1 + N_2F_2$. 
Plugging this into \eqref{eq:sum_intersections} with $i = 1$ and  $i = 2$ gives us \[
N_2 F_1 \cdot F_2 = 2, \qquad F_1 \cdot F_2 = 2 N_2
.\] 
From this we easily see that 
 \[
F_1 \cdot F_2 = 2, \qquad N_2 = 1
.\] 
So $F_1$ intersects $F_2$ twice. Either in two different points or in a single point. This gives Type  $\text{I}_2$ or Type $\text{III}$ respectively. 

\begin{figure}[ht]
    \centering
    \incfig{type-i-and-iii}
    \caption{The special fibers of elliptic curves of type $\text{I}_2$ and $\mathrm{III}$}
    \label{fig:type-i-and-iii}
\end{figure}


\subsection{$\mathscr E _s$ has more irreducible components, i.e.\ $r \ge 3$} \label{sec:S_s_has_more_irreducible_components}

In this case we claim that curves intersect at most once, i.e. 
\begin{claim}
\[
	F_i \cdot F_j \le 1 \text{ for all }  1 \le i, j \le r, i \ne j
\]
\end{claim}
\begin{proof}
	$\mathscr E _s$ is connected and has at least three irreducible components. 
	So there is a third distinct component $F_k$ that intersects $F_i$ or $F_j$. Suppose without loss of generality that  $F_k \cdot F_i \ge 1$. 
	Using \eqref{eq:sum_intersections} on both  $i, j$ yields \[
	N_j F_i \cdot F_j < N_j F_i \cdot  F_k + N_k F_i \cdot  F_k \le 2N_i, \qquad N_i F_i \cdot F_j \le 2 N_j
	.\] 
	Multiplying both inequalities and canceling $N_i N_j$ from both sides yields $(F_i \cdot  F_j)^2 < 4$, from which we see that $F_i \cdot  F_j = 1$ or $F_i \cdot F_j = 0$. 
\end{proof}

Recall that we assume that $F_1$ intersects $F_2$. So $F_1 \cdot  F_2 = 1$. 
Again using  \eqref{eq:sum_intersections} with $i = 1$ we see that \[
N_2 = N_2 F_1 \cdot F_2 \le 2N_1 = 2
.\] 
So $N_2$ is either $1$ or $2$. We split again into two cases. 

\subsubsection{Case $N_2 = 1$} 
In this case we will find that $\mathscr E_s$ is of type $\text{I}_r$ or $\text{IV}$. This follows if we can show any component $F_j$ has multiplicity $N_j$ and intersects exactly two other components (i.e.\ the dual graph cannot split). 

We already know that $F_1 \cdot F_2 = 1$ so $F_2$ intersects $F_1$ once. Using the same formula with $i = 2$ again gives us \[
	\underbrace{N_1 F_1\cdot F_2}_{1} + \sum_{j = 3}^{r}  N_j F_j \cdot F_2 = 2
.\] 
Hence $F_2$ intersects exactly one new component (not $F_1$). We are free to reorder and call that component $F_3$. Then we also read that $N_3 = 1$.  

Now we continue by induction. Suppose that for a string of components, $F_1, F_2, \ldots, F_m$ we have that all components have multiplicity $1$, i.e.\ $n_1 = \ldots = N_{m} = 1$ and that each inner component ($F_2, \ldots, F_{m -1}$) exactly intersects each of its neighbors once. 
Then applying the formula \eqref{eq:sum_intersections} to $F_m$ we see that \[
	\underbrace{N_{m -1} F_{m - 1} \cdot F_m}_1 + \sum_{j= 1, j\ne m}^{r} N_j F_j \cdot F_m  = 2
.\] 
So again $F_m$ intersects exactly one component that we are free to call $F_{m + 1}$ and $N_{m + 1} = 1$. 

Eventually we find an $F_m$ that already occurs in the string, and hence the components of the string form a circle. Because $\mathscr E _s$ is connected, this is the whole special fiber.  Hence $\mathscr E _s $ is of type $\text{IV}$ or $\text{I}_{r}$. 

\begin{figure}[ht]
    \centering
    \incfig{type-iv-and-ir}
    \caption{The special fibers of elliptic curves of type $\text{IV}$ and $\mathrm{I}_r$}
    \label{fig:type-iv-and-ir}
\end{figure}

\subsubsection{Case $N_2 = 2$} 
We are getting pretty comfortable with the yoga of this calculation,  using \eqref{eq:sum_intersections} and renaming components as necessary. 
So from now on we will stop referring the formula explicitly each time and take the liberty to label new components with increasing integers.

This we apply it to $F_1$ and find \[
	\underbrace{N_1 F_1 \cdot  F_2}_{2} + \sum_{j = 3}^{r} N_j F_j\cdot F_1 = 2
.\] 
So we see that $F_1$ cannot intersect any other component. 
Lets figure out what $F_2$ might intersect. 
\[
	\underbrace{N_1 F_1 \cdot F_2}_1 + \sum_{j = 3}^{r} N_j F_j \cdot F_2 = 4
.\] 
We find three possibilities
\begin{itemize}
	\item $F_2$ intersects three new components once $F_3, F_4, F_5$ with $N_3 = N_4 = N_5 = 1$. 
		Then the formula gives \[
			\underbrace{N_2F_2 \cdot  F_3}_{2} + \sum_{j \ne 2,3}^{}n_j F_j \cdot  F_3 = 2 
		.\] 
		So $F_3$ and similarly $F_4, F_5$ intersect no new components. 
		So we find that $E$ has type $I_0^*$.
	\item  $F_2$ intersects two new components. 
		In this case one can use similar arguments to find that $E$ is of type $\mathrm I_n^*$ for some $n \ge 1$. 
	\item $F_2$ intersects one new component. 
		Then using similar arguments as the rest of the proof one can find that $E$ is of type $\mathrm{II}^*, \mathrm{III}^*, \mathrm{IV}^*$.
\end{itemize}
This completes the classification. 




\section{Determining the reduction type of $E$} \label{sec:determining_the_reduction_type_of_e}
It is a natural question to ask, whether given an elliptic curve $E$, one can determine its reduction type. 
It is a well known result that any elliptic curve $E / K$ can be written in the form of a Weierstrass equation \begin{equation}\label{eq:weierstrass_eq_KN_tate}
E: y^2 +a_1 xy + a_3 y= x^3 + a_2 x^2 + a_4 x + a_6
,\end{equation}
with $a_1, \ldots, a_6 \in K$. 
After applying a coordinate transformation $(x', y') = (\pi^{-2n}x, \pi^{-3n} y)$ for sufficiently high $n$, one sees that one may choose  $a_1, \ldots, a_6 \in R$.
Thus \eqref{eq:weierstrass_eq_KN_tate} defines a relative curve (model of $E$) over  $R$. 

Starting from such description of $E$, J.\ Tate found an algorithm that finds the reduction type of $E$ if \eqref{eq:weierstrass_eq_KN_tate} is a minimal Weierstrass equation of  $E$. 
Otherwise it outputs an another Weierstrass equation that can be further reduced. 
Repeatedly applying this algorithm will terminate and result in the reduction type of $E$. 
A description and proof of Tate's algorithm is given in \cite[sec.\ IV.9]{silvermanAdvancedTopicsArithmetic1994}.
We will use Tate's algorithm in \cref{sec:expectations} to determine the reduction type of elliptic curves defined by a specific type of Weierstrass equation.

\medskip

Assume that \eqref{eq:weierstrass_eq_KN_tate} is a reduced Weierstrass equation. 
Let $\mathscr W $ be the $R$-scheme defined by this equation. 
This is a normal, but not necessarily regular model of $E$, with reduced special fiber. 
One can show that $\mathscr E$ can be obtained as a desingularisation of $\mathscr W$. 
In particular one of the irreducible components of $\mathscr E$ is the strict transform of $\mathscr W_s$. 
If $\mathscr W_s$ is a node, with a type $A_n$ singularity, then $E$ is of type $\mathrm I_n$. 







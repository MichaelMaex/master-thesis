
The proof of the classification is very repetitive as it keeps using the same combinatorial techniques to determine all possible configurations of the special fiber. 
So we will only cover a part of the proof to get a feel of the combinatorics involved. 
For a full proof, see \cite[sec.\ IV.8]{silvermanAdvancedTopicsArithmetic1994} or \cite[sec.\ 10.2.1]{liuAlgebraicGeometryArithmetic2002}.

\medskip 
We start with the usual decomposition of the special fiber as in \cref{sec:notation_and_conventions}  \[
	\mathscr E_s = \sum_{i = 0}^{r} N_i F_i
.\]  
By point 1 of \cref{prop:irred_comp_of_prop_model} and reordering the fibers we may assume that $N_1 = 1$ and that $F_1$ contains the image of $0 \in E (K)$ under the reduction map. 
From now on the proof uses a lot of case distinctions. 
\subsection{$\mathscr E _s$ is irreducible, i.e. $r = 1$} \label{sec:C_s_is_irreducible,_i.e._r_=_1}
We have $\mathscr E _s = N_1 F_1 = F_1$. Then by point 4 of \cref{prop:irred_comp_of_prop_model} we see that  $\mathscr E_s$ is a irreducible curve of genus one over $k$. 
Then from general theory of curves of genus 1 we see that $\mathscr E _s$ is either smooth, has a cusp or has a node, which give type $\text{I}_0, \text{I}_1$ and $\text{II}$ respectively.  

\begin{figure}[ht]
    \centering
    \incfig{type-i2-iii}
    \caption{The special fibers of elliptic curves of type $\text{I}_0 \text{, I}_1$ and $\text{II}$,}
    \label{fig:type-i2-iii}
\end{figure}

\bigskip
In what follows we consider cases where $r \ge 2$. We will make frequent use the formula 
\begin{equation}\label{eq:sum_intersections}
	\sum_{1 \le j \le r, j \ne i}^{} N_j F_i \cdot F_j = 2 N_i
\end{equation}
The main clue is that all the terms in this sequence are non negative, which will really restrict the how large the $N_j$ 's and intersection numbers  $F_i \cdot  F_j$ can be.
Recall that we also have that $N_1 = 1$ and that $\mathscr E _s$ is connected. We may relabel such that $F_2$ intersects $F_1$, i.e.\  $F_1 \cdot F_2 > 0$.  
From now on the proof boils down to using combinatorics to find which cases are possible, similar to proofs of A, (B, C), D, E classifications once one has obtained the right symmetric bilinear form (the intersection pairing in our case).  



\subsection{$\mathscr E _s$ has two irreducible components, i.e.\ $r  = 2$} \label{sec:C_s_has_two_irreducible_components}
In this case we have $\mathscr E _s = F_1 + N_2F_2$. 
Plugging this into \eqref{eq:sum_intersections} with $i = 1$ and  $i = 2$ gives us \[
N_2 F_1 \cdot F_2 = 2, \qquad F_1 \cdot F_2 = 2 N_2
.\] 
From this we easily see that 
 \[
F_1 \cdot F_2 = 2, \qquad N_2 = 1
.\] 
So $F_1$ intersects $F_2$ twice. Either in two different points or in a single point. This gives Type  $\text{I}_2$ or Type $\text{III}$ respectively. 

\begin{figure}[ht]
    \centering
    \incfig{type-i-and-iii}
    \caption{The special fibers of elliptic curves of type $\text{I}_2$ and $\mathrm{III}$}
    \label{fig:type-i-and-iii}
\end{figure}


\subsection{$\mathscr E _s$ has more irreducible components, i.e.\ $r \ge 3$} \label{sec:S_s_has_more_irreducible_components}

In this case we claim that curves intersect at most once, i.e. 
\begin{claim}
\[
	F_i \cdot F_j \le 1 \text{ for all }  1 \le i, j \le r, i \ne j
\]
\end{claim}
\begin{proof}
	$\mathscr E _s$ is connected and has at least three irreducible components. 
	So there is a third distinct component $F_k$ that intersects $F_i$ or $F_j$. Suppose without loss of generality that  $F_k \cdot F_i \ge 1$. 
	Using \eqref{eq:sum_intersections} on both  $i, j$ yields \[
	N_j F_i \cdot F_j < N_j F_i \cdot  F_k + N_k F_i \cdot  F_k \le 2N_i, \qquad N_i F_i \cdot F_j \le 2 N_j
	.\] 
	Multiplying both inequalities and canceling $N_i N_j$ from both sides yields $(F_i \cdot  F_j)^2 < 4$, from which we see that $F_i \cdot  F_j = 1$ or $F_i \cdot F_j = 0$. 
\end{proof}

Recall that we assume that $F_1$ intersects $F_2$. So $F_1 \cdot  F_2 = 1$. 
Again using  \eqref{eq:sum_intersections} with $i = 1$ we see that \[
N_2 = N_2 F_1 \cdot F_2 \le 2N_1 = 2
.\] 
So $N_2$ is either $1$ or $2$. We split again into two cases. 

\subsubsection{Case $N_2 = 1$} 
In this case we will find that $\mathscr E_s$ is of type $\text{I}_r$ or $\text{IV}$. This follows if we can show any component $F_j$ has multiplicity $N_j$ and intersects exactly two other components (i.e.\ the dual graph cannot split). 

We already know that $F_1 \cdot F_2 = 1$ so $F_2$ intersects $F_1$ once. Using the same formula with $i = 2$ again gives us \[
	\underbrace{N_1 F_1\cdot F_2}_{1} + \sum_{j = 3}^{r}  N_j F_j \cdot F_2 = 2
.\] 
Hence $F_2$ intersects exactly one new component (not $F_1$). We are free to reorder and call that component $F_3$. Then we also read that $N_3 = 1$.  

Now we continue by induction. Suppose that for a string of components, $F_1, F_2, \ldots, F_m$ we have that all components have multiplicity $1$, i.e.\ $n_1 = \ldots = N_{m} = 1$ and that each inner component ($F_2, \ldots, F_{m -1}$) exactly intersects each of its neighbors once. 
Then applying the formula \eqref{eq:sum_intersections} to $F_m$ we see that \[
	\underbrace{N_{m -1} F_{m - 1} \cdot F_m}_1 + \sum_{j= 1, j\ne m}^{r} N_j F_j \cdot F_m  = 2
.\] 
So again $F_m$ intersects exactly one component that we are free to call $F_{m + 1}$ and $N_{m + 1} = 1$. 

Eventually we find an $F_m$ that already occurs in the string, and hence the components of the string form a circle. Because $\mathscr E _s$ is connected, this is the whole special fiber.  Hence $\mathscr E _s $ is of type $\text{IV}$ or $\text{I}_{r}$. 

\begin{figure}[ht]
    \centering
    \incfig{type-iv-and-ir}
    \caption{The special fibers of elliptic curves of type $\text{IV}$ and $\mathrm{I}_r$}
    \label{fig:type-iv-and-ir}
\end{figure}

\subsubsection{Case $N_2 = 2$} 
We are getting pretty comfortable with the yoga of this calculation,  using \eqref{eq:sum_intersections} and renaming components as necessary. 
So from now on we will stop referring the formula explicitly each time and take the liberty to label new components with increasing integers.

This we apply it to $F_1$ and find \[
	\underbrace{N_1 F_1 \cdot  F_2}_{2} + \sum_{j = 3}^{r} N_j F_j\cdot F_1 = 2
.\] 
So we see that $F_1$ cannot intersect any other component. 
Lets figure out what $F_2$ might intersect. 
\[
	\underbrace{N_1 F_1 \cdot F_2}_1 + \sum_{j = 3}^{r} N_j F_j \cdot F_2 = 4
.\] 
We find three possibilities
\begin{itemize}
	\item $F_2$ intersects three new components once $F_3, F_4, F_5$ with $N_3 = N_4 = N_5 = 1$. 
		Then the formula gives \[
			\underbrace{N_2F_2 \cdot  F_3}_{2} + \sum_{j \ne 2,3}^{}n_j F_j \cdot  F_3 = 2 
		.\] 
		So $F_3$ and similarly $F_4, F_5$ intersect no new components. 
		So we find that $E$ has type $I_0^*$.
	\item  $F_2$ intersects two new components. 
		In this case one can use similar arguments to find that $E$ is of type $\mathrm I_n^*$ for some $n \ge 1$. 
	\item $F_2$ intersects one new component. 
		Then using similar arguments as the rest of the proof one can find that $E$ is of type $\mathrm{II}^*, \mathrm{III}^*, \mathrm{IV}^*$.
\end{itemize}
This completes the classification. 



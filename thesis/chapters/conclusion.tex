In \cref{chap:a_berkovich_approach_to_classifying_elliptic_curves} we succesfully solved the main problem (\cref{prob:main_problem}) in the case where $\ch k = 2$ using Berkovich geometry. 
To achieve this we obtained a new result on the behavior of the metric on Berkovich analytic curves under morphisms, a topic for which results are lacking in the literature. 

When $\ch k = 2$ we described of the image of the essential skeleton of an elliptic curve in the Berkovich projective line. 
This is a useful example to understand the behaviur of weight functions and essential skeleta under wildly ramified morphisms. 
A topic that at the time is not understood very well. 

In \cref{chap:loose_ends} we identified further paths of research which could lead to a better understanding of weight functions and metrics under morphisms. 





\section{Possible leads to follow up.} \label{sec:possible_leads_to_follow_up}

\subsection{An intrinsic definition of length on Berkovich analytic curves} \label{sec:an_intrinsic_definition_of_length_on_berkovich_analytic_curves}

\Cref{lem:number_divisorial_points,lem:equality_lenghts_preserve_mult} suggest that over a discretely valued base field the metric is an intrinsic property of an analytic curve, that can be defined without the use of the analytification map $X\an \to X$ and models of $X$. 
A possible definition might look something like this. 
Let $\mathcal{X} $ be a analytic curve and $p$ a path in $\mathcal X$. 
Let $a(n)$ be the number of divisorial points of degree $n$ on $p$. 
Then the length of $p$ is \[
	\ell(p) = \lim_{n} \frac{a(n)}{\phi(n)}
,\] 
where $\phi$ is the Euler totient function and $n$ becomes more divisible in the limit. 
If this is succesfull, one might think about generalizing this construction of the measure to higher dimensional $K$-analytic spaces. 

\subsection{Widening the class of elliptic curves considered} \label{sec:widening_the_class_of_elliptic_curves_considered}

In \cref{chap:a_berkovich_approach_to_classifying_elliptic_curves} we assumed that $E$ was of the form $y^2 = x(x-1)(x-\lambda)$ and that $v(\lambda) \le 0, v(1-\lambda) = 0$. 
It would be interesting to understand what happens if we choose $\lambda$ differently. 
What if $\lambda$ does not belong to the ground field $K$?
In the most general form we might ask what $E$ looks like if it is given by the equation $y^2 = (x-\alpha)(x-\beta)(x-\gamma)$ where $\alpha, \beta, \gamma$ may not even lay in the ground field $K$. 

\subsection{Generalizing to arbitrary morphisms of curves} \label{sec:generalizing_to_arbitrary_morphisms_of_curves}

Most of the results we found here consider the morphism of genus 1 curve to $\pro^{1}_K$, but many results should generalize to abitrary morphisms between curves. 
As a first step we might want to study hyper-elliptic curves, i.e.\ curves with a degree $2$ morphism to $\pro^{1}_K$.

\subsection{Understanding the log different better} \label{sec:understanding_the_log_different_better}
\todo[inline]{something about pulling back weight functions in the wild ramification case. Maybe similar combinatorial discription as in \cite{bakerWeightFunctionsBerkovich2016} suitable for computation.}





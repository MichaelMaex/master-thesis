Let $\mathscr X$ be a proper model of some $K$-variety proper $X$.
In \cref{sec:the_reduction_map_models} we constructed the reduction map $\red_{\mathscr X_s}: X_\text{cl}  \to \mathscr X_s$. 
The reduction map generalises to a map $\red_{\mathscr X}: X\an \to \mathscr X_s$.
This map is best described in terms of characters. 
\begin{definition}
	Let $\mathscr X$ be a proper model of a proper $K$-variety $X$. 
	Then we define the \emph{reduction map} to 
	\begin{align*}
		\red_{\mathscr X}:  X\an &\longrightarrow \mathscr X_k \\
		(\chi_x: \spec \mathcal H(x) \to X) &\longmapsto (\tilde{\chi_x}: \spec \widetilde{\mathcal H (x)} \to X_s)
	.\end{align*}
\end{definition}
Lets unpack this definition. 
Recall from \cref{sec:berkovich_spaces} that we can describe points from the Berkovich spectrum as characters, i.e.\ maps from the the spectrum of a complete valued field to $X$.
A minimal representative of such character for $x$ is given by $\chi_x: \spec \mathcal{H} (x) \to X \simeq \mathscr X_\eta$. 
This fits into a commutative diagram \[
\begin{tikzcd}
	\spec \mathcal{H} (x) \dar \rar{\chi_x} & X \rar[hookrightarrow] & \mathscr X \dar \\ 
	\spec \mathcal{H} (x)^{o} \ar[dashed]{urr}[']{\chi_x^{o}} \ar{rr} & & \spec R
\end{tikzcd}
.\] 
By the valuative criterion of properness there exists a unique map $\chi_x^{o}: \mathcal{H} (x)^{o} \to \mathscr X$. 
Then we may basechange this morphism to  $k$ to obtain $\widetilde{\chi_x} = \chi_x^{o} \otimes_R K: \widetilde{\mathcal{H} (x)} \to X_s$. 
This is a character of a point in $\mathscr X_s$. 


\begin{proposition}
	Let $X$ be a proper $K$-variety with a proper model $\mathscr X$. 
	Then the reduction map  $\red_{\mathscr X}$ is surjective and the inverse image of a generic point of an irreducible component of $\mathscr X$ is unique. 
\end{proposition}
\begin{proof}
	Let $x \in \mathscr X_s$ and  $\spec A \subset \mathscr X$ be affine open with \[
		A = \frac{R[T_1, \ldots, T_n]}{(f_1, \ldots, f_r)}
	.\] 
	Let \[
		\mathcal{A}  = \frac{K\left<T_1, \ldots, T_n \right>}{(f_1, \ldots, f_r)}
	.\] 
	Then \[
		\widetilde{\mathcal{A} } = \frac{k[T_1,\ldots, T_n]}{(f_1, \ldots, f_r)}
	,\]  
	an affine open of $X_s$. 
	The result now follows from \cite[prop.\ 2.4.4]{berkovichSpectralTheoryAnalytic2012}.
	\question{Klopt dit bewijs? De context in the paper van Berkovich is anders, maar alle andere bronnen verwijzen steeds naar deze stelling zonder verdere uitleg}
\end{proof}



The reduction map as defined above can also be defined for formal schemes $R$ schemes.  
\begin{definition}
	Let $\mathfrak{X} $ be a formal scheme over $R$. 
	Then \emph{reduction map} is 
	\begin{align*}
		\red_{\mathfrak X}:  \mathfrak{X} _\eta &\longrightarrow \mathfrak{X} _s \\
		(\spec \mathcal H(x) \to \mathfrak X) &\longmapsto (\spf \mathcal H(x)^o \to \mathfrak X)
	.\end{align*}
\end{definition}

We call the inverse images of points under the reduction map (either one) formal fibers. 
One approach to understand the analytification of a variety or formal scheme is to determine the formal fibre at any point of the reduction map, and glue these together to obtain the whole analytic space. 

If $\mathscr X$ is a semistable model of $X$, or $\mathfrak{X} $ is a semistable formal scheme then one can show that the formal fiber over a closed point in $\mathscr X_s$ or $\mathfrak{X} _s$ is a open ball, and that the formal fibre over an intersection point is an annulus. 
The formal fibres over generic points of $\mathscr X_s$ or $\mathfrak{X} _s$ are single points. 
So from this information we can piece together $X\an$ or  $\mathfrak{X} _\eta$. 
This worked in detail in \cite{bakerStructureNonarchimedeanAnalytic2013}.




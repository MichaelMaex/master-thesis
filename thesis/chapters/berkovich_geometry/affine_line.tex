Before we describe $\mathcal{M} (K[T])$ it is useful to study $K$ as a topological/metric space first.
\subsection{The topology of $K$} \label{sec:ultrametric_spaces}

Recall that $K$ is a metric space with the metric $d(a, b) = |a - b|$. 
This metric satisfies a stronger version of the triangle inequality. 
Such metrics are known as ultrametrics. 
\begin{definition}
	An \emph{ultrametric space} is a metric space $(X, d)$ where the metric satisfies the non-archimedean triangle inequality. 
	\[
		d(x, y) \le \max \{d(x, z) ,d(z, y)\}, \quad \forall x, y ,z \in X
	.\] 
\end{definition}
\begin{lemma}
	If $d(x,z) \ne d(z,y)$ the inequality is an equality. 
\end{lemma}
\begin{proof}
	Suppose without loss of generality that $d(x, z) \ge d(z,y)$, so 
	\[	
		d(x, y) \le \max \{d(x, z), d(z,y)\} = d(x, z)
	.\] 
	Then $d(x, z) \le \max \{d(x, y), d(y,z)\}$. So $d(x, z) \le d(x, y) \le d(x, z)$. This proves equality.
\end{proof}
\begin{corollary}
	Balls/disks in ultra metric spaces have some unusual behavior
	\begin{itemize}
		\item Any point of a ball (open or closed) is a center for that ball. 
		\item Two balls are either disjoint, or one is contained in the other.  
	\end{itemize}
	See \cref{fig:oddities_of_ultrametric_balls}.
\end{corollary}
\begin{figure}[h]
    \centering
    \incfig{oddities-of-ultrametric-balls}
    \caption{Oddities of disks in ultrametric spaces}
    \label{fig:oddities_of_ultrametric_balls}
\end{figure}
\nomenclature[Bar]{$B(a, r)$}{The closed ball with center $a$ and radius $r$}
\begin{corollary}
	The topology of $X$ is totally disconnected. 
\end{corollary}

\subsection{The Berkovich affine line when $K$ is algebraically closed} \label{sec:the_berkovich_affine_line_K_alg_closed}

We will describe the Berkovich space of the polynomials in one variable  $\mathcal{M} (K[T])$.
We first do this first under the assumption that $K$ is algebraically closed, and then explain how to expand to the case where $K$ is discretely valued.
We call the resulting space $\aff_K^{1, \text{an}}$, the \emph{analytification of the affine line}, or sometimes the \emph{Berkovich affine line}. 
\nomenclature[Ak]{$\aff^{1, \text{an}}_K$}{The Berkovich affine line over $K$}
\nomenclature[Pk]{$\pro^{1, \text{an}}_K$}{The Berkovich projective line over $K$}

\medskip

One way to find norms on $K[T]$ would to take a point $a \in K$ and define $|f|_a = |f(a)|_K$ where $|\cdot |_K$ is the norm on $K$. 
So there is a natural embedding $K \into \mathcal{M}(K[T])$ and we will implicitly identify these points.  

Another way to define a norm is by choosing a closed disk $B(a, r) = \{x \in K \st |x - a| \le r\} $ with center $a$, and radius $r$ and define \[
	|f|_{B(a, r)} = \sup_{x \in B(a, r)} |f(x)|
.\] 
\begin{claim}
	The norm $|f|_{B(a, r)}$ as defined above is a well defined multiplicative seminorm extending the norm on $K$. 
	If $r \in |K^{\times }|$ then the supremum is a maximum and the maximum is obtained for some point on the boundary of $B(a, r)$.
	Moreover, if  $f = \sum_{i = 0}^{n} b_i (T-a)^{i}$ is the Taylor expansion of $f$ around $a$ then for any $r \in [0, \infty)$ the norm can be computed as \begin{equation}\label{eq:norm_disk_taylor}
		|f|_{B(a, r)} = \max_{i}\{  |b_i|r^{i}\}
	.\end{equation} 
\end{claim}
\begin{proof}
	Let $f = \sum_{i = 0}^{n} a_i T^{i}$. 
	Then for $x \in B(a, r)$ we have $|x| \le |a| + r$. 
	So  $|f(x)| \le  \sum_{i = 0}^{n} |a_i| |x|^{i} \le \sum_{i = 0}^{n}|a_i| (|a| + r)^{i} $. 
	This last bound is independent of $x$. Hence the supremum exists. 

	The non-archimedean triangle inequality and extending the norm on $K$ is easily checked. 
	The multiplicativity is a little bit more subtle. 

	Without loss of generality we may translate the disk such that $a = 0$. 
	Lets assume for now that $r \in |K|$.
	Then we may also rescale such that $r = 1$. 

	Clearly the seminorm is multiplicative for elements in  $K$, i.e.\ for every $\lambda \in K, f \in K[t]: |\lambda \cdot f|_{B(a, r)} = |\lambda|\cdot |f|_{B(a, r)}$.  Let $f, g \in K[T]$, which we may rescale by an element of $K$, such that the largest coefficient of $f, g$ has norm $1$. In particular this means that $|f|_{B(0,1)}, |g|_{B(0,1)} \le 1$. 
	Then $f, g \in R[T]$ and their  reductions modulo  $\pi$ $\overline{f}, \overline{g} \in k[T]$ are nonzero.
	So $\overline{f}, \overline{g}$ have only a finite number of roots. In particular these is some $\overline{a} \in k$, represented by $a \in R$ such that $\overline{f}(\overline{a}) \ne 0, \overline{f}(\overline{a})\ne 0$. 
	So $f(a), g(a) \in R = B(0, 1)$ and $|f(a)| = |g(a)| = 1$. 
	In particular \cref{eq:norm_disk_taylor} holds. 
	So the supremum is obtained in $a$ for both polynomials and the norm is multiplicative as the supremum for $f\cdot g$ will also be obtained in $a$. 


	To obtain the result also holds for $r \not\in |K^{\times }|$, we use that any such disk $B(a, r)$ can be written as an increasing union of closed disks with radii in the value group. The result now follows from the continuity of \cref{eq:norm_disk_taylor} and the continuity of multiplication. 
\end{proof}



So the space of closed discs in $K$, which includes points (discs of radius zero), embeds as well in $\mathcal{M} (K[T])$. 
As we will see in a second these will make up almost all of the points on $\mathcal{M} (K[T])$. 
If we ignore these missing points for a moment, this means that $\mathcal{M}(K[T])$ is connected. 
Indeed, let $a \in K$ be any point. Then we can define a line \begin{equation}\label{eq:line_in_A_1}
	\ell_a: [0, \infty) \to  \aff_K^{1, \text{an}}: r\mapsto B(a, r)
\end{equation}
Because in an ultrametric space any point in a ball is a center of ball, we see that for any two points $a, b$ the lines $\ell_a, \ell_b$ coincide precisely when $r \ge |a - b|$. 
So $\aff_K^{1,\text{an}}$ is not only connected, it is path connected!
This is illustrated in \cref{fig:the_lines_la_lb_lc}. 
\begin{figure}[h]
    \centering
    \incfig{the-lines-la-lb-lc}
    \caption{The lines $\ell_a, \ell_b, \ell_c$ join up (i.e.\ define the same disk) when the radius $r$ is greater than the distance between the points.}
    \label{fig:the_lines_la_lb_lc}
\end{figure}

We have been missing a few points on $\aff_K^{1, \text{an}}$. 
The Berkovich classification theorem gives all points. 
\begin{theorem}
	[Berkovich classification theorem] \label{thm:berkovich_classification}
	Assume that $K$ is algebraically closed. 
	Any point $x \in \aff_K^{1,\text{an}}$ is given by a decreasing sequence of closed disks $B_n = B(a_n, r_n)$ in $K$ by the formula \begin{equation}\label{eq:norm_disk_polynomial}
	|f|_x = \lim_{n \to \infty} |f|_{B_n}
	.\end{equation}
	Moreover, two such sequences $B_n, B'_n$ define the same norm if and only if:
	 \begin{itemize}
		\item Both sequences have the same non-emtpy intersection, i.e.\ $\bigcap_{n = 1}^{\infty} B_n = \bigcap_{n = 1}^{\infty} B_n' \ne \emptyset$. 
			In this case the intersection $B = \bigcap_{n \in N} B_n$ is a closed disk (possibly a point) and $x = |\cdot |_B$. 
		\item Both sequences have empty intersection, i.e.\ $\bigcap_{n = 1}^{\infty} B_n = \bigcap_{n = 1}^{\infty} B_n' = \emptyset$  and for every $n$ there is an $m$ such that $B'_m \subset  B_n$  and $B_m \subset  B'_n$.
	\end{itemize}
	This means that we can classify the points in $\aff ^{1,\text{an}}_K$ into four types depending on $B = \bigcap_{n \in \N} B_n$. 
	\begin{description}
		\item[Type 1:] $B = \{a\} $ a point and the $|f|_x = |f(a)|$. 
		\item[Type 2:] $B = B(a, r)$ with $r \in |K^* |$ and $|f|_x = |f|_{B(a, r)}$
		\item[Type 3:] $B = B (a, r)$ with $r \not\in |K^*|$ and $|f|_x = |f|_{B(a, r)}$
		\item[Type 4:] $B = \emptyset$
	\end{description}
\end{theorem}
As we can see our previous discussion has excluded type four points. We will see an example of such point in \cref{ex:type4point}. 

\begin{proof}
	I will only show the proof that every norm is given by such a sequence as the proofs of all other statements are technical and give little insight. 

	Let $x \in \aff_K^{1, \text{an}}$ be a multiplicative seminorm on $K[T]$. 
	As  $K$ is algebraically closed every polynomial factors into linear factors $(T-a)$. 
	Hence $x$ is uniquely determined by its value $|T-a|_x$ for each $a \in K$. 
	We will write $\rho_a = |T-a|_x$.
	Let  $a, b \in K$ such that $\rho_a \le \rho_b$. 
	Then  \begin{equation}\label{eq:proof_berk_class_1}
		|a - b|_x = |(T-b) - (T - a)|_x \le \max \{\rho_a, \rho_b\} = \rho_b 
	.\end{equation} 
	Hence $a \in B_{b, \rho_b}$ and thus  $B(a, \rho_a) \subseteq B(b, \rho_b)$. 

	Define $\rho := \inf_{a \in K} \rho_a$ and let $a_1, a_2, \ldots$ be a sequence such that $(\rho_{a_i})_{i \in \N}  $ is a non-increasing sequence converging to $\rho$. 
	By the previous remark this gives a decreasing sequence of disks \[
		B(a_1, \rho_{a_1}) \supset B(a_2, \rho_{a_2}) \supset B(a_3, \rho_{a_3}) \supset\ldots
	.\] 
	Now we will check that $|\cdot |_x = \lim_{n \to \infty} |\cdot |_{B(a_i, \rho_i)}$ by verifying that both norms agree on $T- b$ for every $b \in K$. 
	 There are two cases. Either $|T - b|_x = \rho_b = \rho$ or $\rho_b \ge \rho_{a_i}$ for some $i$. 

	 If $ \rho_b = \rho$ then by \eqref{eq:proof_berk_class_1} we find $|b - a_i|_x \le \rho_n$ for all $i$.
	 So \[
		 |T - b|_{B(a_i, \rho_i)} = |(T - a_i) - (b_i - a_i)|_{B(a_i, \rho_i)} = \max(\rho_n, |b - a_i|) = \rho_n
	 .\]
	 So $\lim_{i \to \infty} |T- b|_{B(a_i, \rho_i)} = \rho = |T - b|$. 

	 If $\rho_b > \rho$ then for $n \gg 0$ we have $\rho_b > \rho_a$ and thus $|b - a|_x \le \rho_b$, thus \[
		 |T - b|_{B(a_i, \rho_i)} = |(T - a_i) - (b -a_i)|_{B(a_i, \rho_i)} = \max (\rho_n, |b - a_i|) = |b - a_i| = |T - b|_x
	 .\]   
	 Taking the limit yields $\lim_{i \to \infty} |T- b| _{B(a_i, \rho_{i})} = |T - b|_x$.
\end{proof}

Using the Berkovich classification theorem, the properties about ultrametric spaces from \cref{sec:ultrametric_spaces} and a little imagination we can see that the $\mathcal{M} (K[T])$ looks something like \cref{fig:affine_line}. 
Points of type 1-4 are highlighted in green. Note that the closed points of the curve $\spec K[T]$ are contained in  $\aff_K^{1,\text{an}}$ as exactly the type 1 points. 

\begin{figure}[h]
    \centering
    \incfig{berkovich-affine-line}
	\caption{The Berkovich affine line $\aff_K^{1, \text{an}} = \mathcal{M} (K[T])$, with the line $\ell_0$ from \cref{eq:line_in_A_1}. 
	Examples of points of type 1-4 are highlighted in green.}
	\label{fig:affine_line}
\end{figure}
Type 4 points are not rare. It might be counter intuitive that in a complete ring there exists a decreasing sequence of closed disks that has empty intersection. 
But in practice almost all fields have type  4 points. 
The trick is that the radii of these disks do not converge to $0$. 
\begin{example}[type 4 point]\label{ex:type4point}
	Let $K = (\Q_p[\sqrt[2]{p} ,\sqrt[3]{p}, \sqrt[4]{p},\ldots])^{\wedge}  $ for some prime $p$. 
	Define 
	\begin{align*}
		a_n &= \sum_{i = 1}^{n} p^{-\frac{1}{n}}\\
		r_n &= p^{\frac{1}{n + 1}} \\
		B_n &= B(a_n, r_n) 
.\end{align*}
Then $|a_n - a_{n + 1}| = r_n$ and $r_{n + 1} \le r_n$ so the $B_{n + 1} \subset B_n$ and the sequence $(B_n)_n$ is decreasing. 
Now we have to show that the intersection $B = \bigcap_{i = 1} ^{\infty} B_n$ is non-empty. 

Let $x \in \C_p$. Then $x = \sum_{i = 0}^{m} b_i p^{c_i}$ where $m$ is either finite or $m = \infty, |b_i| = 1$ and $c_i$ is an increasing sequence such that $c_i \to \infty $. Suppose that  $x \in B$, then $x \in B_n$ for every $n$, and thus we see that  $b_i = p^{\frac{-1}{i}}$ for any $i = 1, \ldots, n$.  
This is true for any $n $, so in particular we find that $b_i = p^{-\frac{1}{i}}$ for any $i$ .
But this contradicts that $b_i$ is unbounded. 
\todo{do we need to look at this again?}

This example is not algebraically closed.  
The algebraically closed field $\C_p$ also has type 4 points, but the proof is a bit more subtle. See \cite[][sec. 3.4]{robertCoursePadicAnalysis2000}.
\end{example}

\subsection{The Berkovich affine line when $K$ is discretely valued.} \label{sec:the_berkovich_affine_line_when_k_is_discretely_valued.}
The Berkovich classification theorem unfortunately requires that $K$ is algebraically closed. 
However, if $K$ is discretely valued there is a trick, which is to pass to the field $K' = \hat{\overline{K}}$. 
There is an inclusion $K[T] \subset  K'[T]$. 
So any norm on $K'[T]$ extending the norm on $K'$, restricts to a norm on $K[T]$. 
One can easily sees that the obtained norm on $K[T]$ extends the norm on  $K$, as the norm on $K'$ extends the norm one $K$. 
This means that there is a map $\mathcal{M} \left(K'[T]\right) \to \mathcal{M} (K[T])$. 
It turns out that this map is surjective.\question{Is er een snelle mannier om te zien dat deze map surjectief is?} 
This gives the following description of $\mathcal{M} (K[T])$.

\begin{proposition}[Berkovich classification when $K$ is discretely valued.]
	Let $K$ be discretely valued. 
	Let $|\cdot |$ be a norm on $K[t]$ extending the norm on $K$. 
	Then $|\cdot |$ is one of the following.
	\begin{itemize}
		\item  Let $D$ be a disk in $\hat{\overline{K}}$ with center $a$ and radious $r  \ge 0$. 
			Then \[
				|f| = \sup_{z \in D} |f(z)|_{\hat{\overline{K}}} = \max_i a_i r^{i}
			\] 
			where $f$ is any element of $K[T]$ with taylor expansion $f = \sum_{i = 0}^{n} a_i x_i$ around $a$. 
		\item $|\cdot |$ is the limit of norms described above given by a decreasing sequence of disks in $\hat{\overline{K}}$ with empty intersection. 
	\end{itemize}
	Two of these points are the same if the disks are in the same galois orbit of $\gal(\hat{\overline{K}} / K) \act \hat{\overline{K}}$.  
\end{proposition}

This does not really change our intuitive understanding of $\aff_{K}^{1}$ or the picture \cref{fig:affine_line}. 
The type 2 points are still the branch points, type 1 and 4 points are still ``leafs''.
The main differences to keep in mind is that type 2 points can have a radius in $\sqrt{|K^{\times }|} := |\overline{K}^{\times }|$, not just $|K^{\times }|$ and that type $1$ points are not necessarily points in of $K$, but Galois orbits of $\hat{\overline{K}} / K$. 




\section{Types of points in analytic curves} \label{sec:types_of_points_in_analytic_curves}
There are other ways of recognizing type 1-4 points, which will generalize better when we consider Berkovich spectra of general curves.
One is topological and the other one is purely algebraic. 

\begin{proposition}
	Let $x \in \aff_K^{1,\text{an}}$. 
	Let $C = \pi_0(\aff_K^{1, \an}\setminus \{ x\} )$ be the set of connected components of the punctured affine line. 
	\begin{itemize}
		\item If $x$ is of type 1  or 4 then $C$ has only one element.
		\item If $x$ is of type 2 then there is a natural morphism $C \simeq k \cup \{\infty\} $. Recall that $k = \tilde K = K^{o} / K^{oo} $. 
			So $x$ is an branch point where infinitely many branches connect. 
		\item If $x = |\cdot |_{B(a, r)}$ is of type 3 then $|C| = 2$. One component consists of all disks contained in $B(a, r)$, (and type 4 points corresponding to decreasing sequences of disks in $B(a,r)$) and the other one are all the leftover points. 
	\end{itemize}
\end{proposition}
\begin{remark}
	Note that if  $X$ is some other analytic curve, containing a loop and $x$ is a point on that loop, then after removing $x$, the spaces $X \setminus \{x\} $ is still connected. 
	So it actually better to think about the number of \emph{tangent directions} at $x$. 
	One way to make the notion of tangent directions precise is to consider each direction as an equivalence class of paths originating at $x$.
\end{remark}
\begin{figure}[ht]
    \centering
    \incfig{type-2-and-3-points}
    \caption{Recognizing type 2 and type 3 points by connected components/tangent directions.}
    \label{fig:type2_type3}
\end{figure}
\todo{if there is time left, remake \cref{fig:type2_type3}}


The algebraic way to classify the points of $\aff_K^{1,\text{an}}$ is by using Abhyankar's inequality. 
It is a sort of trancendental analogue of \cref{prop:balancing_valuations}.
\begin{definition}
	Let $K \subset L$ be a non-archimedean field extension. 
	Then we define \[
		E_{ L / K} = \dim \frac{|L^{\times }|}{|K^{\times }|} \otimes_\Z \Q, \qquad F_{L / K} = \trdeg(\tilde L / \tilde K)
	.\] 
\end{definition}

\begin{theorem}[Abhyankar's inequality]
	Let $K \subset H \subset L$ be non-archimedean rings extending each others norm. Assume that $L$ is algebraic over $\hat{H}$ and that $H$ is of transcendence degree $n$ over $K$. Then \[
	E_{L / K} + F_{L / K}  \le n
	.\] 
\end{theorem}

If $x \in \aff_K^{1, \text{an}}$ is a norm (not a seminorm) then its residue field $\mathcal{H} (x)$ is a completion on $K(T)$ with respect to the norm induced by $x$. As $K(T)$ is of transcendence degree 1 Abhyankar's inequality states that \[
	E_{\mathcal{H} (x) / K} + F_{\mathcal{H} (x) / K} \le 1
.\] 
This also allows us to classify points, because either both $E_{\mathcal{H} (x) / K}, F_{\mathcal{H} (x) / K}$ are zero, or exactly one is $1$. 
\begin{proposition}[2.3.3.3 in \cite{temkinIntroductionBerkovichAnalytic2010}]\label{def:type_points_curve}
	Let $x \in \aff_K^{1, \text{an}}$, then 
	\begin{itemize}
		\item $x$ is of type 1 if $\mathcal{H} (x) = K$
		\item $x$ is of type 2 if $F_{\mathcal{H} (x) / K} = 1$ and $E_{\mathcal{H} (x) / K} = 0$. 
		\item $x$ is of type 3 if $F_{\mathcal{H} (x) / K} = 0$ and $E_{\mathcal{H} (x) / K} = 1$.
		\item $x$ is of type 4 if $\mathcal{H} (x) \subsetneq K$ and  $F_{\mathcal{H} (x) / K} = E_{\mathcal{H} (x) / K} = 0$.
	\end{itemize}
\end{proposition}

\begin{remark}
	From now on we will use the classification of points from \cref{def:type_points_curve} as the definition of type 1-4 points.
\end{remark}



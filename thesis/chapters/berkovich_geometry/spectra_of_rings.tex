We will first work with a preliminary definition of the Berkovich spectrum of a ring, which for now is just a topological space. 
It will serve to build intuition for Berkovich spaces by exploring the Berkovich spectrum of two rings $\Z$ and $K[T]$.
\medskip 

Recall \cref{rem:ker_norm_ideal} which stated that the kernel of (multiplicative) seminorm is a (prime) ideal. 
In some sense seminorms are analogous to ideals and multiplicative seminorms are analogous to prime ideals. 
This is will be the basis of Berkovich spectra of rings.
Like the ordinary $\spec$ functor we can define the space of all multiplicative norms on a ring. 
Unfortunately we can't just copy all constructions we know for $\spec$. 
For example, we can't define a Zarisky topology on this set. 
In fact we will need two different topologies on this space for the theory to work. 
And for curves we will even add a third in \cref{chap:weight_functions}. 
Don't worry. This is a feature, not a bug! 
One topology will have all the nice properties we can wish for: Hausdorff, (locally) compact, path-connected, locally contractible, \ldots while the other one is great for sheaf theory. 

\begin{definition}
	Let $A$ be a ring. We define the \emph{Berkovich spectrum of $A$} to be \[
		\mathcal{M} (A) = \{x: A \to \R^{+} \st \text{multiplicative seminorm on } A\} 
	.\] 
	This comes equipped with the coarsest topology such that for every $f \in A$ the map $\mathcal{M} (A) \to [0, \infty) : \|\cdot \|_x \mapsto \|f\|_x $ is continuous. 
\end{definition}
\nomenclature[MA]{$\mathcal{M} (A)$}{The Berkovich spectrum of $A$}
	We will almost always deal with a relative case where $A$ is a $k$-algebra over some real valued field $K$. 
\begin{definition}
	Let $A$ be a $K$-algebra, then the \emph{Berkovich spectrum of $A$}
	\[
		\mathcal{M} (A) = \{x : A \to \R^{ + }  \mid \text{multiplicative seminorm on }A \text{ extending the norm on } K\} 
	.\] 
	Again $\mathcal{M} (A)$ comes with the coarsest topology making all maps $\mathcal{M} (A) \to [0, \infty): x \mapsto \|f\|_x, \forall f \in A$ continuous. 
\end{definition}
We think of $\mathcal{M} (A)$ as a geometric space and thus its elements are points, usually denoted by a letter, like $x$. 
For an element $f \in A$ we write $|f|_x$ or $x(f)$ for  $f$ evaluated in the norm $x$. 

\begin{remark}
	$\mathcal{M}$ is not only a map, but it is also a contravariant functor. 
	Let $A, B$ be two rings and $f: A \to B$ a morphism. 
	Then a seminorm $x \in \mathcal{M} (B)$ turns into a seminorm on $\mathcal{M} (A)$ by defining $\mathcal{M} (f)(x): A \to \R^{ +}: a \mapsto |f(a)|_x$.


This is analogous to the $\spec$ functor and there even is a natural transformation $\mathcal{M} \implies \spec$ defined by $\mathcal{M} (A) \to \spec A: |\cdot |_x \mapsto \ker x$. This is well defined by \cref{rem:ker_norm_ideal}. 
\end{remark}

There also is a generalization of the residue field at a prime ideal. 

\begin{definition}\label{def:completed_residue_field}
	Let $A$ be a ring (possibly over $K$) and $x \in \mathcal{M} (A)$ a point in its Berkovich spectrum. 
	Then the \emph{completed residue field at $x$} is defined as \[
		\mathcal{H} (x) = (\ff(A / \ker x), |\cdot |_x)^{\wedge}
	.\] 
	This is the completion of $\ff(A / \ker x)$ with respect to the norm induced by $x$. 
	It comes with a induced morphism $\chi_x: A \to \mathcal{H} (x)$.
\end{definition}
\nomenclature[Hx]{$\mathcal{H} (x)$}{The completed residue field at $x$}

Like $\spec$, the points in $\mathcal{M} $ can be described as \emph{characters}, i.e.\ maps $\chi: A \to L$ (over $K$) where $L$ is a complete valued field subject to the condition that $\chi_1: A \to L_1, \chi_2: A \to L_2$ are equivalent if there exists another map to a complete valued field $\chi: A \to L$  and maps $L \to L_1, L \to L_2$ extending the norm on $L$ such that the diagram 
\[
\begin{tikzcd}
	& A \ar{dl}[']{\chi_1} \ar{dr}{\chi_2} \dar{\chi} \\
	L_1 & L \lar \rar & L_2
\end{tikzcd}
\] 
commutes. 
For any point $x$ the map $\chi_x: A \to \mathcal{H} (x)$ gives a minimal representative for the character corresponding to the point $x$. 


\begin{remark}
	Later we will consider other topologies on $X\an$, which will be necessary to turn $X\an $ into a ringed space. 
	These topologies will not be topologies like we know it. They will be Grothendieck topologies, which unlike ordinary topologies restrict the notion of a cover. 
\end{remark}

\begin{proposition}\label{prop:spec_ring_haussdorf}
	Let $R$ be a ring then $\mathcal{M} (R)$ is Hausdorff. 
\end{proposition}
\begin{proof}
	Let $x, y \in \mathcal{M} (R)$ be two different points. 
	Then there is some $f \in R$ such that $f(x) \ne f(y)$. 
	So there are opens $U_x, U_y \subset  [0, \infty)$ that separate $f(x), f(y)$.
	Recall that $\phi: \mathcal{M} (A) \to [0, \infty): x \mapsto f(x)$ is continuous by definition of the topology on $\mathcal{M} (A)$. 
	So $\phi^{-1}(U_x)$ and $\phi^{-1}(U_y)$ are opens in $\mathcal{M} (R)$ separating $x,y$. 
\end{proof}


\subsection{The Berkovich spectrum of $\Z$} \label{sec:the_berkovich_spectrum_of_z}
Classifying all norms on $\Z$ was already done by Ostrowski.
\begin{theorem}[Ostrowski]\label{thm:ostrowksi}
	Any multiplicative norm on $\Z$ is either
	\begin{itemize}
		\item \emph{the trivial norm} \[
				|\cdot |_0: n \mapsto \begin{cases}
					1 & n \ne 0 \\
					0 & n = 0
				\end{cases}
			\]
		\item an \emph{archimedean norm} \[
				|\cdot |_{\infty, \epsilon}: n \mapsto |n|^{\epsilon},
			\]
			for $0 <  \epsilon \le 1$
		\item a \emph{$p$-adic norm} \[
				|\cdot |_{p, \epsilon}: t \mapsto \epsilon ^{- v_p(t)},
			\]
			for a prime number $p$ and $0 < \epsilon < \infty$. 
		\item a $p$-adic trivial seminorm 
			\[
			|\cdot |_{p, \infty}: n \mapsto \begin{cases}
				1 & p \nmid n \\
				0 & p \mid n
			\end{cases}
			,\] 
			for $p$ prime. 
	\end{itemize}
\end{theorem}
\begin{proof}
	See \cite[thm.\ 1, p.\ 3]{koblitzPadicNumbersPadic1984}
\end{proof}
So for every $p \in \{\text{primes}\} \cup \{\infty\} $ we find a continuous family of norms \[
	\epsilon \mapsto|\cdot |_{p, \epsilon}
\] 
for $\epsilon \in [0, 1]$ if $p = \infty$ and else $\epsilon \in [0, \infty]$. 
When $\epsilon = 0$ these all give the trivial norm. 
So these rays glue together at the point $|\cdot |_0$ of trivial norm.
Hence the Berkovich space $\mathcal{M} (\Z)$ looks like \cref{fig:berkovich-space-of-z}.
We have to be careful with the topology here.
While each of branches is homeomorphic to a closed interval, they are not glued together along an open.
So the topology is not uniquely determined by the topology of each branch separately.
In fact, any open containing the point $|\cdot |_0$ has to contain all but finitely many of the branches.

\begin{figure}[h]
    \centering
    \incfig{berkovich-space-of-z}
    \caption{The Berkovich spectrum of $\Z$}
    \label{fig:berkovich-space-of-z}
\end{figure}


It is a good exercise to verify that 
\begin{itemize}
	\item  $\mathcal{H}(|\cdot |_0) = \Q$ 
	\item $\mathcal{H} (|\cdot |_{\infty, \epsilon}) = \R$ for $\epsilon \in (0, 1]$
	\item $\mathcal{H} (|\cdot |_{p, \epsilon}) = \Q_p$ for $\epsilon \in (0, \infty)$
	\item $\mathcal{H} (|\cdot |_{p, \infty}) = \F_p$ 
\end{itemize}
While it may seem like the completed residue field of different norms can be the same, it is actually not as they will be equipped with different norms.  

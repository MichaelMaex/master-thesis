We might ask ourselves what a good notion for a minimal skeleton might be. 
A wishlist of good properties for such a skeleton $\sk(X)$ is the following. 
\begin{itemize}
	\item The minimal skeleton needs to capture all information about the ``shape'' of $X\an$. So the inclusion $\sk(X) \into X\an$ to be a homotopy equivalence and it has to contain all all divisorial points whose associated minimal component is not rational.
		\todo{explain better why this is important}
	\item The $\sk(X)$ is cannonical. 
	\item The $\sk(X)$ is well behaved under (nice) base change. 
\end{itemize}
For a curve $C$ of postive genus there is a canonical minimal skeleton $\sk(\mathscr C_\text{min} )$, with $\mathscr C_\text{min} $ the minimal snc-model.
However, this does not generalize to higher dimensions as not every variety has a minimal snc-model. 
The $\sk(\mathscr C_\text{min})$ is also not functorial under base change. 

In \cite{mustataWeightFunctionsNonArchimedean2015} Mustaţă and Nicaise construct what they call the \emph{essential skeleton} for any variety with semi-ample cannonical bundle.
In particular every curve of non-zero genus. 
The essential skeleton is cannonical, contains the shape of $X\an$ in the sense of the first wish and it pullsback under tame base-change.  
The main tool in defining it are weight functions. 

\subsection{Weight functions} \label{sec:weight_functions}
Given a variety $X$, a weight function is a piecewise $\Z$-affine linear function corresponding to a rational section $\omega$ of the cannonical linebundle. 
They are important to study skeleta because if $\omega$ is a regular section then the associated weight function $\wt_\omega$ is strictly increasing away from every skeleton associated to a snc-model. 
So the minimal locus $\minloc \wt_\omega$ has to be contained in every skeleton of $X$. 
This gives us a way of identyfing pieces of a skeleton. 
Taking all these pieces together is the essential skeleton. 
We will only work in the context of curves. 

\begin{definition}\label{def:log_cannonical_bundle}
	Let  $\mathscr C$ be a normal proper model of a curve $C $. 
	We define the \emph{log-cannonical bundle of $C$} to be \[
		\Omega_{\mathscr C}^{\text{log}}  = \Omega_{\mathscr C / R}^{\text{can}}(\mathscr C_{s, \text{red}})
	,\] 
	where $\Omega_{\mathscr C} ^{\text{can}}$ is the cannonical bundle of $\mathscr C$ and $\mathscr C_{s, \text{red}}$ is the divisor on $\mathscr C$ that contains every irreducible component of $\mathscr C_s$ with multiplicity $1$.  
\end{definition}

Note that the restiction of $\Omega_{\mathscr C}^{\log}$ to the generic fibre is the cannonical bundle on $C$. 

\begin{remark}
	The log-cannonical bundle is actually a the analogue of the cannonincal bundle for log schemes. 
	In this case the the $\mathscr C$ has a log structure induced by the special fibre. 
	But for our purposes we can work with the explicit definition above. 
\end{remark}
\begin{definition}
	Let $C $ be a curve.
	The $m$-pluricannonical line bundle is the $n$-th tensor power of the ordinary cannonical linebundle, $\Omega_{C / K}^{\otimes m }$. 
	A \emph{$m$-pluricannonical form}  is a non-zero global section $\Omega_{C / K}^{\otimes m }$
\end{definition}
\begin{notation}
	Let $D$ be a cartier divisor on a scheme $X$, and let $F$ be a prime divisor on $D$. 
	Then we write $\mult_{F}(D)$ is the coefficient of $F$ in $D$. 
\end{notation}

We first define the weight function on a divisorial point and then use a sort of expansion by continuity to define it on the whole of $C\an$. 
\begin{definition}\label{def:weight_function_divisorial_point}
	Let $C$ and $\omega$ be rational section of $\Omega_{C / K}^{\otimes m}$.
	Let $x$ be a divisorial point in $C\an$ and $\mathscr C$ be a regular model  with a irreducible component $F$ of multplicity $N$ corresponding to $x$. 
	Let $\omega'$ be the unique extension of $\omega$ to a rational section of $(\Omega_{\mathscr C} ^{\text{log}})^{\otimes m}$, then we define the \emph{weight of $\omega$ at $x$} to be \[
		\wt_\omega(x) = \frac{\mult_F(\divisor(\omega'))}{N}
	.\] 
\end{definition}

\begin{remark}
	If one doens't like to work with the log-cannonical bundle, we can use an alternative definition that puts a minor error term in the definition of the weight. 
	Let $\omega"$ be the extension of $\omega$ to $(\Omega_{\mathscr C / R}^{\text{can}})^{\otimes m}$. Then \[
		\wt_{\omega} = \frac{\mult_F(\divisor (\omega")) + m}{N}
	.\] 
\end{remark}
One can prove that this definition is indepent of the model $\mathscr C$ chosen. 
We extend the weight function to $\mathbb{H}(C)$ as follows 
\begin{definition}\label{def:weight_function}
	Let $C$ be a curve and $\omega$ a rational section of $\Omega_{C / K}^{\otimes m}$. 
	Then we define the \emph{weight function} $\wt_{\omega}: \mathbb{H}(C) \to \R$ to be the unique such that 
	\begin{itemize}
		\item $\wt_{\omega}$ is continuous with respect to the metric topology on $\mathbb{H}(C)$, \question{is this truely equivalent to whats written in Baker Nicaise?}
		\item on divisorial points $x$ the $\wt_{\omega}(x)$ agrees with \cref{def:weight_function_divisorial_point}.
	\end{itemize}
\end{definition}

The weight function can even be extended to the entirety of $C\an$, but then it may take $\pm \infty$ as values on the type 1 points. See \cite[§4.5.4]{mustataWeightFunctionsNonArchimedean2015}

\begin{proposition}
	Let $\mathscr C$ be snc-model of a curve $C$ and $\omega$ a $m$-pluricannnical rational section. 
	Then $\wt_\omega \circ \sk_{\mathscr C}: |\Delta(\mathscr C)| \to \R$ is linear on the edges. 
\end{proposition}

Various tools and techniques for computing weight functions on curves can be found in \cite{bakerWeightFunctionsBerkovich2016}. 
These will be useful in \cref{chap:a_berkovich_approach_to_classifying_elliptic_curves}. 

\subsection{The essential skeleton $\sk(C)$}\label{sec:the_essential_skeleton_sk_c$}

Weight functions are interesting for studying models because when $\omega$ is a $m$-pluricannonical form, i.e.\ a non-zero global section, then $\wt_\omega$ is strictly increasing away from any skeleton associated to a snc-model. 
This is made precise in the following theorem. 
\begin{proposition}\label{prop:weight_function_increase}
	Let $\mathscr C$ be a snc-model of a curve $C$ and $\omega$ be $m$-pluricannical form on $C$ (i.e. a global section).
	Let $\rho_{\mathscr C}: C\an \to \sk(\mathscr C)$ be the retract from \cref{prop:retract_analytification_skeleton}. 
	Let $x \in C\an $ be any point. 
	Then \[
		\wt_\omega(x) \ge \wt_\omega(\rho_{\mathscr C}(x))
	,\] 
	with equality if and only if $x \in \sk(\mathscr C)$. 
\end{proposition}
\begin{proof}
	See \cite[prop.\ 4.4.4]{mustataWeightFunctionsNonArchimedean2015}. 
\end{proof}

\begin{definition}[Kontsevich–Soibelman skeleton]\label{def:KS_skeleton}
	Let $\mathscr C $ be curve and $\omega$ a rational pluricannonical form. 
	Then we define the \emph{Kontsevich-Soibelman skeleton of $\omega$} to be \[
		\sk(C, \omega) = \minloc \wt_{\omega}
	.\]  
	Here $\minloc f$ (the minimal locus) is the set where the function $f$ attians its infiumum.
\end{definition}
\begin{lemma}
	Let $\omega$ be a $m$-pluricannonical form and $\mathscr C$ a snc-model of $C$. 
	Then $\sk(C, \omega) \subset  \sk(\mathscr C)$
\end{lemma}
\begin{proof}
	This follows from \cref{prop:weight_function_increase}, and the definition of the Kontsevich-Soibelman skeleton. 
\end{proof}
Calling the Kontsevich-Soibelman skeleton a skeleton might be slightly misleading as it usually does not have the same homotopy type as $C\an$. 
Note that for general $\omega$ it might even be that $\sk(C, \omega)$ is empty. 
But if $\omega$ is a form then $\sk(C, \omega)$ may be computed on $\sk (\mathscr C)$ for any snc-model $C$ by \cref{prop:weight_function_increase}. 
As $\sk(\mathscr C)$ is compact we know that $\wt_{\omega}$ attains its minimum on $\sk(\mathscr C)$ and thus $\sk(C, \omega)$ is non-empty. 
Still $\sk(C, \omega)$ might only be a small part $\sk(\mathscr C)$. 

The solution to this is to take the union of many Kontsevich-Soibelman skeleton.
\begin{definition}[essential skeleton]
	Let $C$ be a curve. 
	Then the \emph{essential skeleton of $C$} is \[
		\sk(C) = \bigcup_{\omega} \sk(C, \omega)
	\] 
	where $\omega$ runs over all $m$-pluricanonical forms for all $m > 1$. 
\end{definition}

\begin{remark}
	While weight functions are defined for \emph{rational sections} of the pluricannonical linebunde $\Omega_{C / K}^{\otimes m}$, when we define the essential skeleton we consider the weight functiosn where $\omega$ is a \emph{form}, i.e. a global section of $\Omega_{C / K} ^{\otimes m}$. 
	So it is important to make the distinction between \emph{rational sections} and \emph{forms}.
\end{remark}

It is not at all clear from the definition that the essential skeleton actually captures the shape of $C\an$, i.e. has the same homotopy type and contains all the points of non-zero genus. 
If $C = \pro^{1}_K$ then $\sk(C)$ is emptry because there are no pluricannonical sections. Indeed the $m$-th pluricannonincal linebundle is isomorphic to  $\mathcal{O}(-2m)$, which does not have global sections. 
But for curves of non-zero genus he cannonincal linebundle is is ample and thus $\sk(C)$ is nonzero. 

If $C$ is a curve of genus $g(C) > 0$ then there is a way of obtaining $\sk(C)$ from $\sk(\mathscr C_\text{min} )$, where $\mathscr C _\text{min}$ is the minimal regular model and removing some superfluous tails or rational curves.  

\begin{definition}
	Let $\mathscr C$ be a snc-model for a curve $C$. 
	An irreducible component $F$ of $\mathscr C_s$ (vertex in $\Delta(\mathscr C)$) is called \emph{inessesential}, if in the dual graph it belongs to a path of vertices $v_1, \ldots, v_n$ of genus $0$ with the valency of $v(v_1) = 1$, $v(v_{2}) = 2, \ldots, v(v_{n-1}) = 2, v(v_n) > 2$ and $F \ne v_n$. 
	Conversely a component/vertex if called \emph{essential} if it is not inessential. 
\end{definition}
\todo[inline]{figure with examples of inessential components}

\begin{theorem}
	Let $C$ be a curve with genus $g(C) > 0$ and let $\mathscr C_\text{min} $ be the minimal scn-model of $C$. 
	Let $\Delta'$ be the complete subgraph of $\Delta(\mathscr C_\text{min} )$ with the essential vertices. 
	Then the essential skeleton is the image of $|\Delta'|$\[
		\sk(C) = \sk_{\mathscr C_\text{min} }(|\Delta'|)
	.\] 
\end{theorem}
\begin{proof}
	The proof is insightful as to obtain this one needs to understand how handle weight functions, but unfortunately it is too long to put in this thesis. 
	The interested reader in encouraged to look at \cite[thm 3.3.13]{bakerWeightFunctionsBerkovich2016}. 
\end{proof}
\todo[inline]{examples}


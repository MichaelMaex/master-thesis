\todo{ramble on why we needed properness and how to fix it}

As seen in \cref{prop:intersection_pairing_proper_model} properness is need to have a good intersection theory on surfaces. 
If not moving a divisor, to a linearly equivant one, might cause one of the intersection points to 
This is a problem when we want to have a intersection theory on a model or more generally a fibered surface. 
Let $\mathscr C  / R $ be a proper regular model. 
The $\mathscr C$ is proper over $\spec R$, but $\spec R$ itself is not proper. 
So we might, and do run into the problem that the intersection pairing is not invariant under linear equivalence. Consider the following example.

\begin{example}
	Consider $\pro^{1}_R = \proj K[x, y]$ as a model of $\pro^{1}_K$. 
	Take $D = (x), E = (x-\pi y )$, and $E' = E + (\frac{x-y}{x-\pi y})= (x - y)$.
	Then  $D, E$ intersect in origin of special fiber, hence $D\cdot E \ge 1$. 
	But $D, E'$ do not have any common points, hence $D \cdot E' = 0$. 
\end{example}

The problem here is that the two divisors span across the two divisors. They are what we will later define to be horizontal divors. Intuitevely they lay in the direction where the scheme is not proper. 
However the special fibre, and every component in it is a proper divisor. 
This suggets a fix.
If we restrict our selves to the case where $D$ lies in the special fibre (is vertical), then we can compute $D\cdot E$ on the restriction do $D$ and once again obtain that the intersection pairing indenpent of linear equivalence. 

\begin{definition}\label{def:vertical_divisor}
	Let $\mathscr{C}$ a $2$-dimensional, regular, integral scheme proper and flat over $R$.  (i.e.\ a proper regular model of a curve). 
	Then $\divisor(\mathscr{C})_s$ is the set of divisors that are supported in the special fibre. 
	Such a divisor is called a \emph{vertical divisor}.
\end{definition}
\begin{definition}\label{def:horizontal_divisor}
	A divisor on $\mathscr{C} $ is \emph{horizontal} if it is the schemeatic closure of a divisor in the on the generic fibre in $\mathscr C$. 
\end{definition}
\begin{lemma}\label{lem:decomponsition_horizontal_vertical}
	Any divisor $D$ on $\mathscr{C}$ can be uniquely decomposed in a horizontal and vertical part $D =  D_\text{vert}  + D_\text{hor}$. 
	Moreover $D_\text{hor} $ is the schematic closure of $D|_{\mathscr C_\eta}$ in $\mathscr C$.
\end{lemma}
\begin{proof}
	It is sufficient to prove that every prime divisor $D$ is either vertical or horizontal.
	Suppose that $D$ is not vertical. 
	Then by \cite[prop. 8.3.4]{liuAlgebraicGeometryArithmetic2002} $D = \overline{\{x\} }$ for some point $x \in \mathscr{C} _\eta$.
	\todo{should I write the proof out here, or is the citation enough?}
\end{proof}

\begin{definition}
	Let $\mathscr C$ a $2$-dimensional, regular, integral scheme proper and flat over $R$.  (i.e.\ a proper regular model of a curve). 
	Then we define a map
	\begin{align*}
		i: \divisor(\mathscr C) \times \divisor_s(\mathscr C) &\longrightarrow \Z \\
		(D, E) &\longmapsto \sum_{1 \le i \le r} n_i \deg \mathcal O_X(D)|_{\Gamma_i}
	,\end{align*}
	where $E = \sum_{i = 1}^{r} n_i \Gamma_i$ is the decomposition of $E$ in prime divisors. 
\end{definition}
	

\begin{theorem}\label{thm:defining_properties_intersection_pairing}
	The map $i: \divisor(\mathscr C) \times \divisor_s(\mathscr C) \rightarrow \Z $ defined above is the unique bilinear map satisfying
	\begin{enumerate}
		\item If $D \in \divisor(\mathscr{C} )$ and $E \in \divisor_s(X)$ are divisors with no common compononents then  \[
				(D\cdot E) = \sum_{x \in \mathscr{C}_\text{cl} }^{} i_x(D, E)[k(x):k]
			.\] 
		\item The restiction of the pairing to $\divisor _s (X) \times  \divisor_s(X)$ is symmetric.
		\item If $D, D' \in \divisor (\mathscr C)$ and $D \sim D'$ then $D\cdot E = D' \cdot E$. 
		\item If $0 < E \le X_s$ then $D \cdot E = \deg \mathcal{O}_X(D) |_E$. 
	\end{enumerate}
\end{theorem}
\begin{proof}
	Uniqueness follows from the bilinearity and property 4.
	So it suffices to show that $i$ satsifies the properties. 
	\begin{enumerate}
		\item $i$ is bilinear and so is $i_x$. 
			So we may assume that $E, D$ are different effective prime divisors.
			We have \[
				\deg \mathcal{O}_X(D)|_E = \deg [D|_E] = \sum i_x (D, E) [k(x):k]
			.\] 
		\item By bilinearity it is enough to show this in the case when $D, E$ are prime divisors. 
			Then either $D = E$ in which case it's trivial. 
			Or $D \ne E$, in which case it follows from 1.
		\item By linearity we may assume that $E$ is prime. 
			Then $\mathcal{O}_X(D) \simeq \mathcal{O}_X(D')$ hence
			\[
				D \cdot E = \deg \mathcal{O}_X(D)|_E = \deg \mathcal{O}_X(D') |_E = D' \cdot E
			.\] 
		\item \todo{I don't understand this proof}
	\end{enumerate}
\end{proof}

\begin{proposition}
	Let $\mathscr C$ over $R$ as before. 
	For any $E \in \divisor_s$ we have $E\cdot \mathscr C_s = 0$ where we consider $\mathscr C_s$ as a divisor on $\mathscr C$. 
\end{proposition}
\begin{proof}
	Note that $\mathscr C_s$ is principal as it is equal to $(\pi)$. 
	Hence $\mathscr C_s \sim 0$ and by \cref{thm:defining_properties_intersection_pairing} part 2, 3 we find \[
	E\cdot C_s = C_s \cdot E = 0 \cdot E = 0
	.\] 
\end{proof}
This seems like a very minor proposition, but in practice it is very useful as it gives an easy way to compute self intersections. 
\begin{corollary}
	Let $\mathscr C$ as before. 
	Let $\mathscr C_s = \sum_{i =1}^{r} n_i \Gamma_i$ be the decomposition into prime divisors. 
	Then for any $i$ we find \[
		\Gamma^2_i := \Gamma -\frac{1}{n_i} \sum_{j \ne i} \Gamma_i \cdot \Gamma_j
	.\] 
\end{corollary}
\begin{proof}
	By the previous proposition we have \[
	 0 = \Gamma_i \cdot \mathscr C_s = \sum_{j = 1}^{r} n_i \Gamma_i \cdot \Gamma_j
	.\] 
	Solving this equation to $\Gamma_i\cdot \Gamma_i$ yields the desired result. 
\end{proof}

\begin{proposition}\label{prop:intersection_horizontal_point_degree}
	Let $x \in \mathscr C_\eta$ a closed point. 
	Then \[
		\overline{\{x\} }\cdot \mathscr C_s = [K(P): K]
	,\] 
	where $\overline{\{x\} }$ is the Zariski closure of $\{x\} $ in $\mathscr C$ with the reduced structure. 
	\todo{what the hell is $K(P)$?}
\end{proposition}
\begin{proof}
	See \cite[prop.\ 9.1.30]{liuAlgebraicGeometryArithmetic2002}
\end{proof}


\begin{corollary}
	Let $x \in \mathscr C_\eta$ be a $K$-rational point and $D = \overline{\{P\} } $. 
	Then $D$ intersects $\mathscr C_s$ in a single point $p$, which is smooth in $\mathscr C_s$. 
	Thus $p$ lies on a unique irreducible component of $X_s$ which is of multiplicity $1$. 
\end{corollary}
\begin{proof}
	By \cref{prop:intersection_horizontal_point_degree} we obtain that \[
	\sum_{i = 1}^{r}  n_i D\cdot \Gamma_i = D \cdot X_s = 1
	,\] 
	where $\mathscr C_s = \sum_{i =1}^{r} n_i \Gamma_i$ is the decomposition in irreducible components. 
	So can be atmost one $i$ such that $D\cdot \Gamma_i \ne  0$ and $n_i = 1$. 
\end{proof}

Let $\omega_{\mathscr C / R} $ be the cannonical bundle of $\mathscr C$ (relative to $\spec R$).
We now turn to the question how we should compute $K_{\mathscr C / R} \cdot E$ where  $K_{\mathscr C / R}$ is a representative of $\omega_{\mathscr C / R}$ and $E \in \divisor_s \mathscr C$. 
Note that the result does not depend on the choice of $K_{\mathscr C / R}$ as any two choices are linearly independent. 

\begin{proposition}
	Let $\mathscr C$ be as above. 
	Then we have \[
		2p_a(\mathscr C_\eta) - 2 = K_{\mathscr C / R} \cdot \mathscr C_s  
	.\] 
\end{proposition}
\begin{proof}
	From \cref{thm:defining_properties_intersection_pairing} we see that \[
		K_{\mathscr C / R} = \deg \omega_{\mathscr C / R}|_{\mathscr C_s} = \deg \omega _{\mathscr C_s / k}
	.\] 
	The last equality is holds because the cannonical bundle pulls back under basechange. 
	We continue the computation \[
		\deg \omega_{\mathscr C_s / k} = - 2\chi (\mathcal{O}_{\mathscr C_s}) = -2 \chi (\mathcal{O}_{\mathscr C_\eta}) = 2p_a(\mathscr C_\eta) - 2
	.\] 
	The middle equality holds because $\mathscr C \to \spec R$ is flat. 
\end{proof}

\begin{theorem}[Adjunction formula]\label{thm:adjunction_formula}
	Let $\mathscr C$ be as above and $0 < E \le \mathscr C_s$, then \[
		\omega_{E / k} \simeq (\mathcal{O}_{\mathscr C}(E) \otimes \omega_{\mathscr C / R})|_E \quad \text{ and } \quad p_a(E) = 1 + \frac{1}{2}(E^2 + K_{\mathscr C / R}\cdot E)
	.\] 
\end{theorem}
\begin{proof}
	The first equality follows from taking the determinant of the second fundamental SES  \[
		0 \to \mathcal{O}(-E)|_E \to \Omega_{\mathscr C / R}|_E \to \Omega_{E / k}
	.\] 
	To show the second equality, note that \[
		p_a(E) = 1 - \chi(\mathcal{O}_E) = 1  + \frac{\deg \omega_{E / k}}{2} = 1 + \frac{1}{2}\left(E^2 + K_{\mathscr C / R} \cdot E\right)
	\] 
	where we used the first equality. 
\end{proof}

\todo[inline]{Should I cover section 9.2.1 of \cite{liuAlgebraicGeometryArithmetic2002} here?}

For the final part of this thesis it will be very important to understand how given a map $f: \mathscr X \to \mathscr Y$ intersection numbers behave under pushforward and pullback of divisors. 

\begin{theorem}[projection formula]
	Let $f:\mathscr X \to \mathscr Y$ be a dominant morphism of 2-dimensional, regular, integral schemes, proper and flat over $R$. 
	Let $C$ be a divisor on $\mathscr X$ and $D$ divisor on $\mathscr Y$. 
	Then the following holds.
	\begin{enumerate}
		\item If $f(\supp C)$ is finite then $C\cdot f^* D = 0$ 
		\item If either $C$ or $D$ is vertical then \begin{equation}\label{eq:projection_formula}
				C\cdot f^* D = f_* C \cdot D
			.\end{equation}
		\item The extension $K(X) / K(Y)$ is finite. 
			If $F$ is a vertical divisor on $\mathscr Y$ then $f^* F$ is also vertical and \[
				f^* F \cdot f^* D = [K(X): K(Y)] F \cdot D
			.\] 
	\end{enumerate}
\end{theorem}
\begin{proof}
	\begin{enumerate}
		\item Note that $C$ is necessarily vertical under this assumption. 
			By linearity we may assume that $C$ is prime and thus $f(C)$ is a single point, $y$. 
			Let $U$ be a neighboord of  $y$ on which $\mathcal{O}(D)$ is trivial. 
			Then $f^*(\mathcal{O}(D))$ on $f^{-1}(U)$ is trivial as well. 
			Hence $f^*(\mathcal{O}(D))|_E$ is trivial, from which the result follows.
		\item By linearity we may assume that $C$ is a prime divisor.
			If $f(C)$ is a point then by definition $f_*(C) = 0$. Then the equality follows from 1. 
			For the other case see \cite[thm.\ 9.2.12]{liuAlgebraicGeometryArithmetic2002}. 
			\todo{understand this proof}
		\item 
			We know that $f_* f^* F = [K(\mathscr X): K(\mathscr Y)] F$. Hence  \[
				f^* F \cdot  f^* D = f_* f^* F \cdot D = [K(\mathscr X) : K(\mathscr Y)] F \cdot  D
			.\] 

	\end{enumerate}
\end{proof}

\begin{remark}
	If $f: \mathscr X \to \mathscr Y$ is projective and surjective suppose that $f^{-1}D$ does not contain any components of $C$ then we even have the identity  \[
		f_*(C . f^*D) = (f_*C \cdot f^* D)
	.\] 
	See \cite[rem.\ 9.2.13]{liuAlgebraicGeometryArithmetic2002} for more details. 
\end{remark}

\begin{definition}
	Let $\mathscr X$ be a $R$-scheme. 
	Then we define 
	\begin{align*}
		\mathscr X_\eta &:= \mathscr X_K = \mathscr X \times_R \spec K \\
		\mathscr X_s &:= \mathscr X_k = \mathscr X \times_R \spec k 
	.\end{align*}
	We say that $\mathscr X_\eta$ is the \emph{generic fibre of $\mathscr X$} and that $\mathscr X_s$ is the \emph{special fibre of $X$}. 
\end{definition}
\begin{definition}
	Let $X$ be a finite type scheme over $K$.  
	A model of $X$ is a flat, separated, finite type $R$-scheme, $\mathscr X$ together with an isomorphism $i: \mathscr X_\eta \simeq X$.
	\todo{check the details of this definition}
\end{definition}
Alternatively we may consider $i$ as open immersion $X\into \mathscr X$. 
We will almost always omit specifying the isomorphism, as it is usually clear from the context. 
Depending on the context we will almost always work with specific types models like requiring $\mathscr X$ to be proper or regular or normal.

\begin{example}\label{ex:first_models}
	\begin{enumerate}
		\item Let $\pro^{1}_K = \proj(K[x, y])$ be the projective line. 
			Then $\mathscr P = \proj\left(\frac{R[x, y, z]}{(x\cdot z - \pi y^2)}\right)$ is a model of $\pro^{1}_K$. 
			The special fibre is $\proj\left( \frac{k[x, y, z]}{(xz)} \right) $ which consists of two irreducible components, each isomorphic to $\pro^{1}_k$.  
		\item Let $E$ be the elliptic curve given by the Weierstrass equation \[
				y^2 = x (x +1)(x-\pi)
		.\] 
		Then $\mathscr E = \proj \left( \frac{K[x, y, z]}{(zy^2 - x(x+z)(x-\pi z)} \right) $ is a model of $E$ and the special fibre is given by the Weierstrass equation $y^2 = x^2 (x+1)$, hence it is a node. 
	\end{enumerate}
	See \cref{fig:first_examples_of_models}.
\end{example}

\begin{figure}[h]
    \centering
    \incfig{first-examples-of-models}
    \caption{The models from \cref{ex:first_models}.}
    \label{fig:first_examples_of_models}
\end{figure}
\todo{figure out why this is rendering incorrectly}

\begin{definition}
	Let $i_1: X \into \mathscr X_1$, $i_2: X \into \mathscr X_2$ be two models of $X$. 
	Then a morphism of models of $X$ is a morphism of $R$-schemes $f: \mathscr X_1 \to \mathscr X_2$ such that \[
	\begin{tikzcd}
		\mathscr X_1 \ar{rr}{f} & & \mathscr X_2 \\
					& X \ar[hookrightarrow]{ul}{i_1}  \ar[hookrightarrow]{ur}{i_2}
	\end{tikzcd}
	\] 
	commutes.
\end{definition}

\begin{example}\label{ex:morphism_models_1}
	The most common way to construct morphisms of models is by blowing up closed subschemes in the special fibre. 
	Let $E, \mathscr E$ be the elliptic curve and model from \cref{ex:first_models}. 
	Suppose that $\ch k \ne 2$. 
	Let $\xi$ be the singular point of $\mathscr E _s$, with coordinates $(x, y,\pi) = (0,0,0)$. 
	Then  $\mathscr E' := \bl_\xi \mathscr E $ is another model of $E$ and the morphism $p:\mathscr E' \to \mathscr E $ is a morphism of models. 
	In this case one can show that the exceptional divisor of $\mathscr E' \to \mathscr E$ is a rational curve of multiplicity $1$ which intersects the strict tranform of $\mathscr E _s$ in two different points. 
	See \cref{fig:morphism_models_blowup_1}.
\end{example}

\begin{figure}[h]
    \centering
    \incfig{morphism-models-blowup-1}
    \caption{The morphism between models of an elliptic curve from \cref{ex:morphism_models_1}.}
    \label{fig:morphism_models_blowup_1}
\end{figure}

\todo[inline]{reduction map}


\todo{something on domination and contraction}

\todo{regular models}
\todo{minimal models}
\todo{what types and properties of morphisms do i need to introduce?}




\begin{definition}
	Let $\mathscr X$ be a $R$-scheme. 
	Then we define 
	\begin{align*}
		\mathscr X_\eta &:= \mathscr X_K = \mathscr X \times_R \spec K \\
		\mathscr X_s &:= \mathscr X_k = \mathscr X \times_R \spec k 
	.\end{align*}
	We say that $\mathscr X_\eta$ is the \emph{generic fiber of $\mathscr X$} and that $\mathscr X_s$ is the \emph{special fiber of $X$}. 
\end{definition}
\nomenclature[m]{$\mathscr X, \mathscr Y, \mathscr C, \mathscr E$}{Models of $X, Y, C, E$ respectively}
\nomenclature[mfg]{$\mathscr X_\eta$}{The generic fiber of a model}
\nomenclature[mfs]{$\mathscr X_s$}{The special fiber of a model}

\begin{definition}
	Let $X$ be a variety over $K$.  
	A model of $X$ is a flat, separated, finite type $R$-scheme, $\mathscr X$ together with an isomorphism $i: \mathscr X_\eta \simeq X$.
\end{definition}
Alternatively we may consider $i$ as open immersion $X\into \mathscr X$. 
We will almost always omit specifying the isomorphism, as it is usually clear from the context. 
Depending on the context we will almost always work with specific types models like requiring $\mathscr X$ to be proper or regular or normal.

\begin{example}\label{ex:first_models}
	\begin{enumerate}
		\item Let $\pro^{1}_K = \proj(K[x, y])$ be the projective line. 
			Then $\mathscr P = \proj\left(\frac{R[x, y, z]}{(x\cdot z - \pi y^2)}\right)$ is a model of $\pro^{1}_K$. 
			The special fiber is $\proj\left( \frac{k[x, y, z]}{(xz)} \right) $ which consists of two irreducible components, each isomorphic to $\pro^{1}_k$.  
		\item Suppose $\ch k \ne 2$. Let $E$ be the elliptic curve given by the Weierstrass equation \[
				y^2 = x (x -1)(x-\pi)
		.\] 
		Then $\mathscr E = \proj \left( \frac{R[x, y, z]}{(zy^2 - x(x0z)(x-\pi z)} \right) $ is a model of $E$ and the special fiber is given by the Weierstrass equation $y^2 = x^2 (x-1)$, hence it is a node. 
	\end{enumerate}
	See \cref{fig:first_examples_of_models}.
\end{example}

\nomenclature[kappa]{$\kappa(x)$}{The residue field at $x$ in a scheme $X$}
\begin{figure}[h]
    \centering
    \incfig{first-examples-of-models}
    \caption{The models from \cref{ex:first_models}.}
    \label{fig:first_examples_of_models}
\end{figure}

\begin{definition}
	Let $i_1: X \into \mathscr X_1$, $i_2: X \into \mathscr X_2$ be two models of $X$. 
	Then a morphism of models of $X$ is a morphism of $R$-schemes $f: \mathscr X_1 \to \mathscr X_2$ such that \[
	\begin{tikzcd}
		\mathscr X_1 \ar{rr}{f} & & \mathscr X_2 \\
					& X \ar[hookrightarrow]{ul}{i_1}  \ar[hookrightarrow]{ur}[']{i_2}
	\end{tikzcd}
	\] 
	commutes.
\end{definition}

\begin{example}\label{ex:morphism_models_1}
	The most common way to construct morphisms of models is by blowing up closed subschemes in the special fiber. 
	Let $E, \mathscr E$ be the elliptic curve and model from \cref{ex:first_models}. 
	Suppose that $\ch k \ne 2$. 
	Let $\xi$ be the singular point of $\mathscr E _s$, with coordinates $(x, y,\pi) = (0,0,0)$. 
	Then  $\mathscr E' := \bl_\xi \mathscr E $ is another model of $E$ and the morphism $p:\mathscr E' \to \mathscr E $ is a morphism of models. 
	In this case one can show that the exceptional divisor of $\mathscr E' \to \mathscr E$ is a rational curve of multiplicity $1$ which intersects the strict transform of $\mathscr E _s$ in two different points. 
	See \cref{fig:morphism_models_blowup_1}.
\end{example}
\nomenclature[bl]{$\bl_x (X)$}{The blowup of $X$ in $x$}

\begin{figure}[h]
    \centering
    \incfig{morphism-models-blowup-1}
    \caption{The morphism between models of an elliptic curve from \cref{ex:morphism_models_1}.}
    \label{fig:morphism_models_blowup_1}
\end{figure}

Morphisms of models are birational morphisms, as they are isomorphisms on the generic fiber. So if $\mathscr C_1, \mathscr C_2$ are models of $C$, then a morphism of models $f: \mathscr C_1 \to \mathscr C_2$ is necessarily unique, if it exists. 

\begin{definition}
	Let $\mathscr C_1, \mathscr C_2$ be models of $C$. 
	Then we say that $\mathscr C_1$ \emph{dominates} $\mathscr C_2$ if there is a surjective morphism of models $\mathscr C_1 \to \mathscr C_2$.
\end{definition}

If $\mathscr C_1 $ dominates $\mathscr C_2$, and both are proper then we can blowup a closed point in $\mathscr C_2$, where the map is not locally isomorphic, to add an one of the exceptional divisors of $\mathscr C_1$ to $\mathscr C_2$. 
Repeating this operation we eventually get the whole of $\mathscr C_1$. 
This leads to the following theorem. The details of this construction can be found in \cite[sec.\ 9.1.2]{liuAlgebraicGeometryArithmetic2002}.
\begin{theorem}[factorisation theorem]\label{thm:factorisation_theorem}
	Let $\mathscr C_1, \mathscr C_2$ be two proper models of a curve $C$ such that $f: \mathscr C_1 \to \mathscr C_2$ dominates. 
	Then $f$ can be decomposed into a sequence of blowups in closed points. 
\end{theorem}

Domination induces a partial order on the models of $\mathscr C$. 
\begin{definition}
	Let $C$ be a curve and $\mathscr C_1, \mathscr C_2$ be two models of $C$. 
	If $\mathscr C_1$ dominates $\mathscr C_2$ then we say $\mathscr C_2 \le \mathscr C_1$.
\end{definition}





\subsection{Weil versus Cartier divisors} \label{sec:weil_versus_cartier_divisors}

Recall the following proposition. 
\begin{proposition}\label{prop:weil_vs_cartier}
	Let $X$ be a regular, integral, Noetherian scheme. Then the groups of Weil and Cartier divisors are isomorphic, and so are the Weil and Cartier divisor classgroups. 
\end{proposition}
\begin{proof}
	This is \cite[prop.\ 7.2.16]{liuAlgebraicGeometryArithmetic2002}. 
\end{proof}

In the next sections we will almost always work with regular models of a curve over $K$, which satisfy the assumptions of \cref{prop:weil_vs_cartier}. So we may freely move between Cartier and Weil divisors. 

\subsection{The reduction map} \label{sec:the_reduction_map_models}
One can think about models as a way of taking looking at varieties modulo $\pi$. 
This is made precise by the reduction map. 
Let $ X\subset \pro^{n}_R$ is some projective variety variety over $R$ and $x \in X$ be a $K$-valued points. 
We may choose coordinate $[a_0:\ldots:a_n] = x$ such that all $a_0, \ldots, a_n \in R$ and at least one of them is not zero mod $\pi$.  
Then $[\overline{a_0}: \ldots :\overline{a_n}]$ defines a point in $X_s \subset  \pro^{n}_k$, which we call $\tilde x$ or the reduction of $x$.  
The idea of the reduction map is to give a more scheme theoretic version of this map. 

\begin{definition}\label{def:reduction_map_R_scheme}
	Let $X$ be a variety proper over $R$. 
	Then there is a map 
	\begin{align*}
		\red_X: X_{\eta}(K) &\longrightarrow X_{s}(k) \\
		x &\longmapsto \tilde x
	,\end{align*}
	which is defined as follows. 
	Let $x: \spec K \to X_\eta$ be $K$-rational point.
	Then by the valuative criterion of properness there is a unique map \[
	\begin{tikzcd}
		K \rar{x} \dar & X \dar \\
		R \rar \ar[dashed]{ur}{x'} & R
	\end{tikzcd}
	\] 
	$x': R \to X$. 
	The base change to $k$ gives a $k$-rational point on $X_s$, $\tilde x = x' \otimes_R k: k \to X_s $.
\end{definition}
\begin{remark}
Let $K$ be a complete field, and $x \in X_{\eta, \text{cl}}$ be any closed point in the special fiber.  Then by \cref{thm:norm_finite_field_ext} $\kappa(x)$ is valued field. 
So by the same procedure as in \cref{def:reduction_map_R_scheme} we may turn the map $\spec: \kappa(x) \to X_{\eta}$ to a map $\widetilde{\kappa(x)} \to X_s$ and extend the reduction map to all closed points \[
\red_X: X_{\eta, \text{cl}} \to X_{s, \text{cl}}
.\] 
In this case one can also show that $\red_X$ is surjective. 
\end{remark}

So for any proper model $\mathscr X$ of a proper $K$-variety $X$ (e.g.\ a proper model of a curve) there is a reduction map \[
	\red_{\mathscr X}: \mathscr X(K) \to \mathscr X(k)
.\] 
\nomenclature[red]{$\red_{\mathscr X}$}{The reduction map from $\mathscr X_\eta$ or $X\an$ to  $\mathscr X_s$}

\begin{remark}
	This reduction map is just a function between sets. 
	It is \emph{not} a morphism of schemes. 
\end{remark}

This chapter will be mostly based on \cite[chap.\ 8, 9, 10]{liuAlgebraicGeometryArithmetic2002}, with the difference that in instead of working over a arbitrary dedekind scheme $S$, I will work over $\spec R$, where $R$ is a discrete valuation ring with fraction field $K$ and residue field $k$. This is a dedekind scheme with only one closed point.  
Occasionaly \cite[appendix A]{hartshorneAlgebraicGeometry1977} and \cite[sec.\ III.7-8]{silvermanAdvancedTopicsArithmetic1994} have been consulted. 

Models are a way to extend a variety over $K$ to a variety over $R$. 
Models of varieties, in particular curves, have long been of interest in arithmetic geometry . 
We study models because they are tightly linked to the Berkovich analytification of a variety as we will see in \cref{chap:intro_berkovich,chap:weight_functions}. 
Intersection is useful to study and define special type of models, which are of special interest to Berkovich geometry.  

\section{Models and the reduction map} \label{sec:models}
\begin{definition}
	Let $\mathscr X$ be a $R$-scheme. 
	Then we define 
	\begin{align*}
		\mathscr X_\eta &:= \mathscr X_K = \mathscr X \times_R \spec K \\
		\mathscr X_s &:= \mathscr X_k = \mathscr X \times_R \spec k 
	.\end{align*}
	We say that $\mathscr X_\eta$ is the \emph{generic fibre of $\mathscr X$} and that $\mathscr X_s$ is the \emph{special fibre of $X$}. 
\end{definition}
\begin{definition}
	Let $X$ be a variety over $K$.  
	A model of $X$ is a flat, separated, finite type $R$-scheme, $\mathscr X$ together with an isomorphism $i: \mathscr X_\eta \simeq X$.
\end{definition}
Alternatively we may consider $i$ as open immersion $X\into \mathscr X$. 
We will almost always omit specifying the isomorphism, as it is usually clear from the context. 
Depending on the context we will almost always work with specific types models like requiring $\mathscr X$ to be proper or regular or normal.

\begin{example}\label{ex:first_models}
	\begin{enumerate}
		\item Let $\pro^{1}_K = \proj(K[x, y])$ be the projective line. 
			Then $\mathscr P = \proj\left(\frac{R[x, y, z]}{(x\cdot z - \pi y^2)}\right)$ is a model of $\pro^{1}_K$. 
			The special fiber is $\proj\left( \frac{k[x, y, z]}{(xz)} \right) $ which consists of two irreducible components, each isomorphic to $\pro^{1}_k$.  
		\item Suppose $\ch k \ne 2$. Let $E$ be the elliptic curve given by the Weierstrass equation \[
				y^2 = x (x -1)(x-\pi)
		.\] 
		Then $\mathscr E = \proj \left( \frac{R[x, y, z]}{(zy^2 - x(x0z)(x-\pi z)} \right) $ is a model of $E$ and the special fiber is given by the Weierstrass equation $y^2 = x^2 (x-1)$, hence it is a node. 
	\end{enumerate}
	See \cref{fig:first_examples_of_models}.
\end{example}

\begin{figure}[h]
    \centering
    \incfig{first-examples-of-models}
    \caption{The models from \cref{ex:first_models}.}
    \label{fig:first_examples_of_models}
\end{figure}

\begin{definition}
	Let $i_1: X \into \mathscr X_1$, $i_2: X \into \mathscr X_2$ be two models of $X$. 
	Then a morphism of models of $X$ is a morphism of $R$-schemes $f: \mathscr X_1 \to \mathscr X_2$ such that \[
	\begin{tikzcd}
		\mathscr X_1 \ar{rr}{f} & & \mathscr X_2 \\
					& X \ar[hookrightarrow]{ul}{i_1}  \ar[hookrightarrow]{ur}[']{i_2}
	\end{tikzcd}
	\] 
	commutes.
\end{definition}

\begin{example}\label{ex:morphism_models_1}
	The most common way to construct morphisms of models is by blowing up closed subschemes in the special fiber. 
	Let $E, \mathscr E$ be the elliptic curve and model from \cref{ex:first_models}. 
	Suppose that $\ch k \ne 2$. 
	Let $\xi$ be the singular point of $\mathscr E _s$, with coordinates $(x, y,\pi) = (0,0,0)$. 
	Then  $\mathscr E' := \bl_\xi \mathscr E $ is another model of $E$ and the morphism $p:\mathscr E' \to \mathscr E $ is a morphism of models. 
	In this case one can show that the exceptional divisor of $\mathscr E' \to \mathscr E$ is a rational curve of multiplicity $1$ which intersects the strict transform of $\mathscr E _s$ in two different points. 
	See \cref{fig:morphism_models_blowup_1}.
\end{example}

\begin{figure}[h]
    \centering
    \incfig{morphism-models-blowup-1}
    \caption{The morphism between models of an elliptic curve from \cref{ex:morphism_models_1}.}
    \label{fig:morphism_models_blowup_1}
\end{figure}

Morphisms of models are birational morphisms, as they are isomorphisms on the generic fiber. So if $\mathscr C_1, \mathscr C_2$ are models of $C$, then a morphisms of models $f: \mathscr C_1 \to \mathscr C_2$ is necessarily unique, if it exists. 

\begin{definition}
	Let $\mathscr C_1, \mathscr C_2$ be models of $C$. 
	Then we say that $\mathscr C_1$ \emph{dominates} $\mathscr C_2$ if there is a surjective morphism of models $\mathscr C_1 \to \mathscr C_2$.
\end{definition}

If $\mathscr C_1 $ dominates $\mathscr C_2$, and both are proper then we can blowup a closed point in $\mathscr C_2$, where the map is not locally isomorphic, to add an one of the exceptional divisors of $\mathscr C_1$ to $\mathscr C_2$. 
Repeating this operation we eventually get the whole of $\mathscr C_1$. 
This leads to the following theorem. The details of this construction can be found in \cite[sec.\ 9.1.2]{liuAlgebraicGeometryArithmetic2002}.
\begin{theorem}[factorisation theorem]\label{thm:factorisation_theorem}
	Let $\mathscr C_1, \mathscr C_2$ be two proper models a curve $C$ such that $f: \mathscr C_1 \to \mathscr C_2$ dominates. 
	Then $f$ can be decomposed into a sequence of blowups in closed points. 
\end{theorem}

Domination induces a partial order on the models of $\mathscr C$. 
\begin{definition}
	Let $C$ be a curve and $\mathscr C_1, \mathscr C_2$ be two models of $C$. 
	If $\mathscr C_1$ dominates $\mathscr C_2$ then we say $\mathscr C_2 \le \mathscr C_1$
\end{definition}





\subsection{Weil versus Cartier divisors} \label{sec:weil_versus_cartier_divisors}

Recall the following proposition. 
\begin{proposition}\label{prop:weil_vs_cartier}
	Let $X$ be a regular, integral, Noetherian scheme. Then the groups of Weil and Cartier divisors are isomorphic, and so are the Weil and Cartier divisor classgroups. 
\end{proposition}
\begin{proof}
	This is \cite[prop.\ 7.2.16]{liuAlgebraicGeometryArithmetic2002}. 
\end{proof}

In the next sections we will almost always work with regular models of a curve over $K$, which satisfy the assumptions of \cref{prop:weil_vs_cartier}. So we may freely move between Cartier and Weil divisors. 

\subsection{The reduction map} \label{sec:the_reduction_map_models}
One can think about models as a way of taking looking at varieties modulo $\pi$. 
This is made precise by the reduction map. 
Let $ X\subset \pro^{n}_R$ is some projective variety variety over $R$ and $x \in X$ be a $K$-valued points. 
We may choose coordinate $[a_0:\ldots:a_n] = x$ such that all $a_0, \ldots, a_n \in R$ and at least one of them is not zero mod $\pi$.  
Then $[\overline{a_0}: \ldots :\overline{a_n}]$ defines a point in $X_s \subset  \pro^{n}_k$, which we call the $\tilde x$ or the reduction of $x$.  
The idea of reduction map is to give a more scheme theoretic version of this map. 

\begin{definition}\label{def:reduction_map_R_scheme}
	Let $X$ be a variety proper over $R$. 
	Then there is a map 
	\begin{align*}
		\red_X: X_{\eta}(K) &\longrightarrow X_{s}(k) \\
		x &\longmapsto \tilde x
	,\end{align*}
	which is defined as follows. 
	Let $x: \spec K \to X_\eta$ be $K$-rational point.
	Then by the valuative criterion of properness there is a unique map \[
	\begin{tikzcd}
		K \rar{x} \dar & X \dar \\
		R \rar \ar[dashed]{ur}{x'} & R
	\end{tikzcd}
	\] 
	$x': R \to X$. 
	The base change to $k$ gives a $k$-rational point on $X_s$, $\tilde x = x' \otimes_R k: k \to X_s $.
\end{definition}
\begin{remark}
If $K$ is a complete field, and $x \in X_{\eta, \text{cl}}$ be any closed point in the special fiber.  Then by \cref{thm:norm_finite_field_ext} $\kappa(x)$ is valued field. 
So by the same procedure as in \cref{def:reduction_map_R_scheme} we may turn the map $\spec: \kappa(x) \to X_{\eta}$ to a map $\widetilde{\kappa(x)} \to X_s$ and extend the reduction map to all closed points \[
\red_X: X_{\eta, \text{cl}} \to X_{s, \text{cl}}
.\] 
In this case one can also show that $\red_X$ is surjective. 
\end{remark}

So for any proper model $\mathscr X$ of a proper $K$-variety $X$ (e.g.\ a proper model of a curve) there is a reduction map \[
	\red_{\mathscr X}: \mathscr X(K) \to \mathscr X(k)
.\] 

\begin{remark}
	This reduction map is just a function between sets. 
	It is \emph{not} a morphism of schemes. 
\end{remark}


\section{Intersections on surfaces} \label{sec:intersections_on_surfraces}

\todo{some introduction here}
Let $X$ be regular Noetherian connected scheme of dimension 2. 
\begin{definition}
	Suppose $D, E$ are effective cartier divisors on $X$ that do not share any irreducible components, cut out by the sheaf of ideals $\mathcal{I} _D, \mathcal{I} _E$ respectively.

	Let $x \in X$ be closed point. Then the \emph{intersection number of $D, E$ in $x$} is \[
		i_x(D, E) = \length_{\mathcal{O}_{X, x}} \frac{\mathcal{O}_{X, x}}{\mathcal{I}_{D,x} + \mathcal{I}_{E,x}} 
	.\]  
\end{definition}

This functions counts how many times $ D$ and $E$ intersect in $x$. 
As you may expect when  $x$ not in the support of $D$ (resp. $E$), then $i_x(D, E) = 0$. 
Indeed, in that case the inclusion $\mathcal{I}_{D, x} \subset  \mathcal{O}_{X, x}$ (resp. $\mathcal{I} _{E, x}\subset \mathcal{O}_{X,x}$) is an isomorphism. 
Hence $\mathcal{I}_{D, x} + \mathcal{I} _{E, x} = \mathcal{O}_{X, x}$ and the lenght is $0$ and the module of which the length we need to compute is trivial. 

If $x \in \supp D \cap \supp E$ then by the Nullstellensatz we have that $\mathfrak{m} _x \subset  \sqrt{\mathcal{I} _{D, x} + \mathcal{I} _{E, x}} $. 
Hence ${\mathcal{O}_{X, x}} \frac{\mathcal{O}_{X, x}}{\mathcal{I}_{D,x} + \mathcal{I}_{E,x}}$ is Artinian, thus $i_x(E, D)$ is finite. 
\todo{why is it non-zero in this case?}. 

\begin{definition}\label{def:intersection_number_effective}
	Let $D, E$ be effective cartier divisors on $X$ that do not share any irreducible components.
	Then we define the \emph{intersection number of $D, E$} to be \[
		D\cdot E = \sum_{x \in X_\text{cl} }i_x (D, E)
	.\] 
\end{definition}
As $\supp D \cap \supp E$ is finite, the terms in the sum above are $0$ for all but finitely many points. Hence $D\cdot E$ is a finite integer. 
\begin{example}
	Let $F$ be any field and $X = \spec K[x, y]$. 
	Let  $D = V(y)$ and $E = V(y - x^r)$ for some $r \ge 1$
	Then  $\supp D \cap \supp E = \{(0, 0)\} $ and $\mathcal{I} _{D, (0,0)} + \mathcal{I} _{E, (0,0)} = (y, y-x^r) = (y, x^r)$.
	So \[
		i_{(0,0)}(D, E) = \length \frac{K[x, y]_{(0, 0)}}{(y, x^{r})} = r
	.\] 
	As this is the only intersection point also have that $D\cdot E = r$. 
\end{example}

\begin{lemma}\label{lem:intersection_surfaces_restriction_mult_1}
	Let $D, B$ be effective cartier divisors with no common components. And let $i: D \into X, j: E \into X$ be the corresponding closed immersions in $X$. 
	Let $x \in E$. Then  
	\begin{align*}
		i_x(D, E) &= \mult_x D|_E   \\
		(D\cdot E) &= \deg D|_E 
	.\end{align*}
\end{lemma}
\begin{proof}
	Without loss of generality it suffices to show that $i_x(D, E) = \mult_x D|_E$. Then $(D\cdot E) =\deg D|_E$ follows from the definition of the degree.  
	We have \[
		i_x(D, E) = \length \frac{\mathcal{O}_{X, x}}{\mathcal{I} _{D, x} + \mathcal{I} _{E, x}} =\length \frac{\mathcal{O}_{E, x}}{\mathcal{O}(-D|_E)_x} = \mult_x D|_E \\
	.\] 
\end{proof}
\begin{lemma}
	Let $D, F$ be effective cartier divisors that do not share any components with a effective cartier divisor $E$. Let $x \in E$
	Then  \[
		i_x (D + F, E) = i_x(D, E) + i_x(F, E), \quad (D + F) \cdot E = D\cdot E + F \cdot E
	.\] 
\end{lemma}
\begin{proof}
	Using \cref{lem:intersection_surfaces_restriction_mult_1} we see that 
	\begin{align*}
		i_x (D + F, E) &= \mult_x (D + F)|_E = \mult_x (D|_E + F|_E) \\
			       &= \mult_x D|_E + \mult_x F|_E \\
			       &= i_x(D + F, E) 
	.\end{align*}
	From this $(D + F) \cdot E = D\cdot E + F \cdot E$ follows.
\end{proof}

\todo{put the results from \cite{liuAlgebraicGeometryArithmetic2002} 9.1.9 abound recognising normal crossings with intersection numbers somewhere}

\begin{definition}
	Let $D, E$ be divisors (not necessarily effective) without common components. 
	Let $E = E_1 - E_2$ and $D = D_1 - D_2$ such that $E_1, E_2, D_1, D_2$ are effective divisors with piarwise no common components. 
	Then we can extend the \cref{def:intersection_number_effective} 
	\[
	D\cdot E = D_1 \cdot E_1 - D_2 \cdot E_1 - D_1 \cdot E_2 + D_2 \cdot  E_2
	.\] 
	The map $(D, E )\mapsto D\cdot E$ is bilinear where it is defined and is called the \emph{intersection pairing}.
\end{definition}

To extend the set of pairs of divisors on which the intersection pairing is defined, as well as prove nice properties of it we need to add extra assumptions to the scheme $X$. 


\todo{check the precise conditions in the following two lemmas}
\begin{lemma}[moving lemma]
	Let $X$ be a connected, normal, separated, Noetherian scheme, and $D, E$ a divisor on $X$. 
	Then there is a $f \in K(X)$ such that $D + (f)$ and $E$ do not share any components. 
\end{lemma}
\begin{proof}
	See \cite[]{liuAlgebraicGeometryArithmetic2002}
\end{proof}

\begin{lemma}
	Suppose that moverover $X$ is proper. 
	Let  $D$ be any divisor and $f \in K(X)$ be such that $D$ and $(f)$ do not share any components. 
	Then $(f)\cdot D = 0$. 
\end{lemma}
\begin{proof}
	
\end{proof}





\section{Intersections on models} \label{sec:intersections_on_models}
\todo{ramble on why we needed properness and how to fix it}

As seen in \cref{prop:intersection_pairing_proper_model} properness is need to have a good intersection theory on surfaces. 
If not moving a divisor, to a linearly equivant one, might cause one of the intersection points to 
This is a problem when we want to have a intersection theory on a model or more generally a fibered surface. 
Let $\mathscr C  / R $ be a proper regular model. 
The $\mathscr C$ is proper over $\spec R$, but $\spec R$ itself is not proper. 
So we might, and do run into the problem that the intersection pairing is not invariant under linear equivalence. Consider the following example.

\begin{example}\label{ex:first_model}
	Consider $\pro^{1}_R = \proj K[x, y]$ as a model of $\pro^{1}_K$. 
	Take $D = (x), E = (x-\pi y )$, and $E' = E + (\frac{x-y}{x-\pi y})= (x - y)$.
	Then  $D, E$ intersect in origin of special fiber, hence $D\cdot E \ge 1$. 
	But $D, E'$ do not have any common points, hence $D \cdot E' = 0$. 
\end{example}

The problem here is that the two divisors span across the two divisors. They are what we will later define to be horizontal divors. Intuitevely they lay in the direction where the scheme is not proper. 
However the special fibre, and every component in it is a proper divisor. 
This suggets a fix.
If we restrict our selves to the case where $D$ lies in the special fibre (is vertical), then we can compute $D\cdot E$ on the restriction do $D$ and once again obtain that the intersection pairing indenpent of linear equivalence. 

\begin{definition}\label{def:vertical_divisor}
	Let $\mathscr{C}$ a $2$-dimensional, regular, integral scheme proper and flat over $R$.  (i.e.\ a proper regular model of a curve). 
	Then $\divisor(\mathscr{C})_s$ is the set of divisors that are supported in the special fibre. 
	Such a divisor is called a \emph{vertical divisor}.
\end{definition}
\begin{definition}\label{def:horizontal_divisor}
	A divisor on $\mathscr{C} $ is \emph{horizontal} if it is the schemeatic closure of a divisor in the on the generic fibre in $\mathscr C$. 
\end{definition}
\begin{lemma}\label{lem:decomponsition_horizontal_vertical}
	Any divisor $D$ on $\mathscr{C}$ can be uniquely decomposed in a horizontal and vertical part $D =  D_\text{vert}  + D_\text{hor}$. 
	Moreover $D_\text{hor} $ is the schematic closure of $D|_{\mathscr C_\eta}$ in $\mathscr C$.
\end{lemma}
\begin{proof}
	It is sufficient to prove that every prime divisor $D$ is either vertical or horizontal.
	Suppose that $D$ is not vertical. 
	Then by \cite[prop. 8.3.4]{liuAlgebraicGeometryArithmetic2002} $D = \overline{\{x\} }$ for some point $x \in \mathscr{C} _\eta$.
	\todo{should I write the proof out here, or is the citation enough?}
\end{proof}

\begin{definition}
	Let $\mathscr C$ a $2$-dimensional, regular, integral scheme proper and flat over $R$.  (i.e.\ a proper regular model of a curve). 
	Then we define a map
	\begin{align*}
		i: \divisor(\mathscr C) \times \divisor_s(\mathscr C) &\longrightarrow \Z \\
		(D, E) &\longmapsto \sum_{1 \le i \le r} n_i \deg \mathcal O_X(D)|_{\Gamma_i}
	,\end{align*}
	where $E = \sum_{i = 1}^{r} n_i \Gamma_i$ is the decomposition of $E$ in prime divisors. 
\end{definition}
	

\begin{theorem}\label{thm:defining_properties_intersection_pairing}
	The map $i: \divisor(\mathscr C) \times \divisor_s(\mathscr C) \rightarrow \Z $ defined above is the unique bilinear map satisfying
	\begin{enumerate}
		\item If $D \in \divisor(\mathscr{C} )$ and $E \in \divisor_s(X)$ are divisors with no common compononents then  \[
				(D\cdot E) = \sum_{x \in \mathscr{C}_\text{cl} }^{} i_x(D, E)[k(x):k]
			.\] 
		\item The restiction of the pairing to $\divisor _s (X) \times  \divisor_s(X)$ is symmetric.
		\item If $D, D' \in \divisor (\mathscr C)$ and $D \sim D'$ then $D\cdot E = D' \cdot E$. 
		\item If $0 < E \le X_s$ then $D \cdot E = \deg \mathcal{O}_X(D) |_E$. 
	\end{enumerate}
\end{theorem}
\begin{proof}
	Uniqueness follows from the bilinearity and property 4.
	So it suffices to show that $i$ satsifies the properties. 
	\begin{enumerate}
		\item $i$ is bilinear and so is $i_x$. 
			So we may assume that $E, D$ are different effective prime divisors.
			We have \[
				\deg \mathcal{O}_X(D)|_E = \deg [D|_E] = \sum i_x (D, E) [k(x):k]
			.\] 
		\item By bilinearity it is enough to show this in the case when $D, E$ are prime divisors. 
			Then either $D = E$ in which case it's trivial. 
			Or $D \ne E$, in which case it follows from 1.
		\item By linearity we may assume that $E$ is prime. 
			Then $\mathcal{O}_X(D) \simeq \mathcal{O}_X(D')$ hence
			\[
				D \cdot E = \deg \mathcal{O}_X(D)|_E = \deg \mathcal{O}_X(D') |_E = D' \cdot E
			.\] 
		\item \todo{I don't understand this proof}
	\end{enumerate}
\end{proof}

\begin{proposition}
	Let $\mathscr C$ over $R$ as before. 
	For any $E \in \divisor_s$ we have $E\cdot \mathscr C_s = 0$ where we consider $\mathscr C_s$ as a divisor on $\mathscr C$. 
\end{proposition}
\begin{proof}
	Note that $\mathscr C_s$ is principal as it is equal to $(\pi)$. 
	Hence $\mathscr C_s \sim 0$ and by \cref{thm:defining_properties_intersection_pairing} part 2, 3 we find \[
	E\cdot C_s = C_s \cdot E = 0 \cdot E = 0
	.\] 
\end{proof}
This seems like a very minor proposition, but in practice it is very useful as it gives an easy way to compute self intersections. 
\begin{corollary}
	Let $\mathscr C$ as before. 
	Let $\mathscr C_s = \sum_{i =1}^{r} n_i \Gamma_i$ be the decomposition into prime divisors. 
	Then for any $i$ we find \[
		\Gamma^2_i := \Gamma -\frac{1}{n_i} \sum_{j \ne i} \Gamma_i \cdot \Gamma_j
	.\] 
\end{corollary}
\begin{proof}
	By the previous proposition we have \[
	 0 = \Gamma_i \cdot \mathscr C_s = \sum_{j = 1}^{r} n_i \Gamma_i \cdot \Gamma_j
	.\] 
	Solving this equation to $\Gamma_i\cdot \Gamma_i$ yields the desired result. 
\end{proof}

\begin{proposition}\label{prop:intersection_horizontal_point_degree}
	Let $x \in \mathscr C_\eta$ a closed point. 
	Then \[
		\overline{\{x\} }\cdot \mathscr C_s = [K(P): K]
	,\] 
	where $\overline{\{x\} }$ is the Zariski closure of $\{x\} $ in $\mathscr C$ with the reduced structure. 
	\todo{what the hell is $K(P)$?}
\end{proposition}
\begin{proof}
	See \cite[prop.\ 9.1.30]{liuAlgebraicGeometryArithmetic2002}
\end{proof}


\begin{corollary}
	Let $x \in \mathscr C_\eta$ be a $K$-rational point and $D = \overline{\{P\} } $. 
	Then $D$ intersects $\mathscr C_s$ in a single point $p$, which is smooth in $\mathscr C_s$. 
	Thus $p$ lies on a unique irreducible component of $X_s$ which is of multiplicity $1$. 
\end{corollary}
\begin{proof}
	By \cref{prop:intersection_horizontal_point_degree} we obtain that \[
	\sum_{i = 1}^{r}  n_i D\cdot \Gamma_i = D \cdot X_s = 1
	,\] 
	where $\mathscr C_s = \sum_{i =1}^{r} n_i \Gamma_i$ is the decomposition in irreducible components. 
	So can be atmost one $i$ such that $D\cdot \Gamma_i \ne  0$ and $n_i = 1$. 
\end{proof}

Let $\omega_{\mathscr C / R} $ be the cannonical bundle of $\mathscr C$ (relative to $\spec R$).
We now turn to the question how we should compute $K_{\mathscr C / R} \cdot E$ where  $K_{\mathscr C / R}$ is a representative of $\omega_{\mathscr C / R}$ and $E \in \divisor_s \mathscr C$. 
Note that the result does not depend on the choice of $K_{\mathscr C / R}$ as any two choices are linearly independent. 

\begin{proposition}
	Let $\mathscr C$ be as above. 
	Then we have \[
		2p_a(\mathscr C_\eta) - 2 = K_{\mathscr C / R} \cdot \mathscr C_s  
	.\] 
\end{proposition}
\begin{proof}
	From \cref{thm:defining_properties_intersection_pairing} we see that \[
		K_{\mathscr C / R} = \deg \omega_{\mathscr C / R}|_{\mathscr C_s} = \deg \omega _{\mathscr C_s / k}
	.\] 
	The last equality is holds because the cannonical bundle pulls back under basechange. 
	We continue the computation \[
		\deg \omega_{\mathscr C_s / k} = - 2\chi (\mathcal{O}_{\mathscr C_s}) = -2 \chi (\mathcal{O}_{\mathscr C_\eta}) = 2p_a(\mathscr C_\eta) - 2
	.\] 
	The middle equality holds because $\mathscr C \to \spec R$ is flat. 
\end{proof}

\begin{theorem}[Adjunction formula]\label{thm:adjunction_formula}
	Let $\mathscr C$ be as above and $0 < E \le \mathscr C_s$, then \[
		\omega_{E / k} \simeq (\mathcal{O}_{\mathscr C}(E) \otimes \omega_{\mathscr C / R})|_E \quad \text{ and } \quad p_a(E) = 1 + \frac{1}{2}(E^2 + K_{\mathscr C / R}\cdot E)
	.\] 
\end{theorem}
\begin{proof}
	\todo{check this proof}
	The first equality follows from taking the determinant of the second fundamental SES  \[
		0 \to \mathcal{O}(-E)|_E \to \Omega_{\mathscr C / R}|_E \to \Omega_{E / k}
	.\] 
	To show the second equality, note that \[
		p_a(E) = 1 - \chi(\mathcal{O}_E) = 1  + \frac{\deg \omega_{E / k}}{2} = 1 + \frac{1}{2}\left(E^2 + K_{\mathscr C / R} \cdot E\right)
	\] 
	where we used the first equality. 
\end{proof}

\todo[inline]{Should I cover section 9.2.1 of \cite{liuAlgebraicGeometryArithmetic2002} here?}

For the final part of this thesis it will be very important to understand how given a map $f: \mathscr X \to \mathscr Y$ intersection numbers behave under pushforward and pullback of divisors. 

\begin{theorem}[projection formula]
	Let $f:\mathscr X \to \mathscr Y$ be a dominant morphism of 2-dimensional, regular, integral schemes, proper and flat over $R$. 
	Let $C$ be a divisor on $\mathscr X$ and $D$ divisor on $\mathscr Y$. 
	Then the following holds.
	\begin{enumerate}
		\item If $f(\supp C)$ is finite then $C\cdot f^* D = 0$ 
		\item If either $C$ or $D$ is vertical then \begin{equation}\label{eq:projection_formula}
				C\cdot f^* D = f_* C \cdot D
			.\end{equation}
		\item The extension $K(X) / K(Y)$ is finite. 
			If $F$ is a vertical divisor on $\mathscr Y$ then $f^* F$ is also vertical and \[
				f^* F \cdot f^* D = [K(X): K(Y)] F \cdot D
			.\] 
	\end{enumerate}
\end{theorem}
\begin{proof}
	\begin{enumerate}
		\item Note that $C$ is necessarily vertical under this assumption. 
			By linearity we may assume that $C$ is prime and thus $f(C)$ is a single point, $y$. 
			Let $U$ be a neighboord of  $y$ on which $\mathcal{O}(D)$ is trivial. 
			Then $f^*(\mathcal{O}(D))$ on $f^{-1}(U)$ is trivial as well. 
			Hence $f^*(\mathcal{O}(D))|_E$ is trivial, from which the result follows.
		\item By linearity we may assume that $C$ is a prime divisor.
			If $f(C)$ is a point then by definition $f_*(C) = 0$. Then the equality follows from 1. 
			For the other case see \cite[thm.\ 9.2.12]{liuAlgebraicGeometryArithmetic2002}. 
			\todo{understand this proof}
		\item 
			We know that $f_* f^* F = [K(\mathscr X): K(\mathscr Y)] F$. Hence  \[
				f^* F \cdot  f^* D = f_* f^* F \cdot D = [K(\mathscr X) : K(\mathscr Y)] F \cdot  D
			.\] 

	\end{enumerate}
\end{proof}

\begin{remark}
	If $f: \mathscr X \to \mathscr Y$ is projective and surjective suppose that $f^{-1}D$ does not contain any components of $C$ then we even have the identity  \[
		f_*(C . f^*D) = (f_*C \cdot f^* D)
	.\] 
	See \cite[rem.\ 9.2.13]{liuAlgebraicGeometryArithmetic2002} for more details. 
\end{remark}


\section{Models and blowups} \label{sec:models_and_blowups}
This section contains results about how models and the intersection pairing on them change after blowing up in a closed point of the special fiber.  

\begin{proposition}\label{prop:E_blowup_properties}
	Let $\mathscr C$ be a proper regular model of a curve $C$ and $x \in \mathscr C_s$ a closed point. 
	Let $p: \mathscr C' = \bl_x \mathscr C$ and $E$ be the exceptional divisor of the blowup. 
	Then $E \simeq \pro^{1}_{\kappa(x)}$ and \[
		\mathcal{O}(E)|_E = \mathcal{O}_E(-1), \quad E^2 = -[\kappa (x): k]
	.\] 
\end{proposition}
\begin{proof}
	That $E$ is isomorphic to $\pro^{1}_{\kappa x}$, and $\omega_{E / X} = \mathcal{O}(E)|_E = \mathcal{O}_E(-1)$ are standard theorem about blowups in regular points. 
	We still have to compute $E^2$. 
	\[
		E^2 = \deg_k \mathcal{O}_E(-1) = [\kappa(x): k] \deg_{\kappa (x)} \mathcal{O}_E(-1) = -[\kappa(x): k]
	.\] 
\end{proof}

These properties also precisely characterize a blowup, i.e.\ any irreducible components satisfying the conclusion of \cref{prop:E_blowup_properties}, can be ``blown-down'' to a regular model.
This is made precise in Castelnuovo's criterion. 

\begin{theorem}\label{thm:castelnuovo}
	[Castelnuovo's criterion]
	Let $\mathscr C$ be a proper regular model of a curve $C$.
	Let $E$ be a irreducible component of $\mathscr C_s$ and $k' = H^{0}(E, \mathcal{O}_E)$. 
	There exists a regular projective model $\mathscr C'$ and a closed point $x \in \mathscr C'$ such that $\mathscr C = \bl_x \mathscr C'$ with exceptional divisor $E$ if and only if $E \simeq \pro_{k'}^{1}$ and $E^2 = -[k': k]$. 
\end{theorem}
\begin{proof}
	One direction is exactly \cref{prop:E_blowup_properties}. 
	The other direction is long and technical so we refer the reader to \cite[sec.\ 9.3.1]{liuAlgebraicGeometryArithmetic2002}.
\end{proof}

\subsection{Strict normal crossings models} \label{sec:strict_normal_crossings_models}

\begin{definition}
	Let $D$ be an effective Cartier divisor on a regular Noetherian scheme $X$. 	Let $x \in X$ be a closed point. 
	We say that $D$ has \emph{normal crossings at $x$ } if there are $f_1, \ldots, f_n$ with $n = \dim \mathcal{O}_{X, x}$ such that $\mathfrak{m} _x \mathcal{O}_{X, x} = (f_1, \ldots, f_n)$ and $\mathcal{O}(-D)_{x} = (f_1^{r_1}, \ldots, f_m^{r_m})$ for some $0 \le m \le n$ and $r_i \in \Z_{> 0}$. 

	We say $D$ has \emph{normal crossings} or $D$ is a \emph{normal crossings divisor} if $D$ has normal crossings at every closed point  $x \in X$. 

	Moreover  $D$ has \emph{strict normal crossings} or $D$ is a \emph{strict normal crossings divisor} if $D$ has normal crossings and every irreducible component of $D$ is regular. 
\end{definition}
Normal crossings and strict normal crossings often get abbreviated as \gls{nc} or \gls{snc} respectively.
You should think of a normal crossings at $x$ as all components of $D$ meeting transversely at $x$. 
Strictness ensures that there are no ``hidden crossings'' which exist étale locally on a component, e.g.\ a node (see \cref{fig:normal_crossings_divisors}).

\begin{figure}[ht]
    \centering
    \incfig{normal-crossings-divisors}
    \caption{Recognizing (strict) normal crossings divisors}
    \label{fig:normal_crossings_divisors}
\end{figure}
\begin{definition}
	A \emph{(strict) normal crossings model} or \emph{(scn-model) nc-model}  of $X$ is a regular model $\mathscr X$ such that $\mathscr X_s = V(\pi) \subset  \mathscr X$ is a (strict) normal crossings divisor. 
\end{definition}

Strict normal crossings models are in some sense the least singular models, as they are regular, normal, and the intersection points are as well behaved as we can hope for. 
The only step up would be semistable models (which we will introduce in \cref{sec:semistable_curves_and_models}) where we require that the special fiber is reduced, i.e.\ every component of the special fiber occurs with multiplicity 1. 
However these don't exist for every curve, whereas snc-models do.

We can recognize whether a model has normal crossings by looking at the intersections of irreducible components of the special fiber. 

\begin{proposition}
	A proper, regular model $\mathscr C$ of $C$ has strict normal crossings if for every if and only if every non-smooth $x \in \mathscr C_s$, belongs to exactly two irreducible components $F, G$ of $\mathscr C_s$ and $i_x(F, G) = 1$.  
\end{proposition}
\begin{proof}
	This is essentially \cite[prop.\ 9.1.8.(b)]{liuAlgebraicGeometryArithmetic2002} in our context of models. 

	\ltr Let $x \in \mathscr C_s$ be not smooth. 
	As the irreducible components are smooth, it follows that $x$ lies in multiple irreducible components, say $F_1, \ldots, F_r$, locally cut out by $f_1, \ldots, f_r$ with multiplicities $n_1, \ldots, n_r$. 
	By the definition of normal crossings this means that  $\mathscr C_s$ is locally cut out by the ideal $(f_1^{n_i}\cdots f_r^{n_r})$ with $r \le 3$. Hence  $r = 2$. 
	As by \cite[prop.\ 9.1.8.(b)]{liuAlgebraicGeometryArithmetic2002} we know that $i_x(F_1, F_2) = 1$. 

	\rtl Let's first argue that $\mathscr C_s$ is a normal crossings divisor and then argue that it is strict.
	Let $x \in \mathscr C_s$ a point. If $x$ is smooth in $\mathscr C_s$ then it belongs to a single irreducible component, which is regular at $x$. 
	Hence $\mathscr C_s$ is nc at $s$. 
	Suppose  $x$ is not smooth. Then it lays on the intersection of exactly two components $F, G$ with $i_x(F, G) = 1$ and the result follows from \cite[prop.\ 9.1.8.(b)]{liuAlgebraicGeometryArithmetic2002}.
	It also follows that $F, G$ are regular at $x$.
\end{proof}

\begin{theorem}\label{thm:desingularisation}
	Let $\mathscr C$ be a regular projective model of a curve $C$.
	Then there exists a snc-model $\mathscr C'$ of $C$ with dominating morphism of models $\mathscr C' \to \mathscr C$. 
	Moreover $\mathscr C'$ can be chosen to be the minimal snc-model dominating $\mathscr C$. 
	Then $\mathscr C'$ is called the \emph{minimal desingularisation of $\mathscr C$}. 
\end{theorem}
\begin{proof}
	This follows from embedded resolution. 
	The idea is to repeatedly blow up points in $\mathscr C$ to make all irreducible components regular, and all intersections normal crossings. 
	See \cite[sec.\ 9.2.4]{liuAlgebraicGeometryArithmetic2002} for more details.
	That a minimal desingularisation exists is the content of \cite[prop.\ 9.3.32]{liuAlgebraicGeometryArithmetic2002}.
\end{proof}
Via the factorization theorem we also see that this desingularisation can be obtained via finite sequence of blowups in closed points. 
In \cref{fig:snc_model_curve_type_ii} we illustate this with the minimal regular model of an elliptic curve of reduction type II. 
The reduction types of elliptic curves will be discussed in \cref{chap:kodaira_neron_classification_of_elliptic_curves}.

\begin{figure}[ht]
    \centering
    \incfig{snc-model-curve-type-ii}
    \caption{Blowing up points in the special fiber of the minimal regular model of an elliptic curve of type II, until as snc-model is obtained. 
    For every step the blue point is the point where we blow up, and the red component is the exceptional divisor. }
    \label{fig:snc_model_curve_type_ii}
\end{figure}


\begin{lemma}\label{lem:blowup_snc}
	Let $\mathscr C$ be a snc-model of a curve $C$, and $x$ a closed point in $\mathscr C_s$. 
	Then $\bl_x\mathscr C$ is an snc-model of $C$. Let $E$ be the exceptional divisor. 
	Moreover 
	\begin{enumerate}
		\item if $x$ is a smooth point on $\mathscr C_s$ laying on the irreducible component $F$ with multiplicity $n$, then $E$ is a rational curve of multiplicity $n$ intersecting $\tilde F$ in one point. 
		\item If $x$ is an intersection point laying on two irreducible components $F, G$ with multiplicity  $m, n$ respectively, then $E$ is a rational curve of multiplicity $n + m$ intersecting $\tilde F, \tilde G$ in 1 point each. 
	\end{enumerate}
	Here $\tilde F$ denotes the strict transform of $F$. 
	See \cref{fig:blowup-snc}
\end{lemma}
\begin{proof}
	As the components are smooth we have $\mu_x(F_i) = 1 $ for every closed point $x \in \mathscr C_s$ and irreducible component  $F$ containing $x$. 
	Then the result is exactly \cite[exercise 9.2.9(a)]{liuAlgebraicGeometryArithmetic2002}.
\end{proof}
\begin{figure}[ht]
    \centering
    \incfig{blowup-snc}
    \caption{Blowing up snc-models in a smooth point (left) and intersection point (right)}
    \label{fig:blowup-snc}
\end{figure}




\begin{theorem}\label{thm:minimal_snc_model}
	Let $C$ be a curve with  $g(C) > 0$, then $C$ has a minimal snc-model $\mathscr C_\text{min} $, i.e.\ any snc-model $\mathscr C$ dominates $\mathscr C_\text{min} $. 
\end{theorem}
\begin{proof}
	See \cite[prop.\ 9.3.36]{liuAlgebraicGeometryArithmetic2002}.
\end{proof}



\section{Semistable curves and models} \label{sec:semistable_curves_and_models}
A particularly nice class of models, are the semistable models.
In short they are models where all the irreducible components in the special fiber are of multiplicity $1$. 
Unfortunately, they do not exists for every curve. 
But a deep theorem about curves states that they do always exist after basechange to a finite extension of the ground field. 

In this section we will assume that the residue field $k$ of $R$ is algebraically closed. 

\begin{definition}\label{def:semi_stable_curve}
	Let $C $ be curve over an algebraically closed field $F$. 
	Then $C$ is called \emph{semistable} if $C$ is reduced and its singular points are ordinary double points. 
\end{definition}


Semistable curves are nice, because we can very easily determine their genus. 
\begin{comment}
	
\begin{lemma}
	Let $C$ be a semistable curve over an algebraically closed field $F$. 
	Let $C'$ be the normalisation of $C$ and $f: C' \to C$ the induced morphism.  
	Then there is an exact sequence \[
	0 \to f_* \omega_{C' / F} \to \omega_{C / F} \to \mathcal{F}  \to 0
	.\] 
	where $\mathcal{F} $ is the skyscraper sheaf which vanishes in the smooth locus of $C$ and for $x$ in the singular locus of $C$ we have $\mathcal{F} _x  = F$. 
\end{lemma}
\todo{Do we need this lemma???}
\begin{proof}
	Intuitively, $C'$ is the disjoint union of the normalisations of the irreducible components of $C$. 
	So $C'$ consists of the same irreducible components, but they are no longer connected along the intersection points. 
	So along the smooth locus of $C$, the map $f_* \omega_{C' / F} \to \omega_{C / F}$ is an isomorphism, and thus $\mathcal{F} $ vanishes on the smooth locus of $C$. 

	Lets now consider a singular point $x$ on $C$. 
	Then $f^{-1}(x)$ are two smooth points.
	Hence the morphism on residue fields $(f_* \omega_{C' / F})_x \to (\omega_{C  /F})_x$ is $F \to F^2$ and thus $\mathcal{F} _x \simeq F$ \todo{this does not make sense?}
	
	For a more detailed argument see \cite[lem.\ 10.3.12]{liuAlgebraicGeometryArithmetic2002}. 
\end{proof}
\end{comment}

\begin{lemma}\label{lem:genus_semi_stable_curve}
	Let $C$ be a semistable curve over $F$. 
	Let $m$ be the number of singular points on $C$. 
	Let $F_1, \ldots, F_n$ be the irreducible components of $C$ and $F_1', \ldots, F_n'$ be their respective normalisations.
	Then \[
		g(C) = \beta(C) + \sum_{i = 1}^{n} g(F_i')
	\] 
	where \[
		\beta(C) = m - n + 1
	\]
	is the \emph{Betti number} of $C$. 
\end{lemma}
\begin{proof}
	Let $C'$ be the normalisation of $C$ and $f: C' \to C$ be the induced morphism. 
	We see that  $C'$ is the disjoint union of $F_1', \ldots, F_n'$. 

	There is  an exact sequence \[
	0 \to \mathcal{O}_C \to f_* \mathcal{O}_{C'} \to \mathcal{F} \to 0 
	.\] 
	where $\mathcal{F} $ is the skyscraper sheaf vanishing on the smooth points of $C$ and for every singular points $x$ we have $\mathcal{F} _x = k$.  
	Then by additivity of the Euler characteristic we find
	\begin{align*}
		\chi(\mathcal O_C) &= \chi(\mathcal O_{C'}) - \chi(\mathcal F) \\
				   &= \sum_{i = 1}^{n} \chi(\mathcal O_{F_i'}) - m\\
				   &= n -  \sum_{i = 1}^{n} g(F_i') - m\\
				   &= -\beta(C) - \sum_{i = 1}^{n} g(F_i') + 1 
	.\end{align*}
	We now get the result because \[
		g(C) = 1 - \chi(\mathcal{O}_C) = \beta(C) + \sum_{i = 1}^{n} g(F_i')  
	.\] 
\end{proof}
\begin{remark}\label{rem:genus_semi_stable_curve}
	The \emph{Betti number} can be interpreted as the number of ``holes'' in the dual graph of $C$. 
	That is the graph where the vertices are the irreducible components $F_1, \ldots, F_n$ and every singular point $x$ on $C$ is an edge between the components that contain $x$.
	If $x$ lies on only one component we add a loop in the dual graph.
	So the genus of $C$ is the sum of the topological genus of the dual graph and the genera of the irreducible components making up $C$.
	The dual graph will make another appearance in \cref{chap:weight_functions}. 
\end{remark}

\begin{definition}\label{def:semi_stable_model}
	Let  $\mathscr C$ be a proper model of a curve $C$. 
	Then $\mathscr C$ is a \emph{semistable model of $C$} if the special fiber $\mathscr C_s$ is a stable curve. 
\end{definition}

Semistable curves are nice because we can easily read of the genus of $C$ in the special fiber. 
\begin{lemma}\label{lem:genus_semi_stable_model}
	Let $\mathscr C$ be a semistable model of  $C$. 
	Let $F_1, \ldots, F_n$ be the irreducible components in the special fiber and  $F_1', \ldots, F_n'$ be their respective normalisation. 
	Then \[
		g(C) = \beta(\mathscr C_s) + \sum_{i = 1}^{n} g(F_i')
	.\] 
\end{lemma}
\begin{proof}
	As $\mathscr C \to \spec R$ is flat we have that $g(C) = g(\mathscr C_\eta) = g(\mathscr C_s)$. 
	The result now follows from \cref{lem:genus_semi_stable_curve}. 
\end{proof}

Not every curve has a semistable model.
As we will see in \cref{chap:kodaira_neron_classification_of_elliptic_curves} there are many elliptic curves without semistable models. 
\begin{definition}
	A curve $C$ over $K$ has \emph{semistable} reduction there exists a semistable model of $C$. 
\end{definition}
To make matters more complicated, a semistable model is not necessarily regular. 
Take for example the model $\mathscr P$ of $\pro^{1}_K$ that on one affine chart is given by $\spec \frac{R[x, y]}{(xy  - \pi^{2})}$. 
Then $\mathscr P_s$ is two projective lines intersection in an ordinary double.
But that double point is not regular in the larger scheme $\mathscr P$. 
Fortunately there are the following results. 
\begin{proposition}\label{prop:design_ss_model}
	Let $\mathscr C$ be a semistable model of $C$ and let $p:\mathscr C' \to \mathscr C$
	be its minimal desingularisation. 
	Then the inverse image $p^{-1}(x)$ of a singular point $x$, consists of a chain of projective lines of multiplicity $1$. 
	In particular $\mathscr C'$ is semistable. 
\end{proposition}
\begin{proof}
	This is (a consequence of) \cite[cor.\ 10.3.25]{liuAlgebraicGeometryArithmetic2002}.
\end{proof}
\begin{corollary}
	Let $C$ be a curve with  $g(C) > 0$.
	Then $C$ has semistable reduction if and only if the minimal regular model  $\mathscr C_\text{reg} $ is semistable. 
\end{corollary}
\begin{proof}
	\ltr Let $\mathscr C$ be be a semistable model of $C$. 
	Let $\mathscr C'$ be the minimal desingulariation of $C$. 
	By \cref{prop:design_ss_model} we know that  $\mathscr C'$ is semistable as well. 
	By the minimality of $\mathscr C_\text{reg} $ we see that it is a contraction of $\mathscr C'$. 
	In particular $\mathscr C_{\text{reg}, s}$ is reduced curve, and it also has ordinary double points as intersection as $\mathscr C_\text{reg} $ is regular.
	\rtl This is immediate. 
\end{proof}

We can also check whether $C$ as semistable reduction by looking at a relatively minimal nc-model. 
\begin{lemma}\label{lem:semistable_nc_model}
	$C$ has semistable reduction if and only if there exists a nc-model of  $C$ has reduced special fiber. 
\end{lemma}
\begin{proof}
	\ltr Let $ \mathscr C$ be a semistable model of $C$, and $\mathscr C'$ be its minimal desingularisation. 
	Then $\mathscr C'$ has is semistable and nc-crossings as the lines meet transversely by \cite[cor.\ 10.3.25]{liuAlgebraicGeometryArithmetic2002}
	\rtl Let  $\mathscr C$ be such nc-model. 
	Then the completion of a local ring at a singular point in $\mathscr C_s$ looks like $R\llbracket x, y \rrbracket /(xy - \pi)$, and thus is ordinary double points in $\mathscr C_s$.
\end{proof}



While not all curves have semistable reduction, we can always basechange to a field such that a given curve  $C$ has semistable reduction. 

\begin{theorem}[Semistable reduction theorem]
	Let $C$ be a curve over $K$. 
	Then there is a finite extension $L$ be a finite extension of $K$ such that $C_L  = C \times _K \spec L$ has semistable reduction, i.e.\
	it has a model $\mathscr C$ over $\mathcal{O}(L)$ where $\mathcal{O}(L)$ is the normalisation of $R$ in $L$. 
\end{theorem}
\begin{proof}
	This is a deep result with a long and involved proof. 
	The first proof was given by Deligne and Mumford in \cite{deligneIrreducibilitySpaceCurves1969}. 
	A proof using rigid analytic spaces (another theory of $K$-analytic spaces, similar to Berkovich spaces) is given in \cite{arzdorfAnotherProofSemistable2012}.
\end{proof}





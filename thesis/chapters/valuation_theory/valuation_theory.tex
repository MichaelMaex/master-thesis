On rings like $\Z$ and $\Q$ and $\C$ the absolute value function gives away of measuring how low \emph{large} each element is. 
We can axiomatize properties of an absolute value function and look for them on arbitrary rings. 
These types of functions are called (semi)norms, absolute values or valuations, depending on the specific proprieties and context.  
In \cref{chap:intro_berkovich} we will use norms as the points that make up Berkovich spaces, which are one of the main objects studied in this thesis. 
So in this chapter we will make sure that we have a solid understanding of norms, as well as to fix some definitions and conventions.

\nomenclature[qp]{$\Q_p$}{The $p$-adic rationals}
\nomenclature[zp]{$\Z_p$}{The $p$-adic integers}

\section{Norm, seminorms and valuations} \label{sec:norm,_seminorms_and_valuations}
Let $R$ be any ring. On rings like $\Z$ and $\Q$ and $\C$ the absolute value function gives away of measuring how low \emph{large} each element is. 
These types of functions are called (semi)norms, absolute values or valuations. 
\begin{definition}
	Let $R$ be a ring. A \emph{absolute value on $R$} is a function $|\cdot |: R \to \R^{+}$ satisfying for all $a, b \in R$
	\begin{enumerate}
		\item $|a| = 0 \iff a = 0$ 
		\item $|a \cdot b| = |a| \cdot |b|$ 
		\item $|a + b| \le |a| + |b|$
	\end{enumerate}
\end{definition}
Note that $R$ is not the zero ring, then $|1| = 1$ because $|1|^2 = |1| $ and $|1|\ne 0 $. 
\begin{example}
	The usual definition of absolute value on $\Z, \Q, \R$ is an absolute value. 
\end{example}
\begin{example}
	On any ring we can define the trivial absolute value as \[
	 |a| = \begin{cases}
		 1 & a \ne 0 \\
		 0 & a = 0
	 \end{cases}
	.\] 
\end{example}
\begin{example}
	A weirder absolute value on $\Z$ is the \emph{$p$-adic absolute value}, where $p$ is a prime. 
	Its defined as $|a|_p = p^{-n}$ where $n$ is the largest integers such that $p^{n} \divides a$, if $a \ne 0$ and $|0|_p = 0$. 


	This $p$-adic absolute value can be extended to an absolute value on $\Q$ by defining $|a / b|_p = \frac{|a|_p}{|b|_p}$. Let $\frac{a}{b}$ be any fraction. 
	I like to think about this in the following way. 
	Factor out the largest powers of $p$ out of $a$ and $b$ and write $a = p^{n} a', b = p^{m} b'$. Then $\frac{a}{ b} = p^{n - m} \frac{a'}{ b'}$ and $|a / b|_p = p^{m - n}$. 
	So the $p$-adic absolute values only cares about what powers of $p$ occur and doesn't care about numbers coprime to $p $.
\end{example}
As we will see in \cref{sec:berkovich_spectrum_of_Z}, we have esssentially listed all absolute values on $\Z$ and $\Q$ as examples above.  

Absolute values are a special case of (semi)norms
\begin{definition}
	Let $R$ be any ring. A \emph{seminorm on $R$} is a function $|\cdot |: R \to \R^{+}$ satisfying for all $a, b \in R$ 
	\begin{enumerate}
		\item $|0| = 0, |1| = 1$ 
		\item $|a \cdot b| \le |a | \cdot |b|$  (submultiplicativity)
		\item $|a + b| \le |a + b|$ (triangle inequality)
	\end{enumerate}
	A seminorm is called \emph{multiplicative} if it also satisfies a stronger version of 2. 
	\begin{enumerate}
		\item [4.]  $|a \cdot  b| = |a | \cdot |b|$
	\end{enumerate}
	A seminorm that is only satisfies $2$ is sometimes called a \emph{submultiplicative seminorm} if we need to stress that the norm is not required to be multiplicative. 
\end{definition}

\begin{definition}
	The \emph{kernel} of a seminorm $|\cdot |: R \to \R^{+}$ is defined as \[
		\ker(|\cdot |) = \{a \in \R \st |a| = 0\} 
	.\] 
\end{definition}
\begin{remark}\label{rem:ker_norm_ideal}
	Let $ |\cdot |$ be a norm on $R$. Then $\ker |\cdot |$ is an prime ideal of $R$. If $|\cdot |$ is multiplicative then $\ker |\cdot |$ is even a prime ideal!
\end{remark}
\begin{proof}
	Let $a, b \in \ker |\cdot |, c \in R$. Then $0 \le |a + b| \le |a|+|b| = 0$. So  $a + b \in \ker |\cdot |$. 
	Also $0 \le |c \cdot a| \le |c|\cdot |a| = 0$. So $c \cdot a \in \ker |\cdot |$. 
	This shows that $\ker |\cdot  |$ is an ideal. 
	
	Suppose that $|\cdot |$ is multiplicative. 
	Let $a, b\in R $ such that $a\cdot b \in \ker |\cdot |$ then  $|a|\cdot |b| = |a\cdot b| = 0$. 
	Hence either $|a| = 0$ or  $|b|=0$. 
	So either  $a \in \ker |\cdot |$ or $b \in \ker |\cdot |$. 
\end{proof}

In some sense seminorms are analogous to ideals and and multiplicative seminorsm are analogous to prime ideals. This is will be the basis of Berkovich spectra of Rings.
Like the ordinary $\spec$ functor we can define the space of all multiplicative norms on a ring. 
But hold your horses trying to define a suitable zarisky topology on this space. 
That won't work. 
In fact we will need two different topologies on this space for the theory to work. 
Don't worry, this is a feature, not a bug. 
One topology will have all the nice properties you can think of: Haussdorf, compact, path-connected, locally contractible, \ldots while the other one is great for sheaf theory. 


\begin{definition}
	Let $R$ be a ring and $|\cdot |$ are seminorm. 
	Then $|\cdot |$ is called a \emph{norm} if $\ker |\cdot | = 0$, i.e.\ $|a| = 0 \iff a = 0$. 
\end{definition}

So an absolute value is a multiplicative norm. 

\begin{lemma}
	Any seminorm $|\cdot |$ on a field $F$ is a norm, i.e.\ has trivial kernel. 
\end{lemma}
\begin{proof}
	We have that $|1| = 1$. So $1 \not\in \ker |\cdot |$ and thus $\ker |\cdot |$ is an ideal of $F$ different from $F$. 
	As $F$ is a field this means that $\ker |\cdot | = (0)$. So $|\cdot |$ is norm. 
\end{proof}

\begin{definition}
	A (semi)norm $\|\cdot \|$ on $R$ is called \emph{non-archimedean} if it satisies the following stronger version of the triangle inequality, \[
	\forall a, b \in \R: \|a + b\| \le \max \{\|a\|, \|b\|\} 
	.\] 

	Confusingly a \emph{Archimedean (semi)norm} is a norm that is not non-archimedean.
\end{definition}

\begin{exercise}[\cite{nicaiseNONARCHIMEDEANGEOMETRY}]
	Let $k$ be a field.
	Show that a absolute value  $|\cdot |$ on $k$ is non-Archemedean if and only if the image of $\Z$ in $k$ is bounded. 
\end{exercise}
\begin{proof}
	Suppose $|\cdot |$ is non-archimedean. Then for any positive integers $n$ we have \[
	|n| = \left|\sum_{i = 1}^{n} 1 \right| \le \max \{1, 1, \ldots, 1\}  = 1
	.\] 
	And for negative integers we have $|n| = |-n| \le 1$. So $\|\cdot \|$ is bounded on the image of $\Z$ by $1$. 


	Suppose that $|\cdot |$ is bounded on the image of $\Z$ by $c$. 
	Let $x, y \in k$. Then \[
		|(x + y)^{n}| = \left| \sum_{i = 0}^{n} \binom{i}{n} x ^{i} y ^{n-i}\right| \le n\cdot c \max(x, y)^{n}
	.\] 
	So \[
		|x + y| \le \sqrt[n]{c\cdot n}  \max(x, y)
	.\] 
	Taking $n \to \infty$ yields the non-archimedean triangle inequality. 
\end{proof}

\begin{corollary}
	Let $A \subset  B$ an inclusion of rings  equipped with multiplicative such that the norm on $B$ extends the norm on $A$. 
	Then $A$ is non-archimedean if and only if $B$ is non-archimedean. 
\end{corollary}

\begin{remark}
	Suppose $R$ is non-archimedean and we have $a, b \in R$ such that $|a| \ne |b|$ then the triangle inequality is an equality \[
	|a + b| = \max \{|a|, |b|\} 
	.\] 
	So in some sense equality is the generic case. 
\end{remark}
It may sound like a really strong property to have, and that not many norms will be non-archimedean. 
But as you will see in this thesis, most norms that pop up are non-archimedean. 
In fact the only complete valued fields with archimedean norms are $\R$ and $\C$ \todo{find reference}.



\begin{definition}
	equivalent norm\todo{define this}
	{\color{red} This is a problem because equivalent norms and equivalent valuations are different as far as I understand}
\end{definition}

Note that the definition a non-archimedean norm only depends on the multiplicative structure on $\R^{+}$ and no longer depends on the additive structure. 
Sometimes it is more natural define norms in terms the isomorphic group $\R, +$. 
These are typically called valuationas. 


\begin{definition}
	Let $R$ be a ring. A valution on $R$ is a function $v:R \to \R \cup \{\infty\} $ such that 	
	\begin{itemize}
		\item $v(0) = \infty$, $v(1) = 0$. 
		\item $v(a\cdot b) = v(a) + v(b)$
		\item $v(a + b) \le \min\{v(a), v(b)\}$
	\end{itemize}
\end{definition}

\begin{remark}
	Note that there is essentially no difference between a non-archimedean absolute value and a valuation because one can take for absolute value $|\cdot |$ defines a valution $x \mapsto -\log |x|$ and a valution $v$ defines a non-archimedean absolute value $x\mapsto e^{- v(x)}$. 
\end{remark}



\subsection{Non-Archimedean rings and fields} \label{sec:non-archimedean_rings_and_fields}


\begin{theorem}\label{thm:norm_finite_field_ext}
	If $L$ is a finite field extension of a complete non-archimedean field $K$, then there is a unique extension of the norm of $K$ to $L$ and this extension is also non-archimedean.
\end{theorem}
\begin{proof}
	This a very technical result and for the proof we refer to \cite[][appendix A]{boschLecturesFormalRigid2014}. 
\end{proof}

\begin{corollary}
	If $K$ is a complete non-archimedean field then the norm of  $K$ extends uniquely to its algebraic closure $\overline{K}$.
\end{corollary}
\begin{proof}
	Recall that \[
	\overline{K} = \varprojlim_{L \supset K} L
	.\] 
	where $L$ ranges over all algebraic extensions of $K$. 
	As all these algebraic extensions have unique norms extending  the norm on $K$ and for $L_1 \supset L_2 \supset K$ the norm on  $L_1$ must restrict to norm on $L_2$ by uniqueness, it is clear that the norm functions $L \to \R^{+}$ glue to a map $\|\cdot \|: \overline{K} \to \R^{+}$. 

	It is easy to verify that this is a norm on $\overline{K}$ extending the norm on $K$. 
\end{proof}

\begin{theorem}
	[Krasner's Lemma]
	Let $K$ be a non-archimedean algebraically closed field. 
	Then its completion, $\widehat K$ is algebraically closed as well. 
\end{theorem}
\begin{proof}
	This follows from the continuity of roots. See \cite[][lem A.6]{boschLecturesFormalRigid2014} for a detailed argument. 
\end{proof}

This show that we can cannonically turn any non-archimedean field $K$ into an algebraically closed field by considering $\widehat{\overline{K}}$, which is first taking the algebraic closure of $K$ and then completing it. 

\begin{definition}
	For a prime $p$ we define the \emph{p-adic complex numbers} as \[
	\C_p = \widehat{\overline{\Q_p}}
	.\] 
\end{definition}

This is useful because Berkovich geometry works best when the base field is algebraically closed and complete. 

\begin{definition}
	Let $R$ be a non-archimedean ring. Then we write 
	\begin{align*}
		R^{0} &= \{a \in k \st |k| \le 1\}  \\
		R^{00} &=  \{a \in k \st |k| < 1\}  \\
		\tilde R &= R^{0} / R^{00}
	.\end{align*}
\end{definition}
If $k$ is a field then $k^{0}$ is a valuation ring and $ k^{00}$ is its maximal ideal. 

We will useally write $K$ for a non-arcimedean field, $R = K^{0}$ for its valuation ring, $k = \tilde K$ for its residue field. 

\begin{definition}
	Let $K$ be a non-archimedean ring. 
	The \emph{value group of $K$} is \[
	|K| = \{|a| \st \} 
	.\] 
\end{definition}


\subsection{Ultrametric spaces} \label{sec:ultrametric_spaces}

\begin{definition}
	A \emph{ultrametric space} is a topological spaces $(X, d)$ where the metric satisfies the non-archimedean triangle inequality. 
	\[
		d(x, y) \le \max \{d(x, z) ,d(z, y)\}, \quad \forall x, y ,z \in X
	.\] 
\end{definition}
\begin{exercise}
	If $d(x,z) \ne d(z,y)$ the inequality is an equality. 
\end{exercise}
\begin{proof}
	Suppose without loss of generality that $d(x, z) \ge d(z,y)$, so $d(x, y) \le \max \{d(x, z), d(z,y)\} = d(x, z) $. 
	Then $d(x, z) \le \max \{d(x, y), d(y,z)\}$. So $d(x, z) \le d(x, y) \le d(x, z)$. So we find equality.
\end{proof}
\begin{corollary}
	Any point of a ball (open or closed) is a center for that ball. 
\end{corollary}
\begin{corollary}
	If two balls have non-empty intersection that one must be included in the other. 
\end{corollary}
\begin{corollary}
	The topology of $X$ is totally disconneced. 
\end{corollary}

Non-archimedean fields are ultrametric spaces with the metric $d(x, y) = |x - y|$. 



\section{Maps between discrete valuation rings} \label{sec:maps_between_discrete_valuation_rings}
In this section we gather results relating norms on fields behave under finite extension, maps between discrete valuation rings, and normalisations of rings of integers. 

\begin{definition}
	A \emph{discrete valuation ring (DVR)} is a local ring that is also a principal ideal domain. 
\end{definition}
So discrete valuation rings are one dimensional noetherian local rings. 
In particular, for regular or normal schemes, the local ring at a point of codimension $1$ is a DVR.
If $R$ is a DVR with maximal ideal $\mathfrak{m} $, then $\mathfrak{m} $ is generated by one element. 
\begin{definition}
	Let $R$ be a DVR, and $\pi \in R$ such that $\mathfrak{m}  = (\pi)$. 
	Then $\pi$ is called a \emph{uniformiser of $R$}. 
\end{definition}
It is clear that any two uniformisers are the same up to multiplication by a unit in $R$. 
\begin{lemma}
	Let $K$ be a field, equipped with a discrete valuation $v$. 
	Then its valuation ring $K^{o} = \{a \in K\st v(a) \ge 0\} $ is a DVR. 
\end{lemma}
\begin{proof}
	Let $\pi$ be element in ${K^{o}}^{\times }$ of minimal valuation. 
	Clearly $\pi \in \mathfrak{m} $. 
	Let $a$ be any non-zero element in $\mathfrak{m} $. 
	Then $v(a/ \pi) = v(a) - v(\pi) \ge 0$ by the minimality assumption on $\pi$. 
	Thus $a / \pi \in K^{o}$ and $a \in (\pi)$. 
	So $(\pi) = \mathfrak{m} $. 
\end{proof}

Conversely a DVR  $R$ is always the valuation ring of its field of fractions $\mathrm{Frac}(R)$. 
As schemes DVR's consists just of two points, the generic point, and the unique closed point, with residue fields $\mathrm{Frac}(R)$ and $R / \mathfrak{m} $ respectively. 

\begin{definition}
	Let $A, B$ be DVRs with maximal ideals $\mathfrak{m}_A, \mathfrak{m} _B$ and valuations $v_a, v_b$ respectively.
	Let $f: A \to B$ be a local ring morphism (i.e. $f^{-1}(\mathfrak{m} _B) = \mathfrak{m} _A)$), then the \emph{ramification degree of $f$} or the \emph{ramification degree of $B$ over $A$} is the number \[
		e_f = e_{B / A}= v_B(f(\pi))
	,\] 
	where $\pi$ is a uniformiser for $A$. 
\end{definition}

\begin{definition}
	Let $K$ be a discretely valued field and $L$ a finite extension of $K$, such that the valuation on $L$ extends the valuation on $K$. 
	Then the \emph{ramfication of  $L$ over  $K$} is the ramification degree of the induced map between valuation rings, or equivalently it is 
	\[
		e_{L / K}  = v_L(\pi)
	,\] 
	where $\pi$ is a uniformiser of $K$. 
\end{definition}

It is easy to see that the ramification degree of a map between DVRs / discretely valued fields, $f: A \to B$ can be computed as $v_B(f(a)) / v_A(a)$ where $a$ is any element of $A$ with nonzero valuation. 
\medskip 

\begin{definition}
	Let $f: A \to B$ be a local map between DVRs. 
	Let $k, \ell$ be the residue fields of $A, B$ respectively and $p = \ch k$. 
	Then the ramification of $f$ is called \emph{tame} if $p \nmid e_f$  and it is said to be \emph{wild} if $p \nmid e_f$. 
\end{definition}
As we will see later in this thesis, morphisms between DVRs with tame ramification degree are much better behaved than those with wild ramification. 

\begin{lemma}\label{lem:generator_canonical_bundle_DVR}
	Let $f: A \into B$ be a map of DVRs with respective fraction fields $K, L$ such that $L / K$ is a finite separable extension. 
	Let $\pi$ be a uniformiser of $B$. 
	Then $d(\pi)$ is a generator of $\Omega_{B / A}$. 
\end{lemma}
\begin{proof}
	Any element $b \in B$ can be written as $b = u \cdot \pi^{n} $ for some unit $u \in B$ and $n \ge 0$. 
	Let $f = \sum_{i = 0}^{n}  a_i x^{i} \in K[x]$ be a monic polynomial such that $f(u) = 0, f'(u) \ne 0$. 
	Then  \[
		0 = v(u^{n}) = v(- \sum_{i = 0}^{n-1}) \le v(a_i u^{i}) = v(a_i) 
	.\] 
	So $f \in R[x]$. 
	Then we compute differentials \[
		0 = d(0) = d(f(u)) = f'(u) d(u)
	.\] 
	As $f'(u) \ne 0$ we find $d(u)$ must be zero. 
	Then  \[
		d(b) = d(u\cdot \pi^{n}) = \pi^{n}\cdot d(u) + u\cdot  n \pi^{n-1}d(\pi) = un\pi^{n-1} d(\pi)
	.\] 
	So we find that $d(b)$ is a multiple of $d(\pi)$. 
	The result now follows because $\Omega_{B / A}$ is generated by $\{d(b) \st b \in B\} $.  
\end{proof}

The following lemma shows that computing the normalization of a DVR $R$ in a field $L$, is really the same as finding all the valuations on $L$ that extend the valuation on $R$. 
\begin{lemma}\label{lem:normalisation_extension_norm}
	Let $K$ be a discretely valued field with valuation $v$, and $L$ a finite extension of $K$. 
	Let $R = K^{o}$ be the value field and $R'$ be the normalisation of $R$ in $L$. 
	Then there is a natural bijection between the maximal ideals of $R'$ and the extensions $v $ to $L$. 
\end{lemma}
\begin{proof}
	As normalisation is finite and local we know that $R'$, as a scheme has a generic points with fraction field is $L$, and a finite number of closed points, corresponding to maximal ideals $\mathfrak{m}_1, \ldots, \mathfrak{m} _n $ in $R'$. 
	Let $x_1, \ldots, x_n$ be the corresponding closed points in $\spec R'$. 
	Then $\mathcal{O}_{x_i, \spec R'}$ is a DVR with fraction field $L$, and thus induces a valuation on $L$. 
	We easily see that (with correct scaling) this valuation extends the valuation on $K$. 

	Conversely, let $v_L$ be a valuation on $L$ extending the valuation on $K$. 
	Let $b$ be an element in $R'$, and $f = \sum_{i = 0}^{r} a_i t^{i}$ be a monic polynomial in $R[t]$ such that $f(b) = 0$. 
	Then \[
		rv_L(b) = v_L(a^{b}) = v_L \left(-\sum_{i = 0}^{r-1} a_i b^{i}\right) \ge \min_{i} v(a_i) + iv_b(b)
	.\] 
	From this it follows that $v_L(b) \ge 0$. 
	So $R' \subset L^{o}$.  

	Let $\mathfrak{p}  = \{a \in R' \st v_L(a) > 0\} $. 
	Then clearly $\mathfrak{p} $ is an ideal of $R'$. 
	As $R \to R'$ is finite, $\mathfrak{p} $ is a maximal ideal of $R$. 
\end{proof}

\begin{lemma}
	Let $K$ be a complete discretely valued field and $L$ be a finite extension of $K$.	Let $R = K^{o}$  be the value field of $K$ and $R'$ be the normalisation of $R$ in $L$. 
	Then $R' = L^{o}$. 
\end{lemma}
\begin{proof}
	These maximal ideals induce different discrete valuations on $R'$ that, which extend to different valuations on the fraction field $L$. 
	As the extension of the norm to $L$ is unique (\cref{thm:norm_finite_field_ext}) we see that $R'$ has exactly one maximal ideal $\mathfrak{m}_{R'}$. 
	In particular the valuation induced by $\mathfrak{m}_{R'} $ is the same as the one induced by $L^{oo} \subset L^{o}$. 
	So both $R'$ and $L^{o}$ are the rings of all elements with non-negative valuation in $L$. 
	\todo{work this out some more.}
\end{proof}

\begin{proposition}\label{prop:balancing_valuations}
	Let $K$ be a discretely valued field with valuation $v_K$,  value ring $R$ and residue fields $k$. 
	Let $L$ be a finite extension of $K$ and $v_1, \ldots, v_n$ be the extensions of $v_R$ to $L$. 
	Let $L_1^{o}, \ldots, L_n^{o}$ be the value rings of $L$ equiped with the norms $v_1, \ldots, v_n$ respectively, and let $\ell_1, \ldots, \ell_n$ be their respective residue fields. 
	Then \[
		[L: K] = \sum_{i= 1}^{n} e_{L_i^{o} / R} \cdot [\ell_i: k]
	.\] 
\end{proposition}
\begin{proof}
	Let $\mathfrak{m} = (\pi) $ be the maximal ideal of $R$
	Let $R'$ be the normalisation of $R$ in $L$. 
	Then this is exactly \stacks{09E8}.
\end{proof}




 



On rings like $\Z$ and $\Q$ and $\C$ the absolute value function gives a way of measuring how \emph{large} each element is. 
We can axiomatize properties of an absolute value function and look for them on arbitrary rings. 
These types of functions are called (semi)norms, absolute values or valuations, depending on the specific proprieties and context.  
In \cref{chap:intro_berkovich} we will use norms as the points that make up Berkovich spaces, which are one of the main objects studied in this thesis. 
So in this chapter we will make sure that we have a solid understanding of norms, as well as to fix some definitions and conventions.

\nomenclature[qp]{$\Q_p$}{The $p$-adic rationals}
\nomenclature[zp]{$\Z_p$}{The $p$-adic integers}

\section{Norm, seminorms and valuations} \label{sec:norm,_seminorms_and_valuations}
On rings like $\Z$ and $\Q$ and $\C$ the absolute value function gives away of measuring how low \emph{large} each element is. 
We can axiomatize proterties of an absolute value function and look for them on arbitrary rings. 
These types of functions are called (semi)norms, absolute values or valuations, depending on the specific propreties and context.  
\begin{definition}
	Let $R$ be a ring. A \emph{absolute value on $R$} is a function $|\cdot |: R \to \R^{+} \cup \{0\} $ satisfying for all $a, b \in R$
	\begin{enumerate}
		\item $|a| = 0 \iff a = 0$ 
		\item $|a \cdot b| = |a| \cdot |b|$ 
		\item $|a + b| \le |a| + |b|$
	\end{enumerate}
\end{definition}
Note that $R$ is not the zero ring, then $|1| = 1$ because $|1|^2 = |1| $ and $|1|\ne 0 $. 
\begin{example}
	The usual definition of absolute value on $\Z, \Q, \R$ is an absolute value. 
\end{example}
\begin{example}
	On any integral domain we can define the trivial absolute value as \[
	 |a| = \begin{cases}
		 1 & a \ne 0 \\
		 0 & a = 0
	 \end{cases}
	.\] 
\end{example}
\begin{example}
	A weirder absolute value on $\Z$ is the \emph{$p$-adic absolute value}, where $p$ is a prime. 
	Its defined as $|a|_p = p^{-n}$ where $n$ is the largest integers such that $p^{n} \divides a$, if $a \ne 0$ and $|0|_p = 0$. 


	This $p$-adic absolute value can be extended to an absolute value on $\Q$ by defining $|a / b|_p = \frac{|a|_p}{|b|_p}$. Let $\frac{a}{b}$ be any fraction. 
	I like to think about this in the following way. 
	Factor out the largest powers of $p$ out of $a$ and $b$ and write $a = p^{n} a', b = p^{m} b'$. Then $\frac{a}{ b} = p^{n - m} \frac{a'}{ b'}$ and $|a / b|_p = p^{m - n}$. 
	So the $p$-adic absolute values only cares about what powers of $p$ occur and doesn't care about numbers coprime to $p $.
\end{example}
As we will see in \cref{sec:the_berkovich_spectrum_of_z}, we have esssentially listed all absolute values on $\Z$ and $\Q$ as examples above.  

\begin{proposition}
	A ring $R$ with an absolute value is a metric space with metric $d(x, y) = |x - y|$. 
\end{proposition}
\begin{definition}
	A ring $R$ with absolute value is \emph{complete} if all Cauchy sequences are covergent. 
\end{definition}
\begin{definition}
	Any ring $R$ with absolute value can be completed, which is defined as \[
		\widehat{R} = \{(a_n)_n \text{ Cauchy sequence in } R\} / \sim
	\] 
	where two sequence $(a_n)_n, (b_n)_n$ are equivalent if $(a_n - b_n)_n \to 0$ as $n \to \infty$.
	Given  $a, b \in R$ with representivis $(a_n)_n, (b_n)_n$, the addition and multiplication are defined as $a + b = [(a_n + b_n)], a\cdot b = [(a_n \cdot b_n)]$. 
\end{definition}

Absolute values are a special case of (semi)norms
\begin{definition}
	Let $R$ be any ring. A \emph{seminorm on $R$} is a function $|\cdot |: R \to \R^{+} \cup \{0\} $ satisfying for all $a, b \in R$ 
	\begin{enumerate}
		\item $|0| = 0, |1| = 1$ 
		\item $|a \cdot b| \le |a| \cdot |b|$  (submultiplicativity)
		\item $|a + b| \le |a| + |b|$ (triangle inequality)
	\end{enumerate}
	A seminorm is called \emph{multiplicative} if it also satisfies a stronger version of 2. 
	\begin{enumerate}
		\item [4.]  $|a \cdot  b| = |a| \cdot |b|$
	\end{enumerate}
\end{definition}
Sometimes we will stress the fact that a (semi)norm might not be multiplicative and call it a \emph{submultiplicative norm}. 

\begin{lemma}
	If $|\cdot |$ is a multiplicative seminorm on $R$ then  $|a | = |-a|$ for all $a \in R$. 
\end{lemma}
\begin{proof}
	We have $|a|^2 = |a^2| = |-a|^2$. As all norms are positive we find that $|a| = |-a|$. 
\end{proof}


\begin{definition}
	The \emph{kernel} of a seminorm $|\cdot |: R \to \R^{+}$ is defined as \[
		\ker(|\cdot |) = \{a \in \R \st |a| = 0\} 
	.\] 
\end{definition}
\begin{remark}\label{rem:ker_norm_ideal}
	Let $ |\cdot |$ be a norm on $R$. Then $\ker |\cdot |$ is an ideal of $R$. If $|\cdot |$ is multiplicative then $\ker |\cdot |$ is even a prime ideal!
\end{remark}
\begin{proof}
	Let $a, b \in \ker |\cdot |, c \in R$. Then $0 \le |a + b| \le |a|+|b| = 0$. So  $a + b \in \ker |\cdot |$. 
	Also $0 \le |c \cdot a| \le |c|\cdot |a| = 0$. So $c \cdot a \in \ker |\cdot |$. 
	This shows that $\ker |\cdot  |$ is an ideal. 
	
	Suppose that $|\cdot |$ is multiplicative. 
	Let $a, b\in R $ such that $a\cdot b \in \ker |\cdot |$ then  $|a|\cdot |b| = |a\cdot b| = 0$. 
	Hence either $|a| = 0$ or  $|b|=0$. 
	So either  $a \in \ker |\cdot |$ or $b \in \ker |\cdot |$. 
\end{proof}

In some sense seminorms are analogous to ideals and and multiplicative seminorsm are analogous to prime ideals. This is will be the basis of Berkovich spectra of rings.
Like the ordinary $\spec$ functor we can define the space of all multiplicative norms on a ring. 
Unfortunately you can't just copy all constructions we know for $\spec$. 
For example, we can't define a Zarisky topology on this set. 
In fact we will need two different topologies on this space for the theory to work. 
Don't worry, this is a feature, not a bug. 
One topology will have all the nice properties you can think of: Haussdorf, compact, path-connected, locally contractible, \ldots while the other one is great for sheaf theory. 


\begin{definition}
	Let $R$ be a ring and $|\cdot |$ a seminorm on $R$. 
	Then $|\cdot |$ is called a \emph{norm} if $\ker |\cdot | = 0$, i.e.\ $|a| = 0 \iff a = 0$. 
\end{definition}

So an absolute value is a multiplicative norm. 

\begin{lemma}
	Any seminorm $|\cdot |$ on a field $F$ is a norm, i.e.\ has trivial kernel. 
\end{lemma}
\begin{proof}
	We have that $|1| = 1$. So $1 \not\in \ker |\cdot |$ and thus $\ker |\cdot |$ is an ideal of $F$ different from $F$. 
	As $F$ is a field this means that $\ker |\cdot | = (0)$. So $|\cdot |$ is norm. 
\end{proof}

\subsection{Non-Archimedean rings and fields} \label{sec:non-archimedean_rings_and_fields}
\begin{definition}
	A (semi)norm $|\cdot |$ on $R$ is called \emph{non-archimedean} if it satisies the following stronger version of the triangle inequality, \[
	\forall a, b \in \R: |a + b| \le \max \{|a|, |b|\} 
	.\] 

	Confusingly a \emph{Archimedean (semi)norm} is a norm that is not non-archimedean.
\end{definition}

\begin{definition}
	A \emph{non-archimedean ring} (resp.\ \emph{field}) is a ring (resp.\ field) equipped with multiplicative non-archimedean norm. 
\end{definition}

\begin{exercise}[\cite{nicaiseNONARCHIMEDEANGEOMETRY}]
	Let $k$ be a field.
	Show that a multiplicative seminorm $|\cdot |$ on $k$ is non-archimedean if and only if the image of $\Z$ in $k$ is bounded. 
\end{exercise}
\begin{proof}
	Suppose $|\cdot |$ is non-archimedean. Then for any positive integers $n$ we have \[
	|n| = \left|\sum_{i = 1}^{n} 1 \right| \le \max \{1, 1, \ldots, 1\}  = 1
	.\] 
	And for negative integers we have $|n| = |-n| \le 1$. So $\|\cdot \|$ is bounded on the image of $\Z$ by $1$. 


	Suppose that $|\cdot |$ is bounded on the image of $\Z$ by $c$. 
	Let $x, y \in k$. Then \[
		|(x + y)^{n}| = \left| \sum_{i = 0}^{n} \binom{i}{n} x ^{i} y ^{n-i}\right| \le (n + 1)\cdot c \max(x, y)^{n}
	.\] 
	So \[
		|x + y| \le \sqrt[n]{c\cdot (n + 1)}  \max(x, y)
	.\] 
	Taking $n \to \infty$ yields the non-archimedean triangle inequality. 
\end{proof}

\begin{corollary}
	Let $A \subset  B$ an inclusion of rings  equipped with multiplicative norms such that the norm on $B$ extends the norm on $A$. 
	Then $A$ is non-archimedean if and only if $B$ is non-archimedean. 
\end{corollary}

\begin{remark}
	Suppose $R$ is a multiplicative non-archimedean norm and we have $a, b \in R$ such that $|a| \ne |b|$ then the triangle inequality is an equality \[
	|a + b| = \max \{|a|, |b|\} 
	.\] 
	So in some sense equality is the generic case. 
\end{remark}
\begin{proof}
	Suppose without loss of generality that  $|a| > |b|$, and assume by contradiction that $|a + b| < \max \{|a|, |b|\}  = |a|$. 
	Then \[
		|a| = |(a + b) - b| \le \max \{|a + b|, |- b|\} = \max \{|a + b|, |b|\}  
	.\] 
	But this contradicts that $|a| > |b|$ and $|a| > |a + b| $. 
\end{proof}
The non-archimedean triangle inequality sounds like a really strong property and thus it might seem that not many norms will be non-archimedean. 
But as you will see in this thesis, most norms that pop up are non-archimedean. 
In fact the only complete valued fields with archimedean norms are $\R$ and $\C$.\footnote{This is a consequence of Gel'fand-Mazur theorem and Ostowki's classification (see \cref{thm:ostrowksi}) of norms on $\Q$.}



Note that the definition a non-archimedean norm only depends on the multiplicative structure on $\R^{+}$ and no longer depends on the additive structure. 
Sometimes it is more natural define norms in terms the isomorphic group $\R, +$ (and add $\infty$ instead of $0$ to the group). 
These are typically called valuations. 


\begin{definition}
	Let $R$ be a ring. A \emph{valuation on $R$} is a function $v:R \to \R \cup \{\infty\} $ such that 	
	\begin{itemize}
		\item $v(0) = \infty$, $v(1) = 0$. 
		\item $v(a\cdot b) = v(a) + v(b)$
		\item $v(a + b) \le \min\{v(a), v(b)\}$
	\end{itemize}
\end{definition}

\begin{remark}
	Note that there is essentially no difference between a non-archimedean absolute value and a valuation because because they are in bijective correspondence with one another.  For absolute value $|\cdot |$ defines a valution $x \mapsto -\log |x|$ and given a valution $v$ one defines a non-archimedean absolute value $x\mapsto e^{- v(x)}$. 
\end{remark}





\begin{theorem}\label{thm:norm_finite_field_ext}
	If $L$ is a finite field extension of a complete non-archimedean field $K$, then there is a unique extension of the norm of $K$ to $L$ and this extension is also non-archimedean.
\end{theorem}
\begin{proof}
	This a very technical result and for the proof we refer to \cite[][appendix A]{boschLecturesFormalRigid2014}. 
\end{proof}

\begin{corollary}
	If $K$ is a complete non-archimedean field then the norm of  $K$ extends uniquely to its algebraic closure $\overline{K}$.
\end{corollary}
\begin{proof}
	Recall that \[
	\overline{K} = \varprojlim_{L \supset K} L
	.\] 
	where $L$ ranges over all algebraic extensions of $K$. 
	As all these algebraic extensions have unique norms extending  the norm on $K$ and for $L_1 \supset L_2 \supset K$ the norm on  $L_1$ must restrict to norm on $L_2$ by uniqueness, it is clear that the norm functions $L \to \R^{+}$ glue to a map $\|\cdot \|: \overline{K} \to \R^{+}$. 

	It is easy to verify that this is a norm on $\overline{K}$ extending the norm on $K$. 
\end{proof}

\begin{theorem}
	[Krasner's Lemma]
	Let $K$ be a non-archimedean algebraically closed field. 
	Then its completion, $\widehat K$ is algebraically closed as well. 
\end{theorem}
\begin{proof}
	This follows from the continuity of roots. See \cite[][lem A.6]{boschLecturesFormalRigid2014} for a detailed argument. 
\end{proof}

This show that we can cannonically turn any non-archimedean field $K$ into an algebraically closed and complete field in a cannonical way by considering $\widehat{\overline{K}}$, which is first taking the algebraic closure of $K$ and then completing it. 

\begin{definition}
	For a prime $p$ we define the \emph{p-adic complex numbers} as \[
	\C_p = \widehat{\overline{\Q_p}}
	.\] 
\end{definition}

This is useful because Berkovich geometry works best when the base field is algebraically closed and complete. 

\begin{definition}
	Let $R$ be a non-archimedean ring. Then we write 
	\begin{align*}
		R^{o} &= \{a \in k \st |k| \le 1\}  \\
		R^{oo} &=  \{a \in k \st |k| < 1\}  \\
		\widetilde R &= R^{o} / R^{oo}
	.\end{align*}
\end{definition}
If $k$ is a field then $k^{0}$ is a valuation ring and $ k^{00}$ is its maximal ideal. 

We will useally write $K$ for a non-arcimedean field, $R = K^{0}$ for its valuation ring, $k = \widetilde K$ for its residue field. 

\begin{definition}
	Let $K$ be a non-archimedean field. 
	The \emph{value group of $K$} is \[
	|K^{\times }| = \{|a| \st  a \in K^{\times }  \} 
	.\] 
	This is a subgroup of $(\R^{ +}, \cdot )$. 
\end{definition}


\subsection{Ultrametric spaces} \label{sec:ultrametric_spaces}

\begin{definition}
	A \emph{ultrametric space} is a topological spaces $(X, d)$ where the metric satisfies the non-archimedean triangle inequality. 
	\[
		d(x, y) \le \max \{d(x, z) ,d(z, y)\}, \quad \forall x, y ,z \in X
	.\] 
\end{definition}
\begin{exercise}
	If $d(x,z) \ne d(z,y)$ the inequality is an equality. 
\end{exercise}
\begin{proof}
	Suppose without loss of generality that $d(x, z) \ge d(z,y)$, so $d(x, y) \le \max \{d(x, z), d(z,y)\} = d(x, z) $. 
	Then $d(x, z) \le \max \{d(x, y), d(y,z)\}$. So $d(x, z) \le d(x, y) \le d(x, z)$. So we find equality.
\end{proof}
\begin{corollary}
	Any point of a ball (open or closed) is a center for that ball. 
\end{corollary}
\begin{corollary}
	If two balls have non-empty intersection that one must be included in the other. 
\end{corollary}
\begin{corollary}
	The topology of $X$ is totally disconneced. 
\end{corollary}

Non-archimedean fields are ultrametric spaces with the metric $d(x, y) = |x - y|$. 



\section{Maps between discrete valuation rings} \label{sec:maps_between_discrete_valuation_rings}
In this section we gather results relating to how norms on fields behave under finite extension, maps between discrete valuation rings, and normalisations of rings of integers. 

\begin{definition}
	A \emph{discrete valuation ring (\gls{DVR})} is a local ring that is also a principal ideal domain. 
\end{definition}

So discrete valuation rings are one dimensional local rings. 
In particular, for regular or normal schemes, the local ring at a point of codimension $1$ is a DVR.
If $R$ is a DVR with maximal ideal $\mathfrak{m} $, then $\mathfrak{m} $ is generated by one element. 

DVRs get their name because there is a canonical valuation on them. 
\begin{definition}
	Let $R$ be a DVR with maximal ideal $\mathfrak{m}$. 
	Then we define \[
		v(a) = \max \{n \in \Z_{\ge 0 } \st a \in \mathfrak{m} ^{n} \} 
	,\] 
	for all $a \in R, a \ne 0$. 
\end{definition}
It is a good exercise to check that this is a well defined valuation. 
\begin{definition}
	Let $R$ be a DVR, and $\pi \in R$ such that $\mathfrak{m}  = (\pi)$. 
	Then $\pi$ is called a \emph{uniformiser of $R$}. 
\end{definition}
It is clear that the uniformiser $\pi$ is unique up to multiplication by a unit in $R$. 
\begin{remark}
	In the definition above we see that $v(\pi) = 1$. 
	This is the usual convention, but sometimes $v$ might be scaled differently. 
\end{remark}

\begin{lemma}
	Let $K$ be a field, equipped with a discrete valuation $v$. 
	Then its valuation ring $K^{\circ} = \{a \in K\st v(a) \ge 0\} $ is a DVR. 
\end{lemma}
\begin{proof}
	Any element $a \in K^{\circ} \setminus K^{\circ \circ} $ has $v(a) = 0$. 
	So $v(a^{-1}) = 0$, thus $a \in K^{\circ}$ as well. 
	So  $a$ is a unit of $K^{\circ}$. 
	From this it follows that $K^{\circ\circ}$ is a unique maximal ideal. 
	
	It remains to show that $K^{\circ \circ }$ is principal.  
	Let $\pi$ be element in ${K^{\circ \circ}}$ of minimal valuation. 
	Let $a$ be any non-zero element in $K^{\circ \circ}$. 
	Then $v(a/ \pi) = v(a) - v(\pi) \ge 0$ by the minimality assumption on $\pi$. 
	Thus $a / \pi \in K^{o}$ and $a \in (\pi)$. 
	So $(\pi) = K^{\circ \circ} $. 
\end{proof}

Conversely a DVR  $R$ is always the valuation ring of its field of fractions $\mathrm{Frac}(R)$. 
The spectrum of a DVR consists just of two points, the generic point, and the unique closed point, with residue fields $\mathrm{Frac}(R)$ and $R / \mathfrak{m} $ respectively. 

\begin{definition}
	Let $A, B$ be DVRs with maximal ideals $\mathfrak{m}_A, \mathfrak{m} _B$ and valuations $v_A, v_B$ respectively.
	Let $f: A \to B$ be a local ring morphism (i.e. $f^{-1}(\mathfrak{m} _B) = \mathfrak{m} _A)$), then the \emph{ramification index of $f$} or the \emph{ramification index of $B$ over $A$} is the number \[
		e_f = e_{B / A}= \frac{v_B(f(\pi))}{v_B(\tau)}
	,\] 
	where $\pi, \tau$ are uniformiser for $A, B$ respectively. 
\end{definition}

\begin{definition}
	Let $K$ be a discretely valued field and $L$ a finite extension of $K$, such that the valuation on $L$ extends the valuation on $K$. 
	Then the \emph{ramfication degree of  $L$ over  $K$} is the ramification index of the induced map between valuation rings, or equivalently it is 
	\[
	e_{L / K}  = \frac{v_L(\pi)}{v_L(\tau)}	,\] 
	where $\pi, \tau$ are uniformiser of $K, L$ respectively. 
\end{definition}

It is easy to see that the ramification index of a map between DVRs / discretely valued fields, $f: A \to B$ can be computed as $v_B(f(a)) / v_A(a)$ where $a$ is any element of $A$ with nonzero valuation, if the valuations are scaled such that the valuation of a uniformiser is 1. 

\begin{lemma}\label{lem:multiplicative_ramification_degree}
	Let $A \xrightarrow f B \xrightarrow g C$ be local maps of DVRs (or discretely valued fields) with respective valuations $v_A, v_B, v_C$. 
	Then $e_{g \circ f} = e_g \cdot e_f$.
\end{lemma}
\begin{proof}
	Let $\pi_A, \pi_B, \pi_C$ be uniformisers for $A, B, C$ respectively. 
	For the sake of simplicity we may assume that all valuations are scaled such that the valuations of their uniformisers are $1$.  
	Then 
	\[
		e_{g \circ f} = v_C(g(f(\pi_A)) = e_g \cdot  v_B(f(\pi_A)) = e_g\cdot e_f
	.\] 
\end{proof}

\begin{definition}
	Let $f: A \to B$ be a local map between DVRs. 
	Let $k, \ell$ be the residue fields of $A, B$ respectively and $p = \ch k$.
	Let $e$ be the ramification index of $f$.
	Then if
	\begin{itemize}
		\item $e = 1$, and $\ell / k$ is separable then $f$ is \emph{unramified}, 
		\item $e > 1$, then $f$ is \emph{ramified},
	\end{itemize}
	Moreover, if $f$ is ramified, $p \nmid e$ and $\ell / k$ is separable, then $f$ is \emph{tamely ramified}. 
	If $p \mid e$, $f$ is said to be \emph{wildly ramified}. 
\end{definition}
As we will see later in this thesis, morphisms between DVRs with tame ramification are much better behaved than those with wild ramification. 

\begin{lemma}\label{lem:generator_canonical_bundle_DVR}
	Let $f: A \into B$ be a map of DVRs with respective fraction fields $K, L$ and residue fields $k, \ell$ such that $\ell / k$ and $K / L$ are finite separable extension. 
	Let $\pi$ be a uniformiser of $B$. 
	Then $d(\pi)$ is a generator of $\Omega_{B / A}$. 
\end{lemma}
\todo{fix this lemma}
\begin{proof}
	Any element $b \in B$ can be written as $b = u \cdot \pi^{n} $ for some unit $u \in B$ and $n \ge 0$. 
	Let $f = \sum_{i = 0}^{n}  a_i x^{i} \in K[x]$ be a monic polynomial such that $f(u) = 0, f'(u) \ne 0$. 
	Then  \[
		0 = v(u^{n}) = v\left(- \sum_{i = 0}^{n-1}\right) \le v(a_i u^{i}) = v(a_i) 
	.\] 
	So $f \in A[x]$. 
	Then we compute differentials \[
		0 = d(0) = d(f(u)) = f'(u) d(u)
	.\] 
	As $f'(u) \ne 0$ we find $d(u)$ must be zero. 
	Thus \[
		d(b) = d(u\cdot \pi^{n}) = \pi^{n}\cdot d(u) + u\cdot  n \pi^{n-1}d(\pi) = un\pi^{n-1} d(\pi)
	.\] 
	So we find that $d(b)$ is a multiple of $d(\pi)$. 
	The result now follows because $\Omega_{B / A}$ is generated by $\{d(b) \st b \in B\} $.  
\end{proof}
\nomenclature{$\Omega_{X / Y}$}{The relative sheaf of Kähler differentials of a morphism of schemes $X \to Y$}

The following lemma shows that computing the normalization of a DVR $R$ in a field $L$, is really the same as finding all the valuations on $L$ that extend the valuation on $R$. 
\begin{lemma}\label{lem:normalisation_extension_norm}
	Let $K$ be a discretely valued field with valuation $v$, such that $R = K^{\circ}$ is Nagata. 
	Let $L$ be a finite extension of $K$ and $R'$ be the normalisation of $R$ in $L$. 
	Then there is a natural bijection between the maximal ideals of $R'$ and the extensions $v $ to $L$. 
\end{lemma}
\begin{proof}
	As normalisation is finite (\stacks{0AVK}) and local we know that $R'$, as a scheme consists of a generic point with fraction field is $L$, and a finite number of closed points, corresponding to maximal ideals $\mathfrak{m}_1, \ldots, \mathfrak{m} _n $ in $R'$. 
	Let $x_1, \ldots, x_n$ be the corresponding closed points in $\spec R'$. 
	Then $\mathcal{O}_{x_i, \spec R'}$ is a DVR with fraction field $L$, and thus induces a valuation on $L$. 
	We easily see that (with correct scaling) this valuation extends the valuation on $K$. 

	Conversely, let $v_L$ be a valuation on $L$ extending the valuation on $K$. 
	Let $b$ be an element in $R'$, and $f = \sum_{i = 0}^{r} a_i t^{i}$ be a monic polynomial in $R[t]$ such that $f(b) = 0$. 
	Then \[
		rv_L(b) = v_L(a^{b}) = v_L \left(-\sum_{i = 0}^{r-1} a_i b^{i}\right) \ge \min_{i} v(a_i) + iv_L(b)
	.\] 
	From this it follows that $v_L(b) \ge 0$. 
	So $R' \subset L^{o}$.  

	Let $\mathfrak{p}  = \{a \in R' \st v_L(a) > 0\} $, which we see to be non-empty and different from $(0)$. 
	Then clearly $\mathfrak{p} $ is a prime ideal.
	Indeed for $a, b \in R$  such $v(a \cdot b) = v(a) + v(b) > 0$, we find either $v(a) > 0$ or $v(b) > 0$, $a, b \in L^{o}$. 
	As $R \to R'$ is finite, we obtain that $\mathfrak{p} $ is a maximal ideal of $R$. 
\end{proof}

\begin{lemma}
	Let $K$ be a complete discretely valued field and $L$ be a finite extension of $K$.	Let $R = K^{o}$  be the value field of $K$ and $R'$ be the normalisation of $R$ in $L$. 
	Then $R' = L^{o}$. 
\end{lemma}
\begin{proof}
	These maximal ideals induce different discrete valuations on $R'$ that, which extend to different valuations on the fraction field $L$. 
	As the extension of the norm to $L$ is unique (\cref{thm:norm_finite_field_ext}) we see that $R'$ has exactly one maximal ideal $\mathfrak{m}_{R'}$. 
	In particular the valuation induced by $\mathfrak{m}_{R'} $ is the same as the one induced by $L^{oo} \subset L^{o}$. 
	So both $R'$ and $L^{o}$ are the rings of all elements with non-negative valuation in $L$. 
\end{proof}

Complete local DVRs have another nice property. They satisfy a version of Hensel's lemma. 
\begin{lemma}
	Let $R$ be a complete DVR. 
	Then $R$ is a hensilian ring.
	That means that for any monic polynomial $f \in R[x]$ and  $a \in k$, such that $\overline{f}(a) = 0$ and $\overline{f'}(a)$, there is a element $b \in R$ such that $f(b) = 0$ and $\overline{b} = a$.
	Here $k$ is the residue field of $R$ and $\overline{b}, \overline{f}$ are images in of $b, f$ in $k, k[x ]$ respectively. 
\end{lemma}
\begin{proof}
	See \stacks{04GE}.
\end{proof}


\begin{proposition}\label{prop:balancing_valuations}
	Let $K$ be a discretely valued field with valuation $v_K$,  value ring $R$ and residue fields $k$. 
	Let $L$ be a finite extension of $K$ and $v_1, \ldots, v_n$ be the extensions of $v_R$ to $L$. 
	Let $L_1^{o}, \ldots, L_n^{o}$ be the value rings of $L$ equipped with the norms $v_1, \ldots, v_n$ respectively, and let $\ell_1, \ldots, \ell_n$ be their respective residue fields. 
	Then \[
		[L: K] = \sum_{i= 1}^{n} e_{L_i^{o} / R} \cdot [\ell_i: k]
	.\] 
\end{proposition}
\begin{proof}
	Let $\mathfrak{m} = (\pi) $ be the maximal ideal of $R$
	Let $R'$ be the normalisation of $R$ in $L$. 
	Then this is exactly \stacks{09E8}.
\end{proof}

Sometimes it is useful to turn a discretely valued field into a valued field with algebraically closed residue field, without changing the value group, as taking the algebraic closure would do. 
For this the following notion is right. 
\begin{definition}
	Let $K$ be a discretely valued field. 
	Then we write $K^{\text{un}}$ for a maximally unramified extension of $K$. 
\end{definition}
If $K$ is complete, then $K^{\text{un}}$ might no longer be complete. But we can resolve this by simply considering $\hat{K^{\text{un}}}$. 
 



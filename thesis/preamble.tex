\RequirePackage{fix-cm}
\documentclass[12pt,a4paper,oneside]{book}

\usepackage{graphicx,xcolor,textpos}
\usepackage{helvet}
\usepackage[utf8]{inputenc}
\usepackage[T1]{fontenc}
\usepackage{textcomp}
\usepackage{amsmath, amssymb, amsthm}
%\usepackage{MnSymbol}
\usepackage{tikz-cd}
\usepackage{mdframed}
\usepackage{longtable}
\usepackage{microtype}
\usepackage{hyperref}
\usepackage{cleveref}
\usepackage{stmaryrd}
\usepackage{mathrsfs}
\usepackage{comment}
\usepackage{float}
\usepackage{rotating}
\usepackage[left,final]{showlabels}
\renewcommand{\showlabelsetlabel}[1]
{\begin{turn}{-100}\showlabelfont #1\end{turn}}
\renewcommand{\showlabelfont}{\scriptsize\ttfamily}
\usepackage{nomencl} 
\renewcommand{\nomname}{List of Symbols}
\newcommand{\myprintnomenclature}{%
  \cleardoublepage%
  \printnomenclature%
  \chaptermark{\nomname}
  %\addcontentsline{toc}{chapter}{\nomname} %% comment to exclude from TOC
}
\makenomenclature

\usepackage{glossaries} % For list of abbreviations
\newcommand{\glossname}{List of Abbreviations}
\newcommand{\myprintglossary}{%
  \renewcommand{\glossaryname}{\glossname}
  \cleardoublepage%
  \printglossary[title=\glossname]
  \chaptermark{\glossname}
  %\addcontentsline{toc}{chapter}{\glossname} %% comment to exclude from TOC
}
\makeglossaries%

\usepackage[backend = biber, style = alphabetic, backref=true]{biblatex}
\addbibresource{references.bib}



\topmargin -10mm
\textwidth 160truemm
\textheight 240truemm
\oddsidemargin 0mm
\evensidemargin 0mm
% ---------------------- textpos settings ----------------------------
% Some additional settings for the cover
% --------------------------------------------------------------------

\definecolor{green}{RGB}{172,196,0}
\definecolor{bluetitle}{RGB}{29,141,176}
\definecolor{blueaff}{RGB}{0,0,128}
\definecolor{blueline}{RGB}{82,189,236}
\setlength{\TPHorizModule}{1mm}
\setlength{\TPVertModule}{1mm}


% figure support
\usepackage{import}
\usepackage{xifthen}
\setlength{\marginparwidth}{2cm}
\usepackage[textsize=tiny,colorinlistoftodos]{todonotes}
\newcommand{\question}[1]{\todo[color=green!40]{#1}}
\newcommand{\sidecomment}[1]{\todo[color=yellow!70]{#1}}
\newcommand{\todoin}[1]{\todo[inline, noinlinepar]{#1}}
\usepackage{pdfpages}
\usepackage{transparent}
\newcommand{\incfig}[1]{%
	\def\svgwidth{\columnwidth}
	\subimport{./figures/}{#1.pdf_tex}
}
\newcommand{\incfigsmall}[1]{%
	\def\svgwidth{.15\columnwidth}
	\subimport{./figures/}{#1.pdf_tex}
}
\pdfsuppresswarningpagegroup=1

\newcommand{\shorthat}[1]{\hat{#1}}
\renewcommand{\hat}[1]{\widehat{#1}}


\newcommand{\N}{\mathbb{N}}
\newcommand{\Z}{\mathbb{Z}}
\newcommand{\Q}{\mathbb{Q}}
\newcommand{\C}{\mathbb{C}}
\newcommand{\R}{\mathbb{R}}
\newcommand{\F}{\mathbb{F}}
\newcommand{\pro}{\mathbb{P}}
\newcommand{\aff}{\mathbb{A}}
\newcommand{\ltr}{\par \noindent \framebox[1\width]{ $\implies$ } \hspace{.2cm}}
\newcommand{\rtl}{\par \noindent \framebox[1\width]{ $\impliedby$ } \hspace{.2cm} }
\let\oldphi\phi
\let\phi\varphi
\let\varphi\oldphi
\let\oldeps\epsilon
\let\epsilon\varepsilon
\let\varepsilon\epsilon
\newcommand{\birat}{\dashrightarrow}




\DeclareMathOperator{\coker}{coker}
\DeclareMathOperator{\id}{Id}

\DeclareMathOperator{\im}{Im}
\DeclareMathOperator{\spec}{Spec}
\DeclareMathOperator{\maxspec}{MaxSpec}
\DeclareMathOperator{\spf}{Spf}
\DeclareMathOperator{\proj}{Proj}
\DeclareMathOperator{\red}{red}
\DeclareMathOperator{\trop}{trop}
\DeclareMathOperator{\trdeg}{TrDeg}
\DeclareMathOperator{\divisor}{div}
\DeclareMathOperator{\Divisor}{Div}
\DeclareMathOperator{\cov}{Cov}
\DeclareMathOperator{\eval}{eval}
\DeclareMathOperator{\wt}{wt}
\DeclareMathOperator{\sk}{Sk}
\DeclareMathOperator{\ch}{char}
\DeclareMathOperator{\ra	m}{Ram}
\DeclareMathOperator{\mult}{mult}
\DeclareMathOperator{\bl}{Bl}
\DeclareMathOperator{\length}{length}
\DeclareMathOperator{\supp}{supp}
\DeclareMathOperator{\ram}{Ram}
\DeclareMathOperator{\val}{val}
\DeclareMathOperator{\minloc}{minloc}
\DeclareMathOperator{\ord}{ord}
\DeclareMathOperator{\lcm}{lcm}
\DeclareMathOperator{\ff}{Frac}
\DeclareMathOperator{\gal}{Gal}



\newcommand{\into}{\hookrightarrow}
\newcommand{\onto}{\twoheadrightarrow}
\newcommand{\Gan}{\ensuremath{\mathbb{G} _m ^{\mathrm{an}}}}
\newcommand{\an}{{}^{\text{an}}}
\newcommand{\bir}{{}^{\text{bir}}}
\newcommand{\act}{\curvearrowright}

\newcommand{\cox}{\widehat{\otimes}}

\newcommand{\st}{%
  \nonscript\;
  \ifnum\currentgrouptype=16
    \,\middle|\,
  \else
    \,|\,
  \fi
  \nonscript\;}

\newcommand{\step}[1]{\medskip \par\noindent\textit{Step #1.} \;}

\theoremstyle{definition}
\newtheorem{definition}{Definition}[section]
\newtheorem{notation}[definition]{Notation}
\newtheorem{remark}[definition]{Remark}
\newtheorem{theorem}[definition]{Theorem}
\newtheorem{example}[definition]{Example}
\newtheorem{lemma}[definition]{Lemma}
\newtheorem{claim}[definition]{Claim}
\newtheorem{proposition}[definition]{Proposition}
\newtheorem{exercise}[definition]{Exercise}
\newtheorem{corollary}[definition]{Corollary}
\newtheorem{problem}[definition]{Problem}
\newtheorem{observation}[definition]{Observation}
\newmdenv[linecolor=black,innertopmargin=1em,innerbottommargin=1em, rightmargin=1em, , leftmargin=1em]{tateconclusion}

\newcommand{\stacks}[1]{\cite[\href{https://stacks.math.columbia.edu/tag/#1}{Tag #1}]{stacks-project}}


\newglossaryentry{DVR}
{
name={DVR},
description={discrete valuation ring}
}
\newglossaryentry{nc}
{
name={nc},
description={normal crossings}
}
\newglossaryentry{snc}
{
name={snc},
description={strict normal crossings}
}

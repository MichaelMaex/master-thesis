\question{Als je je afvraagt waarom het gerbuik van mathmode hier heel spaars is. Ik moet dit als plain text kunnen uploaden bij het indienen van de thesis.}
The goal of this thesis is to study the reduction type elliptic curves over complete discretely valued fields using Berkovich geometry. 
To do this we first give introductions to the prerequisites that go beyond basic scheme theory.
This is split up in two parts. First we cover all the prerequisites from algebraic and arithmetic geometry in chapter 2-4. 
Then we introduce Berkovich geometry in chapter 5 and 6. 

\medskip

In chapter 2 we cover valuation theory. In particular the definition of norms and valuations on rings, non-archimedean norms, discretely valued fields, maps between discretely valued fields, the relation between normalisation of rings and extensions of norms, and basic theorems about these objects. 

Chapter 3 covers intersection theory on general and arithmetic surfaces. 
In particular we use this to define different types of models of curves, which will be very useful for studying Berkovich geometry later on. Most importantly we cover minimal regular and minimal strict normal crossings models.

Chapter 4 describes the Kodaira-Neron classification of elliptic curves, which classifies elliptic curves by the combinatorial properties of the special fiber of their minimal regular model. 
This is important because our end goal is to understand the relation between the Kodaira-Neron type of an elliptic curve and its Berkovich analytification.

\medskip


In chapter 5 we aim to give a general introduction to Berkovich geometry without fixating on a specific goal. 
We start by motivating Berkovich geometry as good theory to analytify varieties over other fields than $\R$ and $\C$. 
Then we get comfortable with the notion of a Berkovich analytification using a purely topological definition for the Berkovich spectrum of a ring. 
We work this out for $\Z$ and the polynomial ring $K[T]$, which gives the analytification of the affine line and a good opportunity to understand the classification of points of the analytification of a curve. 
We introduce the theory of affinoid algebras to define a structure sheaf on the Berkovich spaces and introduce the full notion of Berkovich analytic spaces. 
We then study the Berkovich analytification of a variety and the generic fiber of a formal scheme as particular analytic spaces coming from algebraic spaces. 

In chapter 6 we specialize our understanding of Berkovich geometry to purely to curves over discretely valued fields. 
In particular we describe the analytification a curve as a limit of dual graphs of snc-model. 
Doing so we obtain a metric on the analytification and a first type of skeleta.
We then introduce weight functions and the essential skeleton, which are one of the main objects of study in this thesis. 
They will play a key role in our research.  

\medskip


After introducing the prerequisites we report on new research in chapter 7. 
The main question is determine the reduction type of an elliptic curve, given by the Weierstrass equation \[
	E: y^2 = x(x-1)(x-\lambda)
,\] 
where $v(\lambda) \ge 0$ and $v(1 + \lambda) = 0$ using Berkovich geometry.
We first use Tate's algorithm to obtain the reduction, which is useful to compare our results to. 
Then we study the degree two cover $f: E \to \pro^{1}_K$ using weight functions and skeleta to solve the problem when the residue characteristic of the ground field is not 2.
We call this case the tamely ramified case. 
When the residue characteristic of the ground field is 2, we say that we are in the wildly ramified case. 
In this case the behavior of the weight functions over the map $f$ are not well understood. 
So we attempt to solve the converse problem. 
We use the explicit blowups in the proof of Tate's algorithm to determine the image of the skeleton of $E$ in the Berkovich projective line.
We then use this to make conjectures about the behavior of the weight functions over the wildly ramified map $f$. 

In chapter 8 we gather some loose ends that we think would be worth studying further. 



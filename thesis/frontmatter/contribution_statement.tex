The main problem (\cref{prob:main_problem}) as well as the idea to use weight functions to compare the skeleta in the tamely ramified case (\cref{lem:sk_E_pulls_back}) as a starting point was given to me by Johannes Nicaise and Art Waeterschoot.
This is an idea that also appeared in \cite[\S 6.1]{brownEssentialSkeletonProduct2019}.
\Cref{prop:weightfunction_fullback} and its generalization in \cref{rem:weightfunction_fullback_art} about the behavior of weight functions under finite morphisms were handed to me by Johannes and Art to whom the first result (\cref{prop:weightfunction_fullback}) was already known before the start of the thesis. 
The generalization (\cref{rem:weightfunction_fullback_art}) was found towards the end of this thesis by Art.
\medskip

My own contributions are all computations and results from \cref{chap:a_berkovich_approach_to_classifying_elliptic_curves} not mentioned above, in particular
\begin{itemize}
	\item the idea to use the coordinate transformations performed in Tate's algorithm to describe the image of the essential skeleton of $E$ in $\pro^{1, \text{an}}_K$ which we use to 
		\begin{itemize}
			\item give a Berkovich geometric argument to deterimine whether $E$ is of type $\mathrm I_m$ or of type $\mathrm I_m^*$ (\cref{sec:distinguishing_between_im_and_ims}) in the tamely ramified case,
			\item describe the map $\sk(E) \to \pro^{1, \text{an}}_K$ in the wildly ramified case for a certain family of curves $E$ (\cref{sec:case_v_lambda_>_2_v_2_and_v_2_is_even}),
		\end{itemize}
	\item a way of comparing the metrics under morphisms of curves away from the branch locus by counting the number of divisorial points of certain multiplicities, (\cref{lem:equality_lenghts_preserve_mult,rem:length_edge_skeleton_cover}), which we use to determine $m$ when $E$ has reduction type $\mathrm I_m$ via Berkovich geometry, in the tame ramification case, 

	\item a description the skeleton of how a chain of multiplicity 2 curves can map onto the skeleton of a chain of multiplicity 1 curves (\cref{sec:determining_m_when_e_is_of_type_ims}), which gives a Berkovich geometric way to determine $m$ when $E$ has reduction type $\mathrm I_m^*$ in the tame ramification case,

	\item a complete Berkovich geometric answer to \cref{prob:main_problem} in the tame ramification case using the results above,

	\item and a partial description of the image of $\sk(E)$ in $\pro^{1, \text{an}}_K$ in the wild ramification case (\cref{sec:wild_ramification}), which led to some conjectures about the behavior of weight functions under wildly ramified covers (\cref{sec:weight_functions_and_wild_ramification}). 
\end{itemize}


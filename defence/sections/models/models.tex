\subsection{Normed fields} \label{sec:normed_fields}

\begin{frame}[fragile]
	\begin{minipage}[t]{.60\textwidth}
	\begin{definition}
		A \emph{normed ring} is a ring  $A$ with a function \[ 
			|\cdot |:A  \longrightarrow \R_{\ge 0}
		.\]
	such that 
	\begin{itemize}
		\item $|a| = 0 \iff a= 0$
		\item $|a \cdot b| = |a | \cdot |b|$ 
		\item $|a + b| \le |a| + |b|$
	\end{itemize}

	\medskip

	\only<2>{ $A$ is \emph{non-archimedean} if moreover}
	\only<1>{\phantom{ $A$ is \emph{non-archimedean} if moreover}}
	\begin{itemize}
		\item<2-> $|a + b| \le \max(|a|, |b|)$
\end{itemize}
	\end{definition}
	\end{minipage}
	\;\;
	\begin{minipage}[t]{.35\textwidth}
		\begin{examples}
		\only<1>{\phantom{archimedean:}}
		\only<2>{archimedean:}
			\[
				\begin{aligned}
			\C,\quad \R,\quad \Z,\quad \Q\\
			\mathrm{abs}:a \mapsto |a|
				\end{aligned}
		\]

		\bigskip

		\only<1>{\phantom{non-archimedean:}}
		\only<2>{non-archimedean:}
		\begin{itemize}
				\item<2-> $\Z_p,\quad |a\cdot p^{n}| = p^{-n}$ $\leadsto \Q_p$
				\item<2-> $\C\llbracket t \rrbracket,\quad |a\cdot t ^{n}| = e^{-n}$ $\leadsto \C((t))$
		\end{itemize}	
	\end{examples}
	\end{minipage}
\end{frame}
\begin{frame}
	\begin{minipage}[t]{.70 \textwidth}
	\begin{notation}
	\begin{itemize}
		\item $K$ non-archimedean discretely valued complete normed field with $\ch K = 0$
		\item $R = \{a \in K \st |a| \le 1\} $ 
		\item $\mathfrak{m} = \{a \st |a| < 1\} = (\pi)$, $\pi$ is \emph{uniformiser}
		\item $k = \frac{R}{\mathfrak{m} }$ is algebraically closed
	\end{itemize}
	\end{notation}
	\end{minipage}
	\;\;
	\begin{minipage}[t]{.25\textwidth}
		\begin{example}
			\begin{itemize}
				\item $K = \C((t))$ 
				\item $R = \C\llbracket t \rrbracket $ 
				\item $\mathfrak{m}  = (t)$ 
				\item $k = \C$
			\end{itemize}
		\end{example}
		
	\end{minipage}
	\pause
	\bigskip

	\note{In language of algebraic geometry\ldots}

	Geometrically:
\begin{figure}
    \centering
    \incfig{inclusion-of-k-in-r}
\end{figure}

\end{frame}

\subsection{Models of curves} \label{sec:models_of_curves}
\begin{frame}
	\begin{definition}
		A \emph{curve} is a geometrically connected smooth projective $1$-dimensional $K$-variety.\\
		\only<1>{\phantom{A \emph{model} of a curve $C$  is a normal proper flat $R$-scheme $\sC$ such that $\sC_\eta \simeq C$.}}
		\only<2>{A \emph{model} of a curve $C$  is a proper flat $R$-scheme $\sC$ such that $\sC_\eta \simeq C$.}
	\end{definition}
	
	\only<1>{\incfig{first-model_1}}
	\only<2>{\incfig{first-model}}
\end{frame}


\begin{frame}{Terminology}
\begin{figure}[ht]
    \centering
    \incfig{terminology}
\end{figure}
\end{frame}


\begin{frame}
	What kinds of models exist?\[
		C = \proj \frac{K[X_1, \ldots, X_n]}{(f_1, \ldots, f_r)}
		\leadsto \sC = \proj \frac{R[X_1, \ldots, X_n]}{(f_1, \ldots, f_r)}
	.\] 

	\pause
	Resolution of singularities $\leadsto $ regular model 
	\[
		\sC _\text{reg}  \longrightarrow \sC 
	.\]

	\note{Choose equations for C.

	Classical result of AG.

Regular is a nice class, but}



\end{frame}
\begin{frame}
	\begin{definition}
		A strict normal crossing moddel (\emph{snc-model}) is 
		\begin{itemize}
			\item regular
			\item each irreducible component of $\sC_s$ with reduced scheme structure is regular 
			\item in each intersection point, exactly 2 components meet transversally
		\end{itemize}
	\end{definition}
	Further blowups: $\mathscr \sC_\text{snc} \longrightarrow \mathscr \sC_\text{reg} \longrightarrow \mathscr \sC$

    \incfig{snc-models}
\end{frame}


\begin{frame}
    \incfig{dual-graphs}
    Edges of $\Delta(C)$ have a length $\leadsto$ metric 
\end{frame}
\begin{frame}
	If $\sC_1, \sC_2$ are regular models of $C$, $f: \sC_1 \to \sC_2$.
\incfig{contraction}
	$\leadsto$ partial order the set of models of $C$. 
	\[
	\mathscr C_2 \ge \mathscr C_1
	.\] 

	\pause
\begin{theorem}
		If $C$ is a curve with $g(C) \ge 1$ then $C$ has a minimal regular model and minimal snc-model.
\end{theorem}
\end{frame}





\subsection{Kodaira-Neron classification} \label{sec:kodaira-neron_classification}

\begin{frame}{Kodaira-Neron Classification of Elliptic Curves}
	Classification of elliptic curves by the dual graphs of their minimal snc model
		\pause
	\begin{figure}[ht]
    \centering
    \incfig{kn-dual-graphs}
\end{figure}
\end{frame}

\begin{frame}
	Each edge in the dual graph has a length
\begin{figure}[ht]
    \centering
    \incfig{type-in-and-ins}
\end{figure}
\end{frame}






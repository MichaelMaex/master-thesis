\subsection{Problem statement} \label{sec:problem_statement}
\begin{frame}
	\begin{minipage}[t]{.7\textwidth}
	Elliptic curve $E$ over $K$
	\[
		E: y^2 = x(x-1)(x-\lambda), \quad \lambda = u\cdot \pi^{n}
	.\] 
	Consider the projection to $x$-axis 
	\[
	f: E \to \pro^{1}_K
	.\] 
	\end{minipage}
	\begin{minipage}[t]{.29\textwidth}
    \incfig{ramified-map}
	\end{minipage}

	\pause

	\begin{question}
	Find KN type of $E$/understand $E\an$ via \[
	\phi = f\an : E\an \to \pro^{1, \text{an}}_K
	.\] 
	\end{question}

\end{frame}
\begin{frame}<presentation:0>[t,noframenumbering]
	${\color{orange} \Gamma_\lambda} = $ convex hull of $0, \lambda, 1, \infty$ in $\pro^{1, \text{an}}_K$.
\begin{figure}[ht]
    \centering
    \incfig{gamma-lambda}
\end{figure}
\end{frame}



\begin{frame}
	\only<1>{\incfig{skeleton-pair-0}}
	\only<2>{\incfig{skeleton-pair}}
	\pause
Line $[a, b] = \sk(\mathscr P)$ has length  $n$. 
\end{frame}
\begin{frame}{Two options}
	\begin{minipage}[t]{.48\textwidth}
		\begin{enumerate}
		\item If $\ch k \ne 2$, \emph{tame ramification}: \begin{align*}
				\underbrace{\wt_{f^*\sigma}}_{\text{on }E\an} &= \underbrace{\wt_{\sigma}}_{\text{on }\pro^{1, \text{an}}_K} \mathbin{\circ} \phi
			\end{align*}
			\pause
			\[
				\implies \sk(E) = \phi^{-1}(\sk(\sP))
			\]
	\end{enumerate}
	\pause

	\end{minipage}
	\;
	\begin{minipage}[t]{.48\textwidth}
	\begin{enumerate}
		\setcounter{enumi}{1}
		\item If $\ch k = 2$, \emph{wild ramification}:
			\[
				\wt_{f^* \sigma} = \wt_{\sigma} \circ \phi + \underbrace{\mathfrak{d} _f}_{\text{error}}
			.\] 
			\pause
			\[
				\redoutb{\text{$\implies \sk(E) = \phi^{-1}(\sk(\sP))$}}
			.\] 
	\end{enumerate}
	\end{minipage}
	\pause
	\medskip
	\begin{block}{Assumption}
		For now $\ch k \ne 2$
	\end{block}

\end{frame}

\subsection{Tame ramification} \label{sec:tame_ramification}

\begin{frame}
	\begin{block}{Fact}
			$|\phi^{-1}(y)| = 1 \text{ or } 2$
		\end{block}

    \incfig{sk-e-to-p-2}

    \pause
		\only<1,2>{
		\begin{align*}
			n > 0 &\iff E \text{ has type } \mathrm I_m \text{ or } \mathrm I_{m}^*, m > 0 \\
			n = 0 &\iff E \text{ has type } \mathrm{I}_0, \mathrm I_0^*, \mathrm{II}, \mathrm{II}^*,\mathrm{III}, \mathrm{III}^*, \mathrm{IV}, \mathrm{IV}^*
	.\end{align*}}
		\only<3>{
		\begin{align*}
			n > 0 &\iff E \text{ has type } \mathrm I_{2n}  \\
			n = 0 &\iff E \text{ has type } \mathrm{I}_0
	.\end{align*}}
\end{frame}
\begin{frame}<presentation:0>[t,noframenumbering]
	\begin{minipage}[t]{.7\textwidth}
		
		\begin{block}{Fact}
			$|\phi^{-1}(y)| = 1 \text{ or } 2$
		\end{block}
		\begin{minipage}[t]{.49\textwidth}
			$\sk(E)$ is either
			\begin{itemize}
				\item circle if $\mathrm I_m, m >0$
				\item line if $\mathrm I_{m}^*, m > 0$
				\item point otherwise
			\end{itemize}
		\end{minipage}
		\begin{minipage}[t]{.49\textwidth}
			$\sk(\mathscr P)$ is either
			\begin{itemize}
				\item line if $n > 0$ 
				\item point if $n = 0$
			\end{itemize}
		\end{minipage}
		
		\only<2>{
		\begin{align*}
			n > 0 &\iff E \text{ has type } \mathrm I_m \text{ or } \mathrm I_{m}^*, m > 0 \\
			n = 0 &\iff E \text{ has type } \mathrm{I}_0, \mathrm I_0^*, \mathrm{II}, \mathrm{II}^*,\mathrm{III}, \mathrm{III}^*, \mathrm{IV}, \mathrm{IV}^*
	.\end{align*}}
		\only<3>{
		\begin{align*}
			n > 0 &\iff E \text{ has type } \mathrm I_{2n}  \\
			n = 0 &\iff E \text{ has type } \mathrm{I}_0
	.\end{align*}}
	\end{minipage}
	\begin{minipage}[t]{.29\textwidth}
    \incfig{sk-e-to-p}
	\end{minipage}
\end{frame}
\begin{frame}<presentation:0>[t,noframenumbering]
	If $E$ has type $\mathrm I_m$, i.e. $\sk(E)$ is a circle.
	\begin{figure}
		\incfig{argument-in}
	\end{figure}
	\pause
    $\leadsto$ $E$ is of type $\mathrm I_{2n}$.
\end{frame}

\subsection{Wild ramification} \label{sec:wild_ramification}

\begin{frame}
	{Wild ramification, $\ch k = 2$}

	Problem: \[
		\wt_{f^*\sigma} = \wt_{\sigma} \circ \phi {\color{orange} +  \mathfrak{d} _f}
	.\] 
	\medskip

	\pause

	Approach:
	\begin{itemize}
		\item Determine reduction type of $E $ using Tate's algorithm.
		\item Carefully analyse proof of Tate's algorithm to determine $\sk(E) \to \pro^{1, \text{an}}_{K}$
	\end{itemize}
	\pause 
	\bigskip

	Goal (longterm):
	\begin{itemize}
		\item Understand $\mathfrak{d} _f$.
	\end{itemize}
\end{frame}
\begin{frame}
	$v(2) =$ maximal $n$ such that $\pi^{n} \mid 2$.  

\begin{figure}[ht]
    \centering
    \incfig{wild-skeleta}
\end{figure}

\end{frame}
\begin{frame}
	If $n < 2v(2)$, excluding ``edge cases''
\begin{figure}[ht]
    \centering
    \incfig{wild-skeleta-2}
\end{figure}
\end{frame}

\begin{frame}
	{Questions \& main references}



	
	Model theory:
	\begin{itemize}
		\item \S 8-10 in \emph{Algebraic Geometry and Arithmetic geometry}, by {Q.\ Liu}
		\item \S III.8-9, \S IV.4-9 in \emph{Advanced topics in the Advanced topics in the arithmetic of elliptic curves}, by {J.\ Silverman}
	\end{itemize}

	Berkovich geometry:
	\begin{itemize}
		\item \emph{An introduction to Berkovich analytic spaces and non-archimedean potential theory on curves}, by  {M.\ Baker}
		\item \emph{Introduction to Berkovich Analytic Spaces}, by {M. Temkin}
	\end{itemize}

	Weight functions:
	\begin{itemize}
		\item \emph{Weight functions on non-Archimedean analytic spaces and the Kontsevich-Soibelman skeleton}, by {J.\ Nicaise, M.\ Mustaţă}
		\item \emph{Weight functions on Berkovich curves}, by {M.\ Baker, J.\ Nicaise}
	\end{itemize}
\end{frame}

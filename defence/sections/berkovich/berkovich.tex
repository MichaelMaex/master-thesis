
\subsection{Motivation}\label{sec:motivation}
\begin{frame}{Berkovich geometry}
	Complex geometry: interplay algebra over $\C$ \& analysis over $\C$. 
	\begin{align*}
		(\cdot )\an : \C\text{-variety} &\longrightarrow \C\text{-manifold} \\
		X &\longmapsto X\an = \text{closed points of }X 
	.\end{align*}
	\pause

	$K$ is complete normed field $\leadsto$ analysis over $K$ 
	\pause
	\only<1,2,3>{\begin{align*}
		(\cdot )\an: K\text{-curve} \longrightarrow& K\text{-analytic space} \\
		C \longmapsto& \text{$C\an =$ closed points of $C$ \phantom{$\cup $ norms on $K(C)$}}
  .\end{align*}}
	\only<4>{\begin{align*}
		(\cdot )\an: K\text{-curve} \longrightarrow& K\text{-analytic space} \\
		C \longmapsto& \text{$C\an =$ closed points of $C$ {\color{orange} $\cup $ norms on $K(C)$}}  
.\end{align*}}

\only<1,2,3>{\phantom{[VB 1990]}}
\only<4>{[VB 1990]}
\end{frame}

\begin{frame}{Projective Line $\pro^{1, \text{an}}_{K}$}
    \incfig{projective-line}
    \note{Too large to draw, imagination

    Closed points, disconnected

    In interior points

    Toplogically hausdorff, compact, path connected

    Kinda a graph}
\end{frame}

\subsection{Dual graphs and skeleta} \label{sec:dual_graphs_and_skeleta}
\begin{frame}{Skeleta}
	\begin{minipage}{.49\textwidth}
	 $\sC$ an snc model of  $C$. 
	 \begin{align*}
		 i_{\sC}: \Delta(\sC) \into C\an
	 .\end{align*}
	\end{minipage}\;\;
	\begin{minipage}{.48\textwidth}
	 \begin{definition}
		 The \emph{skeleton} of $\mathscr C$
	 	\[
			\sk(\mathscr C) = \im (i_{\mathscr C})
	 	.\] 
	 \end{definition}
	\end{minipage}

	\bigskip
	\pause

	 Conversely, \[
		 C\an = \varprojlim_{\mathscr C \text{ snc-model}} \sk(\mathscr C)
	 .\] 

	 \medskip 
	 Bonus: ``$C\an$ is a metric space''
\end{frame}


\begin{frame}
\begin{figure}[ht]
    \centering
    \incfig{embedding-skeleta}
\end{figure}
\pause
$i_{\mathscr C}, r_{\sC}$ are homotopy inverses.
\begin{block}{Slogan}
	$\sk(\sC)$ controls the homotopy tye of $C\an$
\end{block}
\end{frame}
\subsection{Weight functions and the essential skeleton} \label{sec:weight_functions_and_essential_skeleta}

\begin{frame}
	\begin{minipage}{.68\textwidth}
		
		Weight functions {\footnotesize[K,S,06, JN,MM,15]}: $\sigma \in \omega_{C}^{\otimes n}(C), \sigma \ne 0$ \[
		\wt_{\sigma}: C\an \to \left(-\infty, \infty\right]	
	.\]
	\vspace{-.7cm}
	\begin{block}{Slogan}
		$\wt_{\sigma}$ increases away from $\sk(\sC)$ for every $\sC$.
	\end{block}
	\pause
	$\leadsto  \minloc(\wt_{\sigma})$ is a piece of every skeleton. 

	\pause
	
	\begin{definition}[JN, MM]
		If $g(C) > 0$ the \emph{essential skeleton of $C$} is \[
			\sk(C) = \bigcup_{\sigma } \minloc(\wt_{\sigma})
		.\] 
		\vspace{-.4cm}
	\end{definition}
	\end{minipage}
	\;
	\begin{minipage}{.29\textwidth}
\begin{figure}[ht]
    \centering
    \incfig{essential-skeleton}
    \vspace{-.5cm}
    \caption{$\sk(C)$ from  $\sC_\text{min}$ [JN, MB, 2016]}
    \label{fig:essential-skeleton}
\end{figure}

	\end{minipage}
\end{frame}

\begin{frame}
	\begin{table}[htpb]
		\centering
		\caption{The essential skeleton $\sk(E)$ for elliptic curves by KN type }
		\label{tab:label}
		\begin{tabular}{l|c|c|c}
		type & $\mathrm I_{n}, n > 0$ & $\mathrm I_{n}^*, n > 0$ & others \\
		\hline
		$\sk(E)$ & \incfigsmall{circle} & \incfigsmall{line}&\incfigsmall{point}
		\end{tabular}
	\end{table}
\end{frame}



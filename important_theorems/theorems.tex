\documentclass[a4paper]{report}

\usepackage[utf8]{inputenc}
\usepackage[T1]{fontenc}
\usepackage{textcomp}
%\usepackage[dutch]{babel}
\usepackage{amsmath, amssymb, amsthm}
\usepackage{geometry}
\usepackage{tikz-cd}
\usepackage{mdframed}
\usepackage{microtype}
\usepackage{hyperref}
\usepackage{todonotes}
\usepackage{libertine}

% bib(la)tex
\usepackage[backend=biber]{biblatex}
\addbibresource{references.bib}

% figure support
\usepackage{import}
\usepackage{xifthen}
\usepackage{todonotes}
\usepackage{pdfpages}
\usepackage{transparent}
\newcommand{\incfig}[1]{%
	\def\svgwidth{\columnwidth}
	\import{./figures/}{#1.pdf_tex}
}

\pdfsuppresswarningpagegroup=1


\newcommand{\N}{\mathbb{N}}
\newcommand{\Z}{\mathbb{Z}}
\newcommand{\Q}{\mathbb{Q}}
\newcommand{\C}{\mathbb{C}}
\newcommand{\R}{\mathbb{R}}
\newcommand{\ltr}{\par \noindent \framebox[1\width]{ $\implies$ } \hspace{.2cm}}
\newcommand{\rtl}{\par \noindent \framebox[1\width]{ $\impliedby$ } \hspace{.2cm} }
\newenvironment{exercise}{
\begin{mdframed}
}
{
\end{mdframed}
}

\DeclareMathOperator{\coker}{coker}
\DeclareMathOperator{\id}{Id}
\DeclareMathOperator{\im}{Im}
\DeclareMathOperator{\divisor}{Div}
\DeclareMathOperator{\spec}{Spec}
\DeclareMathOperator{\maxspec}{MaxSpec}
\DeclareMathOperator{\proj}{Proj}

\newcommand{\into}{\hookrightarrow}
\newcommand{\onto}{\twoheadrightarrow}

\theoremstyle{definition}
\newtheorem*{theorem}{Theorem}
\newtheorem*{lemma}{Lemma}
\newtheorem*{definition}{Definition}
\newtheorem*{proposition}{Proposition}
\newtheorem*{corollary}{Corollary}
\newtheorem{question}{Question}

\author{Micha\"el Maex}
\date{\today}
\title{Important theorems}
\begin{document}

\maketitle
{\color{red} I will just copy verbatim into this document if I do not feel like rephrasing. Be carefull when copying from here with regards to plagiarism.}
\tableofcontents

\pagebreak
\chapter{Non-archimedian geometry} \label{chap:non-archimedian_geometry}
\section{Affinoid algebras} \label{sec:affinoid_algebras}
Affinoid algebras are a type of rings, which are complete with respect to their norm. 
They will form the basis building block of a theory that mimics the scheme theory of ordinary rings. 

\subsection{Basics on norms} \label{sec:basics_on_norms}
\begin{definition}
	Let $k$ be normed field and $A$ a $k$-algebra. 
	A \emph{seminorm} on $A$ is a map $\|*\|:A \to \R_{> 0}$ such that for all $x, y \in A$ and $\lambda \in k$
	\begin{itemize}
		\item  $\|0\| = 0$. 
		\item $\|1\| = 1$ 
		\item $\|\lambda \cdot x\| = |\lambda| \cdot \|x\|$
		\item $\|x + y\| \le \|x\| + \|y\|$
		\item $\|x \cdot y\| \le \|x\| \cdot  \|y\|$
	\end{itemize}
	

	If a seminorm satisfies $\|x\| = 0 \implies x = 0$ then it is called a \emph{norm}.

	If a seminorm satisies $\|x \cdot  y\| = \|x\| \cdot \|y\|$ we say it is multiplicative. 

\end{definition}

\begin{definition}
	[2.1.2, \cite{conrad2008several}]
	let $(a, \|\cdot\|)$ be a $k$-banach algebra. (usually we take $a = k$). 
	the tate algebra over $a$ in $n$-variables \[
	a \left<y_1, \ldots, y_n \right> = \left\{\sum_{\nu \in \n^{n}}^{} a_\nu y^{\nu} \st a_\nu \to 0, \text{ as } \nu \to \infty\right\} 
	.\] 

	this can be thought of as the completing of the polynomial ring $a[x_1, \ldots, x_n]$ with the norm defined as \[
	\left\| \sum_{\nu \in \n^{n}} a_\nu y^{\nu}\right\| = \max_{\nu \in \n^{n}} \|a_\nu\|
	.\] 
	this naturally gives a similarly defined norm on $a\left<y_1, \ldots, y_n \right>$,but there are many other interesting norms on $a \left<y_1, \ldots, y_n \right>$. 
\end{definition}

\subsection{affinoid domains} \label{sec:affinoid_domains}

there are different types of affinoid domains that we often encounter. 

\begin{definition}[multiple definitions in \cite{conrad2008several}]
		there are \emph{weierstrass domains, laurent domains} and \emph{rational domains}. 
	\begin{description}
		\item[weierstrass domain] let $a_1, \ldots, a_n$ be in $a$. 
			\[
				a\left<a_1, \ldots, a_n \right> := \frac{a\left<x_1, \ldots, x_n \right>}{(x_1-a_1, \ldots, x_n - a_n)}
			.\] 
			unlike for polynomial rings, this is not an evaulation map. you should think of this as forcing the $a_i$'s to become powerbounded. 
		\item [laurent domains]
			let $a_1, \ldots, a_n, b_1, \ldots, b_m \in a$ the \[
				a\left<a_1, \ldots, a_n, b_1^{-1}, \ldots, b_m^{-1} \right> := \frac{a\left<x_1, \ldots, x_n,y_1, \ldots, y_m  \right>}{(x_1 - a_1, \ldots, x_n - a_n, b_1 y_1 - 1, \ldots, b_m y_m - 1)}
			.\] 
		\item[rational domain] with $a_1, \ldots, a_n, a' \in a$ with no common zeros, i.e. there is no $\mathfrak{m} $ that contains both of these elements.   
			\[
				a \left<\frac{a_1}{a}, \ldots, \frac{a_n}{a'} \right> := \frac{a\left<x_1, \ldots, x_n \right>}{(a' x_1 - a_1, \ldots, a' x_n - a_n)}
			.\] 
			this is essentially the affinoid version of localisation.
			but it does more than just make $a'$ invertible. 
			is also makes $\frac{a_i}{ a'}$ act like it is power bounded in some sense. 
		\item 
	\end{description}

\end{definition}
many authors define these domains by describing their space of points, be it max spec or the berkovich functor. 
the following lemma gives shows that we can use these descriptions interchangeably
\begin{lemma}
	[2.1.8 in \cite{conrad2008several}]
	suppose $a$ is a $k$-affinoid algebra and $a_1, \ldots, a_n, a_0 \in a$ be elements with no common zero (i.e. at no point in $ x \in \maxspec a$ are all $a_{i} \in \mathfrak{m}$).
	 then a map of $k$-affinoid algebras $\phi: a \to b$ factors through $a \left<\frac{a_1}{a_0}, \ldots, \frac{a_n}{a_0} \right>$ like in the diagram \[
	 \begin{tikzcd}
		 a \left<\frac{a_1}{a_0}, \ldots, \frac{a_n}{a_0} \right> \drar[dashed]  \\
		 a \uar \rar[']{\phi} & b
	 \end{tikzcd}
	 \] 
	 if and only if  $m\phi: \maxspec b \to \maxspec a$ factors through the subset $x \in m(a)$ such that $|a_j(x)| \le |a'(x)|$ for all $j$. 
\end{lemma}

\subsection{admissible opens, covers and $g$-topology} \label{sec:admissible_opens,_covers_and_g-topology}

\begin{definition}
	[2.2.6, \cite{conrad2008several}]
	let $a$ be a  $k$ affinoid algebra. 
	a subset $u$ of $\maxspec(a)$ is an \emph{admissible open} if is has a cover $\{u_i\} $ of affinoid subdomains $u_i$ satisfying the following property: 
	for any map of $k$-affinoid algebras $\phi: a \to b$ such that the induced map $f: m(b) \to m(a)$ lands in $u$, the pullback of the cover $\{f^{-1}(u_i)\} $ has a finite subcovering. 


	a covering $v = \bigcup_{i \in  i} v_i$ is an \emph{admissible cover} if for any any map $\phi:a \to b$  of $k$-affinoid algebras that the image of  $f: m(b) \to m(a)$ lies in $v$, the pullback cover $\{f^{-1}(v_i)\} $ has a refinement by a covering consisting of finitely many affinoid subdomains.  
\end{definition}
\begin{proof}
	\todo{look at this proof}
\end{proof}

we need this to define the notion of a $g$-topology, which is a grothendieck topology on the set $\maxspec(a)$. 
i'm pretty sure later we will scrap $\maxspec$ in favour of the berkovich space.




\section{Models} \label{sec:models}


This will be mostly theorems from \cite{liu}, \cite{silvermanarithmetic}.


\begin{definition}
	[fibered surface, 8.3.1 \cite{liu}]
	A \emph{fibered surface} is a integral, projective, flat $S$ scheme, $X \to S$ of dimension 2 with $S$ a dedekind scheme. 
\end{definition}
\begin{definition}
	[arithmetic surface, 8.3.14 \cite{liu}]
We will call a regular fibered surface $X \to S$ over a Dedekind scheme $S$ of dimension $1$ an \emph{arithmetic surface}.
\end{definition}

\begin{theorem}
	[Adjuction formula, 9.1.37 \cite{liu}] 
	Let $X \to S$ be a regular fibered surface, $s \in S$ a closed point, and $E \in \divisor_s(X)$	such that $0 < E \le X_s$ (the second inequality an empty condition if $\dim S =0$). 
	Then we have \[
		\omega_{E / k(s)} \simeq (\mathcal{O}_X(E) \otimes \omega_{X / s})|_E
	,\] 
	and if $K_{X / S}$ is a canonical divisor, \[
		p_a(E) = 1 + \frac{1}{2}(E^2 + K_{X / S} \cdot E)
	.\] 

\end{theorem}

The following seems to be a quite technical result, but Art marked it, so I guess it will be useful. 
\begin{lemma}
	[9.4.29 \cite{liu}]
	Let $\pi: W \to S = \spec A$ be a fibrede surface such that $W_\eta = E$, that $O := \overline{\{o\} }$ is contained in the smooth locus of $W$,
	and that $W_s$ is integral for every $s \in S$. 
	Let us consider $O$ as as a Cartier divisor on $W$. 
	Then the following properties are true. 
	\begin{enumerate}
		\item [(a)] for any $n \ge 2$, $\mathcal{O} _W(n O)$ is generated by its global sections. 
		\item [(b)] 
			The fheas $\mathcal{L}  = R^{1}\pi_*\mathcal{O} _W$ is invertible on $S$. 
			Let us suppose that it is free. 
			For any $n \ge 2$, there exists an exact sequence \[
				0 \to \pi_* \mathcal{O} _W((n -1)O) \to \pi_* \mathcal{O}_W(nO) \to \mathcal{L} ^{\otimes n} \to 0
			,\] 
			$\pi_* \mathcal{O}_W(n) $ is free of rank $n$, and the cannonical homomorphism \[
				\bigoplus_{2a + 3b \le n}(\pi_* \mathcal{O}_W(2O)) ^{ \otimes a} \otimes (\pi_* \mathcal{O}_W(3O)) ^{\otimes b} \to \pi_* O_W(nO)
			\] 
			is surjective. 
	\end{enumerate}
\end{lemma}

Definition of a \emph{minimal model}
\begin{definition}[9.4.33 \cite{liu}]
	Let $W $ be a Weierstraß model of $E$ over $S = \spec A$ with discriminant $\Delta_W \in A / A^*$. 
	We sya that $W$ is \emph{minimal} at $s$ if $\nu_s(\Delta_W)$ is the smallest among the valuations of the discriminats of the integral equation of $E$. 

	We say that the integer $\nu_s(\Delta_W)$ is the \emph{minimal discriminant} of $E$ at $s$. 

	$W$ is \emph{minimal} if it is minimal at every $s \in S$. 

\end{definition}

\begin{definition}[10.1.1 \cite{liu}]
	Let $S$ be a one-dimensional Dedekind scheme, with function field $K$.  
	Let $C$ be a normal, projective, connected curve over $K$. 
	Then a model of  $C$ is a fibred surface $\mathcal{C}  \to S$ such that there is an isomorphism $f: \mathcal{C} _\eta \simeq C$ over $K$. 
	I.e. the following diagram \[
	\begin{tikzcd}
		C \rar{f} \dar \ar[phantom]{dr}{\times }& \mathcal{C}  \dar \\
		K \rar & S
	\end{tikzcd}
	\] 
	is cartesian. 
	Note, the morphism $f$ is part of the model. 

	A morphism of two models $\mathcal{C} , \mathcal{C} '$ of $C$ is a morphism of $S$-schemes $\mathcal{C}  \to \mathcal{C} '$ such that the following diagram commutes
	\[
	\begin{tikzcd}[row sep =small ]
		& \mathcal{C}  \ar{dd} \\
		C \ar{ur} \ar{dr} \\
		& \mathcal{C} '
	\end{tikzcd}
	.\] 
\end{definition}

\begin{proposition}
	[VII.2.1 \cite{silvermanarithmetic}]
	There is an exact sequence of abelian groups \[
		0 \to E_1(K) \to E_0(K) \to \tilde E_\text{ns} (k) \to 0
	,\] 
	where  the right hand map is reduction modulo $\pi$ and 
	\begin{align*}
		E_0(K ) &= \{P \in E(K) \mid \tilde  P \in \tilde E_\text{ns} (k)\}\\
		E_1(K) &= \{P \in E(K)  \mid \tilde P = \tilde O\}  
	.\end{align*}
\end{proposition}

\begin{proposition}
	[10.1.8 \cite{liu}]
	Let us suppose $S $ is affine. Lec $C$ be a smooth projective curve of genus $g$ over $K$. 
	Then $C$ admits a relatively  minimal regular model (resp. a regular model with normal crossings) over  $S$. 
	If, moreover, $g \ge 1$ then $C$ admits a unique minimal regular model $C_\text{min}$, and a unique minimal regular model with normal crossings. 
	\todo{Using $C$ and $\mathcal{C} $ for two different curves is a horrible idea.}
\end{proposition}

\begin{lemma}
	[9.3.20 \cite{liu}]
	Let $Y \to S $ be a normal fibered surface. Let us supopse that $Y$ admits two regular models $X_1, X_2$ wihotu relation of domination between them. 
	Then there exists a regular model $Z$ of $Y$ that dominates the $X_i$ and an exceptional divisor $E_1$ on $Z$ contained in the exceptional locus of $Z \to X_1$ (in other words, the image of $E_1$ in $X_1$ is a point) such that 
	\begin{enumerate}
		\item Either the image of $E_1$ in $X_2$ is still an exceptional divisor. 
		\item or there exists another exceptional divisor $E_2$ on $Z$ and an integer $\mu \ge 1$ with $(E_1 + \mu E_2)^2 \ge 0$. 
	\end{enumerate}
\end{lemma}


\begin{corollary}
	[3.27 in \cite{liu}]
	Let $\pi: X \to S$ be a minimal arithmetic surface whose generic fiber is an ellpitic curve. 
	Then $\pi_* \omega_{X / S}$ is an invertible sheaf on $S$, and the canonical homomorphism $\pi^* \pi_* \omega_{X / S} \to \omega_{X / S}$ is an isomorphism. 
\end{corollary}

\pagebreak
\printbibliography

\chapter{Elementary questions} \label{chap:elementary_}
\begin{question}
	Does residue norm preserve multiplicativity?
	Does it preserve sub-multiplicativity?
\end{question}
\begin{question}
	Is de definitie van een admissible cover makkelijker in  berkovich space? Want de grote waarde van de admissible covers in rigid context is de space connected maken.
\end{question}


\end{document}
